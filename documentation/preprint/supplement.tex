% Options for packages loaded elsewhere
\PassOptionsToPackage{unicode}{hyperref}
\PassOptionsToPackage{hyphens}{url}
\documentclass[
]{article}
\usepackage{xcolor}
\usepackage[margin=1in]{geometry}
\usepackage{amsmath,amssymb}
\setcounter{secnumdepth}{-\maxdimen} % remove section numbering
\usepackage{iftex}
\ifPDFTeX
  \usepackage[T1]{fontenc}
  \usepackage[utf8]{inputenc}
  \usepackage{textcomp} % provide euro and other symbols
\else % if luatex or xetex
  \usepackage{unicode-math} % this also loads fontspec
  \defaultfontfeatures{Scale=MatchLowercase}
  \defaultfontfeatures[\rmfamily]{Ligatures=TeX,Scale=1}
\fi
\usepackage{lmodern}
\ifPDFTeX\else
  % xetex/luatex font selection
  \setmainfont[]{TeX Gyre Termes}
  \setmathfont[]{TeX Gyre Termes Math}
\fi
% Use upquote if available, for straight quotes in verbatim environments
\IfFileExists{upquote.sty}{\usepackage{upquote}}{}
\IfFileExists{microtype.sty}{% use microtype if available
  \usepackage[]{microtype}
  \UseMicrotypeSet[protrusion]{basicmath} % disable protrusion for tt fonts
}{}
\makeatletter
\@ifundefined{KOMAClassName}{% if non-KOMA class
  \IfFileExists{parskip.sty}{%
    \usepackage{parskip}
  }{% else
    \setlength{\parindent}{0pt}
    \setlength{\parskip}{6pt plus 2pt minus 1pt}}
}{% if KOMA class
  \KOMAoptions{parskip=half}}
\makeatother
\usepackage{longtable,booktabs,array}
\newcounter{none} % for unnumbered tables
\usepackage{calc} % for calculating minipage widths
% Correct order of tables after \paragraph or \subparagraph
\usepackage{etoolbox}
\makeatletter
\patchcmd\longtable{\par}{\if@noskipsec\mbox{}\fi\par}{}{}
\makeatother
% Allow footnotes in longtable head/foot
\IfFileExists{footnotehyper.sty}{\usepackage{footnotehyper}}{\usepackage{footnote}}
\makesavenoteenv{longtable}
\usepackage{graphicx}
\makeatletter
\newsavebox\pandoc@box
\newcommand*\pandocbounded[1]{% scales image to fit in text height/width
  \sbox\pandoc@box{#1}%
  \Gscale@div\@tempa{\textheight}{\dimexpr\ht\pandoc@box+\dp\pandoc@box\relax}%
  \Gscale@div\@tempb{\linewidth}{\wd\pandoc@box}%
  \ifdim\@tempb\p@<\@tempa\p@\let\@tempa\@tempb\fi% select the smaller of both
  \ifdim\@tempa\p@<\p@\scalebox{\@tempa}{\usebox\pandoc@box}%
  \else\usebox{\pandoc@box}%
  \fi%
}
% Set default figure placement to htbp
\def\fps@figure{htbp}
\makeatother
\setlength{\emergencystretch}{3em} % prevent overfull lines
\providecommand{\tightlist}{%
  \setlength{\itemsep}{0pt}\setlength{\parskip}{0pt}}
% Fix run-in paragraph style for ##### headers
\usepackage{titlesec}
\titleformat{\paragraph}[block]{\normalfont\normalsize\bfseries}{\theparagraph}{1em}{}
\titlespacing*{\paragraph}{0pt}{3.25ex plus 1ex minus .2ex}{0.5em}
\titleformat{\subparagraph}[block]{\normalfont\normalsize\bfseries}{\thesubparagraph}{1em}{}
\titlespacing*{\subparagraph}{0pt}{3.25ex plus 1ex minus .2ex}{0.5em}

% Table packages for tabularx with custom column types
\usepackage{booktabs}
\usepackage{tabularx}
\usepackage{array}
\newcolumntype{Y}{>{\raggedright\arraybackslash}X}

% Appendix lettered numbering for figures and equations.
%
% In this repo, the manuscript uses "##" headings, which Pandoc maps to \subsection.
% There are no \section-level headings, so relying on subsection/section counters
% yields 0.x-style numbering. Instead, we define an appendix-letter counter that
% advances once per appendix (\subsection after \appendix) and resets figure/equation
% numbering within each appendix.
\usepackage{etoolbox}
\newif\ifkcorappendix
\newcounter{kcorappendix}
\pretocmd{\appendix}{%
  \kcorappendixtrue%
  \setcounter{kcorappendix}{0}%
  \renewcommand{\thefigure}{\Alph{kcorappendix}.\arabic{figure}}%
  \renewcommand{\theequation}{\Alph{kcorappendix}.\arabic{equation}}%
  \renewcommand{\thetable}{\Alph{kcorappendix}.\arabic{table}}%
}{}{}
\pretocmd{\subsection}{%
  \ifkcorappendix%
    \stepcounter{kcorappendix}%
    \setcounter{figure}{0}%
    \setcounter{equation}{0}%
    \setcounter{table}{0}%
  \fi%
}{}{}

% Disable TeX hyphenation (reviewer-friendly) and avoid PDF soft-hyphen artifacts.
% Note: this does not prevent manual hyphens, only automatic word-breaking.
\usepackage{ragged2e}
\RaggedRight
\hyphenpenalty=10000
% Allow line breaks at *explicit* hyphens that already exist in the text.
% (Keep automatic hyphenation disabled via \hyphenpenalty above.)
\exhyphenpenalty=50
\pretolerance=10000
\tolerance=2000
\emergencystretch=3em

% Suppress XeLaTeX "Missing character" warnings for Unicode math symbols.
% When math like $\theta$ appears in table cells, XeLaTeX checks if the text font
% has the Unicode math character (U+1D703 = mathematical italic theta), generating
% a warning even though the math font handles rendering correctly.
% \tracinglostchars=0 is the TeX primitive that controls this warning level:
%   0 = no warnings, 1 = warnings in log only, 2 = warnings + terminal (default)
\tracinglostchars=0

\usepackage{bookmark}
\IfFileExists{xurl.sty}{\usepackage{xurl}}{} % add URL line breaks if available
\urlstyle{same}
\hypersetup{
  hidelinks,
  pdfcreator={LaTeX via pandoc}}

\author{}
\date{}

\begin{document}

\section{KCOR Supplementary Information
(SI)}\label{kcor-supplementary-information-si}

% Enforce Supplementary (S) numbering for figures and tables in the SI PDF.
\renewcommand{\thefigure}{S\arabic{figure}}%
\setcounter{figure}{0}%
\renewcommand{\thetable}{S\arabic{table}}%
\setcounter{table}{0}%

This document provides supplementary material supporting the KCOR
methodology described in the main manuscript, including extended
derivations, simulation studies, robustness analyses, and additional
empirical results.

\subsection{S1. Overview}\label{s1.-overview}

This SI is organized as follows:

\begin{itemize}
\tightlist
\item
  \textbf{S2}: Extended diagnostics and failure modes
\item
  \textbf{S3}: Positive controls (injected harm/benefit)
\item
  \textbf{S4}: Control-test specifications and simulation parameters
\item
  \textbf{S5}: Additional figures and diagnostics
\item
  \textbf{S6}: Extended Czech empirical application / illustrative
  registry analysis
\end{itemize}

\subsection{S2. Extended diagnostics and failure
modes}\label{s2.-extended-diagnostics-and-failure-modes}

\subsubsection{Diagnostics and failure modes for KCOR
assumptions}\label{diagnostics-and-failure-modes-for-kcor-assumptions}

This section describes the \textbf{observable diagnostics and failure
modes} associated with each of the five KCOR assumptions (A1--A5). No
additional assumptions are introduced here. KCOR is designed to
\textbf{fail transparently rather than silently}: when an assumption is
violated, the resulting lack of identifiability or model stress
manifests through explicit diagnostic signals rather than spurious
estimates.

A compact summary mapping each assumption to its corresponding
diagnostic signals and recommended actions is provided in Table
\ref{tbl:appendixD_assumptions_diagnostics}.

\newpage

\begin{longtable}[]{@{}
  >{\raggedright\arraybackslash}p{(\linewidth - 8\tabcolsep) * \real{0.2000}}
  >{\raggedright\arraybackslash}p{(\linewidth - 8\tabcolsep) * \real{0.2000}}
  >{\raggedright\arraybackslash}p{(\linewidth - 8\tabcolsep) * \real{0.2000}}
  >{\raggedright\arraybackslash}p{(\linewidth - 8\tabcolsep) * \real{0.2000}}
  >{\raggedright\arraybackslash}p{(\linewidth - 8\tabcolsep) * \real{0.2000}}@{}}
\caption{KCOR assumptions and corresponding
diagnostics.}\label{tbl:appendixD_assumptions_diagnostics}\tabularnewline
\toprule\noalign{}
\begin{minipage}[b]{\linewidth}\raggedright
Assumption
\end{minipage} & \begin{minipage}[b]{\linewidth}\raggedright
What must hold
\end{minipage} & \begin{minipage}[b]{\linewidth}\raggedright
Diagnostic signal
\end{minipage} & \begin{minipage}[b]{\linewidth}\raggedright
Interpretation
\end{minipage} & \begin{minipage}[b]{\linewidth}\raggedright
Action if violated
\end{minipage} \\
\midrule\noalign{}
\endfirsthead
\toprule\noalign{}
\begin{minipage}[b]{\linewidth}\raggedright
Assumption
\end{minipage} & \begin{minipage}[b]{\linewidth}\raggedright
What must hold
\end{minipage} & \begin{minipage}[b]{\linewidth}\raggedright
Diagnostic signal
\end{minipage} & \begin{minipage}[b]{\linewidth}\raggedright
Interpretation
\end{minipage} & \begin{minipage}[b]{\linewidth}\raggedright
Action if violated
\end{minipage} \\
\midrule\noalign{}
\endhead
\bottomrule\noalign{}
\endlastfoot
A1. Fixed cohort at enrollment & Cohort membership does not change over
follow-up & Step changes or discontinuities inconsistent with depletion
& Endogenous selection or reclassification & Redefine cohort at
enrollment; disallow transitions \\
A2. Shared external hazard environment & Cohorts experience the same
background hazard within the comparison window & Divergent slopes during
prespecified quiet periods & Unshared exogenous shocks or
policy/measurement effects & Restrict calendar window, stratify, or use
alternative controls \\
A3. Time-invariant latent frailty & Individual frailty is time-invariant
over follow-up & Systematic residual curvature after normalization &
Time-varying susceptibility or competing selection processes & Shorten
follow-up window; reinterpret as time-varying selection \\
A4. Adequacy of gamma frailty & Gamma family adequately approximates
frailty mixing & Residual curvature or poor fit diagnostics after
inversion & Frailty distribution misspecification & Treat as diagnostic;
avoid over-interpretation \\
A5. Quiet-window validity & No intervention effect during
frailty-estimation window & Slope breaks or non-parallel trends within
quiet window & Contaminated quiet window & Redefine quiet window; rerun
diagnostics \\
\end{longtable}

\subsubsection{S2.1 Diagnostics for Assumption A1 (Fixed cohorts at
enrollment)}\label{s2.1-diagnostics-for-assumption-a1-fixed-cohorts-at-enrollment}

\textbf{Assumption A1} requires that cohorts be fixed at enrollment,
with no post-enrollment switching or censoring in the primary estimand.

\textbf{Diagnostic signals of violation.}

\begin{itemize}
\tightlist
\item
  Inconsistencies in cohort risk sets (e.g., unexplained increases in
  at-risk counts).
\item
  Early-time hazard suppression or inflation inconsistent with selection
  or depletion geometry.
\item
  Dependence of results on as-treated reclassification or censoring
  rules.
\end{itemize}

\textbf{Interpretation.} KCOR is not defined for datasets with
post-enrollment switching or informative censoring in the primary
estimand. Such violations are design-level failures rather than modeling
failures and indicate that KCOR should not be applied without redefining
cohorts.

\subsubsection{S2.2 Diagnostics for Assumption A2 (Shared external
hazard
environment)}\label{s2.2-diagnostics-for-assumption-a2-shared-external-hazard-environment}

\textbf{Assumption A2} requires that all cohorts experience the same
calendar-time external mortality environment.

\textbf{Diagnostic signals of violation.}

\begin{itemize}
\tightlist
\item
  Calendar-time hazard spikes or drops that appear in only one cohort.
\item
  Misalignment of major mortality shocks (e.g., epidemic waves) across
  cohorts.
\item
  Cohort-specific reporting artifacts or administrative discontinuities.
\end{itemize}

\textbf{Interpretation.} External shocks are permitted under KCOR
provided they act symmetrically across cohorts. Cohort-specific shocks
violate comparability and are visible directly in calendar-time hazard
overlays. When detected, such violations limit interpretation of KCOR
contrasts over affected periods.

\subsubsection{S2.3 Diagnostics for Assumption A3 (Selection via
time-invariant latent
frailty)}\label{s2.3-diagnostics-for-assumption-a3-selection-via-time-invariant-latent-frailty}

\textbf{Assumption A3} posits that selection at enrollment operates
primarily through differences in a time-invariant latent frailty
distribution that induces depletion of susceptibles.

\textbf{Diagnostic signals of violation.}

\begin{itemize}
\tightlist
\item
  Strongly structured residuals in cumulative-hazard space inconsistent
  with depletion.
\item
  Instability of fitted frailty parameters not attributable to window
  placement.
\item
  Early-time transients that do not decay and are inconsistent across
  related cohorts.
\end{itemize}

\textbf{Interpretation.} Frailty in KCOR is a geometric construct
capturing unobserved heterogeneity, not a causal mechanism. If dominant
time-varying individual risk unrelated to depletion is present,
curvature attributed to frailty becomes unstable. Such cases are
revealed by residual structure and parameter instability rather than
masked by the model.

\subsubsection{S2.4 Diagnostics for Assumption A4 (Adequacy of gamma
frailty
approximation)}\label{s2.4-diagnostics-for-assumption-a4-adequacy-of-gamma-frailty-approximation}

\textbf{Assumption A4} requires that gamma frailty provides an adequate
approximation to the depletion geometry observed in cumulative-hazard
space over the estimation window.

\textbf{Diagnostic signals of violation.}

\begin{itemize}
\tightlist
\item
  Poor fit of the gamma-frailty cumulative-hazard model during the quiet
  window.
\item
  Systematic residual curvature after frailty normalization.
\item
  Strong sensitivity of results to minor model or window perturbations.
\end{itemize}

Additional internal diagnostics for Assumption A4 include the magnitude,
coherence, and stability of the fitted frailty variance parameter
(\(\theta\)). Values of \(\theta\) approaching zero are expected when
cumulative hazards are approximately linear, while larger values
correspond to visible depletion-induced curvature. Implausible
\(\theta\) estimates---such as large values in the absence of curvature,
sign instability, or extreme sensitivity to small changes in the
estimation window---indicate model stress or misspecification rather
than substantive cohort effects.

\textbf{Interpretation.} Gamma frailty is used as a mathematically
tractable approximation, not as a claim of biological truth. When
depletion geometry deviates substantially from the gamma form, KCOR
normalization can degrade visibly through poor fit and residual
curvature. Such behavior indicates model inadequacy rather than
supporting alternative interpretation.

\subsubsection{S2.5 Diagnostics for Assumption A5 (Quiet-window
validity)}\label{s2.5-diagnostics-for-assumption-a5-quiet-window-validity}

\textbf{Assumption A5} requires the existence of a prespecified quiet
window in which selection-induced depletion dominates other sources of
curvature, permitting identification of frailty parameters.

\textbf{Diagnostic signals of violation.}

\begin{itemize}
\tightlist
\item
  Failure of KCOR(t) trajectories to stabilize or asymptote following
  frailty normalization.
\item
  Persistent nonzero slope in KCOR(t), indicating residual curvature
  after normalization.
\item
  Instability of fitted frailty parameters (\(\theta\)) under small
  perturbations of quiet-window boundaries.
\item
  Failure of depletion-neutralized cumulative hazards to become
  approximately linear during the quiet window.
\item
  Degraded cumulative-hazard fit error concentrated within the nominal
  quiet period.
\end{itemize}

\textbf{Interpretation.} Quiet-window validity is the primary
dataset-specific requirement for KCOR applicability. When this
assumption is violated---e.g., due to overlap with strong treatment
effects or external shocks---KCOR does not amplify spurious signals.
Instead, normalization becomes unstable and KCOR(t) trajectories
attenuate toward unity or may fail to stabilize, explicitly signaling
loss of identifiability.

Under a valid quiet window, depletion-neutralized baseline cumulative
hazards are expected to be approximately linear and \(\mathrm{KCOR}(t)\)
trajectories to stabilize rather than drift. Persistent
\(\mathrm{KCOR}(t)\) slope or \(\hat{\theta}_d\) instability indicates
contamination of the quiet window by external shocks or time-varying
effects and signals loss of identifiability rather than evidence of
cohort differences.

\paragraph{Quiet-window selection protocol (operational
detail)}\label{quiet-window-selection-protocol-operational-detail}

The quiet window is selected prior to estimation using operational
criteria such as:

\begin{enumerate}
\def\labelenumi{\arabic{enumi}.}
\tightlist
\item
  Calendar-time hazard curves exhibit approximate linearity with no
  sustained trend breaks.
\item
  Periods containing epidemic waves, reporting artifacts, or policy
  shocks are excluded.
\item
  The window spans a minimum duration sufficient for stable slope
  estimation.
\item
  Sensitivity is assessed by perturbing the window boundaries (± several
  weeks).
\end{enumerate}

\textbf{Failure signals (do not report KCOR as identified).}\\
Treat the analysis as not identified if any cohort shows: (i) poor fit
in cumulative-hazard space over the quiet window; (ii) persistent
post-normalization nonlinearity within the quiet window; or (iii)
instability of fitted parameters under small boundary perturbations
(e.g., ±4 weeks).

\emph{Practical example.}\\
In COVID-19 mortality analyses, a quiet window may be defined as an
inter-wave period between major variant surges, verified by
approximately linear all-cause cumulative hazards in the general
population and the absence of cohort-differential policy or reporting
shocks. Robustness to small boundary perturbations (e.g., ± several
weeks) is treated as a diagnostic; if fitted depletion parameters or
post-normalization linearity are unstable under such perturbations, the
quiet-window assumption is treated as violated and the analysis as not
identified.

The diagnostics above are designed to detect quiet-window violations
that induce residual curvature or parameter instability. They do not, by
themselves, exclude the possibility of smooth, approximately stationary
cohort-differential hazards within the quiet window that may be absorbed
into fitted frailty parameters without producing obvious drift. For this
reason, when feasible, we additionally recommend split-window and
multi-window stability checks, in which frailty parameters and
post-normalization linearity are assessed for consistency across
sub-windows. Failure of such stability checks is treated as evidence
against Assumption A5.

\subsubsection{S2.6 Diagnostic coherence across
assumptions}\label{s2.6-diagnostic-coherence-across-assumptions}

Several diagnostics operate across assumptions A4 and A5, including
stabilization of KCOR(t) trajectories and coherence of fitted \(\theta\)
parameters with observed cumulative-hazard curvature. These diagnostics
are not assumptions of the KCOR framework; rather, they are observable
consequences of successful frailty normalization. When these behaviors
fail to emerge, KCOR explicitly signals reduced interpretability through
residual curvature, parameter instability, or attenuation toward unity.

\subsubsection{S2.7 Identifiability under sparse
data}\label{s2.7-identifiability-under-sparse-data}

KCOR does not require large sample sizes by assumption; however,
reliable estimation of frailty parameters and depletion-neutralized
cumulative hazards requires sufficient event information within the
identification window. When event counts are very small, frailty
estimates may become unstable, resulting in noisy normalization,
non-linear baseline cumulative hazards, or drifting KCOR(t)
trajectories.

Such failures are diagnosable: sparse-data regimes are characterized by
instability of estimated frailty parameters under small perturbations of
the quiet window, loss of post-normalization linearity, and
non-stabilizing KCOR(t). In these cases, KCOR signals loss of
identifiability rather than producing spurious effects. Applicability
should therefore be assessed via diagnostic stability rather than
nominal sample size thresholds.

\subsubsection{S2.8 Summary: Diagnostic enforcement rather than
assumption
inflation}\label{s2.8-summary-diagnostic-enforcement-rather-than-assumption-inflation}

KCOR relies on exactly five assumptions (A1--A5), stated exhaustively in
the main manuscript. This section demonstrates that each assumption has
\textbf{explicit, observable diagnostics} and \textbf{well-defined
failure modes}. When assumptions are violated, KCOR signals reduced
interpretability through instability, poor fit, or residual structure
rather than producing misleading cumulative contrasts. This diagnostic
enforcement is a core design feature of the KCOR framework.

\subsection{S3. Positive controls (injected
harm/benefit)}\label{s3.-positive-controls-injected-harmbenefit}

\subsubsection{S3.1 Positive controls: detect injected
harm/benefit}\label{s3.1-positive-controls-detect-injected-harmbenefit}

The effect window is a simulation construct used solely for
positive-control validation and does not represent a real-world
intervention period or biological effect window.

Positive controls are constructed by starting from a negative-control
dataset and injecting a known effect into the data-generating process
for one cohort, for example by multiplying the \emph{baseline} hazard by
a constant factor \(r\) over a prespecified interval:

\begin{equation}\protect\phantomsection\label{eq:pos-control-injection}{
h_{0,\mathrm{treated}}(t) = r \cdot h_{0,\mathrm{control}}(t) \quad \text{for } t \in [t_1, t_2],
}\end{equation}

with \(r>1\) for harm and \(0<r<1\) for benefit.

After gamma-frailty normalization (inversion), KCOR should deviate from
1 in the correct direction and with magnitude consistent with the
injected effect (up to discretization and sampling noise). Figure
\ref{fig:pos_control_injected} and Table \ref{tbl:pos_control_summary}
confirm this behavior.

\begin{figure}
\centering
\pandocbounded{\includegraphics[keepaspectratio,alt={Positive control validation: KCOR correctly detects injected effects. Left panels show harm scenario (r=1.2), right panels show benefit scenario (r=0.8). Top row displays cohort hazard curves with effect window shaded. Bottom row shows \textbackslash mathrm\{KCOR\}(t) deviating from 1.0 in the expected direction during the effect window. Uncertainty bands (95\% bootstrap intervals) are shown.}]{figures/fig_pos_control_injected.png}}
\caption{Positive control validation: KCOR correctly detects injected
effects. Left panels show harm scenario (r=1.2), right panels show
benefit scenario (r=0.8). Top row displays cohort hazard curves with
effect window shaded. Bottom row shows \(\mathrm{KCOR}(t)\) deviating
from 1.0 in the expected direction during the effect window. Uncertainty
bands (95\% bootstrap intervals) are
shown.}\label{fig:pos_control_injected}
\end{figure}

\subsection{S4. Control-test specifications and simulation
parameters}\label{s4.-control-test-specifications-and-simulation-parameters}

\subsubsection{S4.1 Control-test
specifications}\label{s4.1-control-test-specifications}

\subsubsection{Reference implementation and default operational
settings}\label{reference-implementation-and-default-operational-settings}

\begin{longtable}[]{@{}
  >{\raggedright\arraybackslash}p{(\linewidth - 6\tabcolsep) * \real{0.2500}}
  >{\raggedright\arraybackslash}p{(\linewidth - 6\tabcolsep) * \real{0.2500}}
  >{\raggedright\arraybackslash}p{(\linewidth - 6\tabcolsep) * \real{0.2500}}
  >{\raggedright\arraybackslash}p{(\linewidth - 6\tabcolsep) * \real{0.2500}}@{}}
\caption{Reference implementation and default operational
settings.}\label{tbl:appendixE_defaults}\tabularnewline
\toprule\noalign{}
\begin{minipage}[b]{\linewidth}\raggedright
Component
\end{minipage} & \begin{minipage}[b]{\linewidth}\raggedright
Setting
\end{minipage} & \begin{minipage}[b]{\linewidth}\raggedright
Default value
\end{minipage} & \begin{minipage}[b]{\linewidth}\raggedright
Notes
\end{minipage} \\
\midrule\noalign{}
\endfirsthead
\toprule\noalign{}
\begin{minipage}[b]{\linewidth}\raggedright
Component
\end{minipage} & \begin{minipage}[b]{\linewidth}\raggedright
Setting
\end{minipage} & \begin{minipage}[b]{\linewidth}\raggedright
Default value
\end{minipage} & \begin{minipage}[b]{\linewidth}\raggedright
Notes
\end{minipage} \\
\midrule\noalign{}
\endhead
\bottomrule\noalign{}
\endlastfoot
Cohort construction & Cohort indexing & Enrollment period × YearOfBirth
group × Dose; plus all-ages cohort (YearOfBirth = -2) & Implementation
detail \\
Quiet-period selection & Quiet window & ISO weeks 2023-01 through
2023-52 & Calendar year 2023 \\
Early-period stabilization (dynamic HVE) & \texttt{SKIP\_WEEKS} & 2 &
Weeks \(t < \mathrm{SKIP\_WEEKS}\) are excluded from hazard accumulation
(set \(\Delta H_d(t)=0\) for those weeks). \\
Frailty estimation & Fit method & Nonlinear least squares in
cumulative-hazard space & Constraints: \(k_d>0\), \(\theta_d \ge 0\) \\
\end{longtable}

\subsubsection{S4.2 Negative control: synthetic gamma-frailty
null}\label{s4.2-negative-control-synthetic-gamma-frailty-null}

The synthetic negative control (Figure \ref{fig:neg_control_synthetic})
is generated using:

\begin{itemize}
\tightlist
\item
  \textbf{Data source}: \texttt{example/Frail\_cohort\_mix.xlsx}
  (pathological frailty mixture)
\item
  \textbf{Generation script}:
  \texttt{code/generate\_pathological\_neg\_control\_figs.py}
\item
  \textbf{Cohort A weights}: Equal weights across 5 frailty groups (0.2
  each)
\item
  \textbf{Cohort B weights}: Shifted weights {[}0.30, 0.20, 0.20, 0.20,
  0.10{]}
\item
  \textbf{Frailty values}: {[}1, 2, 4, 6, 10{]} (relative frailty
  multipliers)
\item
  \textbf{Base weekly probability}: 0.01
\item
  \textbf{Weekly log-slope}: 0.0 (constant baseline during quiet
  periods)
\item
  \textbf{Skip weeks}: 2
\item
  \textbf{Normalization weeks}: 4
\item
  \textbf{Time horizon}: 250 weeks
\end{itemize}

Both cohorts share identical per-frailty-group death probabilities; only
the mixture weights differ. This induces different cohort-level
curvature under the null.

\subsubsection{S4.3 Negative control: empirical age-shift
construction}\label{s4.3-negative-control-empirical-age-shift-construction}

The empirical negative control (Figures \ref{fig:neg_control_10yr} and
\ref{fig:neg_control_20yr}) is generated using:

\begin{itemize}
\tightlist
\item
  \textbf{Data source}: Czech Republic administrative mortality and
  vaccination data, aggregated into KCOR\_CMR format
\item
  \textbf{Generation script}:
  \texttt{test/negative\_control/code/generate\_negative\_control.py}
\item
  \textbf{Construction}: Age strata remapped to pseudo-doses within same
  vaccination category
\item
  \textbf{Age mapping}:

  \begin{itemize}
  \tightlist
  \item
    Dose 0 → YoB \{1930, 1935\}
  \item
    Dose 1 → YoB \{1940, 1945\}
  \item
    Dose 2 → YoB \{1950, 1955\}
  \end{itemize}
\item
  \textbf{Output YoB}: Fixed at 1950 (unvax cohort) or 1940 (vax cohort)
\item
  \textbf{Sheets processed}: 2021\_24, 2022\_06
\end{itemize}

This construction ensures that dose comparisons are within the same
underlying vaccination category, preserving a true null while inducing
10--20 year age differences.

\subsubsection{S4.4 Positive control: injected
effect}\label{s4.4-positive-control-injected-effect}

The positive control (Figure \ref{fig:pos_control_injected} and Table
\ref{tbl:pos_control_summary}) is generated using:

\begin{itemize}
\tightlist
\item
  \textbf{Generation script}:
  \texttt{test/positive\_control/code/generate\_positive\_control.py}
\item
  \textbf{Initial cohort size}: 100,000 per cohort
\item
  \textbf{Baseline hazard}: 0.002 per week
\item
  \textbf{Frailty variance}: \(\theta_0 = 0.5\) (control),
  \(\theta_1 = 1.0\) (treatment)
\item
  \textbf{Effect window}: weeks 20--80
\item
  \textbf{Hazard multipliers}:

  \begin{itemize}
  \tightlist
  \item
    Harm scenario: \(r = 1.2\)
  \item
    Benefit scenario: \(r = 0.8\)
  \end{itemize}
\item
  \textbf{Random seed}: 42
\item
  \textbf{Enrollment date}: 2021-06-14 (ISO week 2021\_24)
\end{itemize}

The injection multiplies the treatment cohort's baseline hazard by
factor \(r\) during the effect window, while leaving the control cohort
unchanged.

\subsubsection{S4.5 Sensitivity analysis
parameters}\label{s4.5-sensitivity-analysis-parameters}

The sensitivity analysis (Figure \ref{fig:sensitivity_overview}) varies:

\begin{itemize}
\tightlist
\item
  \textbf{Baseline weeks}: {[}2, 3, 4, 5, 6, 7, 8{]}
\item
  \textbf{Quiet-start offsets}: {[}-12, -8, -4, 0, +4, +8, +12{]} weeks
  from 2023-01
\item
  \textbf{Quiet-window end}: Fixed at 2023-52
\item
  \textbf{Dose pairs}: 1 vs 0, 2 vs 0, 2 vs 1
\item
  \textbf{Cohorts}: 2021\_24
\end{itemize}

Output grids show KCOR(t) values for each parameter combination.

\subsubsection{S4.6 Tail-sampling / bimodal selection (adversarial
selection
geometry)}\label{s4.6-tail-sampling-bimodal-selection-adversarial-selection-geometry}

We generate a base frailty population distribution with mean 1. Cohort
construction differs by selection rule:

\begin{itemize}
\tightlist
\item
  \textbf{Mid-sampled cohort}: frailty restricted to central quantiles
  (e.g., 25th--75th percentile) and renormalized to mean 1.
\item
  \textbf{Tail-sampled cohort}: mixture of low and high tails (e.g.,
  0--15th and 85th--100th percentiles) with mixture weights chosen to
  yield mean 1.
\end{itemize}

Both cohorts share the same baseline hazard \(h_0(t)\) and no treatment
effect (negative-control version). We also generate positive-control
versions by applying a known hazard multiplier in a prespecified window.
We evaluate (i) KCOR drift, (ii) quiet-window fit RMSE, (iii)
post-normalization linearity, and (iv) parameter stability under window
perturbation.

\begin{itemize}
\tightlist
\item
  \textbf{Generation script}:
  \texttt{test/sim\_grid/code/generate\_tail\_sampling\_sim.py}
\item
  \textbf{Base frailty distribution}: Log-normal with mean 1, variance
  0.5
\item
  \textbf{Mid-quantile cohort}: 25th--75th percentile
\item
  \textbf{Tail-mixture cohort}: {[}0--15th{]} + {[}85th--100th{]}
  percentiles, equal weights
\item
  \textbf{Baseline hazard}: 0.002 per week (constant)
\item
  \textbf{Positive-control hazard multiplier}: \(r = 1.2\) (harm) or
  \(r = 0.8\) (benefit)
\item
  \textbf{Effect window}: weeks 20--80
\item
  \textbf{Random seed}: 42
\end{itemize}

\subsubsection{S4.7 Joint frailty and treatment-effect simulation
(S7)}\label{s4.7-joint-frailty-and-treatment-effect-simulation-s7}

This simulation evaluates KCOR under conditions in which \textbf{both
selection-induced depletion (frailty heterogeneity)} and a \textbf{true
treatment effect (harm or benefit)} are present simultaneously. The
purpose is to assess whether KCOR can (i) correctly identify and
neutralize frailty-driven curvature using a quiet period and (ii) detect
a true treatment effect outside that period without confounding the two
mechanisms.

\paragraph{Design}\label{design}

Two fixed cohorts are generated with identical baseline hazards but
differing frailty variance. Individual hazards are multiplicatively
scaled by a latent frailty term drawn from a gamma distribution with
unit mean and cohort-specific variance. A treatment effect is then
injected over a prespecified time window that does not overlap the quiet
period used for frailty estimation.

Formally, individual hazards are generated as

\begin{equation}\protect\phantomsection\label{eq:appendixB_individual_hazard_with_effect}{
h_i(t) = z_i \cdot h_0(t) \cdot r(t).
}\end{equation}

where \(z_i\) is individual frailty, \(h_0(t)\) is a shared baseline
hazard, and \(r(t)\) is a time-localized multiplicative treatment effect
applied to one cohort only.

\subsection{S5. Additional figures and
diagnostics}\label{s5.-additional-figures-and-diagnostics}

\subsubsection{S5.1 Fit diagnostics}\label{s5.1-fit-diagnostics}

For each cohort \(d\), the gamma-frailty fit produces diagnostic outputs
including:

\begin{itemize}
\tightlist
\item
  \textbf{RMSE in \(H\)-space}: Root mean squared error between observed
  and model-predicted cumulative hazards over the quiet window. Values
  \textless{} 0.01 indicate excellent fit; values \textgreater{} 0.05
  may warrant investigation.
\item
  \textbf{Fitted parameters}: baseline hazard level and frailty
  variance. Very small frailty variance (\textless{} 0.01) indicates
  minimal detected depletion; very large values (\textgreater{} 5) may
  indicate model stress.
\item
  \textbf{Number of fit points}: \(n_{\mathrm{obs}}\) observations in
  quiet window. Larger \(n_{\mathrm{obs}}\) provides more stable
  estimates.
\end{itemize}

Example diagnostic output from the reference implementation:

\begin{verbatim}
KCOR_FIT,EnrollmentDate=2021_24,YoB=1950,Dose=0,
  k_hat=4.29e-03,theta_hat=8.02e-01,
  RMSE_Hobs=3.37e-03,n_obs=97,success=1
\end{verbatim}

\subsubsection{S5.2 Residual analysis}\label{s5.2-residual-analysis}

Fit residuals should be examined for. Define residuals:

\[
r_{d}(t)=H_{\mathrm{obs},d}(t)-H_{d}^{\mathrm{model}}(t;\hat{k}_d,\hat{\theta}_d).
\]

\begin{itemize}
\tightlist
\item
  \textbf{Systematic patterns}: Residuals should be approximately random
  around zero. Systematic curvature in residuals suggests model
  inadequacy.
\item
  \textbf{Outliers}: Individual weeks with large residuals may indicate
  data quality issues or external shocks.
\item
  \textbf{Autocorrelation}: Strong autocorrelation in residuals suggests
  the model is missing time-varying structure.
\end{itemize}

\subsubsection{S5.3 Parameter stability
checks}\label{s5.3-parameter-stability-checks}

Robustness of fitted parameters should be assessed by:

\begin{itemize}
\tightlist
\item
  \textbf{Quiet-window perturbation}: Shift the quiet-window start/end
  by ±4 weeks and re-fit. Stable parameters should vary by \textless{}
  10\%.
\item
  \textbf{Skip-weeks sensitivity}: Vary SKIP\_WEEKS from 0 to 8 and
  verify KCOR(t) trajectories remain qualitatively similar.
\item
  \textbf{Baseline-shape alternatives}: Compare the default constant
  baseline over the fit window to mild linear trends and verify
  normalization is not sensitive to this choice.
\end{itemize}

\subsubsection{S5.4 Quiet-window overlay
plots}\label{s5.4-quiet-window-overlay-plots}

Recommended diagnostic: overlay the prespecified quiet window on hazard
and cumulative-hazard time series plots. The fit window should:

\begin{itemize}
\tightlist
\item
  Avoid major epidemic waves or external mortality shocks
\item
  Contain sufficient event counts for stable estimation
\item
  Span a time range where baseline mortality is approximately stationary
\end{itemize}

Visual inspection of quiet-window placement relative to mortality
dynamics is an essential diagnostic step.

\subsubsection{S5.5 Robustness to age
stratification}\label{s5.5-robustness-to-age-stratification}

This subsection illustrates robustness of \(\mathrm{KCOR}(t)\) to narrow
age stratification by repeating the same fixed-cohort comparison in
three single birth-year cohorts spanning advanced ages (1930, 1940,
1950). Across these strata, the trajectories remain qualitatively stable
after depletion normalization, supporting the claim that the observed
behavior is not an artifact of age aggregation.

\begin{figure}
\centering
\pandocbounded{\includegraphics[keepaspectratio,alt={Birth-year cohort 1930: KCOR(t) trajectories comparing dose 2 and dose 3 to dose 0 for cohorts enrolled in ISO week 2022-26 and evaluated over calendar year 2023. KCOR curves are anchored at t\_0 = 4 weeks (i.e., plotted as \textbackslash mathrm\{KCOR\}(t; t\_0)). This figure is presented as an illustrative application demonstrating estimator behavior on registry data and does not support causal inference.}]{figures/supplement/kcor_realdata_yob1930_enroll2022w26_eval2023.png}}
\caption{Birth-year cohort 1930: KCOR(t) trajectories comparing dose 2
and dose 3 to dose 0 for cohorts enrolled in ISO week 2022-26 and
evaluated over calendar year 2023. KCOR curves are anchored at
\(t_0 = 4\) weeks (i.e., plotted as \(\mathrm{KCOR}(t; t_0)\)). This
figure is presented as an illustrative application demonstrating
estimator behavior on registry data and does not support causal
inference.}\label{fig:appendixC_yob1930}
\end{figure}

\begin{figure}
\centering
\pandocbounded{\includegraphics[keepaspectratio,alt={Birth-year cohort 1940: KCOR(t) trajectories comparing dose 2 and dose 3 to dose 0 for cohorts enrolled in ISO week 2022-26 and evaluated over calendar year 2023. KCOR curves are anchored at t\_0 = 4 weeks (i.e., plotted as \textbackslash mathrm\{KCOR\}(t; t\_0)). This figure is presented as an illustrative application demonstrating estimator behavior on registry data and does not support causal inference.}]{figures/supplement/kcor_realdata_yob1940_enroll2022w26_eval2023.png}}
\caption{Birth-year cohort 1940: KCOR(t) trajectories comparing dose 2
and dose 3 to dose 0 for cohorts enrolled in ISO week 2022-26 and
evaluated over calendar year 2023. KCOR curves are anchored at
\(t_0 = 4\) weeks (i.e., plotted as \(\mathrm{KCOR}(t; t_0)\)). This
figure is presented as an illustrative application demonstrating
estimator behavior on registry data and does not support causal
inference.}\label{fig:appendixC_yob1940}
\end{figure}

\begin{figure}
\centering
\pandocbounded{\includegraphics[keepaspectratio,alt={Birth-year cohort 1950: KCOR(t) trajectories comparing dose 2 and dose 3 to dose 0 for cohorts enrolled in ISO week 2022-26 and evaluated over calendar year 2023. KCOR curves are anchored at t\_0 = 4 weeks (i.e., plotted as \textbackslash mathrm\{KCOR\}(t; t\_0)). This figure is presented as an illustrative application demonstrating estimator behavior on registry data and does not support causal inference.}]{figures/supplement/kcor_realdata_yob1950_enroll2022w26_eval2023.png}}
\caption{Birth-year cohort 1950: KCOR(t) trajectories comparing dose 2
and dose 3 to dose 0 for cohorts enrolled in ISO week 2022-26 and
evaluated over calendar year 2023. KCOR curves are anchored at
\(t_0 = 4\) weeks (i.e., plotted as \(\mathrm{KCOR}(t; t_0)\)). This
figure is presented as an illustrative application demonstrating
estimator behavior on registry data and does not support causal
inference.}\label{fig:appendixC_yob1950}
\end{figure}

\textbf{Additional empirical negative-control variant (20-year age
shift).}\\
For completeness, we include the more extreme 20-year age-shift negative
control referenced in the main text:

\begin{figure}
\centering
\pandocbounded{\includegraphics[keepaspectratio,alt={Empirical negative control with approximately 20-year age difference between cohorts. Even under extreme composition differences, \textbackslash mathrm\{KCOR\}(t) exhibits no systematic drift, consistent with robustness to selection-induced curvature. KCOR curves are anchored at t\_0 = 4 weeks (i.e., plotted as \textbackslash mathrm\{KCOR\}(t; t\_0)). Uncertainty bands (95\% bootstrap intervals) are shown. Data source: Czech Republic mortality and vaccination dataset processed into KCOR\_CMR aggregated format (negative-control construction; see Supplementary Information, SI).}]{figures/fig3_neg_control_20yr_age_diff.png}}
\caption{Empirical negative control with approximately 20-year age
difference between cohorts. Even under extreme composition differences,
\(\mathrm{KCOR}(t)\) exhibits no systematic drift, consistent with
robustness to selection-induced curvature. KCOR curves are anchored at
\(t_0 = 4\) weeks (i.e., plotted as \(\mathrm{KCOR}(t; t_0)\)).
Uncertainty bands (95\% bootstrap intervals) are shown. Data source:
Czech Republic mortality and vaccination dataset processed into
KCOR\_CMR aggregated format (negative-control construction; see
Supplementary Information, SI).}\label{fig:neg_control_20yr}
\end{figure}

\subsection{S6. Extended Czech empirical
application}\label{s6.-extended-czech-empirical-application}

\subsubsection{S6.1 Empirical application with diagnostic validation:
Czech Republic national registry mortality
data}\label{s6.1-empirical-application-with-diagnostic-validation-czech-republic-national-registry-mortality-data}

The Czech results do not validate KCOR; they represent an application
that satisfies all pre-specified diagnostic criteria. Substantive
implications follow only if the identification assumptions hold.
Throughout this subsection, observed divergences are interpreted
strictly as properties of the estimator under real-world selection, not
as intervention effects.

Unless otherwise noted, KCOR curves in the Czech analyses are shown
anchored at \(t_0 = 4\) weeks for interpretability.

\paragraph{S6.1.1 Illustrative empirical context: COVID-19 mortality
data}\label{s6.1.1-illustrative-empirical-context-covid-19-mortality-data}

The COVID-19 vaccination period provides a natural empirical regime
characterized by strong selection heterogeneity and non-proportional
hazards, making it a useful illustration for the KCOR framework. During
this period, vaccine uptake was voluntary, rapidly time-varying, and
correlated with baseline health status, creating clear examples of
selection-induced non-proportional hazards. The Czech Republic national
mortality registry data exemplify this regime: voluntary uptake led to
asymmetric selection at enrollment, with vaccinated cohorts exhibiting
minimal frailty heterogeneity while unvaccinated cohorts retained
substantial heterogeneity. This asymmetric pattern reflects the healthy
vaccinee effect operating through selective uptake rather than
treatment. KCOR normalization removes this selection-induced curvature,
enabling interpretable cumulative comparisons. While these examples
illustrate KCOR's application, the method is general and applies to any
retrospective cohort comparison where selection induces differential
depletion dynamics.

\paragraph{S6.1.2 Frailty normalization behavior under empirical
validation}\label{s6.1.2-frailty-normalization-behavior-under-empirical-validation}

Across examined age strata in the Czech Republic mortality dataset,
fitted frailty parameters exhibit a pronounced asymmetry across cohorts.
Some cohorts show negligible estimated frailty variance:

\begin{equation}\protect\phantomsection\label{eq:appendixC_theta_near_zero}{
\hat{\theta}_d \approx 0.
}\end{equation}

while others exhibit substantial frailty-driven depletion. This pattern
reflects differences in selection-induced hazard curvature at cohort
entry rather than any prespecified cohort identity.

As a consequence, KCOR normalization leaves some cohorts' cumulative
hazards nearly unchanged, while substantially increasing the
depletion-neutralized baseline cumulative hazard for others. This
behavior is consistent with curvature-driven normalization rather than
cohort identity. This pattern is visible directly in
depletion-neutralized versus observed cumulative hazard plots and is
summarized quantitatively in the fitted-parameter logs (see
\texttt{KCOR\_summary.log}).

After frailty normalization, the depletion-neutralized baseline
cumulative hazards are approximately linear in event time. Residual
deviations from linearity reflect real time-varying risk---such as
seasonality or epidemic waves---rather than selection-induced depletion.
This linearization is a diagnostic consistent with successful removal of
depletion-driven curvature under the working model; persistent
nonlinearity or parameter instability indicates model stress or
quiet-window contamination.

Table \ref{tbl:appendixC_diagnostic_gate} summarizes these diagnostic
checks across age strata.

\newpage

\begin{longtable}[]{@{}
  >{\raggedright\arraybackslash}p{(\linewidth - 8\tabcolsep) * \real{0.1702}}
  >{\raggedright\arraybackslash}p{(\linewidth - 8\tabcolsep) * \real{0.1915}}
  >{\raggedright\arraybackslash}p{(\linewidth - 8\tabcolsep) * \real{0.2979}}
  >{\raggedright\arraybackslash}p{(\linewidth - 8\tabcolsep) * \real{0.2021}}
  >{\raggedright\arraybackslash}p{(\linewidth - 8\tabcolsep) * \real{0.1383}}@{}}
\caption{Diagnostic gate for Czech application: KCOR results reported
only where diagnostics
pass.}\label{tbl:appendixC_diagnostic_gate}\tabularnewline
\toprule\noalign{}
\begin{minipage}[b]{\linewidth}\raggedright
Age band (years)
\end{minipage} & \begin{minipage}[b]{\linewidth}\raggedright
Quiet window valid
\end{minipage} & \begin{minipage}[b]{\linewidth}\raggedright
Post-normalization linearity
\end{minipage} & \begin{minipage}[b]{\linewidth}\raggedright
Parameter stability
\end{minipage} & \begin{minipage}[b]{\linewidth}\raggedright
KCOR reported
\end{minipage} \\
\midrule\noalign{}
\endfirsthead
\toprule\noalign{}
\begin{minipage}[b]{\linewidth}\raggedright
Age band (years)
\end{minipage} & \begin{minipage}[b]{\linewidth}\raggedright
Quiet window valid
\end{minipage} & \begin{minipage}[b]{\linewidth}\raggedright
Post-normalization linearity
\end{minipage} & \begin{minipage}[b]{\linewidth}\raggedright
Parameter stability
\end{minipage} & \begin{minipage}[b]{\linewidth}\raggedright
KCOR reported
\end{minipage} \\
\midrule\noalign{}
\endhead
\bottomrule\noalign{}
\endlastfoot
40--49 & Yes & Yes & Yes & Yes \\
50--59 & Yes & Yes & Yes & Yes \\
60--69 & Yes & Yes & Yes & Yes \\
70--79 & Yes & Yes & Yes & Yes \\
80--89 & Yes & Yes & Yes & Yes \\
90--99 & Yes & Yes & Yes & Yes \\
All ages & Yes & Yes & Yes & Yes \\
\end{longtable}

All age strata in the Czech application satisfied the prespecified
diagnostic criteria, permitting KCOR computation and reporting. KCOR
results are not reported for any age stratum where diagnostics indicated
non-identifiability.

\textbf{Interpretation:} Unvaccinated cohorts exhibit frailty
heterogeneity, while Dose 2 cohorts show near-zero estimated frailty
across all age bands, consistent with selective uptake prior to
follow-up:

\begin{equation}\protect\phantomsection\label{eq:appendixC_theta_positive}{
\hat{\theta}_d > 0.
}\end{equation}

for Dose 0 cohorts and

\begin{equation}\protect\phantomsection\label{eq:appendixC_theta_near_zero_dose2}{
\hat{\theta}_d \approx 0.
}\end{equation}

for Dose 2 cohorts. Estimated frailty heterogeneity can appear larger at
younger ages because baseline hazards are low, so proportional
differences across latent risk strata translate into visibly different
short-term hazards before depletion compresses the risk distribution. At
older ages, higher baseline hazard and stronger ongoing depletion can
reduce the apparent dispersion of remaining risk, yielding smaller
fitted \(\theta\) even if latent heterogeneity is not literally smaller.
Frailty variance is largest at younger ages, where low baseline
mortality amplifies the impact of heterogeneity on cumulative hazard
curvature, and declines at older ages where mortality is compressed and
survivors are more homogeneous. Because Table
\ref{tbl:appendixC_frailty_variance} demonstrates selection-induced
heterogeneity, unadjusted cumulative outcome contrasts are expected to
conflate depletion effects with any true treatment differences; see
Table \ref{tbl:appendixC_raw_hazards} for raw cumulative hazards
reported as a pre-normalization diagnostic. KCOR normalization removes
the depletion component, enabling interpretable comparison of the
remaining differences.

These raw contrasts reflect both selection and depletion effects and are
not interpreted causally.

\newpage

\begin{longtable}[]{@{}
  >{\raggedright\arraybackslash}p{(\linewidth - 4\tabcolsep) * \real{0.3077}}
  >{\raggedleft\arraybackslash}p{(\linewidth - 4\tabcolsep) * \real{0.3462}}
  >{\raggedleft\arraybackslash}p{(\linewidth - 4\tabcolsep) * \real{0.3462}}@{}}
\caption{Estimated gamma-frailty variance (fitted frailty variance) by
age band and vaccination status for Czech cohorts enrolled in
2021\_24.}\label{tbl:appendixC_frailty_variance}\tabularnewline
\toprule\noalign{}
\begin{minipage}[b]{\linewidth}\raggedright
Age band (years)
\end{minipage} & \begin{minipage}[b]{\linewidth}\raggedleft
Fitted frailty variance (Dose 0)
\end{minipage} & \begin{minipage}[b]{\linewidth}\raggedleft
Fitted frailty variance (Dose 2)
\end{minipage} \\
\midrule\noalign{}
\endfirsthead
\toprule\noalign{}
\begin{minipage}[b]{\linewidth}\raggedright
Age band (years)
\end{minipage} & \begin{minipage}[b]{\linewidth}\raggedleft
Fitted frailty variance (Dose 0)
\end{minipage} & \begin{minipage}[b]{\linewidth}\raggedleft
Fitted frailty variance (Dose 2)
\end{minipage} \\
\midrule\noalign{}
\endhead
\bottomrule\noalign{}
\endlastfoot
40--49 & 16.79 & \(2.66 \times 10^{-6}\) \\
50--59 & 23.02 & \(1.87 \times 10^{-4}\) \\
60--69 & 13.13 & \(7.01 \times 10^{-18}\) \\
70--79 & 6.98 & \(3.46 \times 10^{-17}\) \\
80--89 & 2.97 & \(2.03 \times 10^{-11}\) \\
90--99 & 0.80 & \(8.66 \times 10^{-16}\) \\
All ages (full population) & 4.98 & \(1.02 \times 10^{-11}\) \\
\end{longtable}

\textbf{Notes:} - The fitted frailty variance quantifies unobserved
frailty heterogeneity and depletion of susceptibles within cohorts.
Near-zero values indicate effectively linear cumulative hazards over the
quiet window and are typical of strongly pre-selected cohorts. - Each
entry reports a single fitted gamma-frailty variance for the specified
age band and vaccination status within the 2021\_24 enrollment cohort. -
The ``All ages (full population)'' row corresponds to an independent fit
over the full pooled age range, included as a global diagnostic. - Table
\ref{tbl:appendixC_raw_hazards} reports raw outcome contrasts for ages
40+ (YOB \(\le 1980\)) where event counts are stable.

\textbf{Diagnostic checks:} - \textbf{Dose ordering:} the fitted frailty
variance is positive for Dose 0 and collapses toward zero for Dose 2
across all age strata, consistent with selective uptake. -
\textbf{Magnitude separation:} Dose 2 estimates are effectively zero
relative to Dose 0, indicating near-linear cumulative hazards rather
than forced curvature. - \textbf{Age coherence:} the fitted frailty
variance decreases at older ages as baseline mortality rises and
survivor populations become more homogeneous; monotonicity is not
imposed. - \textbf{Stability:} No sign reversals, boundary pathologies,
or numerical instabilities are observed. - \textbf{Falsifiability:}
Failure of any one of these checks would constitute evidence against
model adequacy.

\newpage

\begin{longtable}[]{@{}
  >{\raggedright\arraybackslash}p{(\linewidth - 6\tabcolsep) * \real{0.2353}}
  >{\raggedleft\arraybackslash}p{(\linewidth - 6\tabcolsep) * \real{0.3382}}
  >{\raggedleft\arraybackslash}p{(\linewidth - 6\tabcolsep) * \real{0.3529}}
  >{\raggedleft\arraybackslash}p{(\linewidth - 6\tabcolsep) * \real{0.0735}}@{}}
\caption{Ratio of observed cumulative mortality hazards for unvaccinated
(Dose 0) versus fully vaccinated (Dose 2) Czech cohorts enrolled in
2021\_24.}\label{tbl:appendixC_raw_hazards}\tabularnewline
\toprule\noalign{}
\begin{minipage}[b]{\linewidth}\raggedright
Age band (years)
\end{minipage} & \begin{minipage}[b]{\linewidth}\raggedleft
Dose 0 cumulative hazard
\end{minipage} & \begin{minipage}[b]{\linewidth}\raggedleft
Dose 2 cumulative hazard
\end{minipage} & \begin{minipage}[b]{\linewidth}\raggedleft
Ratio
\end{minipage} \\
\midrule\noalign{}
\endfirsthead
\toprule\noalign{}
\begin{minipage}[b]{\linewidth}\raggedright
Age band (years)
\end{minipage} & \begin{minipage}[b]{\linewidth}\raggedleft
Dose 0 cumulative hazard
\end{minipage} & \begin{minipage}[b]{\linewidth}\raggedleft
Dose 2 cumulative hazard
\end{minipage} & \begin{minipage}[b]{\linewidth}\raggedleft
Ratio
\end{minipage} \\
\midrule\noalign{}
\endhead
\bottomrule\noalign{}
\endlastfoot
40--49 & 0.005260 & 0.004117 & 1.2776 \\
50--59 & 0.014969 & 0.009582 & 1.5622 \\
60--69 & 0.045475 & 0.023136 & 1.9655 \\
70--79 & 0.123097 & 0.057675 & 2.1343 \\
80--89 & 0.307169 & 0.167345 & 1.8355 \\
90--99 & 0.776341 & 0.517284 & 1.5008 \\
All ages (full population) & 0.023160 & 0.073323 & 0.3159 \\
\end{longtable}

This table reports unadjusted cumulative hazards derived directly from
the raw data, prior to any frailty normalization or depletion
correction, and is shown to illustrate the magnitude and direction of
selection-induced curvature addressed by KCOR.

Values reflect raw cumulative outcome differences prior to KCOR
normalization and are not interpreted causally due to cohort
non-exchangeability. Cumulative hazards were integrated from cohort
enrollment through the end of available follow-up for the 2021\_24
enrollment window (through week 2024-16), identically for Dose 0 and
Dose 2 cohorts.

\paragraph{S6.1.3 Illustrative application to national registry
mortality
data}\label{s6.1.3-illustrative-application-to-national-registry-mortality-data}

We include a brief illustrative application to demonstrate end-to-end
KCOR behavior on real registry mortality data in a setting that
minimizes timing-driven shocks and window-tuning sensitivity. Cohorts
were enrolled in ISO week 2022-26, and evaluation was restricted to
calendar year 2023, yielding a 26-week post-enrollment buffer before
slope estimation and a prespecified full-year window for assessment.
Frailty parameters were estimated using a prespecified epidemiologically
quiet window (calendar year 2023) to minimize wave-related hazard
variation. This example is intended to illustrate estimator behavior
under real-world selection and heterogeneity and does not support causal
inference.

Figure \ref{fig:appendixC_allages} shows \(\mathrm{KCOR}(t)\)
trajectories for dose 2 and dose 3 relative to dose 0 for an all-ages
analysis. We deliberately present an all-ages analysis as a
high-heterogeneity stress test, since aggregation across age induces
substantial baseline hazard and frailty variation.

\begin{figure}
\centering
\pandocbounded{\includegraphics[keepaspectratio,alt={All-ages stress test: \textbackslash mathrm\{KCOR\}(t) trajectories comparing dose 2 and dose 3 to dose 0 for cohorts enrolled in ISO week 2022-26 and evaluated over calendar year 2023. KCOR curves are anchored at t\_0 = 4 weeks (i.e., plotted as \textbackslash mathrm\{KCOR\}(t; t\_0)). This figure is presented as an illustrative application demonstrating estimator behavior under extreme heterogeneity and does not support causal inference.}]{figures/kcor_realdata_allages_enroll2022w26_eval2023.png}}
\caption{All-ages stress test: \(\mathrm{KCOR}(t)\) trajectories
comparing dose 2 and dose 3 to dose 0 for cohorts enrolled in ISO week
2022-26 and evaluated over calendar year 2023. KCOR curves are anchored
at \(t_0 = 4\) weeks (i.e., plotted as \(\mathrm{KCOR}(t; t_0)\)). This
figure is presented as an illustrative application demonstrating
estimator behavior under extreme heterogeneity and does not support
causal inference.}\label{fig:appendixC_allages}
\end{figure}

\subsection{S7. Computational environment and runtime
notes}\label{s7.-computational-environment-and-runtime-notes}

\textbf{Environment.} Python 3.11; key dependencies include numpy,
scipy, pandas, and lifelines (for Cox-model comparisons), with plotting
via matplotlib.

\textbf{Compute requirements.} The full simulation grid reproduces in
approximately 1 hour 26 minutes on a 20-core CPU with 128 GB RAM;
smaller subsets reproduce in minutes.

\textbf{Reproduction.} Running \texttt{make\ paper} (or the repository's
top-level build command) regenerates all artifacts from a clean
checkout.

\end{document}
