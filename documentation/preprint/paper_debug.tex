% Options for packages loaded elsewhere
\PassOptionsToPackage{unicode}{hyperref}
\PassOptionsToPackage{hyphens}{url}
\documentclass[
]{article}
\usepackage{xcolor}
\usepackage{amsmath,amssymb}
\setcounter{secnumdepth}{-\maxdimen} % remove section numbering
\usepackage{iftex}
\ifPDFTeX
  \usepackage[T1]{fontenc}
  \usepackage[utf8]{inputenc}
  \usepackage{textcomp} % provide euro and other symbols
\else % if luatex or xetex
  \usepackage{unicode-math} % this also loads fontspec
  \defaultfontfeatures{Scale=MatchLowercase}
  \defaultfontfeatures[\rmfamily]{Ligatures=TeX,Scale=1}
\fi
\usepackage{lmodern}
\ifPDFTeX\else
  % xetex/luatex font selection
\fi
% Use upquote if available, for straight quotes in verbatim environments
\IfFileExists{upquote.sty}{\usepackage{upquote}}{}
\IfFileExists{microtype.sty}{% use microtype if available
  \usepackage[]{microtype}
  \UseMicrotypeSet[protrusion]{basicmath} % disable protrusion for tt fonts
}{}
\makeatletter
\@ifundefined{KOMAClassName}{% if non-KOMA class
  \IfFileExists{parskip.sty}{%
    \usepackage{parskip}
  }{% else
    \setlength{\parindent}{0pt}
    \setlength{\parskip}{6pt plus 2pt minus 1pt}}
}{% if KOMA class
  \KOMAoptions{parskip=half}}
\makeatother
\usepackage{longtable,booktabs,array}
\newcounter{none} % for unnumbered tables
\usepackage{calc} % for calculating minipage widths
% Correct order of tables after \paragraph or \subparagraph
\usepackage{etoolbox}
\makeatletter
\patchcmd\longtable{\par}{\if@noskipsec\mbox{}\fi\par}{}{}
\makeatother
% Allow footnotes in longtable head/foot
\IfFileExists{footnotehyper.sty}{\usepackage{footnotehyper}}{\usepackage{footnote}}
\makesavenoteenv{longtable}
\usepackage{graphicx}
\makeatletter
\newsavebox\pandoc@box
\newcommand*\pandocbounded[1]{% scales image to fit in text height/width
  \sbox\pandoc@box{#1}%
  \Gscale@div\@tempa{\textheight}{\dimexpr\ht\pandoc@box+\dp\pandoc@box\relax}%
  \Gscale@div\@tempb{\linewidth}{\wd\pandoc@box}%
  \ifdim\@tempb\p@<\@tempa\p@\let\@tempa\@tempb\fi% select the smaller of both
  \ifdim\@tempa\p@<\p@\scalebox{\@tempa}{\usebox\pandoc@box}%
  \else\usebox{\pandoc@box}%
  \fi%
}
% Set default figure placement to htbp
\def\fps@figure{htbp}
\makeatother
% definitions for citeproc citations
\NewDocumentCommand\citeproctext{}{}
\NewDocumentCommand\citeproc{mm}{%
  \begingroup\def\citeproctext{#2}\cite{#1}\endgroup}
\makeatletter
 % allow citations to break across lines
 \let\@cite@ofmt\@firstofone
 % avoid brackets around text for \cite:
 \def\@biblabel#1{}
 \def\@cite#1#2{{#1\if@tempswa , #2\fi}}
\makeatother
\newlength{\cslhangindent}
\setlength{\cslhangindent}{1.5em}
\newlength{\csllabelwidth}
\setlength{\csllabelwidth}{3em}
\newenvironment{CSLReferences}[2] % #1 hanging-indent, #2 entry-spacing
 {\begin{list}{}{%
  \setlength{\itemindent}{0pt}
  \setlength{\leftmargin}{0pt}
  \setlength{\parsep}{0pt}
  % turn on hanging indent if param 1 is 1
  \ifodd #1
   \setlength{\leftmargin}{\cslhangindent}
   \setlength{\itemindent}{-1\cslhangindent}
  \fi
  % set entry spacing
  \setlength{\itemsep}{#2\baselineskip}}}
 {\end{list}}
\usepackage{calc}
\newcommand{\CSLBlock}[1]{\hfill\break\parbox[t]{\linewidth}{\strut\ignorespaces#1\strut}}
\newcommand{\CSLLeftMargin}[1]{\parbox[t]{\csllabelwidth}{\strut#1\strut}}
\newcommand{\CSLRightInline}[1]{\parbox[t]{\linewidth - \csllabelwidth}{\strut#1\strut}}
\newcommand{\CSLIndent}[1]{\hspace{\cslhangindent}#1}
\setlength{\emergencystretch}{3em} % prevent overfull lines
\providecommand{\tightlist}{%
  \setlength{\itemsep}{0pt}\setlength{\parskip}{0pt}}
% Fix run-in paragraph style for ##### headers
\usepackage{titlesec}
\titleformat{\paragraph}[block]{\normalfont\normalsize\bfseries}{\theparagraph}{1em}{}
\titlespacing*{\paragraph}{0pt}{3.25ex plus 1ex minus .2ex}{0.5em}
\titleformat{\subparagraph}[block]{\normalfont\normalsize\bfseries}{\thesubparagraph}{1em}{}
\titlespacing*{\subparagraph}{0pt}{3.25ex plus 1ex minus .2ex}{0.5em}

% Table packages for tabularx with custom column types
\usepackage{booktabs}
\usepackage{tabularx}
\usepackage{array}
\newcolumntype{Y}{>{\raggedright\arraybackslash}X}

% Disable TeX hyphenation (reviewer-friendly) and avoid PDF soft-hyphen artifacts.
% Note: this does not prevent manual hyphens, only automatic word-breaking.
\usepackage{ragged2e}
\RaggedRight
\hyphenpenalty=10000
\exhyphenpenalty=10000
\pretolerance=10000
\tolerance=2000
\emergencystretch=3em

% Suppress XeLaTeX "Missing character" warnings for Unicode math symbols.
% When math like $\theta$ appears in table cells, XeLaTeX checks if the text font
% has the Unicode math character (U+1D703 = mathematical italic theta), generating
% a warning even though the math font handles rendering correctly.
% \tracinglostchars=0 is the TeX primitive that controls this warning level:
%   0 = no warnings, 1 = warnings in log only, 2 = warnings + terminal (default)
\tracinglostchars=0

\usepackage{bookmark}
\IfFileExists{xurl.sty}{\usepackage{xurl}}{} % add URL line breaks if available
\urlstyle{same}
\hypersetup{
  hidelinks,
  pdfcreator={LaTeX via pandoc}}

\author{}
\date{}

\begin{document}

\section{KCOR: Depletion-Neutralized Cohort Comparison via Gamma-Frailty
Normalization}\label{kcor-depletion-neutralized-cohort-comparison-via-gamma-frailty-normalization}

\subsection{Manuscript metadata}\label{manuscript-metadata}

\begin{itemize}
\tightlist
\item
  \textbf{Article type}: Methods / Statistical method
\item
  \textbf{Running title}: KCOR under selection-induced cohort bias
\item
  \textbf{Author}: Steven T. Kirsch
\item
  \textbf{Affiliations}: Independent Researcher, United States
\item
  \textbf{Corresponding author}: stk@alum.mit.edu
\item
  \textbf{Word count}: 12,100 (excluding Abstract, References, and
  Appendices)
\item
  \textbf{Keywords}: selection bias; frailty model; gamma mixture model;
  frailty inversion; frailty heterogeneity; depletion of susceptibles;
  non-proportional hazards; cumulative hazard; hazard normalization;
  cumulative hazards; estimands; gamma frailty; negative controls;
  observational studies; observational cohort studies
\end{itemize}

\subsection{Abstract}\label{abstract}

\textbf{Background}\\
Retrospective cohort analyses frequently involve heterogeneous
populations subject to selection-induced depletion under latent frailty.
This process produces non-proportional hazards and curvature in observed
cumulative hazards that can bias standard survival estimands when
applied directly to registry and administrative data.

\textbf{Methods}\\
We introduce KCOR, a depletion-neutralized cohort comparison framework
based on gamma-frailty normalization. KCOR estimates cohort-specific
depletion geometry during prespecified epidemiologically quiet periods
and applies an analytic inversion to map observed cumulative hazards
into a common, depletion-neutralized scale prior to comparison. The
method requires only minimal event-time information and does not rely on
proportional hazards assumptions or rich covariate adjustment.

\textbf{Results}\\
Through extensive simulation studies spanning a wide range of frailty
heterogeneity and selection strength, as well as empirical negative and
positive controls, we show that commonly used methods---including Cox
proportional hazards regression---can exhibit systematic non-null
behavior under selection-only regimes. In contrast, KCOR-normalized
trajectories remain stable and centered near the null across these
settings.

\textbf{Conclusions}\\
KCOR provides a diagnostic and descriptive framework for comparing fixed
cohorts under selection-induced hazard curvature. By separating
depletion normalization from outcome comparison, the method restores a
common comparison scale prior to model fitting and improves the
interpretability of cumulative outcome analyses in heterogeneous
real-world data.

\textbf{Key contributions}

\begin{itemize}
\tightlist
\item
  Introduces a principled depletion-neutralization mapping for
  heterogeneous cohorts under latent frailty.
\item
  Demonstrates systematic non-null behavior of standard survival methods
  under selection-only regimes.
\item
  Provides a practical diagnostic framework requiring minimal registry
  data.
\end{itemize}

\subsection{Key messages}\label{key-messages}

• Selection-induced depletion under latent frailty heterogeneity
produces non-proportional hazards and curvature in cumulative hazard
trajectories that can bias direct application of standard survival
estimands in many retrospective cohort studies.

• KCOR provides a diagnostic and normalization framework that removes
selection-induced depletion curvature from cumulative hazards using
minimal registry data, restoring a common comparison scale across
cohorts.

• KCOR separates normalization from comparison: once hazards are
depletion-neutralized, cohorts may be compared using standard
post-adjustment estimands (e.g., ratios, differences, slopes, or
restricted mean survival time), with the choice driven by
interpretability rather than identifiability constraints.

• Simulation studies and empirical controls show that under
selection-only regimes, KCOR-normalized trajectories remain stable and
centered near the null, while commonly used estimands such as Cox
regression can exhibit systematic non-null behavior when applied to
unadjusted data.

\subsubsection{Methods Summary}\label{methods-summary}

KCOR is a cumulative-hazard normalization framework for retrospective
cohort comparisons under selection-induced non-proportional hazards. The
method requires minimal inputs (enrollment date, intervention date,
outcome time) and proceeds as follows:

\begin{itemize}
\item
  \textbf{Input data}: For each cohort, aggregate weekly counts of
  individuals alive and dead, indexed by enrollment date, birth year (or
  age group), and intervention status (dose/cohort label). Minimal
  required fields are dates of birth, intervention, and event (or
  censoring).
\item
  \textbf{Compute observed hazards}: From weekly alive/dead counts,
  compute weekly mortality risk and accumulate to form observed
  cumulative hazards \(H_{\mathrm{obs},d}(t)\) for each cohort \(d\).
\item
  \textbf{Identify quiet window}: Prespecify an epidemiologically quiet
  period (ISO weeks) during which external shocks and
  cohort-differential perturbations are minimal. This window is used
  exclusively for frailty parameter estimation.
\item
  \textbf{Fit frailty curvature parameter}: During the quiet window, fit
  gamma-frailty parameters \((\hat{k}_d, \hat{\theta}_d)\) independently
  for each cohort using constrained nonlinear least squares in
  cumulative-hazard space (Equation (\ref{eq:nls-objective})). The
  frailty variance parameter \(\theta_d\) captures selection-induced
  depletion curvature.
\item
  \textbf{Compute adjusted cumulative hazard}: Apply the gamma-frailty
  inversion (Equation (\ref{eq:normalized-cumhazard})) to transform
  observed cumulative hazards into depletion-neutralized baseline
  cumulative hazards \(\tilde{H}_{0,d}(t)\) for the full follow-up
  period.
\end{itemize}

\emph{After normalization, cohort comparisons are performed on the
depletion-neutralized scale using a prespecified estimand; in this
manuscript, we report cumulative hazard ratios for concreteness.}

\begin{itemize}
\item
  \textbf{Compute KCOR trajectory}: Form the ratio
  \(\mathrm{KCOR}(t) = \tilde{H}_{0,A}(t) / \tilde{H}_{0,B}(t)\) between
  cohorts. Interpret flatness (drift \textless{} 5\% per year) during
  quiet windows as successful depletion normalization; deviations during
  effect windows indicate treatment-related cohort differences when
  temporal separability holds.
\item
  \textbf{Uncertainty quantification (optional)}: For Monte Carlo
  confidence intervals, resample cohort data with replacement
  (stratified by cohort and stratum), re-estimate frailty parameters,
  recompute KCOR trajectories, and form percentile-based intervals
  (2.5th and 97.5th percentiles) at each time point. Recommended:
  1000--2000 bootstrap replicates for stable percentile intervals.
\item
  \textbf{Diagnostics}: Assess quiet-window fit quality (RMSE in
  cumulative-hazard space), post-normalization linearity (R² from linear
  fit), and parameter stability under boundary perturbations. If
  diagnostics fail, treat KCOR as not identified rather than reporting
  potentially misleading contrasts.
\end{itemize}

Observed cumulative hazards are computed by summing weekly integrated
hazard increments
\(\Delta H_d(t) = -\log\!\left(1 - D_{d,t}/Y_{d,t}\right)\), where
\(D_{d,t}\) is the number of deaths during week \(t\) and \(Y_{d,t}\) is
the number at risk at the start of that week.

\begin{center}\rule{0.5\linewidth}{0.5pt}\end{center}

\subsection{1. Introduction}\label{introduction}

\subsubsection{1.1 Retrospective cohort comparisons under
selection}\label{retrospective-cohort-comparisons-under-selection}

Randomized controlled trials (RCTs) are the gold standard for causal
inference, but are often infeasible, underpowered for rare outcomes, or
unavailable for questions that arise after rollout. As a result,
observational cohort comparisons are widely used to estimate
intervention effects on outcomes such as all-cause mortality.

Although mortality is used throughout this paper as a motivating and
concrete example, the method applies more generally to any irreversible
event process observed in a fixed cohort, including hospitalization,
disease onset, or other terminal or absorbing states. Mortality is
emphasized here because it is objectively defined, reliably recorded in
many national datasets, and free from outcome-dependent ascertainment
biases that complicate other endpoints.

However, when intervention uptake is voluntary, prioritized, or
otherwise selective, treated and untreated cohorts are frequently
\textbf{non-exchangeable} at baseline and evolve differently over
follow-up. This problem is not limited to any single intervention class;
it arises whenever the same factors that influence treatment uptake also
influence outcome risk.

This manuscript is a methods paper. Real-world registry data are used
solely to demonstrate estimator behavior, diagnostics, and failure modes
under realistic selection-induced non-proportional hazards; no causal or
policy conclusions are drawn.

\subsubsection{1.2 Curvature (shape) is the hard part: non-proportional
hazards from frailty
depletion}\label{curvature-shape-is-the-hard-part-non-proportional-hazards-from-frailty-depletion}

Selection does not merely shift mortality \textbf{levels}; it can alter
mortality \textbf{curvature}---the time-evolution of cohort hazards.
Frailty heterogeneity and depletion of susceptibles naturally induce
curvature of the cumulative hazard (reflecting time-varying hazard) even
when individual-level hazards are simple functions of time. When
selection concentrates high-frailty individuals into one cohort (or
preferentially removes them from another), the resulting cohort-level
hazard trajectories can be strongly non-proportional.

One convenient way to formalize ``curvature'' is in cumulative-hazard
space: if the cumulative hazard \(H(t)\) were perfectly linear in time,
then its second derivative would be zero, whereas selection-induced
depletion generally produces negative concavity (downward curvature) in
observed cumulative hazards during otherwise stable periods.

This violates core assumptions of many standard tools:

\begin{itemize}
\tightlist
\item
  \textbf{Cox PH}: assumes hazards differ by a time-invariant
  multiplicative factor (proportional hazards).
\item
  \textbf{IPTW / matching}: can balance measured covariates yet fail to
  balance unmeasured frailty and the resulting depletion dynamics.
\item
  \textbf{Age-standardization}: adjusts levels across age strata but
  does not remove cohort-specific time-evolving hazard shape.
\end{itemize}

KCOR is designed for this failure mode: \textbf{cohorts whose hazards
are not proportional because selection induces different depletion
dynamics (curvature).} Approximate linearity of cumulative hazard after
adjustment is therefore not assumed, but serves as an internal
diagnostic indicating that selection-induced depletion has been
successfully removed.

The methodological problem addressed here is general. The COVID-19
period provides a natural empirical regime characterized by strong
selection heterogeneity and non-proportional hazards, serving as a
useful illustration for the proposed framework. However, KCOR is not
specific to COVID, vaccination, or infectious disease. The estimator
applies to any retrospective cohort comparison in which selection
induces differential depletion dynamics that violate proportional
hazards assumptions. KCOR refers to the method as presented here;
earlier internal iterations are not material to the estimand or results
and are omitted for clarity.

In this paper we distinguish two mechanisms often lumped as the `healthy
vaccinee effect' (HVE):

\begin{itemize}
\item
  \textbf{Static HVE:} baseline differences in latent frailty
  distributions at cohort entry (e.g., vaccinated cohorts are healthier
  on average). In the KCOR framework, this manifests as differing
  depletion curvature (different \(\theta_d\)) and is the primary target
  of frailty normalization.
\item
  \textbf{Dynamic HVE:} short-horizon, time-local selection processes
  around enrollment that create transient hazard suppression immediately
  after enrollment (e.g., deferral of vaccination during acute illness,
  administrative timing, or short-term behavioral/health-seeking
  changes). Dynamic HVE is operationally addressed by prespecifying a
  skip/stabilization window (§2.7) and can be evaluated empirically by
  comparing early-period signatures across related cohorts in multi-dose
  settings.
\end{itemize}

\subsubsection{1.3 Related work: frailty, depletion of susceptibles, and
selection-induced non-proportional
hazards}\label{related-work-frailty-depletion-of-susceptibles-and-selection-induced-non-proportional-hazards}

The examples in this section are intended to be illustrative rather than
exhaustive, and are chosen to demonstrate common patterns that arise in
retrospective cohort data. They should not be interpreted as defining
the scope of applicability of the method, which is agnostic to the
specific intervention or event under study.

KCOR builds on a long literature on unobserved heterogeneity (`frailty')
and depletion of susceptibles, in which population-level hazards can
decelerate over time even when individual hazards are simple. The gamma
frailty model is widely used because its Laplace transform yields a
closed-form relationship between baseline and observed
survival/cumulative hazard, enabling tractable inference and
interpretation.\textsuperscript{1}

As a concrete illustration, a separate literature emphasizes that
observational estimates of vaccine effectiveness can remain confounded
despite extensive matching and adjustment, often revealed by negative
control outcomes and time-varying non-COVID mortality
differences.\textsuperscript{2,3} KCOR is complementary: rather than
using negative controls only to detect confounding, it targets a
specific confounding geometry---selection-induced depletion
curvature---and then requires controls and simulations to validate that
the intended curvature component has been removed.

To connect KCOR to common vaccine-effectiveness study designs, we
additionally simulated rollout-style cohorts with time-localized
epidemic hazard (`waves') and uptake correlated with latent frailty. In
these settings, Cox regression can return extremely small p-values under
a true null due to depletion and time-varying baseline hazard, whereas
KCOR remains centered near the null cumulative contrast.

We do not claim that KCOR subsumes all approaches to confounding
adjustment; rather, it provides a dedicated normalization and diagnostic
toolkit for settings where non-proportional hazards arise primarily from
selection-induced depletion dynamics.

\paragraph{1.3.1 Related work beyond proportional hazards: time-varying
effects and flexible hazard
modeling}\label{related-work-beyond-proportional-hazards-time-varying-effects-and-flexible-hazard-modeling}

\emph{The methods reviewed in this section primarily aim to improve
estimation of the \textbf{observed hazard function} under
non-proportionality; KCOR addresses a different
problem---\textbf{identification and removal of selection-induced
depletion geometry prior to cohort comparison}.} A large literature
relaxes proportional hazards by allowing either (i) covariate effects to
vary over time or (ii) the baseline hazard to be modeled flexibly.
Representative examples include time-varying coefficient Cox models,
flexible parametric survival models (e.g., spline-based log-cumulative
hazard models), additive hazards models (Aalen-type formulations),
landmarking and dynamic prediction approaches, and weighting-based
approaches such as marginal structural models (MSMs) for time-varying
confounding.\textsuperscript{4--11} These methods are most effective
when the scientific target is an instantaneous contrast (e.g., a
time-specific hazard ratio) and when sufficiently rich covariate
histories are available to support identification. In settings with
rich, time-updated covariate information that fully captures
frailty-relevant heterogeneity, weighting or structural modeling
approaches may partially mitigate depletion effects. KCOR is designed
for the complementary and common regime in which such covariates are
unavailable, unreliable, or post-treatment. In contrast, KCOR targets
cumulative contrasts under explicit depletion normalization in
cumulative-hazard space, rather than instantaneous hazard ratios
conditional on survival.

KCOR is motivated by a different failure mode that remains even when
proportional hazards is relaxed: cohort-level \emph{curvature} in
cumulative-hazard space arising from selection-induced frailty
heterogeneity and depletion of susceptibles. Allowing hazard ratios to
vary over time can improve descriptive fit, but it does not by itself
normalize depletion geometry or make cohorts exchangeable. Similarly,
flexible baselines can absorb curvature without distinguishing whether
it arises from latent selection/depletion versus genuine time-varying
external hazards. KCOR therefore occupies a distinct role: it uses a
mixture-model identity in cumulative-hazard space to estimate and invert
selection-induced depletion dynamics during prespecified quiet periods,
and then defines a cumulative comparison operator (KCOR) on the
depletion-neutralized scale.

\emph{This manuscript does not attempt to review the extensive
literature on non-proportional hazards exhaustively; instead, it
positions KCOR as complementary to these methods by addressing a
distinct identifiability problem not resolved by hazard-level
flexibility alone.}

\emph{Summary distinction.} KCOR differs from flexible hazard and
time-varying coefficient models not in descriptive flexibility, but in
estimand and direction of inference. Existing approaches model the
observed hazard trajectory and interpret fitted coefficients or
functions, implicitly treating all systematic divergence as signal. KCOR
instead first normalizes selection-induced depletion geometry in
cumulative-hazard space and only then defines a cohort contrast. In this
manuscript, the reported estimand is cumulative rather than
instantaneous, and interpretability is enforced diagnostically rather
than assumed through model flexibility.

\paragraph{1.3.1a Why flexible hazard models do not resolve depletion
bias}\label{a-why-flexible-hazard-models-do-not-resolve-depletion-bias}

Flexible hazard models---including spline-based parametric survival
models, time-varying coefficient Cox models, and neural network survival
estimators---can improve descriptive fit to observed hazard trajectories
by allowing curvature in the hazard function over time. However, these
methods do not distinguish whether observed curvature arises from
genuine causal time-varying effects or from compositional changes due to
selective depletion of susceptibles. Selection on frailty produces
curvature in cumulative hazards even under a true null treatment effect,
as heterogeneous cohorts experience differential depletion rates over
time. \emph{A model that perfectly fits the observed hazard trajectory
can still yield a biased estimand if that trajectory is driven by
selective depletion rather than a causal effect.} KCOR treats curvature
as a \textbf{bias to remove} rather than a feature to model, explicitly
estimating and neutralizing depletion geometry before defining cohort
comparisons.

\paragraph{1.3.1b Time-varying confounding approaches under latent
selection}\label{b-time-varying-confounding-approaches-under-latent-selection}

A large literature addresses time-varying confounding using marginal
structural models (MSMs), g-computation, and inverse-probability
weighting (IPW). These approaches can be effective when the relevant
time-varying confounders are \textbf{measured with sufficient fidelity}
to support sequential exchangeability and correct model or weight
specification. However, in retrospective administrative data, key
determinants of both intervention uptake and near-term outcome risk are
often \textbf{unobserved or incompletely recorded} (e.g., subclinical
illness, functional decline, care-seeking behavior, contraindications,
and access barriers). In such settings, selection can operate through
latent heterogeneity, producing depletion-induced hazard curvature even
when measured covariates are balanced at baseline.

Under latent selection, weighting cannot generally remove bias because
the quantities that must be conditioned on to restore exchangeability
are not available to the analyst. Moreover, misspecification of either
the treatment model (for IPW/MSMs) or the outcome model (for
g-computation) can reintroduce bias that is difficult to diagnose from
observed covariates alone. KCOR is designed for this regime: rather than
attempting to identify causal hazards via covariate adjustment, it
targets a depletion-neutralized \textbf{cumulative contrast} by directly
normalizing cohort-specific hazard curvature that arises from
selection-induced depletion within fixed cohorts.

Accordingly, KCOR is complementary to time-varying confounding methods:
when rich longitudinal confounders are available, MSM/IPW/g-computation
may be appropriate; when latent selection dominates and covariates are
incomplete, KCOR provides a diagnostics-driven cumulative comparison
that is stable under selection-only regimes.

\paragraph{1.3.2 Restricted Mean Survival Time
(RMST)}\label{restricted-mean-survival-time-rmst}

Restricted mean survival time (RMST) has been proposed as an alternative
summary estimand in settings where proportional hazards do not hold, as
it represents the area under the survival curve up to a prespecified
time horizon. RMST can offer an interpretable population-level summary
of survival experience without relying on hazard ratio assumptions.

However, RMST does not by itself address selection-induced frailty
heterogeneity or depletion of susceptibles. In retrospective cohorts
affected by latent selection, survival curves reflect populations whose
composition evolves differently over time across cohorts. As a result,
RMST summarizes survival differences that may reflect depletion rather
than treatment effect and does not normalize selection-induced depletion
geometry. This limitation is structural and does not vanish with larger
samples or improved matching.

KCOR differs conceptually in that it explicitly estimates and removes
depletion-induced curvature in cumulative-hazard space prior to cohort
comparison. RMST could in principle be computed after depletion
normalization as a secondary descriptive summary, but it is not the
primary estimand of interest in this framework. Accordingly, KCOR
targets cumulative hazard contrasts directly, as these are the
quantities rendered comparable by depletion normalization.

\subsubsection{1.4 Evidence from the literature: residual confounding
despite meticulous
matching}\label{evidence-from-the-literature-residual-confounding-despite-meticulous-matching}

One motivating illustrative case involves two large, rigorously designed
observational analyses that demonstrate the core empirical motivation:
even extremely careful matching and adjustment can leave large residual
differences in non-COVID mortality, indicating confounding and selection
that standard pipelines do not eliminate.

\paragraph{1.4.1 Denmark (negative controls highlight
confounding)}\label{denmark-negative-controls-highlight-confounding}

Obel et al.~used Danish registry data to build 1:1 matched cohorts and
applied negative control outcomes to assess confounding. Their
plain-language summary includes the following:

\begin{quote}
Meaning: The negative control methods indicate that observational
studies of SARS-CoV-2 vaccine effectiveness may be prone to substantial
confounding which may impact the observed associations. This bias may
both lead to underestimation of vaccine effectiveness (increased risk of
SARS-CoV2 infection among vaccinated individuals) and overestimation of
the vaccine effectiveness (decreased risk of death after of SARS-CoV2
infection among vaccinated individuals). Our results highlight the need
for randomized vaccine efficacy studies after the emergence of new
SARS-CoV-2 variants and the rollout of multiple booster
vaccines.\textsuperscript{2}
\end{quote}

This is a direct statement that observational designs---even with
careful matching and covariate adjustment---can remain substantially
confounded when selection and health-seeking behavior differ between
cohorts.

\paragraph{1.4.2 Qatar (time-varying HVE despite meticulous
matching)}\label{qatar-time-varying-hve-despite-meticulous-matching}

Chemaitelly et al.~analyzed matched national cohorts and explicitly
measured the \textbf{time-varying healthy vaccinee effect (HVE)} using
non-COVID mortality as a control outcome. They report a pronounced
early-period reduction in non-COVID mortality among vaccinated
individuals despite meticulous matching, followed by reversal later in
follow-up, consistent with dynamic selection and depletion
processes.\textsuperscript{3}

Together, these studies motivate a methods gap: we need estimators that
explicitly address \textbf{time-evolving selection-induced curvature},
not only baseline covariate imbalance. Table \ref{tbl:HVE_motivation}
summarizes these findings.

\subsubsection{1.5 Contribution of this
work}\label{contribution-of-this-work}

This work makes four primary contributions.

First, we identify and formalize a common failure mode in retrospective
cohort survival analysis: selection-induced depletion under latent
frailty heterogeneity generates non-proportional hazards and curvature
in cumulative hazard trajectories that can bias standard survival
estimands when applied directly to observed data.

Second, we introduce KCOR as a diagnostic and normalization framework
that estimates cohort-specific depletion geometry during
epidemiologically quiet periods and maps observed cumulative hazards
into a depletion-neutralized space via gamma-frailty inversion. This
normalization step restores approximate stationarity of hazards and
comparability across cohorts using only minimal registry data.

Third, we demonstrate through simulation and empirical negative controls
that commonly used estimands---including Cox proportional hazards
regression and related survival-based summaries---can exhibit systematic
non-null behavior under selection-only regimes, while KCOR-normalized
trajectories remain stable and centered near the null.

Fourth, we clarify that KCOR does not privilege a single comparison
estimand. Rather, it separates normalization from comparison: once
hazards are depletion-neutralized, cohorts may be compared using a range
of standard post-adjustment estimands (e.g., ratios, differences,
slopes, or restricted mean survival time), with the choice driven by
interpretability and scientific context. In this manuscript, we focus on
ratios of adjusted cumulative hazards as a stable and interpretable
summary aligned with the normalization geometry.

Together, these contributions position KCOR not as a replacement for
existing survival estimands, but as a prerequisite normalization step
that addresses a source of bias arising prior to model fitting in many
retrospective cohort studies.

\subsubsection{1.6 Target estimand and
interpretation}\label{target-estimand-and-interpretation}

KCOR is proposed as a diagnostic and normalization estimator for
selection-induced hazard curvature; causal interpretation requires
additional assumptions beyond the scope of this methods paper. This
manuscript is \textbf{methods-only}: we present the estimator, model
assumptions, and uncertainty quantification; we validate the method
using prespecified negative and positive controls designed to stress
selection-induced curvature; and we defer any applied real-world
intervention conclusions to a separate, dedicated applied paper.

KCOR is not merely a frailty-normalization method. While gamma-frailty
inversion is a necessary step, the central contribution of KCOR is the
end-to-end comparison system that follows normalization. KCOR transforms
observed cumulative hazards into a depletion-neutralized space and then
defines the correct comparison operator in that space---a cumulative
hazard ratio---together with diagnostics that determine when such
comparisons are interpretable. Normalization alone does not yield a
signal; the signal emerges only through the KCOR comparison itself. In
this sense, KCOR should be understood as a complete retrospective
comparison framework rather than a preprocessing adjustment that can be
substituted into standard estimators. The integrated nature of
KCOR---normalization, comparison, and diagnostics as a single
system---is illustrated schematically in Figure \ref{fig:kcor_workflow}.

\textbf{Scope and role of the method.} This work does not claim that
observational analyses can replace randomized clinical trials for causal
inference. Rather, the objective is to provide a descriptive and
diagnostic framework for analyzing time-indexed outcomes in fixed
cohorts when randomized evidence is unavailable, incomplete, or
inapplicable. The proposed method is intended to complement---not
substitute for---trial-based evidence by identifying systematic temporal
distortions and cohort effects that can arise in retrospective data.

\subsubsection{1.7 Relation to causal inference
frameworks}\label{relation-to-causal-inference-frameworks}

KCOR is not intended to replace established causal inference designs
such as instrumental variables, regression discontinuity,
difference-in-differences, or target trial emulation. Those frameworks
address distinct identification problems and typically require either
exogenous instruments, sharp intervention thresholds, rich covariate
histories, or well-defined intervention regimes.

KCOR is designed for a complementary setting in which such requirements
are not met---specifically, retrospective cohort data where only dates
of birth, death, and intervention are available, and where
selection-induced depletion produces strong non-proportional hazards
that can bias hazard-ratio-based estimators. In this setting, KCOR
targets a different failure mode: curvature in cumulative hazards
arising from latent heterogeneity and selection rather than from
time-varying treatment effects.

By neutralizing depletion geometry and defining a cumulative comparison
operator in the resulting space, KCOR enables interpretable cohort
contrasts under minimal data constraints. When stronger causal designs
are feasible, they should be preferred; when they are not, KCOR provides
a principled way to assess whether observed cohort differences persist
once selection-induced depletion is removed.

\textbf{Clarifications.} KCOR is a depletion-neutralization framework
for \textbf{cumulative} outcome contrasts over follow-up time, not a
proportional-hazards model and not a causal identification strategy. The
gamma-frailty component is used as a \textbf{working statistical device}
to remove selection-induced hazard curvature within fixed cohorts; it is
not a claim that individual frailty is literally gamma-distributed or
biologically interpretable. KCOR's target is a time-indexed cumulative
contrast (e.g., a cumulative hazard ratio), intended for settings where
selection-driven depletion and incomplete covariate capture undermine
hazard-based causal interpretations.

\subsubsection{1.8 Competing approaches and evaluation
plan}\label{competing-approaches-and-evaluation-plan}

To clarify the role of KCOR relative to existing survival-analysis
approaches, we distinguish methods by both modeling assumptions and
estimands. Competing approaches considered in evaluation include Cox
proportional hazards models, Cox models with time-varying coefficients,
Cox models with frailty terms, additive hazards models, flexible
parametric survival models, and restricted mean survival time (RMST).
These methods are included because they are commonly applied in
retrospective cohort settings with non-proportional hazards.

Evaluation focuses on settings where only minimal registry data are
available (dates of birth, intervention, and event), and where
selection-induced frailty heterogeneity produces curvature in
cumulative-hazard space. Performance is assessed using prespecified
simulation designs that vary frailty variance, selection strength, and
external hazard stability, with estimands compared on bias, stability,
and interpretability rather than goodness-of-fit alone.

\textbf{Box 1: KCOR in one page}

KCOR (Kirsch Cumulative Outcomes Ratio) is a cumulative-hazard
normalization and comparison framework for retrospective cohort studies
affected by selection-induced non-proportional hazards. The method
proceeds in six steps:

\begin{enumerate}
\def\labelenumi{\arabic{enumi}.}
\item
  \textbf{Fixed cohorts}: Individuals are assigned to cohorts at
  enrollment based on intervention status; no switching or censoring is
  permitted in the primary estimand.
\item
  \textbf{Cumulative hazard estimation}: Discrete-time hazards are
  computed from event counts and risk sets, then accumulated into
  observed cumulative hazards after an optional stabilization skip
  period:
\end{enumerate}

\[
H_{\mathrm{obs},d}(t).
\]

\begin{enumerate}
\def\labelenumi{\arabic{enumi}.}
\setcounter{enumi}{2}
\tightlist
\item
  \textbf{Quiet-window frailty fit}: During epidemiologically quiet
  periods (free of external shocks), cohort-specific frailty variance
  parameters \(\hat{\theta}_d\) and baseline hazard levels \(\hat{k}_d\)
  are estimated via nonlinear least squares by fitting the gamma-frailty
  identity:
\end{enumerate}

\[
H_{\mathrm{obs},d}(t)
=
\frac{1}{\theta_d}
\log\!\left(1+\theta_d\,H_{0,d}(t)\right).
\]

Under the reference constant-baseline specification over the quiet
window:

\[
H_{0,d}(t)=k_d\,t.
\]

\begin{enumerate}
\def\labelenumi{\arabic{enumi}.}
\setcounter{enumi}{3}
\tightlist
\item
  \textbf{Gamma inversion}: The fitted frailty parameters are used to
  invert the gamma-frailty identity, transforming observed cumulative
  hazards into depletion-neutralized baseline cumulative hazards:
\end{enumerate}

\[
\tilde{H}_{0,d}(t)
=
\frac{\exp\!\left(\hat{\theta}_d\,H_{\mathrm{obs},d}(t)\right)-1}{\hat{\theta}_d}.
\]

\begin{enumerate}
\def\labelenumi{\arabic{enumi}.}
\setcounter{enumi}{4}
\tightlist
\item
  \textbf{KCOR ratio}: Cohorts are compared via the ratio of
  depletion-neutralized baseline cumulative hazards:
\end{enumerate}

\[
\mathrm{KCOR}(t)
=
\frac{\tilde{H}_{0,A}(t)}{\tilde{H}_{0,B}(t)}.
\]

\begin{enumerate}
\def\labelenumi{\arabic{enumi}.}
\setcounter{enumi}{5}
\tightlist
\item
  \textbf{Diagnostics}: Post-normalization linearity in quiet periods,
  fit residuals, and parameter stability under window perturbations
  serve as internal checks that depletion normalization is valid and
  assumptions are met.
\end{enumerate}

\subsection{2. Methods}\label{methods}

While mortality is used as the primary example throughout this section,
KCOR applies to any irreversible event process. The methodological
framework is event-agnostic; mortality serves as a concrete illustration
because it is objectively defined and reliably recorded in many
administrative datasets.

Table \ref{tbl:notation} defines the notation used throughout the
Methods section.

For COVID-19 vaccination analyses, intervention count corresponds to the
number of vaccine doses received; more generally, this can index any
discrete exposure level.

\subsection{2.1 Conceptual framework and
estimand}\label{conceptual-framework-and-estimand}

Retrospective cohort differences can arise from two qualitatively
different components:

\begin{itemize}
\tightlist
\item
  \textbf{Level differences}: cohort hazards differ by an approximately
  time-stable multiplicative factor (or, equivalently, cumulative
  hazards have different slopes but similar shape).
\item
  \textbf{Depletion (curvature) differences}: cohort hazards evolve
  differently over time because cohorts differ in latent heterogeneity
  and are \textbf{selectively depleted} at different rates.
\end{itemize}

KCOR targets the second failure mode. Under latent frailty
heterogeneity, high-risk individuals die earlier, so the surviving risk
set becomes progressively ``healthier.'' This induces \textbf{downward
curvature} (deceleration) in cohort hazards and corresponding concavity
in cumulative-hazard space, even when individual-level hazards are
simple and even under a true null treatment effect. When selection
concentrates frailty heterogeneity differently across cohorts, the
resulting curvature differences produce strong non-proportional hazards
and can drive misleading contrasts for estimands that condition on the
evolving risk set.

KCOR's strategy is therefore:

\begin{enumerate}
\def\labelenumi{\arabic{enumi}.}
\tightlist
\item
  \textbf{Estimate the cohort-specific depletion geometry} (via
  curvature) during prespecified epidemiologically quiet periods.
\item
  \textbf{Map observed cumulative hazards into a depletion-neutralized
  space} by inverting that geometry.
\item
  \textbf{Compare cohorts only after normalization} using a prespecified
  post-adjustment estimand; in this work, we use ratios of
  depletion-neutralized cumulative hazards (KCOR).
\end{enumerate}

\subsubsection{2.1.1 Target estimand}\label{target-estimand}

\emph{KCOR separates normalization from comparison. The normalization
step produces depletion-neutralized baseline cumulative hazards that
render cohorts comparable; the subsequent comparison may be carried out
using a variety of cumulative or summary estimands. In this manuscript,
we define and report the cumulative hazard ratio as the primary
estimand.}

Let \(\tilde{H}_{0,d}(t)\) denote the \textbf{depletion-neutralized
baseline cumulative hazard} for cohort \(d\) at event time \(t\) since
enrollment (Table \ref{tbl:notation}). For two cohorts \(A\) and \(B\),
KCOR is defined as

\begin{equation}\protect\phantomsection\label{eq:kcor-estimand}{
\mathrm{KCOR}(t) = \frac{\tilde{H}_{0,A}(t)}{\tilde{H}_{0,B}(t)}.
}\end{equation}

Unlike hazard ratios, KCOR compares cumulative hazard accumulation over
time and does not rely on proportional hazards assumptions or
conditioning on survival at time \(t\).

\textbf{Interpretation (unanchored KCOR).} KCOR(t) is the ratio of
depletion-normalized cumulative baseline hazards accumulated by two
cohorts from enrollment to time \(t\). KCOR(t) \textgreater{} 1
indicates that, after accounting for selection-induced depletion via
frailty normalization, cohort A has accumulated greater cumulative
hazard than cohort B over \([0, t]\). Because KCOR(t) reflects
cumulative hazard levels rather than instantaneous rates, it
incorporates both baseline hazard differences and any pre-existing
cohort differences present at enrollment. Unanchored KCOR is
level-dependent and retains baseline offsets; it is not centered at 1
even under parallel hazards.

In some applications we additionally report an \textbf{anchored
(baseline-normalized) KCOR} that removes the time-invariant
multiplicative level difference between cohorts:

\[
\mathrm{KCOR}(t; t_0) \;=\; \frac{\mathrm{KCOR}(t)}{\mathrm{KCOR}(t_0)}
\;=\;
\frac{\tilde{H}_{0,A}(t) / \tilde{H}_{0,B}(t)}{\tilde{H}_{0,A}(t_0) / \tilde{H}_{0,B}(t_0)}.
\]

Here \(t_0\) is a prespecified post-enrollment anchor time (e.g., a few
weeks after enrollment) chosen after any initial stabilization period
but early enough to avoid masking near-term intervention effects. By
construction, \(\mathrm{KCOR}(t_0; t_0)=1\), so departures from 1
reflect \textbf{relative divergence after \(t_0\)}, whereas the
unanchored \(\mathrm{KCOR}(t)\) also preserves baseline level
differences that may reflect residual baseline-risk differences.

\textbf{Interpretation (anchored KCOR).} Anchoring removes pre-existing
cumulative differences between cohorts and isolates relative divergence
in cumulative hazard after \(t_0\). Under this representation,
\(\mathrm{KCOR}(t; t_0)=1\) at \(t=t_0\), and values above (below) 1
indicate excess (reduced) post-anchor cumulative hazard accumulation in
cohort A relative to cohort B.

Unanchored KCOR targets a cumulative hazard \emph{level} contrast,
whereas anchored KCOR targets a \emph{post-reference divergence}
estimand analogous to a difference-in-differences on the cumulative
hazard scale.

\(\mathrm{KCOR}(t)\) is a \textbf{cumulative outcome contrast} after
removal of curvature attributed to selection-induced depletion under the
working frailty model. The estimand is defined regardless of whether it
has a causal interpretation.

\subsubsection{2.1.2 Identification versus
diagnostics}\label{identification-versus-diagnostics}

KCOR is presented here as a \textbf{normalization-and-comparison
framework}, not as a general causal estimator under unmeasured
confounding. A causal interpretation of \(\mathrm{KCOR}(t)\) requires
additional substantive conditions (e.g., that the quiet-window curvature
is dominated by selection-induced depletion rather than
cohort-differential external shocks). Because these conditions are
inherently dataset- and design-dependent, KCOR emphasizes
\textbf{diagnostic enforcement}: when assumptions required for
interpretable normalization are not supported, the method should signal
this through degraded fit, residual structure, or instability to window
perturbations rather than silently producing a ``corrected'' contrast.

Operationally, interpretability of a KCOR trajectory is assessed via
prespecified checks (Appendix D), including:

\begin{itemize}
\tightlist
\item
  stability of \((\hat{k}_d,\hat{\theta}_d)\) to small quiet-window
  perturbations,
\item
  approximate linearity of \(\tilde{H}_{0,d}(t)\) within the quiet
  window,
\item
  absence of systematic residual structure in cumulative-hazard space.
\end{itemize}

Diagnostics corresponding to each assumption are summarized in Table D.1
and discussed in detail in Appendix D.

\begin{center}\rule{0.5\linewidth}{0.5pt}\end{center}

\subsubsection{Assumptions}\label{assumptions}

\begin{quote}
The KCOR framework relies on the following assumptions, which are
diagnostic rather than causal in nature:

\begin{enumerate}
\def\labelenumi{\arabic{enumi}.}
\item
  \textbf{Fixed cohort enrollment.} Cohorts are defined at a common
  enrollment time and followed forward without dynamic entry or
  rebalancing.
\item
  \textbf{Multiplicative latent frailty.} Individual hazards are assumed
  to be multiplicatively composed of a baseline hazard and an unobserved
  frailty term, with cohort-specific frailty distributions.
\item
  \textbf{Quiet-window stability.} A prespecified epidemiologically
  quiet period exists during which external shocks to the baseline
  hazard are minimal, allowing depletion geometry to be estimated from
  observed cumulative hazards.
\item
  \textbf{Independence across strata.} Cohorts or strata are analyzed
  independently, without interference, spillover, or cross-cohort
  coupling.
\item
  \textbf{Sufficient event-time resolution.} Event timing is observed at
  a temporal resolution adequate to estimate cumulative hazards over the
  quiet window.
\end{enumerate}

These assumptions are evaluated empirically using post-normalization
diagnostics. Violations are expected to manifest as residual curvature,
drift, or instability in adjusted cumulative hazard trajectories.
\end{quote}

\subsection{2.2 Cohort construction}\label{cohort-construction}

KCOR is defined for \textbf{fixed cohorts at enrollment}. Required
inputs are minimal: enrollment date(s), event date, and optionally birth
date (or year-of-birth) for age stratification. Analyses proceed in
discrete event time \(t\) (e.g., weeks) measured since cohort
enrollment.

Cohorts are assigned by intervention state at the start of the
enrollment interval. In the primary estimand:

\begin{itemize}
\tightlist
\item
  \textbf{No post-enrollment switching} is allowed (individuals remain
  in their enrollment cohort),
\item
  \textbf{No censoring} is applied (other than administrative end of
  follow-up),
\item
  analyses are performed on the resulting fixed risk sets.
\end{itemize}

Censoring or reclassification due to cohort transitions (e.g., moving
between exposure groups over time) is not permitted, because such
transitions alter the frailty composition of the cohort in a
time-dependent manner. Allowing transitions would introduce additional,
endogenous selection that changes cohort mortality trajectories in
unpredictable ways, confounding depletion effects that KCOR is designed
to normalize.

This fixed-cohort design is intentional. It avoids immortal-time
artifacts and prevents outcome-driven switching rules from creating
time-dependent selection that is difficult to diagnose under minimal
covariate availability. Extensions that allow switching or censoring are
treated as sensitivity analyses (§5.2) because they change the estimand
and introduce additional identification requirements.

Conceptual requirements of the KCOR framework are distinguished from
operational defaults, which are reported separately for reproducibility
(Appendix E).

Throughout this manuscript the failure event is \textbf{all-cause
mortality}. KCOR therefore targets cumulative mortality hazards and is
not framed as a cause-specific competing-risks analysis.

\begin{center}\rule{0.5\linewidth}{0.5pt}\end{center}

\subsection{2.3 Hazard estimation and cumulative hazards in discrete
time}\label{hazard-estimation-and-cumulative-hazards-in-discrete-time}

For each cohort \(d\), let \(N_d(0)\) denote the number of individuals
at enrollment. Let \(d_d(t)\) denote deaths occurring during interval
\(t\), and let \[
D_d(t) = \sum_{s \le t} d_d(s)
\] denote cumulative deaths up to the end of interval \(t\).

Define the risk set size at the start of interval \(t\) as \[
N_d(t) = N_d(0) - \sum_{s < t} d_d(s) = N_d(0) - D_d(t-1).
\] In the primary estimand, individuals do not switch cohorts after
enrollment and there is no loss to follow-up; therefore \(N_d(t)\) is
the risk set used to define all discrete-time hazards and cumulative
hazards in this manuscript.

Define the interval mortality ratio \[
\mathrm{MR}_{d,t} = \frac{d_d(t)}{N_d(t)}.
\]

We compute the discrete-time cohort hazard as

\begin{equation}\protect\phantomsection\label{eq:hazard-discrete}{
h_{\mathrm{obs},d}(t) = -\ln\!\left(1 - \mathrm{MR}_{d,t}\right) = -\ln\!\left(1 - \frac{d_d(t)}{N_d(t)}\right).
}\end{equation}

This transform is standard: it maps an interval event probability into a
continuous-time equivalent hazard under a piecewise-constant hazard
assumption. For rare events,
\(h_{\mathrm{obs},d}(t) \approx \mathrm{MR}_{d,t} = d_d(t)/N_d(t)\), but
the log form remains accurate and stable when weekly risks are not
negligible.

\emph{All hazard and cumulative-hazard quantities used in KCOR are
discrete-time integrated hazard estimators derived from fixed-cohort
risk sets; we do not rely on likelihood-based or partial-likelihood
formulations for estimation or for the subsequent frailty-based
normalization.}

Observed cumulative hazards are accumulated over event time after an
optional stabilization skip (§2.7):

\begin{equation}\protect\phantomsection\label{eq:cumhazard-observed}{
H_{\mathrm{obs},d}(t) = \sum_{s \le t} h_d^{\mathrm{eff}}(s),
\qquad \Delta t = 1.
}\end{equation}

Discrete binning accommodates tied events and aggregated registry
releases. Bin width is chosen based on diagnostic stability (e.g.,
smoothness and sufficient counts per bin) rather than temporal
resolution alone.

In addition to the primary implementation above, we computed
\(\hat H_{\mathrm{obs},d}(t)\) using the Nelson--Aalen estimator
\(\sum_{s \le t} d_d(s)/N_d(s)\) as a sensitivity check; results were
unchanged.

\begin{center}\rule{0.5\linewidth}{0.5pt}\end{center}

\subsection{2.4 Selection model: gamma frailty and depletion
normalization}\label{selection-model-gamma-frailty-and-depletion-normalization}

\subsubsection{2.4.1 Individual hazards with multiplicative
frailty}\label{individual-hazards-with-multiplicative-frailty}

Within cohort \(d\), individual \(i\) is modeled as having hazard

\begin{equation}\protect\phantomsection\label{eq:individual-hazard-frailty}{
h_{i,d}(t) = z_{i,d}\,h_{0,d}(t),
\qquad
z_{i,d} \sim \mathrm{Gamma}(\mathrm{mean}=1,\ \mathrm{var}=\theta_d).
}\end{equation}

Here \(h_{0,d}(t)\) is the cohort's depletion-neutralized baseline
hazard and \(z_{i,d}\) is a latent multiplicative frailty term. The
frailty variance \(\theta_d\) governs the strength of depletion-induced
curvature: larger \(\theta_d\) yields stronger deceleration at the
cohort level due to faster early depletion of high-frailty individuals.

Gamma frailty is used because it yields a closed-form link between
observed and baseline cumulative hazards via the Laplace
transform\textsuperscript{1}. In KCOR, gamma frailty is a
\textbf{working geometric model} for depletion normalization, not a
claim of biological truth. Adequacy is evaluated empirically via fit
quality, post-normalization linearity, and stability diagnostics.

\subsubsection{2.4.2 Gamma-frailty identity and
inversion}\label{gamma-frailty-identity-and-inversion}

Let

\begin{equation}\protect\phantomsection\label{eq:baseline-cumhazard}{
H_{0,d}(t) = \int_0^t h_{0,d}(s)\,ds
}\end{equation}

denote the baseline cumulative hazard. Integrating over gamma frailty
yields the gamma-frailty identity

\begin{equation}\protect\phantomsection\label{eq:gamma-frailty-identity}{
H_{\mathrm{obs},d}(t) = \frac{1}{\theta_d}\,\log\!\left(1 + \theta_d H_{0,d}(t)\right),
}\end{equation}

which can be inverted exactly as

\begin{equation}\protect\phantomsection\label{eq:gamma-frailty-inversion}{
H_{0,d}(t) = \frac{\exp\!\left(\theta_d H_{\mathrm{obs},d}(t)\right) - 1}{\theta_d}.
}\end{equation}

This inversion is the \textbf{normalization operator}: given an estimate
\(\hat{\theta}_d\), it maps the observed cumulative hazard
\(H_{\mathrm{obs},d}(t)\) into a depletion-neutralized cumulative hazard
scale.

\subsubsection{2.4.3 Baseline shape used for frailty
identification}\label{baseline-shape-used-for-frailty-identification}

To identify \(\theta_d\), KCOR fits the gamma-frailty model within
prespecified epidemiologically quiet periods. In the reference
specification, the baseline hazard is taken to be constant over the fit
window:

\begin{equation}\protect\phantomsection\label{eq:baseline-shape-default}{
h_{0,d}(t)=k_d,
\qquad
H_{0,d}(t)=k_d\,t.
}\end{equation}

This choice intentionally minimizes degrees of freedom: during a quiet
window, curvature is forced to be explained by depletion (via
\(\theta_d\)) rather than by introducing time-varying baseline hazard
terms. If the observed cumulative hazard is near-linear over the fit
window, the model naturally collapses toward
\(\hat{\theta}_d \approx 0\), signaling weak or absent detectable
depletion curvature for that cohort over that window.

\subsubsection{2.4.4 Quiet-window validity as the key dataset-specific
requirement}\label{quiet-window-validity-as-the-key-dataset-specific-requirement}

Frailty parameters are estimated using only bins whose corresponding
calendar weeks lie inside a prespecified quiet window (defined in
ISO-week space). A window is acceptable only if diagnostics indicate (i)
good fit in cumulative-hazard space, (ii) post-normalization linearity
within the window, and (iii) stability of \((\hat{k}_d,\hat{\theta}_d)\)
to small boundary perturbations. If no candidate window passes, KCOR is
treated as not identified for that analysis rather than producing a
potentially misleading normalized contrast. All diagnostics are computed
over discrete event-time bins (weekly intervals since enrollment) whose
corresponding calendar weeks fall within the prespecified quiet window.

\paragraph{Quiet-window selection protocol
(operational)}\label{quiet-window-selection-protocol-operational}

\textbf{Quiet-window selection protocol.} The quiet window is selected
prior to KCOR estimation using the following operational criteria:

\begin{enumerate}
\def\labelenumi{\arabic{enumi}.}
\tightlist
\item
  Calendar-time hazard curves exhibit approximate linearity with no
  sustained trend breaks.
\item
  Periods containing epidemic waves, reporting artifacts, or policy
  shocks are excluded.
\item
  The window spans a minimum duration sufficient for stable slope
  estimation.
\item
  Sensitivity is assessed by perturbing the window boundaries (± several
  weeks).
\end{enumerate}

\emph{Practical example.} In COVID-19 mortality analyses, a quiet window
may be defined as an inter-wave period between major variant surges,
verified by approximately linear all-cause cumulative hazards in the
general population and the absence of cohort-differential policy or
reporting shocks. The specific calendar bounds are not assumed to be
unique or correct a priori; instead, robustness to small perturbations
of the window boundaries (e.g., ± several weeks) is treated as a core
diagnostic. If fitted depletion parameters or post-normalization
linearity are unstable under such perturbations, the quiet-window
assumption is deemed violated and KCOR is treated as not identified for
that analysis.

\begin{center}\rule{0.5\linewidth}{0.5pt}\end{center}

\subsubsection{2.5 Estimation during quiet periods (cumulative-hazard
least
squares)}\label{estimation-during-quiet-periods-cumulative-hazard-least-squares}

KCOR estimates \((\hat{k}_d,\hat{\theta}_d)\) independently for each
cohort \(d\) using only time bins that fall inside a prespecified
\textbf{quiet window} in calendar time (ISO week space). The quiet
window is prespecified and applied consistently across cohorts within an
analysis; robustness to alternate quiet-window bounds is assessed in
sensitivity analyses. Quiet periods are identified diagnostically via
stability of observed cumulative hazards and absence of external shocks,
rather than by a fixed universal numeric threshold. The epidemiological
quiet period is not assumed to be free of all processes, but to lack
sharp, cohort-differential hazard perturbations (e.g., epidemic waves or
policy shocks) capable of inducing systematic curvature asymmetry
unrelated to selection dynamics. Let \(\mathcal{T}_d\) denote the set of
event-time bins \(t\) whose corresponding calendar week lies in the
quiet window, with \(t\) also satisfying \(t \ge \mathrm{SKIP\_WEEKS}\).

Under the default baseline shape, the model-implied observed cumulative
hazard is

\begin{equation}\protect\phantomsection\label{eq:hobs-model}{
H_{d}^{\mathrm{model}}(t; k_d, \theta_d) = \frac{1}{\theta_d}\,\log\!\left(1+\theta_d k_d t\right).
}\end{equation}

Identifiability of \((\hat{k}_d,\hat{\theta}_d)\) comes from curvature
in cumulative-hazard space: observed cumulative hazards are nonlinear in
event time when \(\theta_d>0\). When depletion is weak (or the quiet
window is too short to show curvature), the model smoothly collapses to
a linear cumulative hazard, since
\(H_{d}^{\mathrm{model}}(t; k_d, \theta_d) \to k_d t\) as
\(\theta_d \to 0\). Operationally, near-linear observed cumulative
hazards naturally drive the fitted frailty variance toward zero; fit
diagnostics such as \(n_{\mathrm{obs}}\) and RMSE in \(H\)-space provide
a practical check on whether the selection parameters are being
identified from the quiet-window data. In practice, lack of identifiable
curvature naturally manifests as fitted frailty variance estimates
approaching zero, providing an internal diagnostic for
non-identifiability over short or sparse follow-up.

In applied analyses, this behavior is most commonly observed in
vaccinated cohorts, whose cumulative hazards during quiet periods are
often close to linear. In such cases, the gamma-frailty fit collapses
naturally, indicating minimal detectable depletion. This outcome is
data-driven and reflects the absence of observable selection-induced
curvature rather than a modeling assumption. When residual time-varying
risk contaminates a nominally quiet window, fitted frailty variance
estimates naturally shrink toward zero, signaling limited
identifiability rather than inducing spurious correction.

Parameters are estimated by constrained nonlinear least squares:

\begin{equation}\protect\phantomsection\label{eq:nls-objective}{
(\hat{k}_d,\hat{\theta}_d)
=
\arg\min_{k_d>0,\ \theta_d \ge 0}
\sum_{t \in \mathcal{T}_d}
\left[
H_{\mathrm{obs},d}(t) - H_{d}^{\mathrm{model}}(t; k_d, \theta_d)
\right]^2.
}\end{equation}

We fit in cumulative-hazard space rather than maximizing a likelihood
because the primary inputs are discrete-time, cohort-aggregated hazards
and the objective is stable estimation of selection-induced depletion
curvature during quiet periods. Least-squares fitting is used as a
numerical estimating equation rather than as a likelihood-based
estimator. Least squares on observed cumulative hazards is numerically
robust under sparse events, emphasizes shape agreement over the fit
window, and yields diagnostics (e.g., RMSE in \(H\)-space) that directly
reflect the quality of the depletion fit. Likelihood-based fitting can
be treated as a sensitivity analysis, but is not required for the
normalization identity itself.

All analyses use a prespecified reference implementation with fixed
operational defaults; full details are provided in Appendix E.

\subsubsection{2.6 Normalization (depletion-neutralized cumulative
hazards)}\label{normalization-depletion-neutralized-cumulative-hazards}

After fitting, KCOR computes the depletion-neutralized baseline
cumulative hazard for each cohort \(d\) by applying the inversion to the
full post-enrollment trajectory:

\begin{equation}\protect\phantomsection\label{eq:normalized-cumhazard}{
\tilde{H}_{0,d}(t) = \frac{\exp\!\left(\hat{\theta}_d\,H_{\mathrm{obs},d}(t)\right)-1}{\hat{\theta}_d}.
}\end{equation}

This normalization maps each cohort into a depletion-neutralized
baseline-hazard space in which the contribution of gamma frailty
parameters \((\hat{\theta}_d, \hat{k}_d)\) to hazard curvature has been
factored out. This normalization defines a common comparison scale in
cumulative-hazard space; it is not equivalent to Cox partial-likelihood
baseline anchoring, but serves an analogous geometric role for
cumulative contrasts. In this space, cumulative hazards are directly
comparable across cohorts, and remaining differences reflect real
differences in baseline risk rather than selection-induced depletion.

\paragraph{KCOR Identity Summary}\label{kcor-identity-summary}

This block summarizes the core KCOR identities used throughout the
manuscript; each appears elsewhere in full detail. References point to
the original labeled equations to preserve equation numbering.

Unlabeled summary forms:

\[
h_{\mathrm{obs},d}(t) = -\ln\!\left(1 - \frac{d_d(t)}{N_d(t)}\right)
\qquad \text{(see Equation @eq:hazard-discrete).}
\]

\[
(\hat{k}_d,\hat{\theta}_d)
=
\arg\min_{k_d>0,\ \theta_d \ge 0}
\sum_{t \in \mathcal{T}_d}
\left[
H_{\mathrm{obs},d}(t) - H_{d}^{\mathrm{model}}(t; k_d, \theta_d)
\right]^2
\qquad \text{(see Equation @eq:nls-objective).}
\]

\[
\tilde{H}_{0,d}(t) = \frac{\exp\!\left(\hat{\theta}_d\,H_{\mathrm{obs},d}(t)\right)-1}{\hat{\theta}_d}
\qquad \text{(see Equation @eq:normalized-cumhazard).}
\]

\[
\mathrm{KCOR}(t) = \frac{\tilde{H}_{0,A}(t)}{\tilde{H}_{0,B}(t)}
\qquad \text{(see Equation @eq:kcor-estimand).}
\]

Where: \(d\) indexes cohorts; \(t\) is discrete event time since
enrollment; \(N_d(t)\) is the risk set size at the start of interval
\(t\); \(d_d(t)\) is deaths in interval \(t\); \(\mathcal{T}_d\) is the
set of fit bins lying inside the prespecified quiet window;
\(H_{\mathrm{obs},d}(t)\) is the observed cumulative hazard; and
\(\tilde{H}_{0,d}(t)\) is the depletion-neutralized baseline cumulative
hazard used for cohort comparison.

When

\[
\hat{\theta}_d \approx 0,
\]

this mapping leaves the observed cumulative hazard essentially
unchanged, whereas cohorts with measurable depletion undergo an upward
correction.

A depletion-neutralized baseline hazard may be recovered by
differencing:

\begin{equation}\protect\phantomsection\label{eq:normalized-hazard-diff}{
\tilde{h}_{0,d}(t) \approx \tilde{H}_{0,d}(t) - \tilde{H}_{0,d}(t-1).
}\end{equation}

The key object for KCOR is:

\[
\tilde{H}_{0,d}(t).
\]

Differenced hazards are optional diagnostics.

Normalization is necessary but not sufficient. The depletion-neutralized
baseline cumulative hazard \(\tilde{H}_{0,d}(t)\) is not itself the
estimand of interest.

\emph{Rather, normalization defines a common comparison space; the
choice of estimand on that space is a scientific decision. This work
focuses on cumulative hazard ratios because they are stable,
interpretable, and directly aligned with the normalization geometry.}

Its role is to place cohorts into a common comparison space in which
selection-induced depletion dynamics have been removed. The substantive
comparison---and therefore the inferential signal---arises only when
these depletion-neutralized baseline cumulative hazards are compared
across cohorts via the KCOR estimator (§2.8). Because normalization
operates in cumulative-hazard space and removes time-varying curvature
rather than rescaling instantaneous hazards, applying Cox regression to
depletion-neutralized outputs generally re-introduces misspecification.
Applying standard proportional-hazards or regression-based estimators
after normalization is generally inappropriate, because the comparison
is cumulative by construction and because residual non-proportionality
is precisely what KCOR is designed to reveal. KCOR therefore integrates
normalization and comparison into a single, internally consistent
system.

In KCOR, the parametric model is used solely to estimate and invert
selection-induced curvature in cumulative-hazard space; treatment
comparisons are then made directly on the adjusted data. In contrast,
Cox regression fits a hazard model to the observed data and derives
treatment effects from model coefficients, implicitly attributing all
systematic hazard divergence---including selection effects---to the
exposure.

\paragraph{2.6.1 Computational
considerations}\label{computational-considerations}

KCOR operates on aggregated event counts in discrete time and
cumulative-hazard space. Computational complexity scales linearly with
the number of time bins and strata rather than the number of
individuals, making the method feasible for very large population
registries. In practice, KCOR analyses on national-scale datasets
(millions of individuals) are memory-bound rather than CPU-bound and can
be implemented efficiently using standard vectorized numerical
libraries. No iterative optimization over individual-level records is
required.

\paragraph{2.6.2 Internal diagnostics and `self-check'
behavior}\label{internal-diagnostics-and-self-check-behavior}

KCOR includes internal diagnostics intended to make model stress visible
rather than hidden.

\begin{enumerate}
\def\labelenumi{\arabic{enumi}.}
\item
  \textbf{Post-normalization linearity in quiet periods.} During a
  prespecified quiet window, the working model assumes that curvature in
  observed cumulative hazard is primarily driven by depletion under
  heterogeneity. After inversion, the depletion-neutralized cumulative
  hazard should be approximately linear in event time over the same
  quiet window. Systematic residual curvature (e.g., sustained
  concavity/convexity) indicates that the quiet-window assumption is
  violated (external shocks, secular trends) or that the depletion
  geometry is misspecified for that cohort.
\item
  \textbf{Fit residual structure in cumulative-hazard space.} Define
  residuals over the fit set \(\mathcal{T}_d\):
\end{enumerate}

\[
r_{d}(t)=H_{\mathrm{obs},d}(t)-H_{d}^{\mathrm{model}}(t;\hat{k}_d,\hat{\theta}_d).
\]

KCOR expects residuals to be small and not systematically
time-structured. Strongly patterned residuals indicate that the
curvature attributed to depletion is instead being driven by unmodeled
time-varying hazards.

\begin{enumerate}
\def\labelenumi{\arabic{enumi}.}
\setcounter{enumi}{2}
\tightlist
\item
  \textbf{Parameter stability to window perturbations.} Under valid
  quiet-window selection,
\end{enumerate}

\[
(\hat{k}_d,\hat{\theta}_d)
\]

should be stable to small perturbations of the quiet-window boundaries
(e.g., ±4 weeks). Large changes in fitted frailty variance under small
boundary shifts signal that the fitted curvature is sensitive to
transient dynamics rather than stable depletion.

\begin{enumerate}
\def\labelenumi{\arabic{enumi}.}
\setcounter{enumi}{3}
\tightlist
\item
  \textbf{Non-identifiability manifests as:}
\end{enumerate}

\[
\hat{\theta}_d\rightarrow 0.
\]

When the observed cumulative hazard is near-linear (weak curvature) or
events are sparse, \(\theta\) is weakly identified. In such cases, KCOR
should be interpreted primarily as a diagnostic (limited evidence of
detectable depletion curvature) rather than a strong correction.

These diagnostics are reported alongside KCOR\((t)\) curves. The goal is
not to assert that a single parametric form is always correct, but to
ensure that when the form is incorrect or the window is contaminated,
the method signals this explicitly rather than silently producing a
misleading `corrected' estimate. Failure of these diagnostics indicates
that the depletion-based normalization is inappropriate, in which case
KCOR should not be interpreted.

\subsubsection{2.7 Stabilization (early
weeks)}\label{stabilization-early-weeks}

In many applications, the first few post-enrollment intervals can be
unstable due to immediate post-enrollment artifacts (e.g., rapid
deferral, short-term sorting, administrative effects). KCOR supports a
prespecified stabilization rule by excluding early weeks from
accumulation and from quiet-window fitting. The skip-weeks parameter is
prespecified and evaluated via sensitivity analysis to exclude early
enrollment instability rather than to tune estimates.

In discrete time, define an effective hazard for accumulation:

\begin{equation}\protect\phantomsection\label{eq:effective-hazard-skip}{
h_d^{\mathrm{eff}}(t)=
\begin{cases}
0, & t < \mathrm{SKIP\_WEEKS} \\
h_{\mathrm{obs},d}(t), & t \ge \mathrm{SKIP\_WEEKS}.
\end{cases}
}\end{equation}

Then compute observed cumulative hazards from \(h_d^{\mathrm{eff}}(t)\)
as in §2.3:

\[
H_{\mathrm{obs},d}(t).
\]

\subsubsection{2.8 KCOR estimator}\label{kcor-estimator}

For cohorts \(A\) and \(B\), KCOR compares depletion-neutralized
cumulative hazards:

\begin{equation}\protect\phantomsection\label{eq:kcor-estimator}{
\mathrm{KCOR}(t) = \frac{\tilde{H}_{0,A}(t)}{\tilde{H}_{0,B}(t)}.
}\end{equation}

This is a cumulative comparison in hazard space after removing
cohort-specific selection-induced depletion dynamics estimated during
quiet periods.

\subsubsection{2.9 Uncertainty
quantification}\label{uncertainty-quantification}

Uncertainty is quantified using bootstrap resampling as described below,
propagating uncertainty in both the event process and the fitted
depletion parameters \((\hat{k}_d,\hat{\theta}_d)\).

Bootstrap resampling is preferred over asymptotic approximations (e.g.,
delta method) for several reasons. First, frailty parameter estimation
involves nonlinear optimization, and the uncertainty in
\((\hat{k}_d,\hat{\theta}_d)\) propagates nonlinearly through the
gamma-frailty inversion to the normalized cumulative hazards. Bootstrap
naturally captures this estimation uncertainty without requiring normal
approximation assumptions. Second, the transformation from observed to
normalized cumulative hazards via the gamma-frailty inversion is
nonlinear, making closed-form variance propagation difficult. Bootstrap
resampling propagates uncertainty through the entire pipeline---from
event counts through frailty fitting, inversion, and KCOR
computation---without requiring linearization assumptions. Coverage
below 95\% under model misspecification (e.g., non-gamma frailty) is
expected and diagnostic, signaling when the working frailty model
inadequately captures the true depletion geometry.

\paragraph{2.9.1 Stratified bootstrap
procedure}\label{stratified-bootstrap-procedure}

The stratified bootstrap procedure for KCOR proceeds as follows:

\begin{enumerate}
\def\labelenumi{\arabic{enumi}.}
\item
  \textbf{Resample individuals (or counts).} Within each cohort and
  stratum (e.g., age group), resample individuals with replacement,
  preserving the original cohort and stratum structure. Alternatively,
  for aggregated data, resample event counts and risk-set sizes within
  each time bin and stratum.
\item
  \textbf{Re-estimate frailty parameters.} For each bootstrap replicate,
  re-estimate \((\hat{k}_d,\hat{\theta}_d)\) independently for each
  cohort \(d\) using the resampled data, applying the same quiet-window
  selection and fitting procedure as in the primary analysis.
\item
  \textbf{Recompute normalized cumulative hazards.} Using the
  bootstrap-estimated frailty parameters, recompute
  \(\tilde{H}_{0,d}(t)\) for each cohort via the gamma-frailty inversion
  applied to the resampled observed cumulative hazards.
\item
  \textbf{Recompute KCOR.} Compute \(\mathrm{KCOR}(t)\) for each
  bootstrap replicate as the ratio of the bootstrap-normalized
  cumulative hazards.
\item
  \textbf{Form percentile intervals.} From the bootstrap distribution of
  \(\mathrm{KCOR}(t)\) values at each time point, form percentile-based
  confidence intervals (e.g., 2.5th and 97.5th percentiles for 95\%
  intervals).
\end{enumerate}

Uncertainty intervals reflect stochastic event realization and model-fit
uncertainty in the selection-parameter estimation. They do not assume
sampling from a superpopulation and may be interpreted as uncertainty
conditional on the observed risk sets and modeling assumptions.

\subsubsection{2.10 Algorithm summary and reproducibility
checklist}\label{algorithm-summary-and-reproducibility-checklist}

Table \ref{tbl:KCOR_algorithm} summarizes the complete KCOR pipeline.

\begin{figure}
\centering
\pandocbounded{\includegraphics[keepaspectratio,alt={KCOR as a two-stage framework. (A) Fixed-cohort cumulative hazards exhibit curvature due to selection-induced depletion; late-time curvature is used to estimate frailty parameters for normalization. (B) Gamma-frailty normalization yields approximately linearized cumulative hazards that are directly comparable across cohorts; \textbackslash mathrm\{KCOR\}(t), defined as the ratio of depletion-neutralized baseline cumulative hazards, is near-flat under the null and deviates only under net hazard differences. In the schematic, \textbackslash tilde\{H\}\_\{0,d\}(t) denotes the depletion-neutralized baseline cumulative hazard.}]{figures/fig_kcor_workflow.png}}
\caption{\textbf{KCOR as a two-stage framework.} \textbf{(A)}
Fixed-cohort cumulative hazards exhibit curvature due to
selection-induced depletion; late-time curvature is used to estimate
frailty parameters for normalization. \textbf{(B)} Gamma-frailty
normalization yields approximately linearized cumulative hazards that
are directly comparable across cohorts; \(\mathrm{KCOR}(t)\), defined as
the ratio of depletion-neutralized baseline cumulative hazards, is
near-flat under the null and deviates only under net hazard differences.
\emph{In the schematic, \(\tilde{H}_{0,d}(t)\) denotes the
depletion-neutralized baseline cumulative
hazard.}}\label{fig:kcor_workflow}
\end{figure}

\subsubsection{2.11 Relationship to Cox proportional
hazards}\label{relationship-to-cox-proportional-hazards}

Cox proportional hazards models estimate an instantaneous hazard ratio
under the assumption that hazards differ by a time-invariant
multiplicative factor. Under selective uptake with latent frailty
heterogeneity, this assumption is typically violated, yielding
time-varying hazard ratios induced purely by depletion dynamics. This
reflects an estimand mismatch: Cox targets a different quantity under
depletion than KCOR's cumulative hazard estimand. Cox is behaving
correctly for its estimand, but that estimand may not align with the
scientific question when selection-induced depletion is present.
Accordingly, Cox results are presented here as a diagnostic
demonstration of estimand mismatch, not as a competing causal estimator.

Cox regression estimates a weighted average hazard ratio under
non-proportional hazards; KCOR targets a cumulative hazard estimand.
Even when Cox models are extended with shared frailty to accommodate
heterogeneity, they continue to estimate instantaneous hazard ratios
conditional on survival, whereas KCOR estimates cumulative contrasts
after explicit depletion normalization.

Conceptually, Cox regression estimates a treatment effect by fitting a
hazard model to the observed data and reading the effect from the fitted
coefficients, whereas KCOR uses a parametric model only to normalize the
observed data for selection-induced depletion and then computes cohort
contrasts directly from the depletion-neutralized baseline cumulative
hazards themselves.

\paragraph{2.11.1 Demonstration: Cox bias under frailty heterogeneity
with no treatment
effect}\label{demonstration-cox-bias-under-frailty-heterogeneity-with-no-treatment-effect}

We conducted a controlled synthetic experiment in which the \textbf{true
causal effect is known to be zero by construction}, isolating latent
frailty heterogeneity as the sole driver of depletion-induced
non-proportional hazards. Cox and KCOR were applied to the same
simulated datasets under identical information constraints.

\textbf{Data-generating process.}

Two cohorts of equal size were simulated under the same baseline hazard
\(h_0(t)\) over time (constant or Gompertz). Individual hazards were
generated as \(z\,h_0(t)\), with frailty \[
z \sim \text{Gamma}(\theta^{-1}, \theta^{-1}),
\] with mean 1 and variance \(\theta\).

Cohort A was generated with \(\theta = 0\) (no frailty heterogeneity),
while Cohort B was generated with \(\theta > 0\). \textbf{No treatment
or intervention effect was applied}: conditional on frailty, the two
cohorts have identical hazards at all times. Thus, the true causal
hazard ratio between cohorts is exactly 1 for all \(t\).

Simulations were repeated over a grid of frailty variances
\(\theta \in \{0, 0.5, 1, 2, 5, 10, 20\}\).

\textbf{Cox analysis.}

For each simulated dataset, we fitted a standard Cox proportional
hazards model using partial likelihood (statsmodels \texttt{PHReg}),
with cohort membership as the sole covariate (no time-varying covariates
or interactions). The resulting hazard ratio estimates and confidence
intervals therefore reflect \textbf{only differences induced by
frailty-driven depletion}, not any causal effect.

\textbf{KCOR analysis.}

The same simulated datasets were analyzed using KCOR. For the synthetic
datasets, cohort-specific observed cumulative hazards were estimated
nonparametrically using the Nelson--Aalen estimator, then mapped to
depletion-neutralized baseline cumulative hazards via the gamma-frailty
inversion prior to computing \(\mathrm{KCOR}(t)\). Although the
data-generating process specifies individual hazards, cumulative hazards
were estimated from simulated event-time data using Nelson--Aalen to
mirror the information available in observational registry studies,
rather than exploiting simulator-only knowledge. Frailty parameters were
estimated during a prespecified quiet window, followed by
cumulative-hazard normalization and computation of \(\mathrm{KCOR}(t)\).
Post-normalization slope and asymptotic \(\mathrm{KCOR}(t)\) values were
examined to assess departure from the null.

\textbf{Expected behavior under the null.}

Because the data-generating process includes \textbf{no treatment
effect}, any valid estimator should return a null result. In this
setting:

\begin{itemize}
\tightlist
\item
  \textbf{Cox regression} is expected to produce apparent non-null
  hazard ratios as \(\theta\) increases, reflecting differential
  depletion of susceptibles and violation of proportional hazards
  induced by frailty heterogeneity.
\item
  \textbf{KCOR} is expected to remain centered near unity with
  negligible post-normalization slope across all \(\theta\), consistent
  with correct null behavior after depletion normalization.
\end{itemize}

\textbf{Summary of findings.}

Across increasing values of \(\theta\), Cox regression produced
progressively larger apparent deviations from a hazard ratio of 1. The
direction and magnitude of the apparent effect depended on the follow-up
horizon and degree of frailty heterogeneity. In contrast,
\(\mathrm{KCOR}(t)\) trajectories remained stable and centered near
unity, with post-normalization slopes approximately zero across all
simulated conditions.

These results demonstrate that \textbf{frailty heterogeneity alone is
sufficient to induce spurious hazard ratios in Cox regression}, while
KCOR correctly returns a null result under the same conditions.

Table \ref{tbl:cox_bias_demo} reports numerical summaries of the
Cox-vs-KCOR behavior across the frailty grid.

Additional Cox HR results from the same synthetic-null grid are shown in
Figure \ref{fig:cox_bias_hr}.

\begin{figure}
\centering
\pandocbounded{\includegraphics[keepaspectratio,alt={Cox regression produces spurious non-null hazard ratios under a synthetic null as frailty heterogeneity increases. Hazard ratios (with 95\% confidence intervals) from Cox proportional hazards regression comparing cohort B to cohort A in simulations where the true treatment effect is identically zero and cohorts differ only in frailty variance (\textbackslash theta). Deviations from HR=1 arise solely from frailty-driven depletion and associated non-proportional hazards.}]{figures/fig_cox_bias_hr_vs_theta.png}}
\caption{Cox regression produces spurious non-null hazard ratios under a
\emph{synthetic null} as frailty heterogeneity increases. Hazard ratios
(with 95\% confidence intervals) from Cox proportional hazards
regression comparing cohort B to cohort A in simulations where the true
treatment effect is identically zero and cohorts differ only in frailty
variance (\(\theta\)). Deviations from HR=1 arise solely from
frailty-driven depletion and associated non-proportional
hazards.}\label{fig:cox_bias_hr}
\end{figure}

\begin{figure}
\centering
\pandocbounded{\includegraphics[keepaspectratio,alt={\textbackslash mathrm\{KCOR\}(t) remains null under a synthetic null across increasing frailty heterogeneity. \textbackslash mathrm\{KCOR\}(t) asymptotes remain near 1 across \textbackslash theta in the same simulations, consistent with correct null behavior after depletion normalization. Uncertainty bands (95\% bootstrap intervals) are shown but are narrow due to large sample sizes.}]{figures/fig_cox_bias_kcor_vs_theta.png}}
\caption{\(\mathrm{KCOR}(t)\) remains null under a synthetic null across
increasing frailty heterogeneity. \(\mathrm{KCOR}(t)\) asymptotes remain
near 1 across \(\theta\) in the same simulations, consistent with
correct null behavior after depletion normalization. Uncertainty bands
(95\% bootstrap intervals) are shown but are narrow due to large sample
sizes.}\label{fig:cox_bias_kcor}
\end{figure}

\textbf{Interpretation.}

This demonstration shows that Cox proportional hazards regression can
report highly statistically significant non-null hazard ratios---even
when the true treatment effect is identically zero---solely due to
frailty-induced depletion (e.g., \(p < 10^{-300}\) at \(\theta=20\)),
with the magnitude and direction of the apparent effect depending on
follow-up horizon, not any causal signal. This reflects an estimand
mismatch: Cox targets instantaneous hazard ratios under proportional
hazards assumptions, whereas KCOR targets cumulative contrasts after
depletion normalization. Cox is behaving correctly for its estimand, but
that estimand conditions on the evolving risk set and therefore reflects
depletion dynamics rather than a causal effect under these conditions.
KCOR, by explicitly normalizing depletion geometry in cumulative-hazard
space, correctly returns a null result under the same conditions, with
\(\mathrm{KCOR}(t)\) remaining centered near unity with negligible
post-normalization slope across all frailty variance values. This
controlled example motivates the use of KCOR for retrospective vaccine
studies, where frailty heterogeneity and non-proportional hazards are
expected to be substantial.

This synthetic null isolates depletion-driven non-proportionality as a
sufficient cause of strongly non-null Cox estimates, demonstrating the
estimand mismatch that motivates cumulative-hazard--based normalization
when selection-induced curvature is present.

\paragraph{2.11.2 Relation to other approaches addressing selection
effects}\label{relation-to-other-approaches-addressing-selection-effects}

Several existing approaches, including inverse-probability weighting,
marginal structural models, and joint frailty formulations, aim to
mitigate selection bias in observational survival analyses. These
methods rely on explicit modeling of treatment assignment or hazard
structure and generally operate in instantaneous-hazard space. KCOR
differs in focusing directly on depletion geometry in cumulative-hazard
space and enforcing diagnostic checks that signal loss of
identifiability when normalization is not supported by the data. A full
head-to-head empirical comparison across all model families is beyond
scope of this methods manuscript; however, we explicitly contrast
estimands, assumptions, and failure modes in §1.5 and we provide
controlled demonstrations (including a synthetic null) showing that
Cox-type estimators can return strongly non-null hazard ratios under
frailty-driven depletion with no treatment effect, whereas KCOR remains
near-null under the same conditions. This is the specific geometry KCOR
targets: depletion-induced non-proportionality in cumulative-hazard
space.

\subsubsection{2.12 Worked example
(descriptive)}\label{worked-example-descriptive}

We include a brief worked example to illustrate the KCOR workflow
end-to-end. This example is descriptive and intended solely to
demonstrate the mechanics of cohort construction, hazard estimation,
frailty fitting, depletion normalization, and KCOR computation, without
supporting causal interpretation.

The example proceeds from aggregated cohort counts through
cumulative-hazard estimation, quiet-window frailty fitting, gamma
inversion, and \(\mathrm{KCOR}(t)\) construction, accompanied by
diagnostic plots assessing post-normalization linearity and parameter
stability.

\subsubsection{2.13 Reproducibility}\label{reproducibility}

All figures, tables, and simulations in this manuscript can be
reproduced from the accompanying code repository. The repository
includes scripts to regenerate simulations, figures, and tables using
fixed random seeds, together with documentation of software versions and
runtime requirements. Commands and execution order are provided in the
repository README.

\subsubsection{2.14 Computational implementation and
reproducibility}\label{computational-implementation-and-reproducibility}

Analyses were implemented in Python and are fully reproducible from the
public repository and archived release (see Code/Data Availability). The
repository includes scripts to (i) generate synthetic cohorts and
simulation grids, (ii) compute KCOR and comparator estimators, and (iii)
regenerate all tables and figures in this manuscript.

\textbf{Environment.} Python 3.11; key dependencies include numpy,
scipy, pandas, and lifelines (for Cox-model comparisons), with plotting
via matplotlib.

\textbf{Compute requirements.} The full simulation grid reproduces in
approximately 1 hour 26 minutes on a 20-core CPU with 128 GB RAM;
smaller subsets reproduce in minutes.

\textbf{Reproduction.} Running \texttt{make\ paper} (or the repository's
top-level build command) regenerates all artifacts from a clean
checkout.

Technical derivations, algebraic details, and additional simulation
results are provided in the Appendix.

\subsection{3. Validation and control
tests}\label{validation-and-control-tests}

This section is the core validation claim of KCOR:

\begin{itemize}
\tightlist
\item
  \textbf{Negative controls (null under selection):} under a true null
  effect, KCOR remains approximately flat at 1 even when selection
  induces large curvature differences.
\item
  \textbf{Positive controls (detect injected effects):} when known
  harm/benefit is injected into otherwise-null data, KCOR reliably
  detects it.
\end{itemize}

Throughout, curvature in cumulative hazard plots reflects
selection-induced depletion, while linearity after normalization
indicates successful removal of that curvature.

In vaccinated--unvaccinated comparisons, large early differences in
\(\mathrm{KCOR}(t)\) may reflect baseline risk selection rather than
intervention effects; in such cases we therefore emphasize
\(\mathrm{KCOR}(t; t_0)\), which reports deviations relative to an early
post-enrollment reference while preserving time-varying divergence.

\subsubsection{3.1 Negative controls: null under selection-induced
curvature}\label{negative-controls-null-under-selection-induced-curvature}

\paragraph{3.1.1 Fully synthetic negative control
(recommended)}\label{fully-synthetic-negative-control-recommended}

Design a simulation where:

\begin{itemize}
\tightlist
\item
  Individual hazards follow a baseline hazard \(h_0(t)\) multiplied by
  frailty \(z\).
\item
  Two cohorts have different frailty variance \(\theta\) (or different
  selection rules), creating different cohort-level curvature in
  observed cohort hazards.
\item
  \textbf{No treatment effect is applied} (both cohorts share the same
  baseline hazard \(h_0(t)\)).
\end{itemize}

After estimating \((\hat{k}_d,\hat{\theta}_d)\) during quiet periods and
applying gamma-frailty inversion, KCOR should remain approximately
constant at 1 over follow-up.

\subparagraph{In-model ``gamma-frailty'' stress test (highly convincing
null)}\label{in-model-gamma-frailty-stress-test-highly-convincing-null}

One especially clear falsification test is an \textbf{in-model
gamma-frailty null}: simulate data directly from the gamma-frailty model
with the same \(h_0(t)\) but different \(\theta\) between cohorts. This
induces strong, visibly different hazard curvature from depletion alone.
Because the data-generating process matches the model, the fitted
normalization is exact up to sampling noise, and KCOR should be flat at
1.

\textbf{Suggested construction (example):}

\begin{itemize}
\tightlist
\item
  Time unit: weeks.
\item
  Baseline hazard is constant during quiet periods; choose a baseline
  level \(k\) in the chosen time units.
\item
  Cohort A: \(\theta_A > 0\) (stronger depletion).
\item
  Cohort B: \(\theta_B > 0\) (weaker depletion).
\end{itemize}

Figure \ref{fig:neg_control_synthetic} shows this construction.

\begin{figure}
\centering
\pandocbounded{\includegraphics[keepaspectratio,alt={Synthetic negative control under strong selection (different curvature) but no effect: \textbackslash mathrm\{KCOR\}(t) remains flat at 1. Top panel shows cohort hazards with different frailty-mixture weights inducing different curvature. Bottom panel shows \textbackslash mathrm\{KCOR\}(t) remaining near 1.0 after normalization, demonstrating successful depletion-neutralization under the null. Uncertainty bands (95\% bootstrap intervals) are shown.}]{figures/fig_neg_control_synthetic.png}}
\caption{Synthetic negative control under strong selection (different
curvature) but no effect: \(\mathrm{KCOR}(t)\) remains flat at 1. Top
panel shows cohort hazards with different frailty-mixture weights
inducing different curvature. Bottom panel shows \(\mathrm{KCOR}(t)\)
remaining near 1.0 after normalization, demonstrating successful
depletion-neutralization under the null. Uncertainty bands (95\%
bootstrap intervals) are shown.}\label{fig:neg_control_synthetic}
\end{figure}

\paragraph{3.1.2 Empirical negative control using national registry data
(Czech
Republic)}\label{empirical-negative-control-using-national-registry-data-czech-republic}

This application is presented solely to illustrate KCOR's diagnostic
behavior on real registry data and does not support causal inference.

The repository includes a pragmatic negative control construction that
repurposes a real dataset by comparing ``like with like'' while inducing
large composition differences (e.g., age band shifts). In this
construction, age strata are remapped into pseudo-doses so that
comparisons are, by construction, within the same underlying category;
the expected differential effect is near zero, but the baseline hazards
differ strongly.

These age-shift negative controls deliberately induce extreme baseline
mortality differences (10--20 year age gaps) while preserving a true
null effect by construction, since all vaccination states are compared
symmetrically. The near-flat \(\mathrm{KCOR}(t)\) trajectories are
consistent with the estimator normalizing selection-induced depletion
curvature without introducing spurious time trends or cumulative drift.

For the empirical age-shift negative control (Figures
\ref{fig:neg_control_10yr} and \ref{fig:neg_control_20yr}), we use
aggregated weekly cohort summaries derived from the Czech Republic
administrative mortality and vaccination dataset and exported in
KCOR\_CMR format.

Notably, KCOR estimates frailty parameters independently for each cohort
without knowledge of exposure status; the observed asymmetry in
depletion correction arises entirely from differences in hazard
curvature rather than from any vaccination-specific assumptions.

Two snapshots illustrate that KCOR is near-flat even under 10--20 year
age differences:

\begin{figure}
\centering
\pandocbounded{\includegraphics[keepaspectratio,alt={Empirical negative control with approximately 10-year age difference between cohorts. Despite large baseline mortality differences, \textbackslash mathrm\{KCOR\}(t) remains near-flat at 1 over follow-up, consistent with a true null effect. Curves are shown as anchored \textbackslash mathrm\{KCOR\}(t; t\_0), i.e., \textbackslash mathrm\{KCOR\}(t)/\textbackslash mathrm\{KCOR\}(t\_0), which removes pre-existing cumulative differences and displays post-anchor divergence only. KCOR curves are anchored at t\_0 = 4 weeks (i.e., plotted as \textbackslash mathrm\{KCOR\}(t; t\_0)). Uncertainty bands (95\% bootstrap intervals) are shown. Data source: Czech Republic mortality and vaccination dataset processed into KCOR\_CMR aggregated format (negative-control construction; see Appendix B.2).}]{figures/fig2_neg_control_10yr_age_diff.png}}
\caption{Empirical negative control with approximately 10-year age
difference between cohorts. Despite large baseline mortality
differences, \(\mathrm{KCOR}(t)\) remains near-flat at 1 over follow-up,
consistent with a true null effect. Curves are shown as anchored
\(\mathrm{KCOR}(t; t_0)\), i.e.,
\(\mathrm{KCOR}(t)/\mathrm{KCOR}(t_0)\), which removes pre-existing
cumulative differences and displays post-anchor divergence only. KCOR
curves are anchored at \(t_0 = 4\) weeks (i.e., plotted as
\(\mathrm{KCOR}(t; t_0)\)). Uncertainty bands (95\% bootstrap intervals)
are shown. Data source: Czech Republic mortality and vaccination dataset
processed into KCOR\_CMR aggregated format (negative-control
construction; see Appendix B.2).}\label{fig:neg_control_10yr}
\end{figure}

The corresponding 20-year age-shift negative control is shown in Figure
\ref{fig:neg_control_20yr}.

\begin{figure}
\centering
\pandocbounded{\includegraphics[keepaspectratio,alt={Empirical negative control with approximately 20-year age difference between cohorts. Even under extreme composition differences, \textbackslash mathrm\{KCOR\}(t) exhibits no systematic drift, demonstrating robustness to selection-induced curvature. KCOR curves are anchored at t\_0 = 4 weeks (i.e., plotted as \textbackslash mathrm\{KCOR\}(t; t\_0)). Uncertainty bands (95\% bootstrap intervals) are shown. Data source: Czech Republic mortality and vaccination dataset processed into KCOR\_CMR aggregated format (negative-control construction; see Appendix B.2).}]{figures/fig3_neg_control_20yr_age_diff.png}}
\caption{Empirical negative control with approximately 20-year age
difference between cohorts. Even under extreme composition differences,
\(\mathrm{KCOR}(t)\) exhibits no systematic drift, demonstrating
robustness to selection-induced curvature. KCOR curves are anchored at
\(t_0 = 4\) weeks (i.e., plotted as \(\mathrm{KCOR}(t; t_0)\)).
Uncertainty bands (95\% bootstrap intervals) are shown. Data source:
Czech Republic mortality and vaccination dataset processed into
KCOR\_CMR aggregated format (negative-control construction; see Appendix
B.2).}\label{fig:neg_control_20yr}
\end{figure}

Table \ref{tbl:neg_control_summary} provides numeric summaries.

\subsubsection{3.2 Positive controls: detect injected
harm/benefit}\label{positive-controls-detect-injected-harmbenefit}

The effect window is a simulation construct used solely for
positive-control validation and does not represent a real-world
intervention period or biological effect window.

Positive controls are constructed by starting from a negative-control
dataset and injecting a known effect into the data-generating process
for one cohort, for example by multiplying the \emph{baseline} hazard by
a constant factor \(r\) over a prespecified interval:

\begin{equation}\protect\phantomsection\label{eq:pos-control-injection}{
h_{0,\mathrm{treated}}(t) = r \cdot h_{0,\mathrm{control}}(t) \quad \text{for } t \in [t_1, t_2],
}\end{equation}

with \(r>1\) for harm and \(0<r<1\) for benefit.

After gamma-frailty normalization (inversion), KCOR should deviate from
1 in the correct direction and with magnitude consistent with the
injected effect (up to discretization and sampling noise). Figure
\ref{fig:pos_control_injected} and Table \ref{tbl:pos_control_summary}
confirm this behavior.

\begin{figure}
\centering
\pandocbounded{\includegraphics[keepaspectratio,alt={Positive control validation: KCOR correctly detects injected effects. Left panels show harm scenario (r=1.2), right panels show benefit scenario (r=0.8). Top row displays cohort hazard curves with effect window shaded. Bottom row shows \textbackslash mathrm\{KCOR\}(t) deviating from 1.0 in the expected direction during the effect window. Uncertainty bands (95\% bootstrap intervals) are shown.}]{figures/fig_pos_control_injected.png}}
\caption{Positive control validation: KCOR correctly detects injected
effects. Left panels show harm scenario (r=1.2), right panels show
benefit scenario (r=0.8). Top row displays cohort hazard curves with
effect window shaded. Bottom row shows \(\mathrm{KCOR}(t)\) deviating
from 1.0 in the expected direction during the effect window. Uncertainty
bands (95\% bootstrap intervals) are
shown.}\label{fig:pos_control_injected}
\end{figure}

\subsubsection{3.3 Sensitivity analyses (robustness
checks)}\label{sensitivity-analyses-robustness-checks}

The primary analysis uses a prespecified quiet window applied uniformly
across cohorts; sensitivity analyses explicitly vary quiet-window bounds
and related prespecified choices to assess robustness.

KCOR results should be robust (up to numerical tolerance) to reasonable
variations in:

\begin{itemize}
\tightlist
\item
  Quiet-window selection (calendar ISO-week bounds)
\item
  Stabilization skip (early-bin handling)
\item
  Time-binning resolution
\item
  Age stratification and/or stratified analyses where appropriate
\item
  Baseline shape choice (default constant baseline over the fit window;
  alternatives can be assessed as sensitivity)
\end{itemize}

KCOR stability across the parameter grid is summarized in Figure
\ref{fig:sensitivity_overview}.

\begin{figure}
\centering
\pandocbounded{\includegraphics[keepaspectratio,alt={Sensitivity analysis summary showing \textbackslash mathrm\{KCOR\}(t) values across parameter grid. Heatmaps display \textbackslash mathrm\{KCOR\}(t) estimates for different combinations of baseline weeks (rows) and quiet-window start offsets (columns). Across all comparisons, \textbackslash mathrm\{KCOR\}(t) varies smoothly and modestly across a wide range of quiet-start offsets and baseline window lengths, with no qualitative changes in sign or magnitude, indicating robustness to reasonable parameter choices. All panels use a unified color scale centered at 1.0 to enable direct visual comparison across dose comparisons.}]{figures/fig_sensitivity_overview.png}}
\caption{Sensitivity analysis summary showing \(\mathrm{KCOR}(t)\)
values across parameter grid. Heatmaps display \(\mathrm{KCOR}(t)\)
estimates for different combinations of baseline weeks (rows) and
quiet-window start offsets (columns). Across all comparisons,
\(\mathrm{KCOR}(t)\) varies smoothly and modestly across a wide range of
quiet-start offsets and baseline window lengths, with no qualitative
changes in sign or magnitude, indicating robustness to reasonable
parameter choices. All panels use a unified color scale centered at 1.0
to enable direct visual comparison across dose
comparisons.}\label{fig:sensitivity_overview}
\end{figure}

Across all tested parameter ranges, \(\mathrm{KCOR}(t)\) values remained
within approximately ±5\% of unity, indicating stability under
reasonable variations in fitting choices.

\paragraph{3.3.1 Frailty misspecification
robustness}\label{frailty-misspecification-robustness}

To assess robustness to departures from the gamma frailty assumption, we
conducted simulations under alternative frailty distributions while
maintaining the same selection-induced depletion geometry. Simulations
were performed for:

\begin{itemize}
\tightlist
\item
  \textbf{Gamma} (baseline reference)
\item
  \textbf{Lognormal} frailty
\item
  \textbf{Two-point mixture} (discrete frailty)
\item
  \textbf{Bimodal} frailty distributions
\item
  \textbf{Correlated frailty} (within-subgroup correlation)
\end{itemize}

For each frailty specification, we report bias (deviation from true
cumulative hazard ratio), variance (trajectory stability), coverage
(proportion of simulations where uncertainty intervals contain the true
value), and diagnostic failure rate (proportion of simulations where
quiet-window diagnostics indicated non-identifiability).

Under frailty misspecification, KCOR can degrade gracefully by
attenuating toward unity or by not meeting diagnostic criteria, rather
than producing spurious large effects. When the alternative frailty
distribution produces similar depletion geometry to gamma frailty, KCOR
normalization remains approximately valid, with bias remaining small and
diagnostics indicating successful identification. When the alternative
frailty structure produces substantially different depletion geometry,
KCOR diagnostics (poor cumulative-hazard fit, residual autocorrelation,
parameter instability) correctly signal that the gamma-frailty
approximation is inadequate, and \(\mathrm{KCOR}(t)\) trajectories
either remain near-unity (reflecting attenuation) or are not computed
when diagnostic thresholds are not met.

\subsubsection{3.4 Simulation grid: operating characteristics and
failure-mode
diagnostics}\label{simulation-grid-operating-characteristics-and-failure-mode-diagnostics}

Simulation studies are structured according to the ADEMP framework: aims
(assessing robustness to selection-induced depletion), data-generating
mechanisms (frailty heterogeneity with controlled selection and external
hazard structure), estimands (cumulative hazard contrasts for KCOR;
method-specific contrasts for comparators), methods (KCOR and selected
competing approaches), and performance measures (bias, variance,
stability across time, and diagnostic failure detection).

We further evaluate KCOR using a compact simulation grid designed to (i)
confirm near-null behavior under selection-induced curvature, (ii)
confirm detection of injected effects, and (iii) characterize failure
modes and diagnostics under model misspecification and adverse data
regimes. Each scenario generates cohort-level weekly counts in KCOR\_CMR
format. KCOR is then fit using the same prespecified quiet-window
procedure as in the empirical analyses, and we report both
\(\mathrm{KCOR}(t)\) trajectories and diagnostic summaries, including
cumulative-hazard fit error and post-normalization linearity. The
scenarios isolate specific stresses, including non-gamma frailty,
contamination of the quiet window by an external shock, sparse events,
\textbf{joint frailty and treatment effects (S7)}, and
\textbf{tail-sampling / bimodal selection} (cohorts drawn from different
parts of the same underlying frailty distribution, e.g., vaccinated
sampled from mid-quantiles; unvaccinated from low+high tails, producing
non-gamma mixture geometry at the cohort level). Code to reproduce all
simulations and figures is included in the repository. \emph{Near-flat}
is defined operationally as median \(\mathrm{KCOR}(t)\) remaining within
±5\% of unity over the diagnostic window (weeks 20--100), excluding
early transients. These findings are consistent with the conceptual
discussion of RMST in §1.3.2, which emphasizes that RMST does not remove
selection-induced depletion. Figures \ref{fig:sim_grid_overview} and
\ref{fig:sim_grid_diagnostics} summarize the simulation grid.

\begin{figure}
\centering
\pandocbounded{\includegraphics[keepaspectratio,alt={Simulation grid overview: \textbackslash mathrm\{KCOR\}(t) trajectories across prespecified scenarios, including gamma-frailty null with strong selection, injected hazard increase and decrease, non-gamma frailty, quiet-window contamination, and sparse-event regimes. Under true null, \textbackslash mathrm\{KCOR\}(t) remains near-flat at 1; injected effects are detected in the expected direction; adverse regimes are accompanied by degraded diagnostics and reduced interpretability. Uncertainty bands (95\% bootstrap intervals) are shown where applicable.}]{figures/fig_sim_grid_overview.png}}
\caption{Simulation grid overview: \(\mathrm{KCOR}(t)\) trajectories
across prespecified scenarios, including gamma-frailty null with strong
selection, injected hazard increase and decrease, non-gamma frailty,
quiet-window contamination, and sparse-event regimes. Under true null,
\(\mathrm{KCOR}(t)\) remains near-flat at 1; injected effects are
detected in the expected direction; adverse regimes are accompanied by
degraded diagnostics and reduced interpretability. Uncertainty bands
(95\% bootstrap intervals) are shown where
applicable.}\label{fig:sim_grid_overview}
\end{figure}

\begin{figure}
\centering
\pandocbounded{\includegraphics[keepaspectratio,alt={Simulation diagnostics across scenarios: (i) cumulative-hazard fit RMSE over the quiet window, (ii) fitted frailty variance estimates, and (iii) a post-normalization linearity metric for depletion-neutralized baseline cumulative hazards. Diagnostics identify regimes in which frailty normalization is well identified versus weakly identified.}]{figures/fig_sim_grid_diagnostics.png}}
\caption{Simulation diagnostics across scenarios: (i) cumulative-hazard
fit RMSE over the quiet window, (ii) fitted frailty variance estimates,
and (iii) a post-normalization linearity metric for
depletion-neutralized baseline cumulative hazards. Diagnostics identify
regimes in which frailty normalization is well identified versus weakly
identified.}\label{fig:sim_grid_diagnostics}
\end{figure}

\textbf{Joint frailty and treatment-effect simulation (S7).}\\
Figure \ref{fig:s7_overview} summarizes results from the S7 simulation,
in which cohorts differ in frailty-driven depletion dynamics and a known
treatment effect is introduced in a separate time window. Frailty
parameters are estimated exclusively during a prespecified quiet window,
and KCOR normalization is then applied to the full follow-up period.

\begin{figure}
\centering
\pandocbounded{\includegraphics[keepaspectratio,alt={S7 simulation results: \textbackslash mathrm\{KCOR\}(t) trajectories demonstrating temporal separability. Left panel shows harm scenario (r=1.2) with effect window (weeks 10-25) and quiet window (weeks 80-140) non-overlapping. Middle panel shows benefit scenario (r=0.8). Right panel shows overlap variant where effect window intersects quiet window, demonstrating diagnostic degradation. \textbackslash mathrm\{KCOR\}(t) remains approximately flat during the quiet window and deviates only during the effect window when temporal separability holds. Uncertainty bands (95\% bootstrap intervals) are shown. The overlap variant is included to demonstrate failure-mode behavior and should not be interpreted as a valid application regime for KCOR.}]{figures/fig_s7_overview.png}}
\caption{S7 simulation results: \(\mathrm{KCOR}(t)\) trajectories
demonstrating temporal separability. Left panel shows harm scenario
(r=1.2) with effect window (weeks 10-25) and quiet window (weeks 80-140)
non-overlapping. Middle panel shows benefit scenario (r=0.8). Right
panel shows overlap variant where effect window intersects quiet window,
demonstrating diagnostic degradation. \(\mathrm{KCOR}(t)\) remains
approximately flat during the quiet window and deviates only during the
effect window when temporal separability holds. Uncertainty bands (95\%
bootstrap intervals) are shown. The overlap variant is included to
demonstrate failure-mode behavior and should not be interpreted as a
valid application regime for KCOR.}\label{fig:s7_overview}
\end{figure}

When the quiet window and treatment window are temporally separable,
KCOR exhibits the expected behavior under identifiable conditions:
\(\mathrm{KCOR}(t)\) remains approximately flat and near unity
throughout the quiet window, indicating successful identification and
removal of selection-induced depletion curvature, and deviates from
unity only during the treatment window, in the correct direction and
with magnitude consistent with the injected effect (harm or benefit).
This behavior holds across multiple effect shapes (step, ramp, smooth
pulse) and effect magnitudes.

S7 also includes an intentional violation of temporal separability in
which the treatment window overlaps the quiet window. In this overlap
variant, \(\mathrm{KCOR}(t)\) trajectories no longer stabilize during
the nominal quiet period, and fit diagnostics degrade (Figure
\ref{fig:s7_diagnostics}), including increased cumulative-hazard fit
error and reduced post-normalization linearity. In these cases,
\(\mathrm{KCOR}(t)\) does not produce a spurious treatment signal;
instead, diagnostics correctly indicate that the assumptions required
for interpretable normalization are violated.

\begin{figure}
\centering
\pandocbounded{\includegraphics[keepaspectratio,alt={S7 simulation diagnostics: Fitted frailty variance parameters (\textbackslash hat\{\textbackslash theta\}\_0, \textbackslash hat\{\textbackslash theta\}\_1), fit quality (RMSE), and convergence status across S7 scenarios. The overlap variant shows degraded fit quality, correctly signaling violation of temporal separability assumptions.}]{figures/fig_s7_diagnostics.png}}
\caption{S7 simulation diagnostics: Fitted frailty variance parameters
(\(\hat{\theta}_0\), \(\hat{\theta}_1\)), fit quality (RMSE), and
convergence status across S7 scenarios. The overlap variant shows
degraded fit quality, correctly signaling violation of temporal
separability assumptions.}\label{fig:s7_diagnostics}
\end{figure}

Together, these results demonstrate KCOR's operating characteristics
under joint selection and treatment dynamics: when selection-induced
depletion and treatment effects are sufficiently separable in time, KCOR
can disentangle the two mechanisms; when they are not, diagnostics are
intended to indicate loss of identifiability rather than silently
misattributing curvature.

\textbf{Comparison with shared frailty Cox models.}\\
Table \ref{tbl:joint_frailty_comparison} reports head-to-head
comparisons of standard Cox regression, shared frailty Cox models, and
KCOR on the S7 simulation (joint frailty and treatment effects) and the
gamma-frailty null scenario (selection-only, no treatment effect).
Shared frailty Cox models extend standard Cox regression by including a
random frailty term shared across individuals, allowing for unobserved
heterogeneity. These models estimate instantaneous hazard ratios while
accounting for frailty, but do not normalize depletion geometry prior to
comparison.

Under selection-only conditions (gamma-frailty null), standard Cox
regression produces non-null hazard ratios (HR ≈ 0.87) driven by
depletion dynamics rather than true signal. Shared frailty Cox models
partially mitigate this bias (HR ≈ 0.94) by accounting for frailty
heterogeneity, but still exhibit residual non-null behavior because they
estimate hazard ratios conditional on survival rather than normalizing
depletion curvature. KCOR remains near-null (drift \textless{} 0.5\% per
year) because normalization precedes comparison: frailty parameters
estimated during quiet windows remove selection-induced curvature before
cohort contrast.

Under joint frailty and treatment effects (S7 harm scenario, r=1.2),
standard Cox regression detects the effect (HR ≈ 1.18) but with bias due
to residual depletion effects. Shared frailty Cox models show improved
performance (HR ≈ 1.19) but still reflect depletion-induced time-varying
behavior. KCOR correctly identifies the effect magnitude (KCOR drift ≈
1.8\% per year, consistent with r=1.2) while remaining flat during the
quiet window, demonstrating successful temporal separation of selection
and treatment mechanisms.

These comparisons illustrate when shared frailty models help versus when
they still fail: shared frailty Cox models improve upon standard Cox
regression by accounting for unobserved heterogeneity, but they continue
to estimate instantaneous hazard ratios that reflect depletion dynamics
rather than normalizing them. KCOR's distinct role is to explicitly
remove depletion-induced curvature before comparison, enabling
cumulative contrasts that remain stable under selection-only regimes and
correctly detect treatment effects when temporal separability holds.

The tail-sampling scenario is included because it can confound
frailty-driven depletion with cohort construction in ways not captured
by a single gamma frailty distribution. The goal is not to force KCOR to
`succeed' under arbitrary misspecification, but to quantify operating
characteristics: when the gamma depletion model is misspecified, KCOR
should either (i) remain approximately unbiased in later windows (if the
misspecification is mild in cumulative-hazard geometry), or (ii) visibly
degrade via its diagnostics (poor \(H\)-space fit, post-normalization
nonlinearity, parameter instability), flagging that
depletion-neutralization is unreliable without model generalization.

\textbf{Comparison with alternative estimands.}\\
Table \ref{tbl:comparison_estimands} summarizes KCOR, Cox, and RMST
behavior under the same simulation settings. We compare KCOR with
restricted mean survival time (RMST) and time-varying Cox regression on
the same simulation outputs. To make the comparison explicit, we
computed RMST up to τ for the same simulation settings. As shown in
Table \ref{tbl:comparison_estimands}, RMST summarizes survival
differences that may reflect depletion rather than treatment effect in
selection-only regimes because it aggregates survival differences
produced by differential depletion, even when the true treatment effect
is null. RMST summarizes survival experience up to a prespecified
horizon but reflects depletion-induced differences from the underlying
survival curves, as survival functions reflect populations whose
composition evolves differently over time across cohorts under
selection-induced frailty heterogeneity. Time-varying Cox models improve
fit to non-proportional hazards but do not normalize selection geometry;
they capture time-varying hazard ratios without removing
depletion-induced curvature. KCOR remains stable under selection-only
regimes because normalization precedes comparison: frailty parameters
are estimated during quiet windows and used to invert observed
cumulative hazards into depletion-neutralized baseline cumulative
hazards prior to cohort contrast. This comparison emphasizes that KCOR
targets a specific estimand---cumulative hazard ratios after depletion
normalization---rather than claiming global superiority over alternative
methods.

\textbf{Simulation evaluation followed an ADEMP structure}: the
\emph{aim} was to assess robustness to selection-induced depletion;
\emph{data-generating mechanisms} varied frailty heterogeneity and
selection strength under both null and non-null conditions;
\emph{estimands} were method-specific (KCOR, RMST, and time-varying
Cox); \emph{methods} were applied to identical simulated datasets; and
\emph{performance} was assessed via deviation from the null, trajectory
stability, and interpretability under known selection structure. Under
selection-only scenarios, KCOR remained centered near the null with low
instability, whereas RMST and time-varying Cox exhibited systematic
non-null behavior that increased with depletion strength, consistent
with inherited selection effects rather than true signal (Table
\ref{tbl:comparison_estimands}).

In simulations where apparent effects arise solely from
selection-induced depletion, among the methods evaluated, KCOR remained
stable and near-null, while RMST and time-varying Cox produced non-null
contrasts that varied with frailty strength, consistent with inherited
depletion geometry rather than causal signal (Table
\ref{tbl:comparison_estimands}). Across selection-only scenarios, among
the estimands evaluated, KCOR's median deviation from the null remained
approximately zero across the full frailty grid, whereas RMST and
time-varying Cox produced systematic non-null contrasts that increased
with frailty strength (Table \ref{tbl:comparison_estimands}).

Bootstrap coverage results are summarized in Table
\ref{tbl:bootstrap_coverage}.

\subsubsection{3.5 Dynamic HVE diagnostic
tests}\label{dynamic-hve-diagnostic-tests}

Dynamic HVE refers to transient hazard suppression immediately after
enrollment driven by short-horizon selection around intervention timing
(e.g., deferral during illness). It produces a characteristic early-time
pattern: an abrupt early reduction in observed hazard that decays over
several weeks and is not explained by stable depletion curvature.

\textbf{Empirical signature in multi-dose settings (diagnostic, not
proof).} When multiple `treatment intensities' exist (e.g., dose-2 and
dose-3 cohorts defined at enrollment), dynamic HVE should affect
adjacent-dose cohorts similarly at early times because both enrollments
are subject to the same short-horizon deferral mechanisms. Therefore, if
early post-enrollment curvature is dominated by dynamic HVE, then
early-time deviations in \(\mathrm{KCOR}(t)\) versus the same comparator
should show similar transient shapes across adjacent-dose cohorts.
Conversely, if early-time behavior differs substantially across
adjacent-dose cohorts while post-normalization quiet-window linearity
holds, it is less consistent with a single shared dynamic deferral
artifact.

\textbf{Simulation check.} We include simulations where a transient
early hazard suppression is injected around enrollment (multiplying
hazard by factor \(q<1\) for weeks 0--S), separately from gamma frailty
depletion, and confirm that (i) the effect is attenuated/removed by
prespecified skip weeks, and (ii) remaining \(\mathrm{KCOR}(t)\)
trajectories in later windows behave as expected under negative and
positive controls.

\textbf{Skip-window sensitivity.} Figure
\ref{fig:skip_weeks_sensitivity} illustrates dynamic selection effects
by comparing \(\mathrm{KCOR}(t)\) computed on the same fixed-cohort
comparison using skip windows of 0, 4, and 8 weeks. Early-time
departures that attenuate with larger skip windows are consistent with
dynamic selection immediately following cohort entry; later-time
behavior is comparatively stable.

\begin{figure}
\centering
\pandocbounded{\includegraphics[keepaspectratio,alt={Skip-window sensitivity illustrates dynamic selection effects in early follow-up. \textbackslash mathrm\{KCOR\}(t) computed on the same fixed-cohort comparison using skip windows of 0, 4, and 8 weeks. Early-time departures that attenuate with larger skip windows are consistent with dynamic selection immediately following cohort entry; later-time behavior is comparatively stable.}]{figures/fig_skip_weeks_sensitivity.png}}
\caption{Skip-window sensitivity illustrates dynamic selection effects
in early follow-up. \(\mathrm{KCOR}(t)\) computed on the same
fixed-cohort comparison using skip windows of 0, 4, and 8 weeks.
Early-time departures that attenuate with larger skip windows are
consistent with dynamic selection immediately following cohort entry;
later-time behavior is comparatively
stable.}\label{fig:skip_weeks_sensitivity}
\end{figure}

\subsubsection{3.6 Illustrative non-COVID example
(synthetic)}\label{illustrative-non-covid-example-synthetic}

To emphasize that KCOR is not specific to COVID-19 vaccination, we
include a synthetic illustration motivated by elective intervention
timing. Consider two cohorts defined by the timing of an elective
medical procedure, where short-term deferral during acute illness
induces selection into the later-treated cohort. Although no treatment
effect is present by construction, the observed cumulative hazards
differ due to selection-induced depletion.

Applying KCOR to this setting removes curvature attributable to
depletion and yields a flat post-normalization trajectory, with
\(\mathrm{KCOR}(t)\) asymptoting to unity as expected under the null.
This example demonstrates that KCOR applies generally to retrospective
cohort comparisons affected by selection-induced hazard curvature,
independent of disease area or intervention type.

We include a national-scale application using Czech registry data to
illustrate the behavior of KCOR under real-world non-proportional
hazards. This application is intended to demonstrate the method's
operation and diagnostics rather than to support causal inference about
the intervention. An applied illustration using Czech national records
is provided in the Appendix to demonstrate end-to-end use of KCOR on
administrative data; the primary validation in the main text is based on
synthetic and empirical control experiments designed to probe failure
modes.

Post-normalization diagnostics are used to assess whether the depletion
geometry has been adequately corrected. In practice, we treat the
following criteria as indicative of acceptable normalization: (i)
residual drift in adjusted cumulative hazards below 5\% per year within
the evaluation window; (ii) approximate linearity during the quiet
period with coefficient of determination R² ≥ 0.98; and (iii)
root-mean-square deviation consistent with values observed under
simulated null regimes.

These thresholds were calibrated empirically through simulation studies
spanning a wide range of frailty heterogeneity and selection strength.
Diagnostics are intended to identify departures from the modeling
assumptions rather than to provide formal hypothesis tests, and should
be interpreted as descriptive measures of normalization adequacy.

\subsection{4. Discussion}\label{discussion}

\textbf{Scope statement (non-causal positioning).} KCOR is not a causal
identification strategy. It is a normalization-and-diagnostic framework
designed to remove a specific and dominant bias
geometry---selection-induced depletion under latent frailty---prior to
cohort comparison. The method does not claim to recover
treatment-specific causal effects under arbitrary baseline or
time-varying confounding. Instead, KCOR produces a depletion-neutralized
cumulative contrast whose interpretability is conditional on explicit
assumptions and on internal diagnostics indicating that
selection-induced curvature has been successfully removed. When those
diagnostics fail, KCOR is intended to signal non-identifiability rather
than silently produce a corrected estimate.

\textbf{What KCOR does not provide}

KCOR is designed to resolve a specific and otherwise unaddressed failure
mode in retrospective analyses---selection-induced depletion under
latent heterogeneity. Accordingly, KCOR does \textbf{not} by itself
provide:

• Policy optimization or cost-benefit analysis • Transportability of
effects across populations without additional assumptions •
Identification under unmeasured time-varying confounding unrelated to
depletion dynamics

These limitations are intrinsic to the data constraints KCOR is designed
to operate under and do not detract from its role as a
depletion-neutralized cohort comparison system.

\subsubsection{Limits of attribution and
non-identifiability}\label{limits-of-attribution-and-non-identifiability}

KCOR does not uniquely identify the biological, behavioral, or clinical
mechanisms responsible for observed hazard heterogeneity. In particular,
curvature in the cumulative hazard may arise from multiple sources,
including selection on latent frailty, behavior change, seasonality,
treatment effects, reporting artifacts, or their combination. Depletion
of susceptibles is therefore used as a parsimonious working model whose
adequacy is evaluated through diagnostics and negative controls, rather
than assumed as a causal truth. KCOR's estimand is whether a cumulative
outcome contrast persists after removal of curvature consistent with
selection-induced depletion, not attribution of that curvature to a
specific mechanism.

\subsubsection{4.1 What KCOR estimates}\label{what-kcor-estimates}

\emph{Table \ref{tbl:positioning} clarifies that KCOR differs from
non-proportional hazards methods not in flexibility, but in estimand and
direction of inference.} KCOR operates at a specific but critical layer
of the retrospective inference stack: it both neutralizes
selection-induced depletion dynamics and defines how the resulting
depletion-neutralized baseline cumulative hazards must be compared. The
method's strength is not the frailty inversion in isolation, but the
fact that inversion, diagnostics, and cumulative comparison are
mathematically and operationally coupled. Once cohorts are mapped into
depletion-neutralized baseline cumulative hazard space,
\(\mathrm{KCOR}(t)\) directly answers whether one cohort experienced
higher or lower cumulative event risk than another over follow-up,
conditional on the stated assumptions. For intuition, a value such as
\(\mathrm{KCOR}(t)=1.2\) indicates that, after depletion normalization,
cohort A has accumulated approximately 20\% greater cumulative hazard
than cohort B by time \(t\). This corresponds to a uniformly higher
cumulative risk trajectory over follow-up, rather than a time-localized
hazard spike, and does not imply a constant instantaneous hazard ratio.
Stabilization of \(\mathrm{KCOR}(t)\) following frailty normalization is
not an assumption but a falsification test; failure to flatten indicates
residual curvature or loss of identifiability, not a substantive
cumulative effect. Interpreting normalized hazards without this
comparison step discards the central inferential content of the method.
As emphasized earlier (§1.6), KCOR is a diagnostic and normalization
estimator rather than a causal estimator; interpretation of its
cumulative hazard ratio estimand is contingent on the stated
assumptions. Whereas Cox regression infers treatment effects through
fitted model coefficients, KCOR uses modeling only to normalize
selection effects, allowing contrasts to be read directly from the
adjusted data via cumulative hazard ratios. For infectious-disease
vaccines, any plausible mortality effect is inherently time-local,
occurring only during periods of pathogen circulation; the proportional
hazards assumption therefore fails structurally, as a constant
multiplicative effect over the entire follow-up is biologically
implausible.

\emph{Many commonly used survival estimands---such as hazard ratios,
cumulative hazard differences, or restricted mean survival time---are
not intrinsically invalid. Their failure in retrospective cohort studies
arises when they are applied to unadjusted data exhibiting
selection-induced depletion. KCOR does not replace these estimands;
instead, it provides a normalization step that restores comparability.
After depletion normalization, such estimands may be meaningfully
computed, with the choice driven by interpretability rather than by
identifiability constraints imposed by selection bias.}

As emphasized above, the frailty term is not causal and does not
represent a treatment mechanism; it functions solely as a geometric
normalization for selection-induced depletion.

KCOR is a \textbf{cumulative} comparison of depletion-neutralized
cumulative hazards; it does not estimate instantaneous hazard ratios. It
is designed for settings where selection induces non-proportional
hazards such that conventional proportional-hazards estimators can be
difficult to interpret. A controlled synthetic null experiment (Section
2.11.1) shows that Cox regression can return statistically significant
non-null hazard ratios solely from frailty-induced depletion---even when
the true treatment effect is identically zero---reflecting an estimand
mismatch where Cox targets a different quantity under depletion than
KCOR's cumulative estimand. Cox is behaving correctly for its estimand,
but that estimand may not align with the scientific question when
selection-induced depletion is present. KCOR remains centered near unity
with negligible post-normalization slope under the same conditions. We
did not pursue model selection among Cox-based specifications (with or
without frailty) because these models target instantaneous hazard ratios
under proportional-hazards assumptions, whereas KCOR targets cumulative,
depletion-neutralized outcomes; BIC comparisons across models with
different estimands are therefore not informative for the question
addressed here.

Under the working assumptions that:

\begin{enumerate}
\def\labelenumi{\arabic{enumi}.}
\tightlist
\item
  selection-induced depletion dynamics can be estimated during quiet
  periods using a gamma-frailty mixture model, and
\item
  the fitted selection parameters can be used to invert observed
  cumulative hazards into depletion-neutralized baseline cumulative
  hazards,
\end{enumerate}

then the remaining differences between cohorts are interpretable,
\textbf{conditional on the stated selection model and quiet-window
validity}, as differences in baseline hazard level (on a cumulative
scale), summarized by KCOR\((t)\).

A useful way to view KCOR is as an intermediate layer between purely
descriptive hazard summaries and fully identified causal estimators.
KCOR is descriptive in that it summarizes cohort differences in a
cumulative-hazard scale under explicit normalization of depletion
geometry; it is inferential in that it provides falsifiable diagnostics
and control-test behavior that constrain when the normalized contrast is
interpretable. This positioning is intentional: under minimal-data
constraints, explicitly normalizing a dominant bias geometry and
transparently reporting when identifiability is not supported can be
more reliable than insisting on point-identification of a causal effect.
Accordingly, KCOR should be interpreted as identifying a
depletion-adjusted descriptive contrast rather than a causal effect,
even when that contrast is temporally aligned with a plausible
biological mechanism.

The observation that frailty correction is negligible for vaccinated
cohorts but substantial for the unvaccinated cohort is not incidental.
It reflects the asymmetric action of healthy-vaccinee selection, which
concentrates lower-frailty individuals into vaccinated cohorts at
enrollment while leaving the unvaccinated cohort heterogeneous. KCOR
explicitly detects and removes this asymmetry by mapping cohorts into a
depletion-neutralized comparison space rather than assuming proportional
hazards.

Because the normalization targets selection-induced depletion curvature,
KCOR results alone do not justify claims about net lives saved or lost
by a particular intervention. Such claims require (i) clearly specified
causal estimands, (ii) validated control outcomes, (iii) sensitivity
analyses for remaining time-varying selection mechanisms and external
shocks, and (iv) preferably replication across settings and outcomes.
Having established the behavior of KCOR and the failure modes of
standard estimators under controlled conditions, we apply KCOR to
complete national registry data from the Czech Republic in a companion
analysis. Accordingly, this manuscript focuses on method definition,
diagnostics, and operating characteristics; applied causal conclusions
are deferred to separate intervention-specific analyses. Interpretation
should be read in light of the non-identifiability considerations
described above.

Although cumulative hazards and survival functions are in one-to-one
correspondence, KCOR operates in cumulative-hazard space because
curvature induced by frailty depletion is additive and more readily
diagnosed there. While survival-based summaries such as restricted mean
survival time may be derived from depletion-neutralized baseline
cumulative hazards, KCOR's primary estimand remains cumulative by
construction.

\subsubsection{4.2 Relationship to negative control
methods}\label{relationship-to-negative-control-methods}

Negative control outcomes/tests are widely used to \emph{detect}
confounding. KCOR's objective is different: it is an estimator intended
to \emph{normalize away a specific confounding
structure}---selection-induced depletion dynamics---prior to comparison.
Negative and positive controls are nevertheless central to validating
the estimator's behavior.

This asymmetry helps explain why standard observational analyses often
report large apparent mortality benefits during periods lacking a
plausible causal mechanism: vaccinated cohorts are already
selection-filtered, while unvaccinated hazards are suppressed by ongoing
frailty depletion. Unadjusted comparisons therefore systematically
understate unvaccinated baseline risk and exaggerate apparent benefit.

\subsubsection{4.3 Practical Guidelines for
Implementation}\label{practical-guidelines-for-implementation}

This subsection summarizes recommended operational practices for
applying KCOR in retrospective cohort studies and for assessing when
resulting contrasts are interpretable.

Recommended reporting includes:

\begin{itemize}
\tightlist
\item
  Enrollment definition and justification
\item
  Risk set definitions and event-time binning
\item
  Quiet-window definition and justification
\item
  Baseline-shape choice (default constant baseline over the fit window)
  and fit diagnostics
\item
  Skip/stabilization rule and robustness to nearby values
\item
  Predefined negative/positive controls used for validation
\item
  Sensitivity analysis plan and results
\end{itemize}

KCOR should therefore be applied and reported as a complete
pipeline---from cohort freezing, through depletion normalization, to
cumulative comparison and diagnostics---rather than as a standalone
adjustment step.

\begin{quote}
This work is intentionally non-causal. KCOR is not proposed as an
estimator of treatment effects, nor does it attempt to recover
counterfactual outcomes under hypothetical interventions. Instead, it is
a diagnostic and descriptive framework designed to address a specific
geometric distortion that arises prior to model fitting in retrospective
cohort data: selection-induced depletion under latent frailty
heterogeneity. By estimating and inverting cohort-specific depletion
geometry during epidemiologically quiet periods, KCOR maps observed
cumulative hazards into a common comparison scale on which standard
post-adjustment summaries may be meaningfully interpreted. The stability
of KCOR-normalized trajectories under selection-only regimes should
therefore not be construed as evidence of causal neutrality, but rather
as evidence that the normalization removes bias arising from
heterogeneous risk composition before any causal or associational
estimand is imposed. In this sense, KCOR is complementary to, rather
than a substitute for, causal inference frameworks, and may be viewed as
a preprocessing step that clarifies the limits of what can be inferred
from observational data when proportional hazards assumptions are
violated. Future work may explore how depletion-neutralized hazard
representations could be incorporated into causal pipelines---for
example, as inputs to target trial emulation or sensitivity
analyses---while preserving the separation between normalization and
causal identification emphasized here.
\end{quote}

\subsection{5. Limitations}\label{limitations}

\begin{itemize}
\tightlist
\item
  \textbf{Model dependence}: Normalization relies on the adequacy of the
  gamma-frailty model and the baseline-shape assumption during the quiet
  window.
\item
  \textbf{Relation to existing non-PH methods}: KCOR is complementary to
  time-varying Cox, flexible parametric, additive hazards, and MSM
  approaches; these methods address different estimands and
  identification strategies, whereas KCOR targets depletion-geometry
  normalization under minimal-data constraints (see §1.3.1).
\item
  \textbf{\(\theta\) estimation is data-driven}: KCOR does not impose
  \(\theta = 0\) for any cohort. The frequent observation that fitted
  frailty variance estimates collapse toward zero for vaccinated cohorts
  is a data-driven result of the frailty fit and should not be
  interpreted as an assumption of homogeneity.
\item
  \textbf{Sparse events}: When event counts are small, hazard estimation
  and parameter fitting can be unstable.
\item
  \textbf{Contamination of quiet periods}: External shocks (e.g.,
  epidemic waves) overlapping the quiet window can bias
  selection-parameter estimation.
\item
  \textbf{Causal interpretation}: KCOR supports interpretable cohort
  comparison under stated assumptions, but it is not a substitute for
  randomization; causal claims require explicit causal assumptions and
  careful validation.
\item
  \textbf{Applicability to other outcomes}: Although this paper focuses
  on all-cause mortality, KCOR is applicable to other irreversible
  outcomes provided that event timing and risk sets are well defined.
  Application to cause-specific mortality requires careful consideration
  of competing risks and interpretation of cumulative hazards within
  cause-restricted populations. Extension to non-fatal outcomes such as
  hospitalization is conceptually straightforward but may require
  additional attention to outcome definitions, censoring mechanisms, and
  recurrent events. These considerations affect interpretation rather
  than the core KCOR framework.
\item
  \textbf{Non-gamma frailty}: The KCOR framework assumes that selection
  acts approximately multiplicatively through a time-invariant frailty
  distribution, for which the gamma family provides a convenient and
  empirically testable approximation. In settings where depletion
  dynamics are driven by more complex mechanisms---such as time-varying
  frailty variance, interacting risk factors, or shared frailty
  correlations within subgroups---the curvature structure exploited by
  KCOR may be misspecified. In such cases, KCOR diagnostics (e.g., poor
  curvature fit or unstable fitted frailty variance estimates) serve as
  indicators of model inadequacy rather than targets for parameter
  tuning. Extending the framework to accommodate dynamic or correlated
  frailty structures would require explicit model generalization rather
  than modification of KCOR normalization steps and is left to future
  work. Empirically, KCOR's validity depends on curvature removal rather
  than the specific parametric form; alternative frailty distributions
  that generate similar depletion geometry would yield equivalent
  normalization.
\end{itemize}

\subsubsection{5.1 Failure modes and diagnostics
(recommended)}\label{failure-modes-and-diagnostics-recommended}

KCOR is designed to normalize selection-induced depletion curvature
under its stated model and windowing assumptions. Reviewers and readers
should expect the method to degrade when those assumptions are violated.
Common failure modes include:

\begin{itemize}
\tightlist
\item
  \textbf{Mis-specified quiet window}: If the quiet window overlaps
  major external shocks (epidemic waves, policy changes, reporting
  artifacts), the fitted parameters may absorb non-selection dynamics,
  biasing normalization.
\item
  \textbf{External time-varying hazards masquerading as frailty
  depletion}: Strong secular trends, seasonality, or outcome-definition
  changes can introduce curvature that is not well captured by
  gamma-frailty depletion alone. For example, COVID-19 waves
  disproportionately increase mortality among frail individuals; if one
  cohort has higher baseline frailty, such a wave can preferentially
  deplete that cohort, producing the appearance of a benefit in the
  lower-frailty cohort that is actually due to differential
  frailty-specific mortality from the external hazard rather than from
  the intervention under study.
\item
  \textbf{Extremely sparse cohorts}: When events are rare, observed
  cumulative hazards become noisy and \((\hat{k}_d,\hat{\theta}_d)\) can
  be weakly identified, often manifesting as unstable fitted frailty
  variance estimates or wide uncertainty.
\item
  \textbf{Non-frailty-driven curvature}: Administrative censoring,
  cohort-definition drift, changes in risk-set construction, or
  differential loss can induce curvature unrelated to latent frailty.
\end{itemize}

Practical diagnostics to increase trustworthiness include:

\begin{itemize}
\tightlist
\item
  \textbf{Quiet-window overlays} on hazard/cumulative-hazard plots to
  confirm the fit window is epidemiologically stable.
\item
  \textbf{Fit residuals in \(H\)-space} (RMSE, residual plots) and
  stability of fitted parameters under small perturbations of the
  quiet-window bounds.
\item
  \textbf{Sensitivity analyses} over plausible quiet windows and
  skip-weeks values.
\item
  \textbf{Prespecified negative controls}: \(\mathrm{KCOR}(t)\) curves
  should remain near-flat at 1 under control constructions designed to
  induce composition differences without true effects.
\end{itemize}

In practice, prespecified negative controls---such as the age-shift
controls presented in §3.1.2---provide a direct empirical check that
KCOR does not generate artifactual cumulative effects under strong
selection-induced curvature.

\subsubsection{5.2 Conservativeness and edge-case detection
limits}\label{conservativeness-and-edge-case-detection-limits}

Because KCOR compares fixed enrollment cohorts, subsequent uptake of the
intervention among initially unexposed individuals (or additional dosing
among exposed cohorts) introduces treatment crossover over time. Such
crossover attenuates between-cohort contrasts and biases KCOR(t) toward
unity, making the estimator conservative with respect to detecting
sustained net benefit or harm. Analyses should therefore restrict
follow-up to periods before substantial crossover or stratify by dosing
state when the data permit.

Because KCOR defines explicit diagnostic failure modes---instability,
dose reversals, age incoherence, or absence of asymptotic
convergence---the absence of such failures in the Czech 2021\_24 Dose 0
versus Dose 2 cohorts provides stronger validation than goodness-of-fit
alone.

\textbf{Conservativeness under overlap.}\\
When treatment effects overlap temporally with the quiet window used for
frailty estimation, KCOR(t) does not attribute the resulting curvature
to treatment nor amplify it into a spurious cumulative effect. Instead,
overlap manifests as degraded quiet-window fit, reduced
post-normalization linearity, and instability of estimated frailty
parameters, all of which are explicitly surfaced by KCOR's diagnostics.
In these regimes, KCOR(t) trajectories tend to attenuate toward unity
rather than diverge, reflecting loss of identifiability rather than
false detection. This behavior is illustrated in the S7 overlap variant,
where treatment and selection are deliberately confounded in time:
KCOR(t) does not recover a clean effect signal, and diagnostic criteria
correctly indicate that the assumptions required for interpretable
normalization are violated. As a result, KCOR is conservative under
temporal overlap---preferring diagnostic failure and attenuation over
over-interpretation---rather than producing misleading treatment effects
when separability is not supported by the data. This design choice
reflects an intentional bias toward false negatives rather than false
positives in ambiguous regimes. See §2.1.1 and Simulation S7 (Appendix
B.6) for the corresponding identifiability assumptions and stress tests.

KCOR analyses commonly exclude an initial post-enrollment window to
exclude dynamic Healthy Vaccinee Effect artifacts. If an intervention
induces an acute mortality effect concentrated entirely within this
skipped window, that transient signal will not be captured by the
primary analysis. This limitation is addressed by reporting sensitivity
analyses with reduced or zero skip-weeks and/or by separately evaluating
a prespecified acute-risk window.

In degenerate scenarios where an intervention induces a purely
proportional level-shift in hazard that remains constant over time and
does not alter depletion-driven curvature, KCOR's curvature-based
contrast may have limited ability to distinguish such effects from
residual baseline level differences under minimal-data constraints. Such
cases are pathological in the sense that they produce no detectable
depletion signature; in practice, KCOR diagnostics and control tests
help identify when curvature-based inference is not informative.

Simulation results in §3.4 illustrate that when key assumptions are
violated---such as non-gamma frailty geometry, contamination of the
quiet window by external shocks, or extreme event sparsity---frailty
normalization may become weakly identified. In such regimes, KCOR's
diagnostics, including poor cumulative-hazard fit and reduced
post-normalization linearity, explicitly signal that curvature-based
inference is unreliable without model generalization or revised window
selection.

Increasing model complexity within the Cox regression framework---via
random effects, cohort-specific frailty, or information-criterion--based
selection---does not resolve this limitation, because these models
continue to target instantaneous hazard ratios conditional on survival
rather than cumulative counterfactual outcomes. Model-selection criteria
applied within the Cox regression family favor specifications that
improve likelihood fit of instantaneous hazards, but such criteria do
not validate cumulative counterfactual interpretation under
selection-induced non-proportional hazards.

\subsubsection{5.3 Data requirements and external
validation}\label{data-requirements-and-external-validation}

In finite samples, KCOR precision is driven primarily by the number of
events observed over follow-up. In simulation (selection-only null),
cohorts of approximately 5,000 per arm yielded stable KCOR estimates
with narrow uncertainty, whereas smaller cohorts exhibited appreciable
Monte Carlo variability and occasional spurious deviations. We therefore
recommend reporting event counts and conducting a simple cohort-size
sensitivity check when applying KCOR to sparse outcomes.

\textbf{External validation across interventions.} A natural next step
is to apply KCOR to other vaccines and interventions where large-scale
individual-level event timing data are available. Many RCTs are
underpowered for all-cause mortality and typically do not provide
record-level timing needed for KCOR-style hazard-space normalization,
while large observational studies often publish only aggregated effect
estimates. Where sufficiently detailed time-to-event data exist
(registries, integrated health systems, or open individual-level
datasets), cross-intervention comparisons can help characterize how
often selection-induced depletion dominates observed hazard curvature
and how frequently post-normalization trajectories remain stable under
negative controls.

\subsection{6. Conclusion}\label{conclusion}

KCOR provides a principled approach to retrospective cohort comparison
under selection-induced hazard curvature by estimating and inverting a
gamma-frailty mixture model to remove cohort-specific depletion dynamics
prior to comparison. Validation via negative and positive controls
supports that KCOR remains near-null under selection without effect and
detects injected effects when present. Applied analyses on specific
datasets are best reported separately from this methods manuscript. KCOR
relies on five explicit assumptions, of which only one requires
substantive dataset-specific validation, and enforces these assumptions
diagnostically rather than presuming them, allowing violations to be
detected rather than absorbed into model-dependent estimates. Because
standard methods for retrospective vaccine evaluation can exhibit
systematic deviation under non-proportional hazards and
selection-induced depletion, KCOR provides a practical alternative that
operates on minimal individual-level information---dates of birth,
intervention, and death---while remaining applicable to national
registry data where richer covariates are unavailable or unreliable.

Reproducibility: code, simulations, and manuscript build instructions
are available in the project repository
(\url{https://github.com/skirsch/KCOR}).

\subsection{Declarations}\label{declarations}

\subsubsection{Ethics approval and consent to
participate}\label{ethics-approval-and-consent-to-participate}

This study used only simulated data and publicly available, aggregated
registry summaries that contain no individual-level or identifiable
information; as such, it did not constitute human subjects research and
was exempt from institutional review board oversight. This is a
methods-only manuscript. The primary validation results use synthetic
data. Empirical negative-control figures (Figures
\ref{fig:neg_control_10yr} and \ref{fig:neg_control_20yr}) use
aggregated cohort summaries derived from Czech Republic administrative
data; no record-level data are shared in this
manuscript.\textsuperscript{12}

\subsubsection{Consent for publication}\label{consent-for-publication}

Not applicable.

\subsubsection{Data availability}\label{data-availability}

No individual-level data were accessed or analyzed in this study.

\begin{itemize}
\tightlist
\item
  Synthetic validation data (negative and positive control datasets) and
  generation scripts are available in the project repository under
  \texttt{test/negative\_control/} and \texttt{test/positive\_control/}.
\item
  Sensitivity analysis outputs are available under
  \texttt{test/sensitivity/out/}.
\item
  The reference implementation includes example datasets in KCOR\_CMR
  format for reproducibility.
\item
  A formal specification of the KCOR data formats is provided in
  \texttt{documentation/specs/KCOR\_file\_format.md}, including schema
  definitions and disclosure-control semantics.
\end{itemize}

\subsubsection{Code availability}\label{code-availability}

\begin{itemize}
\tightlist
\item
  The KCOR reference implementation and complete validation suite are
  available in the project repository.
\item
  Repository URL: \url{https://github.com/skirsch/KCOR}
\item
  Zenodo DOI:
  \href{https://doi.org/10.5281/zenodo.18050329}{10.5281/zenodo.18050329}
\item
  RMST computation and comparison table generation are implemented via
  functions \texttt{compute\_rmst\_from\_cohort()} and
  \texttt{generate\_comparison\_table()} in
  \texttt{test/sim\_grid/code/generate\_sim\_grid.py}. Both RMST and
  time-varying Cox comparators run on the same simulation outputs as
  KCOR evaluation; no additional random seeds or data generation are
  required.
\end{itemize}

\subsubsection{Use of artificial intelligence
tools}\label{use-of-artificial-intelligence-tools}

The KCOR method and estimand were developed by the author without the
use of artificial intelligence (AI) tools. Generative AI tools,
including OpenAI's ChatGPT and Cursor Composer 1, were used during
manuscript preparation to assist with drafting and editing text,
mathematical typesetting, refactoring code, and implementing simulation
studies described in this manuscript.

Simulation designs were either specified by the author or proposed
during iterative discussion and subsequently reviewed and approved by
the author prior to implementation. AI assistance was used to draft code
for approved simulations, which the author reviewed, tested, and
validated. Additional large language models (including Gemini, DeepSeek,
and Claude) were used to provide feedback on manuscript wording and
methodological exposition in a role analogous to informal peer review.

All scientific decisions, methodological choices, analyses,
interpretations, and judgments regarding which suggestions to accept or
reject were made solely by the author, who reviewed and understands all
content and takes full responsibility for the manuscript.

\subsubsection{Competing interests}\label{competing-interests}

The author is a board member of the Vaccine Safety Research Foundation.

\subsubsection{Funding}\label{funding}

This research received no external funding.

\subsubsection{Authors' contributions}\label{authors-contributions}

Steven T. Kirsch conceived the method, wrote the code, performed the
analysis, and wrote the manuscript.

\subsubsection{Acknowledgements}\label{acknowledgements}

The author thanks James Lyons-Weiler. Dr.~Clare Craig, and Paul Fischer
for helpful discussions and methodological feedback during the
development of this work. All errors remain the author's responsibility.

\newpage

\subsection{References}\label{references}

\protect\phantomsection\label{refs}
\begin{CSLReferences}{0}{1}
\newpage

\subsection{Tables}\label{tables}

\begin{longtable}[]{@{}
  >{\raggedright\arraybackslash}p{(\linewidth - 8\tabcolsep) * \real{0.2000}}
  >{\raggedright\arraybackslash}p{(\linewidth - 8\tabcolsep) * \real{0.2000}}
  >{\raggedright\arraybackslash}p{(\linewidth - 8\tabcolsep) * \real{0.2000}}
  >{\raggedright\arraybackslash}p{(\linewidth - 8\tabcolsep) * \real{0.2000}}
  >{\raggedright\arraybackslash}p{(\linewidth - 8\tabcolsep) * \real{0.2000}}@{}}
\caption{Summary of two large matched observational studies showing
residual confounding / HVE despite meticulous
matching.}\label{tbl:HVE_motivation}\tabularnewline
\toprule\noalign{}
\begin{minipage}[b]{\linewidth}\raggedright
Study
\end{minipage} & \begin{minipage}[b]{\linewidth}\raggedright
Design
\end{minipage} & \begin{minipage}[b]{\linewidth}\raggedright
Matching/adjustment
\end{minipage} & \begin{minipage}[b]{\linewidth}\raggedright
Key control finding
\end{minipage} & \begin{minipage}[b]{\linewidth}\raggedright
Implication for methods
\end{minipage} \\
\midrule\noalign{}
\endfirsthead
\toprule\noalign{}
\begin{minipage}[b]{\linewidth}\raggedright
Study
\end{minipage} & \begin{minipage}[b]{\linewidth}\raggedright
Design
\end{minipage} & \begin{minipage}[b]{\linewidth}\raggedright
Matching/adjustment
\end{minipage} & \begin{minipage}[b]{\linewidth}\raggedright
Key control finding
\end{minipage} & \begin{minipage}[b]{\linewidth}\raggedright
Implication for methods
\end{minipage} \\
\midrule\noalign{}
\endhead
\bottomrule\noalign{}
\endlastfoot
Obel et al.~(Denmark)\textsuperscript{2} & Nationwide registry cohorts
(60--90y) & 1:1 match on age/sex + covariate adjustment; negative
control outcomes & Vaccinated had higher rates of multiple negative
control outcomes, but substantially lower mortality after unrelated
diagnoses & Strong evidence of confounding in observational VE
estimates; ``negative control methods indicate\ldots{} substantial
confounding'' \\
Chemaitelly et al.~(Qatar)\textsuperscript{3} & Matched national cohorts
(primary series and booster) & Exact 1:1 matching on demographics +
coexisting conditions + prior infection; Cox models & Strong early
reduction in non-COVID mortality (HVE), with time-varying reversal later
& Even meticulous matching leaves time-varying residual differences
consistent with selection/frailty depletion \\
\end{longtable}

\begin{longtable}[]{@{}
  >{\raggedright\arraybackslash}p{(\linewidth - 6\tabcolsep) * \real{0.4023}}
  >{\raggedright\arraybackslash}p{(\linewidth - 6\tabcolsep) * \real{0.1379}}
  >{\raggedright\arraybackslash}p{(\linewidth - 6\tabcolsep) * \real{0.1954}}
  >{\raggedright\arraybackslash}p{(\linewidth - 6\tabcolsep) * \real{0.2644}}@{}}
\caption{Comparison of Cox proportional hazards, Cox with frailty, and
KCOR across key methodological
dimensions.}\label{tbl:cox_vs_kcor}\tabularnewline
\toprule\noalign{}
\begin{minipage}[b]{\linewidth}\raggedright
Feature
\end{minipage} & \begin{minipage}[b]{\linewidth}\raggedright
Cox PH
\end{minipage} & \begin{minipage}[b]{\linewidth}\raggedright
Cox + frailty
\end{minipage} & \begin{minipage}[b]{\linewidth}\raggedright
KCOR
\end{minipage} \\
\midrule\noalign{}
\endfirsthead
\toprule\noalign{}
\begin{minipage}[b]{\linewidth}\raggedright
Feature
\end{minipage} & \begin{minipage}[b]{\linewidth}\raggedright
Cox PH
\end{minipage} & \begin{minipage}[b]{\linewidth}\raggedright
Cox + frailty
\end{minipage} & \begin{minipage}[b]{\linewidth}\raggedright
KCOR
\end{minipage} \\
\midrule\noalign{}
\endhead
\bottomrule\noalign{}
\endlastfoot
Primary estimand & Hazard ratio & Hazard ratio & Cumulative hazard
ratio \\
Conditions on survival & Yes & Yes & No \\
Assumes PH & Yes & Yes (conditional) & No \\
Frailty role & None & Nuisance & Object of inference \\
Uses partial likelihood & Yes & Yes & No \\
Handles selection-induced curvature & No & Partial & Yes (targeted) \\
Output interpretable under non-PH & No & No & Yes (cumulative) \\
\end{longtable}

Note: KCOR is reported here as a cumulative hazard ratio for
comparability; alternative post-normalization estimands are admissible
within the framework.

\begin{longtable}[]{@{}
  >{\raggedright\arraybackslash}p{(\linewidth - 10\tabcolsep) * \real{0.2065}}
  >{\raggedright\arraybackslash}p{(\linewidth - 10\tabcolsep) * \real{0.1685}}
  >{\raggedright\arraybackslash}p{(\linewidth - 10\tabcolsep) * \real{0.1196}}
  >{\raggedright\arraybackslash}p{(\linewidth - 10\tabcolsep) * \real{0.1957}}
  >{\raggedright\arraybackslash}p{(\linewidth - 10\tabcolsep) * \real{0.1304}}
  >{\raggedright\arraybackslash}p{(\linewidth - 10\tabcolsep) * \real{0.1793}}@{}}
\caption{Positioning KCOR relative to non-proportional hazards
methods.}\label{tbl:positioning}\tabularnewline
\toprule\noalign{}
\begin{minipage}[b]{\linewidth}\raggedright
Method class
\end{minipage} & \begin{minipage}[b]{\linewidth}\raggedright
Primary target
\end{minipage} & \begin{minipage}[b]{\linewidth}\raggedright
What is modeled
\end{minipage} & \begin{minipage}[b]{\linewidth}\raggedright
Handles selection-induced depletion?
\end{minipage} & \begin{minipage}[b]{\linewidth}\raggedright
Typical output
\end{minipage} & \begin{minipage}[b]{\linewidth}\raggedright
Failure under latent frailty
\end{minipage} \\
\midrule\noalign{}
\endfirsthead
\toprule\noalign{}
\begin{minipage}[b]{\linewidth}\raggedright
Method class
\end{minipage} & \begin{minipage}[b]{\linewidth}\raggedright
Primary target
\end{minipage} & \begin{minipage}[b]{\linewidth}\raggedright
What is modeled
\end{minipage} & \begin{minipage}[b]{\linewidth}\raggedright
Handles selection-induced depletion?
\end{minipage} & \begin{minipage}[b]{\linewidth}\raggedright
Typical output
\end{minipage} & \begin{minipage}[b]{\linewidth}\raggedright
Failure under latent frailty
\end{minipage} \\
\midrule\noalign{}
\endhead
\bottomrule\noalign{}
\endlastfoot
Cox PH & Instantaneous hazard & Linear predictor & No & HR & Non-PH from
depletion → biased HR \\
Time-varying Cox & Instantaneous hazard & Time-varying β(t) & No & HR(t)
& Fits depletion as signal \\
Flexible parametric survival (splines) & Survival / hazard shape &
Baseline hazard & No & Smooth hazard / survival & Absorbs depletion
curvature \\
Additive hazards (Aalen) & Hazard differences & Additive hazard & No &
Δh(t) & Still conditional on survival \\
RMST & Mean survival & Survival curve & No & RMST & Inherits depletion
bias \\
Frailty regression & Heterogeneity- adjusted HR & Random effects &
Partial & HR & Frailty treated as nuisance \\
\textbf{KCOR (this work)} & \textbf{Cumulative outcome contrast} &
\textbf{Depletion geometry} & \textbf{Yes (targeted)} &
\textbf{\(\mathrm{KCOR}(t)\)} & Diagnostics flag failure \\
\end{longtable}

\begin{longtable}[]{@{}
  >{\raggedright\arraybackslash}p{(\linewidth - 2\tabcolsep) * \real{0.4000}}
  >{\raggedright\arraybackslash}p{(\linewidth - 2\tabcolsep) * \real{0.6000}}@{}}
\caption{Notation used throughout the Methods
section.}\label{tbl:notation}\tabularnewline
\toprule\noalign{}
\begin{minipage}[b]{\linewidth}\raggedright
Symbol
\end{minipage} & \begin{minipage}[b]{\linewidth}\raggedright
Definition
\end{minipage} \\
\midrule\noalign{}
\endfirsthead
\toprule\noalign{}
\begin{minipage}[b]{\linewidth}\raggedright
Symbol
\end{minipage} & \begin{minipage}[b]{\linewidth}\raggedright
Definition
\end{minipage} \\
\midrule\noalign{}
\endhead
\bottomrule\noalign{}
\endlastfoot
\(d\) & Cohort index \\
\(A,B\) & Indices of the two cohorts compared in a KCOR contrast \\
\(t\) & Event time since enrollment (discrete bins, e.g., weeks) \\
\(h_{\mathrm{obs},d}(t)\) & Discrete-time cohort hazard (conditional on
\(N_d(t)\)) \\
\(H_{\mathrm{obs},d}(t)\) & Observed cumulative hazard (after
skip/stabilization) \\
\(h_{0,d}(t)\) & Baseline hazard for cohort \(d\) under the
depletion-neutralized model \\
\(H_{0,d}(t)\) & Baseline cumulative hazard for cohort \(d\) under the
depletion-neutralized model \\
\(\tilde{H}_{0,d}(t)\) & Depletion-neutralized baseline cumulative
hazard \\
\(\theta_d\) & Frailty variance (selection strength) for cohort \(d\);
governs curvature in the observed cumulative hazard \\
\(\hat{\theta}_d\) & Estimated frailty variance from quiet-window
fitting \\
\(k_d\) & Baseline hazard level for cohort \(d\) under the default
baseline shape \\
\(\hat{k}_d\) & Estimated baseline hazard level from quiet-window
fitting \\
\(t_0\) & Anchor time for baseline normalization (prespecified) \\
\(\mathrm{KCOR}(t; t_0)\) & Anchored KCOR:
\(\mathrm{KCOR}(t)/\mathrm{KCOR}(t_0)\) \\
\end{longtable}

\begin{longtable}[]{@{}
  >{\raggedright\arraybackslash}p{(\linewidth - 8\tabcolsep) * \real{0.2000}}
  >{\raggedright\arraybackslash}p{(\linewidth - 8\tabcolsep) * \real{0.2000}}
  >{\raggedright\arraybackslash}p{(\linewidth - 8\tabcolsep) * \real{0.2000}}
  >{\raggedright\arraybackslash}p{(\linewidth - 8\tabcolsep) * \real{0.2000}}
  >{\raggedright\arraybackslash}p{(\linewidth - 8\tabcolsep) * \real{0.2000}}@{}}
\caption{Step-by-step KCOR algorithm (high-level), with recommended
prespecification and
diagnostics.}\label{tbl:KCOR_algorithm}\tabularnewline
\toprule\noalign{}
\begin{minipage}[b]{\linewidth}\raggedright
Step
\end{minipage} & \begin{minipage}[b]{\linewidth}\raggedright
Operation
\end{minipage} & \begin{minipage}[b]{\linewidth}\raggedright
Output
\end{minipage} & \begin{minipage}[b]{\linewidth}\raggedright
Prespecify?
\end{minipage} & \begin{minipage}[b]{\linewidth}\raggedright
Diagnostics
\end{minipage} \\
\midrule\noalign{}
\endfirsthead
\toprule\noalign{}
\begin{minipage}[b]{\linewidth}\raggedright
Step
\end{minipage} & \begin{minipage}[b]{\linewidth}\raggedright
Operation
\end{minipage} & \begin{minipage}[b]{\linewidth}\raggedright
Output
\end{minipage} & \begin{minipage}[b]{\linewidth}\raggedright
Prespecify?
\end{minipage} & \begin{minipage}[b]{\linewidth}\raggedright
Diagnostics
\end{minipage} \\
\midrule\noalign{}
\endhead
\bottomrule\noalign{}
\endlastfoot
1 & Choose enrollment date and define fixed cohorts & Cohort labels &
Yes & Verify cohort sizes/risk sets \\
2 & Compute discrete-time hazards (observed hazards) & Hazard curves &
Yes (binning/transform) & Check for zeros/sparsity \\
3 & Apply stabilization skip and accumulate observed cumulative hazards
& Observed cumulative hazards & Yes (skip rule) & Plot observed
cumulative hazards \\
4 & Select quiet-window bins in calendar ISO-week space & Fit points
\(\mathcal{T}_d\) & Yes & Overlay quiet window on hazard plots \\
5 & Fit \((\hat{k}_d,\hat{\theta}_d)\) via cumulative-hazard least
squares & Fitted parameters & Yes & RMSE, residuals, fit stability \\
6 & Normalize: invert gamma-frailty identity to depletion-neutralized
cumulative hazards & Depletion-neutralized cumulative hazards & Yes &
Compare pre/post shapes; sanity checks \\
7 & Cumulate and ratio: compute \(\mathrm{KCOR}(t)\) &
\(\mathrm{KCOR}(t)\) curve & Yes (horizon) & Flat under negative
controls \\
8 & Uncertainty & CI / intervals & Yes & Coverage on positive
controls \\
\end{longtable}

\begin{longtable}[]{@{}
  >{\raggedleft\arraybackslash}p{(\linewidth - 10\tabcolsep) * \real{0.1364}}
  >{\raggedleft\arraybackslash}p{(\linewidth - 10\tabcolsep) * \real{0.0909}}
  >{\raggedleft\arraybackslash}p{(\linewidth - 10\tabcolsep) * \real{0.0909}}
  >{\raggedleft\arraybackslash}p{(\linewidth - 10\tabcolsep) * \real{0.1667}}
  >{\raggedleft\arraybackslash}p{(\linewidth - 10\tabcolsep) * \real{0.2121}}
  >{\raggedleft\arraybackslash}p{(\linewidth - 10\tabcolsep) * \real{0.3030}}@{}}
\caption{Cox vs KCOR under a synthetic null with increasing frailty
heterogeneity. Two cohorts are simulated with identical baseline hazards
and no treatment effect \emph{(null by construction)}; cohorts differ
only in gamma frailty variance (\(\theta\)). Despite the true hazard
ratio being 1 by construction, Cox regression produces increasingly
non-null hazard ratios as \(\theta\) increases, reflecting
depletion-induced non-proportional hazards. \(\mathrm{KCOR}(t)\) remains
centered near unity with negligible post-normalization slope across
\(\theta\) values. (Exact values depend on simulation seed and follow-up
horizon.)}\label{tbl:cox_bias_demo}\tabularnewline
\toprule\noalign{}
\begin{minipage}[b]{\linewidth}\raggedleft
\(\theta\)
\end{minipage} & \begin{minipage}[b]{\linewidth}\raggedleft
Cox HR
\end{minipage} & \begin{minipage}[b]{\linewidth}\raggedleft
95\% CI
\end{minipage} & \begin{minipage}[b]{\linewidth}\raggedleft
Cox p-value
\end{minipage} & \begin{minipage}[b]{\linewidth}\raggedleft
KCOR asymptote
\end{minipage} & \begin{minipage}[b]{\linewidth}\raggedleft
KCOR post-norm slope
\end{minipage} \\
\midrule\noalign{}
\endfirsthead
\toprule\noalign{}
\begin{minipage}[b]{\linewidth}\raggedleft
\(\theta\)
\end{minipage} & \begin{minipage}[b]{\linewidth}\raggedleft
Cox HR
\end{minipage} & \begin{minipage}[b]{\linewidth}\raggedleft
95\% CI
\end{minipage} & \begin{minipage}[b]{\linewidth}\raggedleft
Cox p-value
\end{minipage} & \begin{minipage}[b]{\linewidth}\raggedleft
KCOR asymptote
\end{minipage} & \begin{minipage}[b]{\linewidth}\raggedleft
KCOR post-norm slope
\end{minipage} \\
\midrule\noalign{}
\endhead
\bottomrule\noalign{}
\endlastfoot
0.0 & 0.988 & {[}0.969, 1.008{]} & 0.234 & 0.988 &
\(7.6 \times 10^{-4}\) \\
0.5 & 0.965 & {[}0.946, 0.985{]} & \(4.9 \times 10^{-4}\) & 0.990 &
\(-3.8 \times 10^{-5}\) \\
1.0 & 0.944 & {[}0.926, 0.963{]} & \(1.7 \times 10^{-8}\) & 0.992 &
\(-3.0 \times 10^{-4}\) \\
2.0 & 0.902 & {[}0.884, 0.921{]} & \(2.4 \times 10^{-23}\) & 0.991 &
\(3.7 \times 10^{-4}\) \\
5.0 & 0.804 & {[}0.787, 0.820{]} & \(1.5 \times 10^{-93}\) & 0.993 &
\(-5.3 \times 10^{-4}\) \\
10.0 & 0.701 & {[}0.686, 0.717{]} & \(<10^{-200}\) & 1.020 &
\(3.2 \times 10^{-4}\) \\
20.0 & 0.551 & {[}0.539, 0.564{]} & \(<10^{-300}\) & 1.024 &
\(-1.6 \times 10^{-4}\) \\
\end{longtable}

\begin{longtable}[]{@{}llrl@{}}
\caption{Example end-of-window \(\mathrm{KCOR}(t)\) values from the
empirical negative control (pooled/ASMR summaries), showing near-null
behavior under large composition differences. (Source:
\protect\texttt{test/negative\_control/out/KCOR\_summary.log})}\label{tbl:neg_control_summary}\tabularnewline
\toprule\noalign{}
Enrollment & Dose comparison & KCOR (pooled/ASMR) & 95\% CI \\
\midrule\noalign{}
\endfirsthead
\toprule\noalign{}
Enrollment & Dose comparison & KCOR (pooled/ASMR) & 95\% CI \\
\midrule\noalign{}
\endhead
\bottomrule\noalign{}
\endlastfoot
2021\_24 & 1 vs 0 & 1.0097 & {[}0.992, 1.027{]} \\
2021\_24 & 2 vs 0 & 1.0213 & {[}1.000, 1.043{]} \\
2021\_24 & 2 vs 1 & 1.0115 & {[}0.991, 1.033{]} \\
2022\_06 & 1 vs 0 & 0.9858 & {[}0.970, 1.002{]} \\
2022\_06 & 2 vs 0 & 1.0756 & {[}1.055, 1.097{]} \\
2022\_06 & 2 vs 1 & 1.0911 & {[}1.070, 1.112{]} \\
\end{longtable}

\begin{longtable}[]{@{}
  >{\raggedright\arraybackslash}p{(\linewidth - 8\tabcolsep) * \real{0.1765}}
  >{\raggedright\arraybackslash}p{(\linewidth - 8\tabcolsep) * \real{0.1765}}
  >{\raggedleft\arraybackslash}p{(\linewidth - 8\tabcolsep) * \real{0.2353}}
  >{\raggedright\arraybackslash}p{(\linewidth - 8\tabcolsep) * \real{0.1765}}
  >{\raggedleft\arraybackslash}p{(\linewidth - 8\tabcolsep) * \real{0.2353}}@{}}
\caption{Positive control results comparing injected hazard multipliers
to detected KCOR deviations. Both scenarios show KCOR deviating from 1.0
in the expected direction, validating that the estimator can detect true
effects.}\label{tbl:pos_control_summary}\tabularnewline
\toprule\noalign{}
\begin{minipage}[b]{\linewidth}\raggedright
Scenario
\end{minipage} & \begin{minipage}[b]{\linewidth}\raggedright
Effect window
\end{minipage} & \begin{minipage}[b]{\linewidth}\raggedleft
Hazard multiplier \(r\)
\end{minipage} & \begin{minipage}[b]{\linewidth}\raggedright
Expected direction
\end{minipage} & \begin{minipage}[b]{\linewidth}\raggedleft
Observed \(\mathrm{KCOR}(t)\) at week 80
\end{minipage} \\
\midrule\noalign{}
\endfirsthead
\toprule\noalign{}
\begin{minipage}[b]{\linewidth}\raggedright
Scenario
\end{minipage} & \begin{minipage}[b]{\linewidth}\raggedright
Effect window
\end{minipage} & \begin{minipage}[b]{\linewidth}\raggedleft
Hazard multiplier \(r\)
\end{minipage} & \begin{minipage}[b]{\linewidth}\raggedright
Expected direction
\end{minipage} & \begin{minipage}[b]{\linewidth}\raggedleft
Observed \(\mathrm{KCOR}(t)\) at week 80
\end{minipage} \\
\midrule\noalign{}
\endhead
\bottomrule\noalign{}
\endlastfoot
Benefit & week 20--80 & 0.8 & \textless{} 1 & 0.825 \\
Harm & week 20--80 & 1.2 & \textgreater{} 1 & 1.107 \\
\end{longtable}

\begin{longtable}[]{@{}
  >{\raggedright\arraybackslash}p{(\linewidth - 14\tabcolsep) * \real{0.0699}}
  >{\raggedright\arraybackslash}p{(\linewidth - 14\tabcolsep) * \real{0.1189}}
  >{\raggedright\arraybackslash}p{(\linewidth - 14\tabcolsep) * \real{0.0559}}
  >{\raggedright\arraybackslash}p{(\linewidth - 14\tabcolsep) * \real{0.1538}}
  >{\raggedright\arraybackslash}p{(\linewidth - 14\tabcolsep) * \real{0.1189}}
  >{\raggedright\arraybackslash}p{(\linewidth - 14\tabcolsep) * \real{0.1469}}
  >{\raggedright\arraybackslash}p{(\linewidth - 14\tabcolsep) * \real{0.1888}}
  >{\raggedright\arraybackslash}p{(\linewidth - 14\tabcolsep) * \real{0.1469}}@{}}
\caption{Comparison of Cox regression, shared frailty Cox models, and
KCOR under selection-only and joint frailty + treatment effect
scenarios. Results are from S7 simulation (joint frailty + treatment)
and gamma-frailty null scenario (selection-only). Standard Cox
regression produces non-null hazard ratios under selection-only
conditions due to depletion dynamics. Shared frailty Cox models
partially mitigate this bias but still exhibit residual non-null
behavior. KCOR remains near-null under selection-only conditions and
correctly detects treatment effects when temporal separability
holds.}\label{tbl:joint_frailty_comparison}\tabularnewline
\toprule\noalign{}
\begin{minipage}[b]{\linewidth}\raggedright
Scenario
\end{minipage} & \begin{minipage}[b]{\linewidth}\raggedright
True effect (r)
\end{minipage} & \begin{minipage}[b]{\linewidth}\raggedright
Cox HR
\end{minipage} & \begin{minipage}[b]{\linewidth}\raggedright
Shared frailty Cox HR
\end{minipage} & \begin{minipage}[b]{\linewidth}\raggedright
KCOR drift/year
\end{minipage} & \begin{minipage}[b]{\linewidth}\raggedright
Cox indicates null?
\end{minipage} & \begin{minipage}[b]{\linewidth}\raggedright
Frailty-Cox indicates null?
\end{minipage} & \begin{minipage}[b]{\linewidth}\raggedright
KCOR indicates null?
\end{minipage} \\
\midrule\noalign{}
\endfirsthead
\toprule\noalign{}
\begin{minipage}[b]{\linewidth}\raggedright
Scenario
\end{minipage} & \begin{minipage}[b]{\linewidth}\raggedright
True effect (r)
\end{minipage} & \begin{minipage}[b]{\linewidth}\raggedright
Cox HR
\end{minipage} & \begin{minipage}[b]{\linewidth}\raggedright
Shared frailty Cox HR
\end{minipage} & \begin{minipage}[b]{\linewidth}\raggedright
KCOR drift/year
\end{minipage} & \begin{minipage}[b]{\linewidth}\raggedright
Cox indicates null?
\end{minipage} & \begin{minipage}[b]{\linewidth}\raggedright
Frailty-Cox indicates null?
\end{minipage} & \begin{minipage}[b]{\linewidth}\raggedright
KCOR indicates null?
\end{minipage} \\
\midrule\noalign{}
\endhead
\bottomrule\noalign{}
\endlastfoot
Gamma-frailty null & 1.0 (null) & 0.87 & 0.94 & \textless{} 0.5\% & No
(HR ≠ 1) & No (HR ≠ 1) & Yes (flat) \\
S7 harm (r=1.2) & 1.2 & 1.18 & 1.19 & +1.8\% & No (detects effect) & No
(detects effect) & No (detects effect) \\
S7 benefit (r=0.8) & 0.8 & 0.83 & 0.82 & -2.1\% & No (detects effect) &
No (detects effect) & No (detects effect) \\
\end{longtable}

\begin{longtable}[]{@{}
  >{\raggedright\arraybackslash}p{(\linewidth - 8\tabcolsep) * \real{0.0792}}
  >{\raggedright\arraybackslash}p{(\linewidth - 8\tabcolsep) * \real{0.1584}}
  >{\raggedright\arraybackslash}p{(\linewidth - 8\tabcolsep) * \real{0.3168}}
  >{\raggedright\arraybackslash}p{(\linewidth - 8\tabcolsep) * \real{0.2079}}
  >{\raggedright\arraybackslash}p{(\linewidth - 8\tabcolsep) * \real{0.2376}}@{}}
\caption{Simulation comparison of KCOR and alternative estimands under
selection-induced non-proportional hazards. Results are summarized
across simulation scenarios (null scenarios: gamma-frailty null,
non-gamma frailty, contamination, sparse events; effect scenarios:
injected hazard increase/decrease). KCOR remains stable under
selection-only regimes, while RMST inherits depletion bias and
time-varying Cox captures non-proportional hazards without normalizing
selection geometry. All methods were applied to identical simulation
outputs.}\label{tbl:comparison_estimands}\tabularnewline
\toprule\noalign{}
\begin{minipage}[b]{\linewidth}\raggedright
Method
\end{minipage} & \begin{minipage}[b]{\linewidth}\raggedright
Target estimand
\end{minipage} & \begin{minipage}[b]{\linewidth}\raggedright
Deviation from null (selection-only scenarios)
\end{minipage} & \begin{minipage}[b]{\linewidth}\raggedright
Variance/instability
\end{minipage} & \begin{minipage}[b]{\linewidth}\raggedright
Interpretability notes
\end{minipage} \\
\midrule\noalign{}
\endfirsthead
\toprule\noalign{}
\begin{minipage}[b]{\linewidth}\raggedright
Method
\end{minipage} & \begin{minipage}[b]{\linewidth}\raggedright
Target estimand
\end{minipage} & \begin{minipage}[b]{\linewidth}\raggedright
Deviation from null (selection-only scenarios)
\end{minipage} & \begin{minipage}[b]{\linewidth}\raggedright
Variance/instability
\end{minipage} & \begin{minipage}[b]{\linewidth}\raggedright
Interpretability notes
\end{minipage} \\
\midrule\noalign{}
\endhead
\bottomrule\noalign{}
\endlastfoot
\textbf{KCOR} & Cumulative hazard ratio (depletion-normalized) & Near
zero (median KCOR ≈ 1.0) & Low (stable trajectory) & Stable under
selection-induced depletion; normalization precedes comparison \\
\textbf{RMST} & Restricted mean survival time & Non-zero (depends on
depletion strength) & Moderate (depends on depletion strength) &
Summarizes survival differences that may reflect depletion rather than
treatment effect; does not normalize selection geometry \\
\textbf{Cox} & Time-varying hazard ratio & Non-zero under frailty
heterogeneity & Moderate (HR instability across time windows) & Improves
fit to non-proportional hazards but does not normalize selection
geometry; inherits depletion structure \\
\end{longtable}

\begin{longtable}[]{@{}
  >{\raggedright\arraybackslash}p{(\linewidth - 6\tabcolsep) * \real{0.1887}}
  >{\raggedright\arraybackslash}p{(\linewidth - 6\tabcolsep) * \real{0.3208}}
  >{\raggedright\arraybackslash}p{(\linewidth - 6\tabcolsep) * \real{0.3585}}
  >{\raggedright\arraybackslash}p{(\linewidth - 6\tabcolsep) * \real{0.1321}}@{}}
\caption{Bootstrap coverage for KCOR uncertainty intervals. Coverage is
evaluated across simulation scenarios using stratified bootstrap
resampling. Nominal 95\% confidence intervals are compared to empirical
coverage (proportion of simulations where the true value lies within the
interval).}\label{tbl:bootstrap_coverage}\tabularnewline
\toprule\noalign{}
\begin{minipage}[b]{\linewidth}\raggedright
Scenario
\end{minipage} & \begin{minipage}[b]{\linewidth}\raggedright
Nominal coverage
\end{minipage} & \begin{minipage}[b]{\linewidth}\raggedright
Empirical coverage
\end{minipage} & \begin{minipage}[b]{\linewidth}\raggedright
Notes
\end{minipage} \\
\midrule\noalign{}
\endfirsthead
\toprule\noalign{}
\begin{minipage}[b]{\linewidth}\raggedright
Scenario
\end{minipage} & \begin{minipage}[b]{\linewidth}\raggedright
Nominal coverage
\end{minipage} & \begin{minipage}[b]{\linewidth}\raggedright
Empirical coverage
\end{minipage} & \begin{minipage}[b]{\linewidth}\raggedright
Notes
\end{minipage} \\
\midrule\noalign{}
\endhead
\bottomrule\noalign{}
\endlastfoot
Gamma-frailty null & 95\% & 94.2\% & Coverage evaluated under
selection-only conditions \\
Injected effect (harm) & 95\% & 93.8\% & Coverage evaluated under known
treatment effect \\
Injected effect (benefit) & 95\% & 93.5\% & Coverage evaluated under
known treatment effect \\
Non-gamma frailty & 95\% & 89.3\% & Coverage under frailty
misspecification \\
Sparse events & 95\% & 87.6\% & Coverage under reduced event counts \\
\end{longtable}

\begin{longtable}[]{@{}
  >{\raggedright\arraybackslash}p{(\linewidth - 4\tabcolsep) * \real{0.3077}}
  >{\raggedleft\arraybackslash}p{(\linewidth - 4\tabcolsep) * \real{0.3462}}
  >{\raggedleft\arraybackslash}p{(\linewidth - 4\tabcolsep) * \real{0.3462}}@{}}
\caption{Table C.1. Estimated gamma-frailty variance (fitted frailty
variance) by age band and vaccination status for Czech cohorts enrolled
in 2021\_24.}\tabularnewline
\toprule\noalign{}
\begin{minipage}[b]{\linewidth}\raggedright
Age band (years)
\end{minipage} & \begin{minipage}[b]{\linewidth}\raggedleft
Fitted frailty variance (Dose 0)
\end{minipage} & \begin{minipage}[b]{\linewidth}\raggedleft
Fitted frailty variance (Dose 2)
\end{minipage} \\
\midrule\noalign{}
\endfirsthead
\toprule\noalign{}
\begin{minipage}[b]{\linewidth}\raggedright
Age band (years)
\end{minipage} & \begin{minipage}[b]{\linewidth}\raggedleft
Fitted frailty variance (Dose 0)
\end{minipage} & \begin{minipage}[b]{\linewidth}\raggedleft
Fitted frailty variance (Dose 2)
\end{minipage} \\
\midrule\noalign{}
\endhead
\bottomrule\noalign{}
\endlastfoot
40--49 & 16.79 & \(2.66 \times 10^{-6}\) \\
50--59 & 23.02 & \(1.87 \times 10^{-4}\) \\
60--69 & 13.13 & \(7.01 \times 10^{-18}\) \\
70--79 & 6.98 & \(3.46 \times 10^{-17}\) \\
80--89 & 2.97 & \(2.03 \times 10^{-11}\) \\
90--99 & 0.80 & \(8.66 \times 10^{-16}\) \\
All ages (full population) & 4.98 & \(1.02 \times 10^{-11}\) \\
\end{longtable}

\textbf{Notes:} - The fitted frailty variance quantifies unobserved
frailty heterogeneity and depletion of susceptibles within cohorts.
Near-zero values indicate effectively linear cumulative hazards over the
quiet window and are typical of strongly pre-selected cohorts. - Each
entry reports a single fitted gamma-frailty variance for the specified
age band and vaccination status within the 2021\_24 enrollment cohort. -
The ``All ages (full population)'' row corresponds to an independent fit
over the full pooled age range, included as a global diagnostic. - Table
C.3 reports raw outcome contrasts for ages 40+ (YOB ≤ 1980) where event
counts are stable.

\textbf{Diagnostic checks:} - \textbf{Dose ordering:} the fitted frailty
variance is positive for Dose 0 and collapses toward zero for Dose 2
across all age strata, consistent with selective uptake. -
\textbf{Magnitude separation:} Dose 2 estimates are effectively zero
relative to Dose 0, indicating near-linear cumulative hazards rather
than forced curvature. - \textbf{Age coherence:} the fitted frailty
variance decreases at older ages as baseline mortality rises and
survivor populations become more homogeneous; monotonicity is not
imposed. - \textbf{Stability:} No sign reversals, boundary pathologies,
or numerical instabilities are observed. - \textbf{Falsifiability:}
Failure of any one of these checks would constitute evidence against
model adequacy.

\begin{longtable}[]{@{}
  >{\raggedright\arraybackslash}p{(\linewidth - 8\tabcolsep) * \real{0.1702}}
  >{\raggedright\arraybackslash}p{(\linewidth - 8\tabcolsep) * \real{0.1915}}
  >{\raggedright\arraybackslash}p{(\linewidth - 8\tabcolsep) * \real{0.2979}}
  >{\raggedright\arraybackslash}p{(\linewidth - 8\tabcolsep) * \real{0.2021}}
  >{\raggedright\arraybackslash}p{(\linewidth - 8\tabcolsep) * \real{0.1383}}@{}}
\caption{Table C.2. Diagnostic gate for Czech application: KCOR results
reported only where diagnostics pass.}\tabularnewline
\toprule\noalign{}
\begin{minipage}[b]{\linewidth}\raggedright
Age band (years)
\end{minipage} & \begin{minipage}[b]{\linewidth}\raggedright
Quiet window valid
\end{minipage} & \begin{minipage}[b]{\linewidth}\raggedright
Post-normalization linearity
\end{minipage} & \begin{minipage}[b]{\linewidth}\raggedright
Parameter stability
\end{minipage} & \begin{minipage}[b]{\linewidth}\raggedright
KCOR reported
\end{minipage} \\
\midrule\noalign{}
\endfirsthead
\toprule\noalign{}
\begin{minipage}[b]{\linewidth}\raggedright
Age band (years)
\end{minipage} & \begin{minipage}[b]{\linewidth}\raggedright
Quiet window valid
\end{minipage} & \begin{minipage}[b]{\linewidth}\raggedright
Post-normalization linearity
\end{minipage} & \begin{minipage}[b]{\linewidth}\raggedright
Parameter stability
\end{minipage} & \begin{minipage}[b]{\linewidth}\raggedright
KCOR reported
\end{minipage} \\
\midrule\noalign{}
\endhead
\bottomrule\noalign{}
\endlastfoot
40--49 & Yes & Yes & Yes & Yes \\
50--59 & Yes & Yes & Yes & Yes \\
60--69 & Yes & Yes & Yes & Yes \\
70--79 & Yes & Yes & Yes & Yes \\
80--89 & Yes & Yes & Yes & Yes \\
90--99 & Yes & Yes & Yes & Yes \\
All ages & Yes & Yes & Yes & Yes \\
\end{longtable}

\begin{longtable}[]{@{}
  >{\raggedright\arraybackslash}p{(\linewidth - 6\tabcolsep) * \real{0.2353}}
  >{\raggedleft\arraybackslash}p{(\linewidth - 6\tabcolsep) * \real{0.3382}}
  >{\raggedleft\arraybackslash}p{(\linewidth - 6\tabcolsep) * \real{0.3529}}
  >{\raggedleft\arraybackslash}p{(\linewidth - 6\tabcolsep) * \real{0.0735}}@{}}
\caption{Table C.3. Ratio of observed cumulative mortality hazards for
unvaccinated (Dose 0) versus fully vaccinated (Dose 2) Czech cohorts
enrolled in 2021\_24.}\tabularnewline
\toprule\noalign{}
\begin{minipage}[b]{\linewidth}\raggedright
Age band (years)
\end{minipage} & \begin{minipage}[b]{\linewidth}\raggedleft
Dose 0 cumulative hazard
\end{minipage} & \begin{minipage}[b]{\linewidth}\raggedleft
Dose 2 cumulative hazard
\end{minipage} & \begin{minipage}[b]{\linewidth}\raggedleft
Ratio
\end{minipage} \\
\midrule\noalign{}
\endfirsthead
\toprule\noalign{}
\begin{minipage}[b]{\linewidth}\raggedright
Age band (years)
\end{minipage} & \begin{minipage}[b]{\linewidth}\raggedleft
Dose 0 cumulative hazard
\end{minipage} & \begin{minipage}[b]{\linewidth}\raggedleft
Dose 2 cumulative hazard
\end{minipage} & \begin{minipage}[b]{\linewidth}\raggedleft
Ratio
\end{minipage} \\
\midrule\noalign{}
\endhead
\bottomrule\noalign{}
\endlastfoot
40--49 & 0.005260 & 0.004117 & 1.2776 \\
50--59 & 0.014969 & 0.009582 & 1.5622 \\
60--69 & 0.045475 & 0.023136 & 1.9655 \\
70--79 & 0.123097 & 0.057675 & 2.1343 \\
80--89 & 0.307169 & 0.167345 & 1.8355 \\
90--99 & 0.776341 & 0.517284 & 1.5008 \\
All ages (full population) & 0.023160 & 0.073323 & 0.3159 \\
\end{longtable}

This table reports unadjusted cumulative hazards derived directly from
the raw data, prior to any frailty normalization or depletion
correction, and is shown to illustrate the magnitude and direction of
selection-induced curvature addressed by KCOR.

Values reflect raw cumulative outcome differences prior to KCOR
normalization and are not interpreted causally due to cohort
non-exchangeability. Cumulative hazards were integrated from cohort
enrollment through the end of available follow-up for the 2021\_24
enrollment window (through week 2024-16), identically for Dose 0 and
Dose 2 cohorts.

\textbf{Table D.1: KCOR assumptions and corresponding diagnostics.}

{\def\LTcaptype{none} % do not increment counter
\begin{longtable}[]{@{}
  >{\raggedright\arraybackslash}p{(\linewidth - 8\tabcolsep) * \real{0.2000}}
  >{\raggedright\arraybackslash}p{(\linewidth - 8\tabcolsep) * \real{0.2000}}
  >{\raggedright\arraybackslash}p{(\linewidth - 8\tabcolsep) * \real{0.2000}}
  >{\raggedright\arraybackslash}p{(\linewidth - 8\tabcolsep) * \real{0.2000}}
  >{\raggedright\arraybackslash}p{(\linewidth - 8\tabcolsep) * \real{0.2000}}@{}}
\toprule\noalign{}
\begin{minipage}[b]{\linewidth}\raggedright
Assumption
\end{minipage} & \begin{minipage}[b]{\linewidth}\raggedright
What must hold
\end{minipage} & \begin{minipage}[b]{\linewidth}\raggedright
Diagnostic signal
\end{minipage} & \begin{minipage}[b]{\linewidth}\raggedright
Interpretation
\end{minipage} & \begin{minipage}[b]{\linewidth}\raggedright
Action if violated
\end{minipage} \\
\midrule\noalign{}
\endhead
\bottomrule\noalign{}
\endlastfoot
A1. Fixed cohort at enrollment & Cohort membership does not change over
follow-up & Step changes or discontinuities inconsistent with depletion
& Endogenous selection or reclassification & Redefine cohort at
enrollment; disallow transitions \\
A2. Shared external hazard environment & Cohorts experience the same
background hazard within the comparison window & Divergent slopes during
prespecified quiet periods & Unshared exogenous shocks or
policy/measurement effects & Restrict calendar window, stratify, or use
alternative controls \\
A3. Time-invariant latent frailty & Individual frailty is time-invariant
over follow-up & Systematic residual curvature after normalization &
Time-varying susceptibility or competing selection processes & Shorten
follow-up window; reinterpret as time-varying selection \\
A4. Adequacy of gamma frailty & Gamma family adequately approximates
frailty mixing & Residual curvature or poor fit diagnostics after
inversion & Frailty distribution misspecification & Treat as diagnostic;
avoid over-interpretation \\
A5. Quiet-window validity & No intervention effect during
frailty-estimation window & Slope breaks or non-parallel trends within
quiet window & Contaminated quiet window & Redefine quiet window; rerun
diagnostics \\
\end{longtable}
}

\newpage

\subsection{Appendix A: Mathematical
derivations}\label{appendix-a-mathematical-derivations}

\subsubsection{A.1 Frailty mixing induces hazard
curvature}\label{a.1-frailty-mixing-induces-hazard-curvature}

Consider a cohort \(d\) in which individual \(i\) has hazard

\[
h_{i,d}(t) = z_{i,d} \, h_{0,d}(t).
\qquad\text{(A.1)}
\]

where the frailty \(z_{i,d}\) is drawn from a distribution with mean 1
and variance \(\theta_d > 0\). Let

\[
S_{i,d}(t) = \exp\!\left(-z_{i,d} H_{0,d}(t)\right).
\qquad\text{(A.2)}
\]

denote the individual survival function, where

\[
H_{0,d}(t) = \int_0^t h_{0,d}(s) \, ds.
\qquad\text{(A.3)}
\]

is the baseline cumulative hazard.

The cohort survival function, obtained by integrating over the frailty
distribution, is

\[
S_d(t) = \mathbb{E}[S_{i,d}(t)] = \mathbb{E}[\exp(-z \, H_{0,d}(t))] = \mathcal{L}_z(H_{0,d}(t)).
\qquad\text{(A.4)}
\]

where \(\mathcal{L}_z(\cdot)\) denotes the Laplace transform of the
frailty distribution. The corresponding cohort-level hazard is

\[
h_d(t) = -\frac{d}{dt} \log S_d(t).
\qquad\text{(A.5)}
\]

Even when \(h_{0,d}(t) = k_d\) is constant (so that
\(H_{0,d}(t) = k_d t\)), the cohort hazard \(h_d(t)\) is generally
time-varying. High-frailty individuals experience events earlier,
progressively depleting the higher-risk portion of the cohort and
shifting the surviving population, conditional on survival, toward lower
frailty over time. This selection-induced depletion is the mechanism by
which frailty heterogeneity induces \textbf{curvature} in cohort-level
hazards.

\subsubsection{A.2 Gamma-frailty identity
derivation}\label{a.2-gamma-frailty-identity-derivation}

For gamma-distributed frailty
\(z \sim \mathrm{Gamma}(\alpha = 1/\theta_d, \beta = 1/\theta_d)\) with
mean 1 and variance \(\theta_d\), the Laplace transform is:

\[
\mathcal{L}_z(s) = \left(1 + \theta_d s\right)^{-1/\theta_d}.
\qquad\text{(A.6)}
\]

The cohort survival function becomes:

\[
S^{\mathrm{cohort}}_{d}(t) = \left(1 + \theta_d H_{0,d}(t)\right)^{-1/\theta_d}.
\qquad\text{(A.7)}
\]

The observed cumulative hazard is defined as:

\[
H_{\mathrm{obs},d}(t) = -\log S^{\mathrm{cohort}}_{d}(t).
\qquad\text{(A.8)}
\]

Substituting the gamma Laplace transform yields the canonical
gamma-frailty identity:

\[
H_{\mathrm{obs},d}(t)
=
\frac{1}{\theta_d}
\log\!\left(1+\theta_d\,H_{0,d}(t)\right).
\qquad\text{(A.9)}
\]

This is the gamma-frailty identity (see Equation
(\ref{eq:gamma-frailty-identity}) in the main text).

\subsubsection{A.3 Inversion formula}\label{a.3-inversion-formula}

Solving for \(H_{0,d}(t)\) from the gamma-frailty identity gives the
canonical inversion:

\[
H_{0,d}(t)
=
\frac{\exp\!\left(\theta_d\,H_{\mathrm{obs},d}(t)\right)-1}{\theta_d}.
\qquad\text{(A.10)}
\]

This inversion recovers the baseline cumulative hazard from the observed
cumulative hazard, conditional on the frailty variance \(\theta_d\).

\subsubsection{A.3a Relationship to the Vaupel--Manton--Stallard gamma
frailty
framework}\label{a.3a-relationship-to-the-vaupelmantonstallard-gamma-frailty-framework}

KCOR's normalization step is grounded in the classical demographic
frailty framework (e.g., Vaupel--Manton--Stallard), in which individual
hazards are multiplicatively scaled by latent frailty and cohort-level
hazards decelerate due to depletion of susceptibles. Under gamma
frailty, the Laplace-transform identity yields a closed-form
relationship between observed cohort cumulative hazard and baseline
cumulative hazard, and the inversion in §A.3 recovers the baseline
cumulative hazard from observed cumulative hazards given \(\theta_d\).

The distinction in KCOR is not the frailty identity itself, but the
\textbf{direction of inference} and the \textbf{estimand}.
Frailty-augmented Cox and related regression approaches embed gamma
frailty within a regression model to estimate covariate effects (hazard
ratios). KCOR instead uses quiet-window curvature to estimate
cohort-specific frailty parameters and then inverts the frailty identity
to obtain depletion-neutralized baseline cumulative hazards, defining
KCOR as a ratio of these cumulative quantities. Thus, KCOR solves an
inverse normalization problem and targets cumulative comparisons under
selection-induced non-proportional hazards rather than instantaneous
hazard-ratio regression parameters.

\subsubsection{A.4 Variance propagation
(sketch)}\label{a.4-variance-propagation-sketch}

For uncertainty quantification, variance in KCOR\((t)\) can be
approximated via the delta method. Define:

\[
KCOR(t)=\frac{\tilde{H}_{0,A}(t)}{\tilde{H}_{0,B}(t)}.
\qquad\text{(A.11)}
\]

If the variance of the depletion-neutralized cumulative hazard is
available (e.g., from bootstrap or analytic propagation through the
inversion), then:

\[
\mathrm{Var}\!\left(KCOR(t)\right) \approx KCOR(t)^2 \left[ \frac{\mathrm{Var}(\tilde{H}_{0,A}(t))}{\tilde{H}_{0,A}(t)^2} + \frac{\mathrm{Var}(\tilde{H}_{0,B}(t))}{\tilde{H}_{0,B}(t)^2} - 2\frac{\mathrm{Cov}(\tilde{H}_{0,A}(t), \tilde{H}_{0,B}(t))}{\tilde{H}_{0,A}(t)\tilde{H}_{0,B}(t)} \right].
\qquad\text{(A.12)}
\]

In practice, Monte Carlo resampling provides a more robust approach that
captures uncertainty from both event realization and parameter
estimation.

\subsection{Appendix B: Control-test
specifications}\label{appendix-b-control-test-specifications}

\subsubsection{B.1 Negative control: synthetic gamma-frailty
null}\label{b.1-negative-control-synthetic-gamma-frailty-null}

The synthetic negative control (Figure \ref{fig:neg_control_synthetic})
is generated using:

\begin{itemize}
\tightlist
\item
  \textbf{Data source}: \texttt{example/Frail\_cohort\_mix.xlsx}
  (pathological frailty mixture)
\item
  \textbf{Generation script}:
  \texttt{code/generate\_pathological\_neg\_control\_figs.py}
\item
  \textbf{Cohort A weights}: Equal weights across 5 frailty groups (0.2
  each)
\item
  \textbf{Cohort B weights}: Shifted weights {[}0.30, 0.20, 0.20, 0.20,
  0.10{]}
\item
  \textbf{Frailty values}: {[}1, 2, 4, 6, 10{]} (relative frailty
  multipliers)
\item
  \textbf{Base weekly probability}: 0.01
\item
  \textbf{Weekly log-slope}: 0.0 (constant baseline during quiet
  periods)
\item
  \textbf{Skip weeks}: 2
\item
  \textbf{Normalization weeks}: 4
\item
  \textbf{Time horizon}: 250 weeks
\end{itemize}

Both cohorts share identical per-frailty-group death probabilities; only
the mixture weights differ. This induces different cohort-level
curvature under the null.

\subsubsection{B.2 Negative control: empirical age-shift
construction}\label{b.2-negative-control-empirical-age-shift-construction}

The empirical negative control (Figures \ref{fig:neg_control_10yr} and
\ref{fig:neg_control_20yr}) is generated using:

\begin{itemize}
\tightlist
\item
  \textbf{Data source}: Czech Republic administrative mortality and
  vaccination data, aggregated into KCOR\_CMR format
\item
  \textbf{Generation script}:
  \texttt{test/negative\_control/code/generate\_negative\_control.py}
\item
  \textbf{Construction}: Age strata remapped to pseudo-doses within same
  vaccination category
\item
  \textbf{Age mapping}:

  \begin{itemize}
  \tightlist
  \item
    Dose 0 → YoB \{1930, 1935\}
  \item
    Dose 1 → YoB \{1940, 1945\}
  \item
    Dose 2 → YoB \{1950, 1955\}
  \end{itemize}
\item
  \textbf{Output YoB}: Fixed at 1950 (unvax cohort) or 1940 (vax cohort)
\item
  \textbf{Sheets processed}: 2021\_24, 2022\_06
\end{itemize}

This construction ensures that dose comparisons are within the same
underlying vaccination category, preserving a true null while inducing
10--20 year age differences.

\subsubsection{B.3 Positive control: injected
effect}\label{b.3-positive-control-injected-effect}

The positive control (Figure \ref{fig:pos_control_injected} and Table
\ref{tbl:pos_control_summary}) is generated using:

\begin{itemize}
\tightlist
\item
  \textbf{Generation script}:
  \texttt{test/positive\_control/code/generate\_positive\_control.py}
\item
  \textbf{Initial cohort size}: 100,000 per cohort
\item
  \textbf{Baseline hazard}: 0.002 per week
\item
  \textbf{Frailty variance}: \(\theta_0 = 0.5\) (control),
  \(\theta_1 = 1.0\) (treatment)
\item
  \textbf{Effect window}: weeks 20--80
\item
  \textbf{Hazard multipliers}:

  \begin{itemize}
  \tightlist
  \item
    Harm scenario: \(r = 1.2\)
  \item
    Benefit scenario: \(r = 0.8\)
  \end{itemize}
\item
  \textbf{Random seed}: 42
\item
  \textbf{Enrollment date}: 2021-06-14 (ISO week 2021\_24)
\end{itemize}

The injection multiplies the treatment cohort's baseline hazard by
factor \(r\) during the effect window, while leaving the control cohort
unchanged.

\subsubsection{B.4 Sensitivity analysis
parameters}\label{b.4-sensitivity-analysis-parameters}

The sensitivity analysis (Figure \ref{fig:sensitivity_overview}) varies:

\begin{itemize}
\tightlist
\item
  \textbf{Baseline weeks}: {[}2, 3, 4, 5, 6, 7, 8{]}
\item
  \textbf{Quiet-start offsets}: {[}-12, -8, -4, 0, +4, +8, +12{]} weeks
  from 2023-01
\item
  \textbf{Quiet-window end}: Fixed at 2023-52
\item
  \textbf{Dose pairs}: 1 vs 0, 2 vs 0, 2 vs 1
\item
  \textbf{Cohorts}: 2021\_24
\end{itemize}

Output grids show KCOR(t) values for each parameter combination.

\subsubsection{B.5 Tail-sampling / bimodal selection (adversarial
selection
geometry)}\label{b.5-tail-sampling-bimodal-selection-adversarial-selection-geometry}

We generate a base frailty population distribution with mean 1. Cohort
construction differs by selection rule:

\begin{itemize}
\tightlist
\item
  \textbf{Mid-sampled cohort}: frailty restricted to central quantiles
  (e.g., 25th--75th percentile) and renormalized to mean 1.
\item
  \textbf{Tail-sampled cohort}: mixture of low and high tails (e.g.,
  0--15th and 85th--100th percentiles) with mixture weights chosen to
  yield mean 1.
\end{itemize}

Both cohorts share the same baseline hazard \(h_0(t)\) and no treatment
effect (negative-control version). We also generate positive-control
versions by applying a known hazard multiplier in a prespecified window.
We evaluate (i) KCOR drift, (ii) quiet-window fit RMSE, (iii)
post-normalization linearity, and (iv) parameter stability under window
perturbation.

\begin{itemize}
\tightlist
\item
  \textbf{Generation script}:
  \texttt{test/sim\_grid/code/generate\_tail\_sampling\_sim.py}
\item
  \textbf{Base frailty distribution}: Log-normal with mean 1, variance
  0.5
\item
  \textbf{Mid-quantile cohort}: 25th--75th percentile
\item
  \textbf{Tail-mixture cohort}: {[}0--15th{]} + {[}85th--100th{]}
  percentiles, equal weights
\item
  \textbf{Baseline hazard}: 0.002 per week (constant)
\item
  \textbf{Positive-control hazard multiplier}: \(r = 1.2\) (harm) or
  \(r = 0.8\) (benefit)
\item
  \textbf{Effect window}: weeks 20--80
\item
  \textbf{Random seed}: 42
\end{itemize}

\subsubsection{B.6 Joint frailty and treatment-effect simulation
(S7)}\label{b.6-joint-frailty-and-treatment-effect-simulation-s7}

This simulation evaluates KCOR under conditions in which \textbf{both
selection-induced depletion (frailty heterogeneity)} and a \textbf{true
treatment effect (harm or benefit)} are present simultaneously. The
purpose is to assess whether KCOR can (i) correctly identify and
neutralize frailty-driven curvature using a quiet period and (ii) detect
a true treatment effect outside that period without confounding the two
mechanisms.

\paragraph{Design}\label{design}

Two fixed cohorts are generated with identical baseline hazards but
differing frailty variance. Individual hazards are multiplicatively
scaled by a latent frailty term drawn from a gamma distribution with
unit mean and cohort-specific variance. A treatment effect is then
injected over a prespecified time window that does not overlap the quiet
period used for frailty estimation.

Formally, individual hazards are generated as

\[
h_i(t) = z_i \cdot h_0(t) \cdot r(t).
\qquad\text{(B.1)}
\]

where \(z_i\) is individual frailty, \(h_0(t)\) is a shared baseline
hazard, and \(r(t)\) is a time-localized multiplicative treatment effect
applied to one cohort only.

\paragraph{Frailty structure}\label{frailty-structure}

\begin{itemize}
\tightlist
\item
  Cohort 0: \(z \sim \text{Gamma}(\theta_0)\)
\item
  Cohort 1: \(z \sim \text{Gamma}(\theta_1)\), with
  \(\theta_1 \neq \theta_0\)
\end{itemize}

Frailty distributions are normalized to unit mean, differing only in
variance, thereby inducing different depletion dynamics and
cumulative-hazard curvature across cohorts in the absence of any
treatment effect.

\paragraph{Treatment effect}\label{treatment-effect}

A known treatment effect is applied to Cohort 1 during a finite window
\([t_{\text{on}}, t_{\text{off}}]\). Three effect shapes are considered:

\begin{enumerate}
\def\labelenumi{\arabic{enumi}.}
\tightlist
\item
  Step change (constant multiplicative factor),
\item
  Linear ramp,
\item
  Smooth pulse (``bump'').
\end{enumerate}

Both harmful (\(r(t) > 1\)) and protective (\(r(t) < 1\)) effects are
evaluated. The treatment window is chosen to lie strictly outside the
quiet period used for frailty estimation.

\paragraph{Quiet period and
estimation}\label{quiet-period-and-estimation}

Frailty parameters are estimated independently for each cohort using
observed cumulative hazards over a prespecified quiet window
\([t_q^{\text{start}}, t_q^{\text{end}}]\) during which \(r(t)=1\) by
construction. KCOR normalization is then applied to the full time
horizon using these estimated parameters.

This design enforces \textbf{temporal separability} between
selection-induced depletion and treatment effects.

\paragraph{Evaluation criteria}\label{evaluation-criteria}

The simulation is considered successful if:

\begin{enumerate}
\def\labelenumi{\arabic{enumi}.}
\tightlist
\item
  KCOR(t) remains approximately flat and near unity during the quiet
  window,
\item
  KCOR(t) deviates in the correct direction and magnitude during the
  treatment window,
\item
  Fit diagnostics (e.g., residual curvature, post-normalization
  linearity) remain stable outside intentionally violated scenarios.
\end{enumerate}

An additional stress-test variant intentionally overlaps the treatment
window with the quiet period. In this case, KCOR diagnostics degrade and
normalized trajectories fail to stabilize, correctly signaling violation
of the identifiability assumptions rather than producing spurious
treatment effects.

\paragraph{Interpretation}\label{interpretation}

This simulation demonstrates that when selection-induced depletion and
treatment effects are temporally separable, KCOR can disentangle the two
mechanisms: frailty parameters are identified from quiet-period
curvature, and true treatment effects manifest as deviations from unity
outside that window. When separability is violated, KCOR does not
silently misattribute effects; instead, diagnostics flag reduced
interpretability.

\newpage

\subsection{Appendix C: Additional figures and
diagnostics}\label{appendix-c-additional-figures-and-diagnostics}

\subsubsection{C.1 Fit diagnostics}\label{c.1-fit-diagnostics}

For each cohort \(d\), the gamma-frailty fit produces diagnostic outputs
including:

\begin{itemize}
\tightlist
\item
  \textbf{RMSE in \(H\)-space}: Root mean squared error between observed
  and model-predicted cumulative hazards over the quiet window. Values
  \textless{} 0.01 indicate excellent fit; values \textgreater{} 0.05
  may warrant investigation.
\item
  \textbf{Fitted parameters}: baseline hazard level and frailty
  variance. Very small frailty variance (\textless{} 0.01) indicates
  minimal detected depletion; very large values (\textgreater{} 5) may
  indicate model stress.
\item
  \textbf{Number of fit points}: \(n_{\mathrm{obs}}\) observations in
  quiet window. Larger \(n_{\mathrm{obs}}\) provides more stable
  estimates.
\end{itemize}

Example diagnostic output from the reference implementation:

\begin{verbatim}
KCOR_FIT,EnrollmentDate=2021_24,YoB=1950,Dose=0,
  k_hat=4.29e-03,theta_hat=8.02e-01,
  RMSE_Hobs=3.37e-03,n_obs=97,success=1
\end{verbatim}

\subsubsection{C.2 Residual analysis}\label{c.2-residual-analysis}

Fit residuals should be examined for. Define residuals:

\[
r_{d}(t)=H_{\mathrm{obs},d}(t)-H_{d}^{\mathrm{model}}(t;\hat{k}_d,\hat{\theta}_d).
\qquad\text{(C.1)}
\]

\begin{itemize}
\tightlist
\item
  \textbf{Systematic patterns}: Residuals should be approximately random
  around zero. Systematic curvature in residuals suggests model
  inadequacy.
\item
  \textbf{Outliers}: Individual weeks with large residuals may indicate
  data quality issues or external shocks.
\item
  \textbf{Autocorrelation}: Strong autocorrelation in residuals suggests
  the model is missing time-varying structure.
\end{itemize}

\subsubsection{C.3 Parameter stability
checks}\label{c.3-parameter-stability-checks}

Robustness of fitted parameters should be assessed by:

\begin{itemize}
\tightlist
\item
  \textbf{Quiet-window perturbation}: Shift the quiet-window start/end
  by ±4 weeks and re-fit. Stable parameters should vary by \textless{}
  10\%.
\item
  \textbf{Skip-weeks sensitivity}: Vary SKIP\_WEEKS from 0 to 8 and
  verify KCOR(t) trajectories remain qualitatively similar.
\item
  \textbf{Baseline-shape alternatives}: Compare the default constant
  baseline over the fit window to mild linear trends and verify
  normalization is not sensitive to this choice.
\end{itemize}

The sensitivity analysis (§3.3 and Figure
\ref{fig:sensitivity_overview}) provides a systematic assessment of
parameter stability.

\subsubsection{C.4 Quiet-window overlay
plots}\label{c.4-quiet-window-overlay-plots}

Recommended diagnostic: overlay the prespecified quiet window on hazard
and cumulative-hazard time series plots. The fit window should:

\begin{itemize}
\tightlist
\item
  Avoid major epidemic waves or external mortality shocks
\item
  Contain sufficient event counts for stable estimation
\item
  Span a time range where baseline mortality is approximately stationary
\end{itemize}

Visual inspection of quiet-window placement relative to mortality
dynamics is an essential diagnostic step.

\subsubsection{C.5 Robustness to age
stratification}\label{c.5-robustness-to-age-stratification}

\begin{figure}
\centering
\pandocbounded{\includegraphics[keepaspectratio,alt={Figure C.1. Birth-year cohort 1930: KCOR(t) trajectories comparing dose 2 and dose 3 to dose 0 for cohorts enrolled in ISO week 2022-26 and evaluated over calendar year 2023. KCOR curves are anchored at t\_0 = 4 weeks (i.e., plotted as \textbackslash mathrm\{KCOR\}(t; t\_0)). This figure is presented as an illustrative application demonstrating estimator behavior on registry data and does not support causal inference.}]{figures/supplement/kcor_realdata_yob1930_enroll2022w26_eval2023.png}}
\caption{Figure C.1. Birth-year cohort 1930: KCOR(t) trajectories
comparing dose 2 and dose 3 to dose 0 for cohorts enrolled in ISO week
2022-26 and evaluated over calendar year 2023. KCOR curves are anchored
at \(t_0 = 4\) weeks (i.e., plotted as \(\mathrm{KCOR}(t; t_0)\)). This
figure is presented as an illustrative application demonstrating
estimator behavior on registry data and does not support causal
inference.}
\end{figure}

\begin{figure}
\centering
\pandocbounded{\includegraphics[keepaspectratio,alt={Figure C.2. Birth-year cohort 1940: KCOR(t) trajectories comparing dose 2 and dose 3 to dose 0 for cohorts enrolled in ISO week 2022-26 and evaluated over calendar year 2023. KCOR curves are anchored at t\_0 = 4 weeks (i.e., plotted as \textbackslash mathrm\{KCOR\}(t; t\_0)). This figure is presented as an illustrative application demonstrating estimator behavior on registry data and does not support causal inference.}]{figures/supplement/kcor_realdata_yob1940_enroll2022w26_eval2023.png}}
\caption{Figure C.2. Birth-year cohort 1940: KCOR(t) trajectories
comparing dose 2 and dose 3 to dose 0 for cohorts enrolled in ISO week
2022-26 and evaluated over calendar year 2023. KCOR curves are anchored
at \(t_0 = 4\) weeks (i.e., plotted as \(\mathrm{KCOR}(t; t_0)\)). This
figure is presented as an illustrative application demonstrating
estimator behavior on registry data and does not support causal
inference.}
\end{figure}

\begin{figure}
\centering
\pandocbounded{\includegraphics[keepaspectratio,alt={Figure C.3. Birth-year cohort 1950: KCOR(t) trajectories comparing dose 2 and dose 3 to dose 0 for cohorts enrolled in ISO week 2022-26 and evaluated over calendar year 2023. KCOR curves are anchored at t\_0 = 4 weeks (i.e., plotted as \textbackslash mathrm\{KCOR\}(t; t\_0)). This figure is presented as an illustrative application demonstrating estimator behavior on registry data and does not support causal inference.}]{figures/supplement/kcor_realdata_yob1950_enroll2022w26_eval2023.png}}
\caption{Figure C.3. Birth-year cohort 1950: KCOR(t) trajectories
comparing dose 2 and dose 3 to dose 0 for cohorts enrolled in ISO week
2022-26 and evaluated over calendar year 2023. KCOR curves are anchored
at \(t_0 = 4\) weeks (i.e., plotted as \(\mathrm{KCOR}(t; t_0)\)). This
figure is presented as an illustrative application demonstrating
estimator behavior on registry data and does not support causal
inference.}
\end{figure}

\subsubsection{C.6 Empirical application with diagnostic validation:
Czech Republic national registry mortality
data}\label{c.6-empirical-application-with-diagnostic-validation-czech-republic-national-registry-mortality-data}

The Czech results do not validate KCOR; they represent an application
that satisfies all pre-specified diagnostic criteria. Substantive
implications follow only if the identification assumptions hold.
Throughout this subsection, observed divergences are interpreted
strictly as properties of the estimator under real-world selection, not
as intervention effects.

Unless otherwise noted, KCOR curves in the Czech analyses are shown
anchored at \(t_0 = 4\) weeks for interpretability.

\subsubsection{C.6.1 Illustrative empirical context: COVID-19 mortality
data}\label{c.6.1-illustrative-empirical-context-covid-19-mortality-data}

The COVID-19 vaccination period provides a natural empirical regime
characterized by strong selection heterogeneity and non-proportional
hazards, making it a useful illustration for the KCOR framework. During
this period, vaccine uptake was voluntary, rapidly time-varying, and
correlated with baseline health status, creating clear examples of
selection-induced non-proportional hazards. The Czech Republic national
mortality registry data exemplify this regime: voluntary uptake led to
asymmetric selection at enrollment, with vaccinated cohorts exhibiting
minimal frailty heterogeneity while unvaccinated cohorts retained
substantial heterogeneity. This asymmetric pattern reflects the healthy
vaccinee effect operating through selective uptake rather than
treatment. KCOR normalization removes this selection-induced curvature,
enabling interpretable cumulative comparisons. While these examples
illustrate KCOR's application, the method is general and applies to any
retrospective cohort comparison where selection induces differential
depletion dynamics.

\subsubsection{C.6.2 Frailty normalization behavior under empirical
validation}\label{c.6.2-frailty-normalization-behavior-under-empirical-validation}

Across examined age strata in the Czech Republic mortality dataset,
fitted frailty parameters exhibit a pronounced asymmetry across cohorts.
Some cohorts show negligible estimated frailty variance:

\[
\hat{\theta}_d \approx 0.
\qquad\text{(C.2)}
\]

while others exhibit substantial frailty-driven depletion. This pattern
reflects differences in selection-induced hazard curvature at cohort
entry rather than any prespecified cohort identity.

As a consequence, KCOR normalization leaves some cohorts' cumulative
hazards nearly unchanged, while substantially increasing the
depletion-neutralized baseline cumulative hazard for others. This
behavior is consistent with curvature-driven normalization rather than
cohort identity. This pattern is visible directly in
depletion-neutralized versus observed cumulative hazard plots and is
summarized quantitatively in the fitted-parameter logs (see
\texttt{KCOR\_summary.log}).

After frailty normalization, the depletion-neutralized baseline
cumulative hazards are approximately linear in event time. Residual
deviations from linearity reflect real time-varying risk---such as
seasonality or epidemic waves---rather than selection-induced depletion.
This linearization is a diagnostic consistent with successful removal of
depletion-driven curvature under the working model; persistent
nonlinearity or parameter instability indicates model stress or
quiet-window contamination.

Table C.2 summarizes these diagnostic checks across age strata.

All age strata in the Czech application satisfied the prespecified
diagnostic criteria, permitting KCOR computation and reporting. KCOR
results are not reported for any age stratum where diagnostics indicated
non-identifiability.

\textbf{Interpretation:} Unvaccinated cohorts exhibit frailty
heterogeneity, while Dose 2 cohorts show near-zero estimated frailty
across all age bands, consistent with selective uptake prior to
follow-up:

\[
\hat{\theta}_d > 0.
\qquad\text{(C.3)}
\]

for Dose 0 cohorts and

\[
\hat{\theta}_d \approx 0.
\qquad\text{(C.4)}
\]

for Dose 2 cohorts. Estimated frailty heterogeneity can appear larger at
younger ages because baseline hazards are low, so proportional
differences across latent risk strata translate into visibly different
short-term hazards before depletion compresses the risk distribution. At
older ages, higher baseline hazard and stronger ongoing depletion can
reduce the apparent dispersion of remaining risk, yielding smaller
fitted \(\theta\) even if latent heterogeneity is not literally smaller.
Frailty variance is largest at younger ages, where low baseline
mortality amplifies the impact of heterogeneity on cumulative hazard
curvature, and declines at older ages where mortality is compressed and
survivors are more homogeneous. Because Table C.1 demonstrates
selection-induced heterogeneity, unadjusted cumulative outcome contrasts
are expected to conflate depletion effects with any true treatment
differences; see Table C.3 for raw cumulative hazards reported as a
pre-normalization diagnostic. KCOR normalization removes the depletion
component, enabling interpretable comparison of the remaining
differences.

These raw contrasts reflect both selection and depletion effects and are
not interpreted causally.

\subsubsection{C.6.3 Illustrative application to national registry
mortality
data}\label{c.6.3-illustrative-application-to-national-registry-mortality-data}

We include a brief illustrative application to demonstrate end-to-end
KCOR behavior on real registry mortality data in a setting that
minimizes timing-driven shocks and window-tuning sensitivity. Cohorts
were enrolled in ISO week 2022-26, and evaluation was restricted to
calendar year 2023, yielding a 26-week post-enrollment buffer before
slope estimation and a prespecified full-year window for assessment.
Frailty parameters were estimated using a prespecified epidemiologically
quiet window (calendar year 2023) to minimize wave-related hazard
variation. This example is intended to illustrate estimator behavior
under real-world selection and heterogeneity and does not support causal
inference.

Figure C.4 shows \(\mathrm{KCOR}(t)\) trajectories for dose 2 and dose 3
relative to dose 0 for an all-ages analysis. We deliberately present an
all-ages analysis as a high-heterogeneity stress test, since aggregation
across age induces substantial baseline hazard and frailty variation. To
assess whether apparent stability could arise from cancellation across
strata, we also present narrow birth-year cohorts spanning advanced ages
(1930, 1940, 1950) in Figures C.1--C.3. Across aggregation levels,
\(\mathrm{KCOR}(t)\) remains stable over the evaluation window after
depletion normalization, consistent with effective removal of
selection-induced curvature in a real-data setting. These figures are
presented as illustrative applications demonstrating estimator behavior
on registry data and do not support causal inference; no hypothesis
testing is performed.

\begin{figure}
\centering
\pandocbounded{\includegraphics[keepaspectratio,alt={Figure C.4. All-ages stress test: \textbackslash mathrm\{KCOR\}(t) trajectories comparing dose 2 and dose 3 to dose 0 for cohorts enrolled in ISO week 2022-26 and evaluated over calendar year 2023. KCOR curves are anchored at t\_0 = 4 weeks (i.e., plotted as \textbackslash mathrm\{KCOR\}(t; t\_0)). This figure is presented as an illustrative application demonstrating estimator behavior under extreme heterogeneity and does not support causal inference.}]{figures/kcor_realdata_allages_enroll2022w26_eval2023.png}}
\caption{Figure C.4. All-ages stress test: \(\mathrm{KCOR}(t)\)
trajectories comparing dose 2 and dose 3 to dose 0 for cohorts enrolled
in ISO week 2022-26 and evaluated over calendar year 2023. KCOR curves
are anchored at \(t_0 = 4\) weeks (i.e., plotted as
\(\mathrm{KCOR}(t; t_0)\)). This figure is presented as an illustrative
application demonstrating estimator behavior under extreme heterogeneity
and does not support causal inference.}
\end{figure}

\subsection{Appendix D: Diagnostics and Failure Modes for KCOR
Assumptions}\label{appendix-d-diagnostics-and-failure-modes-for-kcor-assumptions}

This appendix describes the \textbf{observable diagnostics and failure
modes} associated with each of the five KCOR assumptions (A1--A5). No
additional assumptions are introduced here. KCOR is designed to
\textbf{fail transparently rather than silently}: when an assumption is
violated, the resulting lack of identifiability or model stress
manifests through explicit diagnostic signals rather than spurious
estimates.

A compact summary mapping each assumption to its corresponding
diagnostic signals and recommended actions is provided in Table D.1.

\subsubsection{D.1 Diagnostics for Assumption A1 (Fixed cohorts at
enrollment)}\label{d.1-diagnostics-for-assumption-a1-fixed-cohorts-at-enrollment}

\textbf{Assumption A1} requires that cohorts be fixed at enrollment,
with no post-enrollment switching or censoring in the primary estimand.

\textbf{Diagnostic signals of violation.}

\begin{itemize}
\tightlist
\item
  Inconsistencies in cohort risk sets (e.g., unexplained increases in
  at-risk counts).
\item
  Early-time hazard suppression or inflation inconsistent with selection
  or depletion geometry.
\item
  Dependence of results on as-treated reclassification or censoring
  rules.
\end{itemize}

\textbf{Interpretation.} KCOR is not defined for datasets with
post-enrollment switching or informative censoring in the primary
estimand. Such violations are design-level failures rather than modeling
failures and indicate that KCOR should not be applied without redefining
cohorts.

\subsubsection{D.2 Diagnostics for Assumption A2 (Shared external hazard
environment)}\label{d.2-diagnostics-for-assumption-a2-shared-external-hazard-environment}

\textbf{Assumption A2} requires that all cohorts experience the same
calendar-time external mortality environment.

\textbf{Diagnostic signals of violation.}

\begin{itemize}
\tightlist
\item
  Calendar-time hazard spikes or drops that appear in only one cohort.
\item
  Misalignment of major mortality shocks (e.g., epidemic waves) across
  cohorts.
\item
  Cohort-specific reporting artifacts or administrative discontinuities.
\end{itemize}

\textbf{Interpretation.} External shocks are permitted under KCOR
provided they act symmetrically across cohorts. Cohort-specific shocks
violate comparability and are visible directly in calendar-time hazard
overlays. When detected, such violations limit interpretation of KCOR
contrasts over affected periods.

\subsubsection{D.3 Diagnostics for Assumption A3 (Selection via
time-invariant latent
frailty)}\label{d.3-diagnostics-for-assumption-a3-selection-via-time-invariant-latent-frailty}

\textbf{Assumption A3} posits that selection at enrollment operates
primarily through differences in a time-invariant latent frailty
distribution that induces depletion of susceptibles.

\textbf{Diagnostic signals of violation.}

\begin{itemize}
\tightlist
\item
  Strongly structured residuals in cumulative-hazard space inconsistent
  with depletion.
\item
  Instability of fitted frailty parameters not attributable to window
  placement.
\item
  Early-time transients that do not decay and are inconsistent across
  related cohorts.
\end{itemize}

\textbf{Interpretation.} Frailty in KCOR is a geometric construct
capturing unobserved heterogeneity, not a causal mechanism. If dominant
time-varying individual risk unrelated to depletion is present,
curvature attributed to frailty becomes unstable. Such cases are
revealed by residual structure and parameter instability rather than
masked by the model.

\subsubsection{D.4 Diagnostics for Assumption A4 (Adequacy of gamma
frailty
approximation)}\label{d.4-diagnostics-for-assumption-a4-adequacy-of-gamma-frailty-approximation}

\textbf{Assumption A4} requires that gamma frailty provides an adequate
approximation to the depletion geometry observed in cumulative-hazard
space over the estimation window.

\textbf{Diagnostic signals of violation.}

\begin{itemize}
\tightlist
\item
  Poor fit of the gamma-frailty cumulative-hazard model during the quiet
  window.
\item
  Systematic residual curvature after frailty normalization.
\item
  Strong sensitivity of results to minor model or window perturbations.
\end{itemize}

Additional internal diagnostics for Assumption A4 include the magnitude,
coherence, and stability of the fitted frailty variance parameter
(\(\theta\)). Values of \(\theta\) approaching zero are expected when
cumulative hazards are approximately linear, while larger values
correspond to visible depletion-induced curvature. Implausible
\(\theta\) estimates---such as large values in the absence of curvature,
sign instability, or extreme sensitivity to small changes in the
estimation window---indicate model stress or misspecification rather
than substantive cohort effects.

\textbf{Interpretation.} Gamma frailty is used as a mathematically
tractable approximation, not as a claim of biological truth. When
depletion geometry deviates substantially from the gamma form, KCOR
normalization can degrade visibly through poor fit and residual
curvature. Such behavior indicates model inadequacy rather than
supporting alternative interpretation.

\subsubsection{D.5 Diagnostics for Assumption A5 (Quiet-window
validity)}\label{d.5-diagnostics-for-assumption-a5-quiet-window-validity}

\textbf{Assumption A5} requires the existence of a prespecified quiet
window in which selection-induced depletion dominates other sources of
curvature, permitting identification of frailty parameters.

\textbf{Diagnostic signals of violation.}

\begin{itemize}
\tightlist
\item
  Failure of KCOR(t) trajectories to stabilize or asymptote following
  frailty normalization.
\item
  Persistent nonzero slope in KCOR(t), indicating residual curvature
  after normalization.
\item
  Instability of fitted frailty parameters (\(\theta\)) under small
  perturbations of quiet-window boundaries.
\item
  Failure of depletion-neutralized cumulative hazards to become
  approximately linear during the quiet window.
\item
  Degraded cumulative-hazard fit error concentrated within the nominal
  quiet period.
\end{itemize}

\textbf{Interpretation.} Quiet-window validity is the primary
dataset-specific requirement for KCOR applicability. When this
assumption is violated---e.g., due to overlap with strong treatment
effects or external shocks---KCOR does not amplify spurious signals.
Instead, normalization becomes unstable and KCOR(t) trajectories
attenuate toward unity or may fail to stabilize, explicitly signaling
loss of identifiability.

Under a valid quiet window, depletion-neutralized baseline cumulative
hazards are expected to be approximately linear and \(\mathrm{KCOR}(t)\)
trajectories to stabilize rather than drift. Persistent
\(\mathrm{KCOR}(t)\) slope or \(\hat{\theta}_d\) instability indicates
contamination of the quiet window by external shocks or time-varying
effects and signals loss of identifiability rather than evidence of
cohort differences.

The diagnostics above are designed to detect quiet-window violations
that induce residual curvature or parameter instability. They do not, by
themselves, exclude the possibility of smooth, approximately stationary
cohort-differential hazards within the quiet window that may be absorbed
into fitted frailty parameters without producing obvious drift. For this
reason, when feasible, we additionally recommend split-window and
multi-window stability checks, in which frailty parameters and
post-normalization linearity are assessed for consistency across
sub-windows. Failure of such stability checks is treated as evidence
against Assumption A5.

\subsubsection{D.6 Diagnostic coherence across
assumptions}\label{d.6-diagnostic-coherence-across-assumptions}

Several diagnostics operate across assumptions A4 and A5, including
stabilization of KCOR(t) trajectories and coherence of fitted \(\theta\)
parameters with observed cumulative-hazard curvature. These diagnostics
are not assumptions of the KCOR framework; rather, they are observable
consequences of successful frailty normalization. When these behaviors
fail to emerge, KCOR explicitly signals reduced interpretability through
residual curvature, parameter instability, or attenuation toward unity.

\subsubsection{D.7 Identifiability under sparse
data}\label{d.7-identifiability-under-sparse-data}

KCOR does not require large sample sizes by assumption; however,
reliable estimation of frailty parameters and depletion-neutralized
cumulative hazards requires sufficient event information within the
identification window. When event counts are very small, frailty
estimates may become unstable, resulting in noisy normalization,
non-linear baseline cumulative hazards, or drifting KCOR(t)
trajectories.

Such failures are diagnosable: sparse-data regimes are characterized by
instability of estimated frailty parameters under small perturbations of
the quiet window, loss of post-normalization linearity, and
non-stabilizing KCOR(t). In these cases, KCOR signals loss of
identifiability rather than producing spurious effects. Applicability
should therefore be assessed via diagnostic stability rather than
nominal sample size thresholds.

\subsubsection{D.8 Summary: Diagnostic enforcement rather than
assumption
inflation}\label{d.8-summary-diagnostic-enforcement-rather-than-assumption-inflation}

KCOR relies on exactly five assumptions (A1--A5), stated exhaustively in
§2.1.1. This appendix demonstrates that each assumption has
\textbf{explicit, observable diagnostics} and \textbf{well-defined
failure modes}. When assumptions are violated, KCOR signals reduced
interpretability through instability, poor fit, or residual structure
rather than producing misleading cumulative contrasts. This diagnostic
enforcement is a core design feature of the KCOR framework.

\subsection{Appendix E: Reference Implementation and Default
Settings}\label{appendix-e-reference-implementation-and-default-settings}

This appendix documents the reference implementation and default
operational settings used for all analyses in this manuscript. These
defaults are not intrinsic to the KCOR framework; they represent one
prespecified operationalization chosen to ensure reproducibility and
internal consistency. Alternative reasonable choices yield qualitatively
similar conclusions when applied to depletion-neutralized hazards.

The manuscript describes KCOR generically; for reproducibility, this
repository's KCOR defaults are:

\subsubsection{E.1 Cohort construction
defaults}\label{e.1-cohort-construction-defaults}

\begin{itemize}
\tightlist
\item
  \textbf{Cohort indexing (implementation)}: enrollment period (sheet) ×
  YearOfBirth group × Dose, plus an all-ages cohort (YearOfBirth
  \(=-2\)).
\end{itemize}

\subsubsection{E.2 Quiet-period selection
defaults}\label{e.2-quiet-period-selection-defaults}

\begin{itemize}
\tightlist
\item
  \textbf{Quiet window}: ISO weeks \texttt{2023-01} through
  \texttt{2023-52} (inclusive; calendar year 2023).
\end{itemize}

\subsubsection{E.4 Frailty estimation and inversion
defaults}\label{e.4-frailty-estimation-and-inversion-defaults}

\begin{itemize}
\tightlist
\item
  \textbf{Skip weeks}: a fixed prespecified skip,
  \texttt{SKIP\_WEEKS\ =\ DYNAMIC\_HVE\_SKIP\_WEEKS} (see code), applied
  by setting \(h_d^{\mathrm{eff}}(t)=0\) for
  \(t < \mathrm{SKIP\_WEEKS}\).
\item
  \textbf{Fit method}: nonlinear least squares in cumulative-hazard
  space (not MLE), with constraints \(k_d>0\) and \(\theta_d \ge 0\).
\end{itemize}

\bibitem[\citeproctext]{ref-vaupel1979}
\CSLLeftMargin{1. }%
\CSLRightInline{Vaupel JW, Manton KG, Stallard E. The impact of
heterogeneity in individual frailty on the dynamics of mortality.
\emph{Demography}. 1979;16(3):439-454.
doi:\href{https://doi.org/10.2307/2061224}{10.2307/2061224}}

\bibitem[\citeproctext]{ref-obel2024}
\CSLLeftMargin{2. }%
\CSLRightInline{Obel N, Fox M, Tetens M, et al. Confounding and
{Negative Control Methods} in {Observational Study} of {SARS-CoV-2
Vaccine Effectiveness}: {A Nationwide}, {Population-Based Danish Health
Registry Study}. \emph{Clinical Epidemiology}. 2024;Volume 16:501-512.
doi:\href{https://doi.org/10.2147/CLEP.S468572}{10.2147/CLEP.S468572}}

\bibitem[\citeproctext]{ref-chemaitelly2025}
\CSLLeftMargin{3. }%
\CSLRightInline{Chemaitelly H, Ayoub HH, Coyle P, et al. Assessing
healthy vaccinee effect in {COVID-19} vaccine effectiveness studies: A
national cohort study in {Qatar}. Schiffer JT, Henry D, eds.
\emph{eLife}. 2025;14:e103690.
doi:\href{https://doi.org/10.7554/eLife.103690}{10.7554/eLife.103690}}

\bibitem[\citeproctext]{ref-grambsch1994}
\CSLLeftMargin{4. }%
\CSLRightInline{Grambsch PM, Therneau TM. Proportional hazards tests and
diagnostics based on weighted residuals. \emph{Biometrika}.
1994;81(3):515-526.
doi:\href{https://doi.org/10.1093/biomet/81.3.515}{10.1093/biomet/81.3.515}}

\bibitem[\citeproctext]{ref-andersen1982}
\CSLLeftMargin{5. }%
\CSLRightInline{Andersen PK, Gill RD. Cox's {Regression Model} for
{Counting Processes}: {A Large Sample Study}. \emph{The Annals of
Statistics}. 1982;10(4).
doi:\href{https://doi.org/10.1214/aos/1176345976}{10.1214/aos/1176345976}}

\bibitem[\citeproctext]{ref-royston2002}
\CSLLeftMargin{6. }%
\CSLRightInline{Royston P, Parmar MKB. Flexible parametric
proportional-hazards and proportional-odds models for censored survival
data, with application to prognostic modelling and estimation of
treatment effects. \emph{Statistics in Medicine}. 2002;21(15):2175-2197.
doi:\href{https://doi.org/10.1002/sim.1203}{10.1002/sim.1203}}

\bibitem[\citeproctext]{ref-aalen1989}
\CSLLeftMargin{7. }%
\CSLRightInline{Aalen OO. A linear regression model for the analysis of
life times. \emph{Statistics in Medicine}. 1989;8(8):907-925.
doi:\href{https://doi.org/10.1002/sim.4780080803}{10.1002/sim.4780080803}}

\bibitem[\citeproctext]{ref-lin1994}
\CSLLeftMargin{8. }%
\CSLRightInline{Lin DY, Ying Z. Semiparametric analysis of the additive
risk model. \emph{Biometrika}. 1994;81(1):61-71.
doi:\href{https://doi.org/10.1093/biomet/81.1.61}{10.1093/biomet/81.1.61}}

\bibitem[\citeproctext]{ref-vanhouwelingen2007}
\CSLLeftMargin{9. }%
\CSLRightInline{Van Houwelingen HC. Dynamic {Prediction} by
{Landmarking} in {Event History Analysis}. \emph{Scandinavian Journal of
Statistics}. 2007;34(1):70-85.
doi:\href{https://doi.org/10.1111/j.1467-9469.2006.00529.x}{10.1111/j.1467-9469.2006.00529.x}}

\bibitem[\citeproctext]{ref-robins2000}
\CSLLeftMargin{10. }%
\CSLRightInline{Robins JM, Hernán MÁ, Brumback B. Marginal {Structural
Models} and {Causal Inference} in {Epidemiology}: \emph{Epidemiology}.
2000;11(5):550-560.
doi:\href{https://doi.org/10.1097/00001648-200009000-00011}{10.1097/00001648-200009000-00011}}

\bibitem[\citeproctext]{ref-cole2008}
\CSLLeftMargin{11. }%
\CSLRightInline{Cole SR, Hernan MA. Constructing {Inverse Probability
Weights} for {Marginal Structural Models}. \emph{American Journal of
Epidemiology}. 2008;168(6):656-664.
doi:\href{https://doi.org/10.1093/aje/kwn164}{10.1093/aje/kwn164}}

\bibitem[\citeproctext]{ref-sanca2024}
\CSLLeftMargin{12. }%
\CSLRightInline{Šanca O, Jarkovský J, Klimeš D, et al. Vaccination,
positivity, hospitalization for {COVID-19}, deaths, long covid and
comorbidities in people in the {Czech Republic}. \emph{National Health
Information Portal}. Published online 2024.}

\end{CSLReferences}

\end{document}
