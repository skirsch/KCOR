% Options for packages loaded elsewhere
\PassOptionsToPackage{unicode}{hyperref}
\PassOptionsToPackage{hyphens}{url}
\documentclass[
]{article}
\usepackage{xcolor}
\usepackage[margin=1in]{geometry}
\usepackage{amsmath,amssymb}
\setcounter{secnumdepth}{-\maxdimen} % remove section numbering
\usepackage{iftex}
\ifPDFTeX
  \usepackage[T1]{fontenc}
  \usepackage[utf8]{inputenc}
  \usepackage{textcomp} % provide euro and other symbols
\else % if luatex or xetex
  \usepackage{unicode-math} % this also loads fontspec
  \defaultfontfeatures{Scale=MatchLowercase}
  \defaultfontfeatures[\rmfamily]{Ligatures=TeX,Scale=1}
\fi
\usepackage{lmodern}
\ifPDFTeX\else
  % xetex/luatex font selection
  \setmainfont[]{TeX Gyre Termes}
  \setmathfont[]{TeX Gyre Termes Math}
\fi
% Use upquote if available, for straight quotes in verbatim environments
\IfFileExists{upquote.sty}{\usepackage{upquote}}{}
\IfFileExists{microtype.sty}{% use microtype if available
  \usepackage[]{microtype}
  \UseMicrotypeSet[protrusion]{basicmath} % disable protrusion for tt fonts
}{}
\makeatletter
\@ifundefined{KOMAClassName}{% if non-KOMA class
  \IfFileExists{parskip.sty}{%
    \usepackage{parskip}
  }{% else
    \setlength{\parindent}{0pt}
    \setlength{\parskip}{6pt plus 2pt minus 1pt}}
}{% if KOMA class
  \KOMAoptions{parskip=half}}
\makeatother
\usepackage{longtable,booktabs,array}
\newcounter{none} % for unnumbered tables
\usepackage{calc} % for calculating minipage widths
% Correct order of tables after \paragraph or \subparagraph
\usepackage{etoolbox}
\makeatletter
\patchcmd\longtable{\par}{\if@noskipsec\mbox{}\fi\par}{}{}
\makeatother
% Allow footnotes in longtable head/foot
\IfFileExists{footnotehyper.sty}{\usepackage{footnotehyper}}{\usepackage{footnote}}
\makesavenoteenv{longtable}
\usepackage{graphicx}
\makeatletter
\newsavebox\pandoc@box
\newcommand*\pandocbounded[1]{% scales image to fit in text height/width
  \sbox\pandoc@box{#1}%
  \Gscale@div\@tempa{\textheight}{\dimexpr\ht\pandoc@box+\dp\pandoc@box\relax}%
  \Gscale@div\@tempb{\linewidth}{\wd\pandoc@box}%
  \ifdim\@tempb\p@<\@tempa\p@\let\@tempa\@tempb\fi% select the smaller of both
  \ifdim\@tempa\p@<\p@\scalebox{\@tempa}{\usebox\pandoc@box}%
  \else\usebox{\pandoc@box}%
  \fi%
}
% Set default figure placement to htbp
\def\fps@figure{htbp}
\makeatother
% definitions for citeproc citations
\NewDocumentCommand\citeproctext{}{}
\NewDocumentCommand\citeproc{mm}{%
  \begingroup\def\citeproctext{#2}\cite{#1}\endgroup}
\makeatletter
 % allow citations to break across lines
 \let\@cite@ofmt\@firstofone
 % avoid brackets around text for \cite:
 \def\@biblabel#1{}
 \def\@cite#1#2{{#1\if@tempswa , #2\fi}}
\makeatother
\newlength{\cslhangindent}
\setlength{\cslhangindent}{1.5em}
\newlength{\csllabelwidth}
\setlength{\csllabelwidth}{3em}
\newenvironment{CSLReferences}[2] % #1 hanging-indent, #2 entry-spacing
 {\begin{list}{}{%
  \setlength{\itemindent}{0pt}
  \setlength{\leftmargin}{0pt}
  \setlength{\parsep}{0pt}
  % turn on hanging indent if param 1 is 1
  \ifodd #1
   \setlength{\leftmargin}{\cslhangindent}
   \setlength{\itemindent}{-1\cslhangindent}
  \fi
  % set entry spacing
  \setlength{\itemsep}{#2\baselineskip}}}
 {\end{list}}
\usepackage{calc}
\newcommand{\CSLBlock}[1]{\hfill\break\parbox[t]{\linewidth}{\strut\ignorespaces#1\strut}}
\newcommand{\CSLLeftMargin}[1]{\parbox[t]{\csllabelwidth}{\strut#1\strut}}
\newcommand{\CSLRightInline}[1]{\parbox[t]{\linewidth - \csllabelwidth}{\strut#1\strut}}
\newcommand{\CSLIndent}[1]{\hspace{\cslhangindent}#1}
\setlength{\emergencystretch}{3em} % prevent overfull lines
\providecommand{\tightlist}{%
  \setlength{\itemsep}{0pt}\setlength{\parskip}{0pt}}
% Fix run-in paragraph style for ##### headers
\usepackage{titlesec}
\titleformat{\paragraph}[block]{\normalfont\normalsize\bfseries}{\theparagraph}{1em}{}
\titlespacing*{\paragraph}{0pt}{3.25ex plus 1ex minus .2ex}{0.5em}
\titleformat{\subparagraph}[block]{\normalfont\normalsize\bfseries}{\thesubparagraph}{1em}{}
\titlespacing*{\subparagraph}{0pt}{3.25ex plus 1ex minus .2ex}{0.5em}

% Table packages for tabularx with custom column types
\usepackage{microtype}
\frenchspacing
\usepackage{booktabs}
\usepackage{tabularx}
\usepackage{array}
\usepackage{xcolor}
\usepackage[framemethod=tikz]{mdframed}
\newcolumntype{Y}{>{\raggedright\arraybackslash}X}

% Box all Markdown blockquotes (Pandoc uses the quote environment)
% Use mdframed for stable full borders with no left/right clipping.
\mdfdefinestyle{kcorquote}{%
  linecolor=black,
  linewidth=0.4pt,
  backgroundcolor=white,
  innerleftmargin=6pt,
  innerrightmargin=6pt,
  innertopmargin=6pt,
  innerbottommargin=6pt,
  leftmargin=0pt,
  rightmargin=0pt,
  skipabove=10pt,
  skipbelow=10pt,
  topline=true,
  bottomline=true,
  leftline=true,
  rightline=true,
  roundcorner=0pt,
  splittopskip=0pt,
  splitbottomskip=0pt
}
\renewenvironment{quote}{\begin{mdframed}[style=kcorquote]}{\end{mdframed}}

% Appendix lettered numbering for figures and equations.
%
% In this repo, the manuscript uses "##" headings, which Pandoc maps to \subsection.
% There are no \section-level headings, so relying on subsection/section counters
% yields 0.x-style numbering. Instead, we define an appendix-letter counter that
% advances once per appendix (\subsection after \appendix) and resets figure/equation
% numbering within each appendix.
\usepackage{etoolbox}
\newif\ifkcorappendix
\newcounter{kcorappendix}
\pretocmd{\appendix}{%
  \kcorappendixtrue%
  \setcounter{kcorappendix}{0}%
  \renewcommand{\thefigure}{\Alph{kcorappendix}.\arabic{figure}}%
  \renewcommand{\theequation}{\Alph{kcorappendix}.\arabic{equation}}%
  \renewcommand{\thetable}{\Alph{kcorappendix}.\arabic{table}}%
}{}{}
\pretocmd{\subsection}{%
  \ifkcorappendix%
    \stepcounter{kcorappendix}%
    \setcounter{figure}{0}%
    \setcounter{equation}{0}%
    \setcounter{table}{0}%
  \fi%
}{}{}

% Disable TeX hyphenation (reviewer-friendly) and avoid PDF soft-hyphen artifacts.
% Note: this does not prevent manual hyphens, only automatic word-breaking.
\usepackage{ragged2e}
\RaggedRight
\hyphenpenalty=10000
% Allow line breaks at *explicit* hyphens that already exist in the text.
% (Keep automatic hyphenation disabled via \hyphenpenalty above.)
\exhyphenpenalty=50
\pretolerance=10000
\tolerance=2000
\emergencystretch=3em

% Suppress XeLaTeX "Missing character" warnings for Unicode math symbols.
% When math like $\theta$ appears in table cells, XeLaTeX checks if the text font
% has the Unicode math character (U+1D703 = mathematical italic theta), generating
% a warning even though the math font handles rendering correctly.
% \tracinglostchars=0 is the TeX primitive that controls this warning level:
%   0 = no warnings, 1 = warnings in log only, 2 = warnings + terminal (default)
\tracinglostchars=0

\usepackage{bookmark}
\IfFileExists{xurl.sty}{\usepackage{xurl}}{} % add URL line breaks if available
\urlstyle{same}
\hypersetup{
  hidelinks,
  pdfcreator={LaTeX via pandoc}}

\author{}
\date{}

\begin{document}

\section{KCOR: A depletion-neutralized framework for retrospective
cohort comparison under latent
frailty}\label{kcor-a-depletion-neutralized-framework-for-retrospective-cohort-comparison-under-latent-frailty}

\subsection{Manuscript metadata}\label{manuscript-metadata}

\begin{itemize}
\tightlist
\item
  \textbf{Article type}: Methods / Statistical method
\item
  \textbf{Running title}: KCOR under selection-induced cohort bias
\item
  \textbf{Author}: Steven T. Kirsch
\item
  \textbf{Affiliations}: Independent Researcher, United States
\item
  \textbf{Corresponding author}: stk@alum.mit.edu
\item
  \textbf{Word count}: 10,909 (excluding Abstract and References)
\item
  \textbf{Keywords}: selection bias; frailty model; gamma mixture model;
  frailty inversion; frailty heterogeneity; selection-induced depletion;
  non-proportional hazards; cumulative hazard; hazard normalization;
  cumulative hazards; estimands; gamma frailty; negative controls;
  observational studies; observational cohort studies
\end{itemize}

\subsection{Abstract}\label{abstract}

Selection-induced depletion under latent frailty heterogeneity can
generate non-proportional hazards and curvature in observed cumulative
hazards, biasing standard survival estimands in retrospective cohort
studies using registry and administrative data. KCOR is a
depletion-neutralized cohort comparison framework based on gamma-frailty
normalization. It estimates cohort-specific depletion geometry during
prespecified quiet periods and applies an analytic inversion to map
observed cumulative hazards into a common comparison scale prior to
computing cumulative contrasts. Across simulations spanning frailty
heterogeneity and selection strength and across negative and positive
controls, Cox proportional hazards regression can exhibit systematic
non-null behavior under selection-only regimes. In contrast,
KCOR-normalized trajectories remain stable and centered near the null
while detecting injected effects. KCOR provides a diagnostic and
descriptive framework for comparing fixed cohorts under
selection-induced hazard curvature by separating depletion normalization
from outcome comparison and improving interpretability of cumulative
outcome analyses under minimal-data constraints.

\subsection{1. Introduction}\label{introduction}

\subsubsection{1.1 Retrospective cohort comparisons under
selection}\label{retrospective-cohort-comparisons-under-selection}

Randomized controlled trials (RCTs) are the gold standard for causal
inference, but are often infeasible, underpowered for rare outcomes, or
unavailable for questions that arise after rollout. As a result,
observational cohort comparisons are widely used to estimate
intervention effects on outcomes such as all-cause mortality.

Although mortality is used throughout this paper as a motivating and
concrete example, the method applies more generally to any irreversible
event process observed in a fixed cohort, including hospitalization,
disease onset, or other terminal or absorbing states. Mortality is
emphasized here because it is objectively defined, reliably recorded in
many national datasets, and free from outcome-dependent ascertainment
biases that complicate other endpoints.

However, when intervention uptake is voluntary, prioritized, or
otherwise selective, treated and untreated cohorts are frequently
\textbf{non-exchangeable} at baseline and evolve differently over
follow-up. This problem is not limited to any single intervention class;
it arises whenever the same factors that influence treatment uptake also
influence outcome risk.

This manuscript is a methods paper. Real-world registry data are used
solely to demonstrate estimator behavior, diagnostics, and failure modes
under realistic selection-induced non-proportional hazards; no policy
conclusions are drawn.

\subsubsection{1.2 Curvature (shape) is the hard part: non-proportional
hazards from frailty
depletion}\label{curvature-shape-is-the-hard-part-non-proportional-hazards-from-frailty-depletion}

Selection does not merely shift mortality \textbf{levels}; it can alter
mortality \textbf{curvature}---the time-evolution of cohort hazards.
Frailty heterogeneity and selection-induced depletion (depletion of
susceptibles) naturally induce curvature of the cumulative hazard
(reflecting time-varying hazard) even when individual-level hazards are
simple functions of time. When selection concentrates high-frailty
individuals into one cohort (or preferentially removes them from
another), the resulting cohort-level hazard trajectories can be strongly
non-proportional.

This violates core assumptions of many standard tools:

\begin{itemize}
\tightlist
\item
  \textbf{Cox PH}: assumes hazards differ by a time-invariant
  multiplicative factor (proportional hazards).
\item
  \textbf{IPTW / matching}: can balance measured covariates yet fail to
  balance unmeasured frailty and the resulting depletion dynamics.
\item
  \textbf{Age-standardization}: adjusts levels across age strata but
  does not remove cohort-specific time-evolving hazard shape.
\end{itemize}

KCOR is designed for this failure mode: \textbf{cohorts whose hazards
are not proportional because selection induces different depletion
dynamics (curvature).} Approximate linearity of cumulative hazard after
adjustment is therefore not assumed, but serves as an internal
diagnostic indicating that selection-induced depletion has been
successfully removed.

The methodological problem addressed here is general. The COVID-19
period provides a natural empirical regime characterized by strong
selection heterogeneity and non-proportional hazards, serving as a
useful illustration for the proposed framework. However, KCOR is not
specific to COVID, vaccination, or infectious disease. KCOR refers to
the method as presented here; earlier internal iterations are not
material to the estimand or results and are omitted for clarity.

Two mechanisms often lumped as the `healthy vaccinee effect' (HVE) are
distinguished here:

\begin{itemize}
\item
  \textbf{Static HVE:} baseline differences in latent frailty
  distributions at cohort entry (e.g., vaccinated cohorts are healthier
  on average). In the KCOR framework, this manifests as differing
  depletion curvature (different \(\theta_d\)) and is the primary target
  of frailty normalization.
\item
  \textbf{Dynamic HVE:} short-horizon, time-local selection processes
  around enrollment that create transient hazard suppression immediately
  after enrollment (e.g., deferral of vaccination during acute illness,
  administrative timing, or short-term behavioral/health-seeking
  changes). Dynamic HVE is operationally addressed by prespecifying a
  skip/stabilization window (§2.7) and can be evaluated empirically by
  comparing early-period signatures across related cohorts in multi-dose
  settings.
\end{itemize}

\begin{quote}
\textbf{Box 1. Two fundamentally different strategies for cohort
comparability}

\begin{itemize}
\item
  \textbf{Traditional matching and regression approaches:} attempt to
  construct comparable cohorts by matching or adjusting
  \emph{characteristics of living individuals} at baseline or over
  follow-up, and then estimating effects via a fitted hazard model
  (e.g., Cox proportional hazards). This implicitly assumes that
  sufficiently rich covariate information can render cohorts
  exchangeable with respect to unobserved mortality risk.
\item
  \textbf{Problem under latent frailty:} even meticulous 1:1 matching on
  observed covariates can fail to equalize mortality risk trajectories.
  In such settings, cohort differences arise not from mismeasured
  covariates, but from \textbf{selection-induced depletion of
  susceptibles}, which alters hazard curvature over time.
\item
  \textbf{KCOR strategy:} rather than equating cohorts based on
  characteristics of the living, KCOR equates cohorts based on how they
  die in aggregate. It estimates cohort-specific depletion geometry from
  observed cumulative mortality during epidemiologically quiet periods,
  removes that geometry via analytic inversion, and compares cohorts on
  the resulting depletion-neutralized cumulative hazard scale.
\item
  \textbf{Inferential distinction:} Cox-type methods are
  \textbf{model-based and individual-level}, conditioning on survival
  and fitting covariate effects, whereas KCOR is
  \textbf{measurement-based and cohort-level}, operating directly on
  aggregated mortality trajectories without fitting covariate models.
  The inferential target is cumulative outcome accumulation rather than
  an instantaneous hazard ratio conditional on survival.
\end{itemize}
\end{quote}

\subsubsection{1.3 Related work (brief
positioning)}\label{related-work-brief-positioning}

KCOR builds on the frailty and selection-induced depletion literature in
which unobserved heterogeneity induces deceleration of cohort-level
hazards over follow-up (a standard working model is gamma
frailty).\textsuperscript{1} KCOR's distinct contribution is not
additional hazard flexibility, but a \textbf{diagnostics-driven
normalization} of selection-induced depletion geometry in
cumulative-hazard space prior to defining a cumulative cohort contrast.
Related approaches that address non-proportional hazards (time-varying
effects, flexible parametric hazards, additive hazards) or time-varying
confounding (MSM/IPW/g-methods) target different estimands and typically
require richer longitudinal covariates than are available in minimal
registry data.\textsuperscript{2--9} Additional discussion is provided
in the Supplementary Information (SI).

Time-varying coefficient Cox models allow hazards to change over time
but do not neutralize frailty-induced depletion because estimation
remains conditional on survival; they therefore address
non-proportionality without removing selection-induced curvature.

Marginal structural models target causal effects under exchangeability
using longitudinal covariates and weighting; KCOR instead targets
descriptive cumulative contrasts under minimal data, so the estimands,
assumptions, and failure modes differ.

Flexible parametric survival models improve baseline fit but do not
resolve depletion-induced selection bias when frailty heterogeneity is
present.

\subsubsection{1.4 Evidence from the literature: residual confounding
despite meticulous
matching}\label{evidence-from-the-literature-residual-confounding-despite-meticulous-matching}

Motivating applied studies show that even careful matching and
adjustment can leave substantial residual differences in non-COVID
mortality and time-varying ``healthy vaccinee effect'' signatures,
consistent with selection and depletion dynamics not captured by
measured covariates.\textsuperscript{10,11}

\subsubsection{1.5 Contribution of this
work}\label{contribution-of-this-work}

This work makes four primary contributions: (i) it formalizes
selection-induced depletion under latent frailty heterogeneity as a
source of non-proportional hazards and curvature that can bias common
survival estimands; (ii) it defines a diagnostics-first normalization
that fits depletion geometry in quiet periods and maps observed
cumulative hazards into a depletion-neutralized space; (iii) it
validates operating characteristics using synthetic and empirical
controls, including a synthetic null under selection-only regimes; and
(iv) it separates normalization from comparison by permitting standard
post-normalization cumulative estimands.

A central implication is identifiability: in minimal-data retrospective
cohorts, interpretability depends on an epidemiologically quiet window
and on internal diagnostics that indicate depletion geometry has been
estimated and removed, rather than absorbed into a time-varying effect
estimate.

Together, these contributions position KCOR not as a replacement for
existing survival estimands, but as a prerequisite normalization step
that addresses a source of bias arising prior to model fitting in many
retrospective cohort studies.

\subsubsection{1.6 Target estimand and scope
(non-causal)}\label{target-estimand-and-scope-non-causal}

\begin{quote}
\textbf{Box 2. Target estimand and scope (non-causal)}

\begin{itemize}
\tightlist
\item
  \textbf{Primary estimand (KCOR)}: For two fixed enrollment cohorts
  \(A\) and \(B\), it is defined as \[
  \mathrm{KCOR}(t)=\tilde H_{0,A}(t)/\tilde H_{0,B}(t),
  \] where \(\tilde H_{0,d}(t)\) is cohort \(d\)'s
  \textbf{depletion-neutralized baseline cumulative hazard} obtained by
  fitting depletion geometry in a prespecified quiet window and applying
  the gamma-frailty inversion (Methods §2).
\item
  \textbf{Operational summary}: KCOR proceeds by (i) estimating
  selection-induced depletion geometry during an epidemiologically quiet
  period, (ii) inverting that geometry to obtain depletion-neutralized
  cumulative hazards, and (iii) comparing cohorts on that normalized
  cumulative scale.
\item
  \textbf{Interpretation}: KCOR is a time-indexed \textbf{cumulative}
  contrast on the depletion-neutralized scale. Values above/below 1
  indicate greater/less cumulative event accumulation in cohort \(A\)
  than \(B\) by time \(t\) after depletion normalization. KCOR is not an
  instantaneous hazard ratio.
\item
  \textbf{What it is not}: KCOR is \textbf{not} a causal effect
  estimator (no ATE/ATT) and does not recover counterfactual outcomes
  under hypothetical interventions.
\item
  \textbf{When interpretable}: Interpretation is conditional on explicit
  assumptions (fixed cohorts; shared external hazard environment;
  adequacy of the working frailty model; existence of an
  epidemiologically quiet window) \textbf{and} on internal diagnostics
  (quiet-window fit quality; post-normalization linearity within the
  quiet window; parameter stability to small window perturbations).
\item
  \textbf{If diagnostics indicate non-identifiability}: the analysis is
  treated as not identified and KCOR is not reported as a ``corrected
  effect''.
\end{itemize}
\end{quote}

\subsubsection{1.7 Paper organization and supporting information
(SI)}\label{paper-organization-and-supporting-information-si}

The main text introduces the KCOR estimator, provides a canonical
demonstration of Cox bias under frailty-driven depletion, and presents
two primary validation examples (a negative control and a stress test).
Additional validations---including positive controls---along with
extended diagnostics, empirical registry-based nulls, and detailed
simulation specifications are provided in the Supplementary Information
(SI; Sections S2 and S4.2--S4.3; Tables
\ref{tbl:si_assumptions}--\ref{tbl:si_identifiability}).

\subsection{2. Methods}\label{methods}

Mortality is used as the primary example throughout this section because
it is objectively defined and reliably recorded in many administrative
datasets.

Table \ref{tbl:notation} defines the notation used throughout the
Methods section.

For COVID-19 vaccination analyses, intervention count corresponds to the
number of vaccine doses received; more generally, this can index any
discrete exposure level.

\subsubsection{2.1 Conceptual framework and
estimand}\label{conceptual-framework-and-estimand}

Retrospective cohort differences can arise from two qualitatively
different components:

\begin{itemize}
\tightlist
\item
  \textbf{Level differences}: cohort hazards differ by an approximately
  time-stable multiplicative factor (or, equivalently, cumulative
  hazards have different slopes but similar shape).
\item
  \textbf{Depletion (curvature) differences}: cohort hazards evolve
  differently over time because cohorts differ in latent heterogeneity
  and are \textbf{selectively depleted} at different rates.
\end{itemize}

This framework targets the second failure mode. Under latent frailty
heterogeneity, high-risk individuals die earlier, so the surviving risk
set becomes progressively ``healthier.'' This induces \textbf{downward
curvature} (deceleration) in cohort hazards and corresponding concavity
in cumulative-hazard space, even when individual-level hazards are
simple and even under a true null treatment effect. When selection
concentrates frailty heterogeneity differently across cohorts, the
resulting curvature differences produce strong non-proportional hazards
and can drive misleading contrasts for estimands that condition on the
evolving risk set.

The strategy is therefore:

\begin{enumerate}
\def\labelenumi{\arabic{enumi}.}
\tightlist
\item
  \textbf{Estimate the cohort-specific depletion geometry} (via
  curvature) during prespecified epidemiologically quiet periods.
\item
  \textbf{Map observed cumulative hazards into a depletion-neutralized
  space} by inverting that geometry.
\item
  \textbf{Compare cohorts only after normalization} using a prespecified
  post-adjustment estimand; ratios of depletion-neutralized cumulative
  hazards (KCOR) are used here.
\end{enumerate}

All analyses are performed using discrete weekly time bins;
continuous-time notation is used solely for expositional convenience.
See Table \ref{tbl:notation} for the full symbol list.

\textbf{Notation preview.}\\
Throughout this paper, \(t\) denotes event time since cohort enrollment
and \(d\) indexes cohorts. Let \(H_{\mathrm{obs},d}(t)\) denote the
observed cohort-level cumulative hazard, computed from fixed risk sets,
and let \(\tilde H_{0,d}(t)\) denote the corresponding
\emph{depletion-neutralized baseline cumulative hazard} obtained after
frailty normalization. Individual hazards are modeled using a latent
multiplicative frailty term \(z\), with cohort-specific variance
\(\theta_d\), which governs the strength of selection-induced depletion
and resulting curvature in observed cumulative hazards. Full notation is
summarized in Table \ref{tbl:notation}.

\paragraph{2.1.1 Target estimand}\label{target-estimand}

Scope and interpretation are summarized in Box 2 (§1.6); the formal
definition used throughout is provided here.

Let \(\tilde H_{0,d}(t)\) denote the \textbf{depletion-neutralized
baseline cumulative hazard} for cohort \(d\) at event time \(t\) since
enrollment (Table \ref{tbl:notation}). For two cohorts \(A\) and \(B\),
KCOR is defined as

\begin{equation}\protect\phantomsection\label{eq:kcor-estimand}{
\mathrm{KCOR}(t) = \frac{\tilde H_{0,A}(t)}{\tilde H_{0,B}(t)}.
}\end{equation}

For visualization, an \textbf{anchored KCOR} is sometimes reported to
show post-reference divergence: \[
\mathrm{KCOR}(t; t_0) = \mathrm{KCOR}(t)/\mathrm{KCOR}(t_0),
\] with prespecified \(t_0\) (e.g., 4 weeks). Anchoring is used only
when explicitly stated to remove pre-existing level differences and
emphasize post-reference divergence.

\paragraph{2.1.2 Identification versus
diagnostics}\label{identification-versus-diagnostics}

Scope and interpretation are summarized in Box 2 (§1.6).

Interpretability of a KCOR trajectory is assessed via prespecified
diagnostics (Supplementary Information §S2; Tables
\ref{tbl:si_assumptions}--\ref{tbl:si_identifiability}). When
diagnostics indicate non-identifiability, the analysis is treated as not
identified and results are not reported. Checks include:

\begin{itemize}
\tightlist
\item
  stability of \((\hat{k}_d,\hat{\theta}_d)\) to small quiet-window
  perturbations,
\item
  approximate linearity of \(\tilde H_{0,d}(t)\) within the quiet
  window,
\item
  absence of systematic residual structure in cumulative-hazard space.
\end{itemize}

Diagnostics corresponding to each assumption are summarized in
Supplementary Information §S2 (Tables
\ref{tbl:si_assumptions}--\ref{tbl:si_identifiability}).

\paragraph{2.1.3 KCOR assumptions and
diagnostics}\label{kcor-assumptions-and-diagnostics}

These assumptions define when KCOR normalization is interpretable.

The KCOR framework relies on the following assumptions, which are framed
diagnostically:

\begin{enumerate}
\def\labelenumi{\arabic{enumi}.}
\item
  \textbf{Fixed cohort enrollment.} Cohorts are defined at a common
  enrollment time and followed forward without dynamic entry or
  rebalancing.
\item
  \textbf{Multiplicative latent frailty.} Individual hazards are assumed
  to be multiplicatively composed of a baseline hazard and an unobserved
  frailty term, with cohort-specific frailty distributions.
\item
  \textbf{Quiet-window stability.} A prespecified epidemiologically
  quiet period exists during which external shocks to the baseline
  hazard are minimal, allowing depletion geometry to be estimated from
  observed cumulative hazards.
\item
  \textbf{Independence across strata.} Cohorts or strata are analyzed
  independently, without interference, spillover, or cross-cohort
  coupling.
\item
  \textbf{Sufficient event-time resolution.} Event timing is observed at
  a temporal resolution adequate to estimate cumulative hazards over the
  quiet window.
\end{enumerate}

If no candidate quiet window satisfies diagnostic criteria, KCOR is not
interpretable and analysis should terminate without reporting contrasts.
Multiple disjoint quiet windows may be pooled provided fitted depletion
parameters are stable and diagnostics are consistent across windows.
Quiet-window validity diagnostics are summarized in Supplementary
Information §S2 (Tables
\ref{tbl:si_assumptions}--\ref{tbl:si_identifiability}).

These assumptions are evaluated empirically using post-normalization
diagnostics. Violations are expected to manifest as residual curvature,
drift, or instability in adjusted cumulative hazard trajectories.

\subsubsection{2.2 Cohort construction}\label{cohort-construction}

KCOR is defined for \textbf{fixed cohorts at enrollment}. Required
inputs are minimal: enrollment date(s), event date, and optionally birth
date (or year-of-birth) for age stratification. Analyses proceed in
discrete event time \(t\) (e.g., weeks) measured since cohort
enrollment.

Cohorts are assigned by intervention state at the start of the
enrollment interval. In the primary estimand:

\begin{itemize}
\tightlist
\item
  \textbf{No post-enrollment switching} is allowed (individuals remain
  in their enrollment cohort),
\item
  \textbf{No censoring} is applied (other than administrative end of
  follow-up),
\item
  analyses are performed on the resulting fixed risk sets.
\end{itemize}

Censoring or reclassification due to cohort transitions (e.g., moving
between exposure groups over time) is not permitted, because such
transitions alter the frailty composition of the cohort in a
time-dependent manner. Allowing transitions would introduce additional,
endogenous selection that changes cohort mortality trajectories in
unpredictable ways, confounding depletion effects that KCOR is designed
to normalize.

This fixed-cohort design is intentional. It avoids immortal-time
artifacts and prevents outcome-driven switching rules from creating
time-dependent selection that is difficult to diagnose under minimal
covariate availability. Extensions that allow switching or censoring are
treated as sensitivity analyses (§5.2) because they change the estimand
and introduce additional identification requirements.

Conceptual requirements of the KCOR framework are distinguished from
operational defaults, which are reported separately for reproducibility
(Supplementary Section S4).

Throughout this manuscript the failure event is \textbf{all-cause
mortality}. KCOR therefore targets cumulative mortality hazards and is
not framed as a cause-specific competing-risks analysis.

\subsubsection{2.3 Hazard estimation and cumulative hazards in discrete
time}\label{hazard-estimation-and-cumulative-hazards-in-discrete-time}

For each cohort \(d\), let \(N_d(0)\) denote the number of individuals
at enrollment. Let \(d_d(t)\) denote deaths occurring during interval
\(t\), and let \[
D_d(t) = \sum_{s \le t} d_d(s)
\] denote cumulative deaths up to the end of interval \(t\).

Define the risk set size at the start of interval \(t\) as \[
N_d(t) = N_d(0) - \sum_{s < t} d_d(s) = N_d(0) - D_d(t-1).
\] In the primary estimand, individuals do not switch cohorts after
enrollment and there is no loss to follow-up; therefore \(N_d(t)\) is
the risk set used to define all discrete-time hazards and cumulative
hazards in this manuscript.

Define the interval mortality ratio \[
\mathrm{MR}_{d,t} = \frac{d_d(t)}{N_d(t)}.
\]

The discrete-time cohort hazard is computed as

\begin{equation}\protect\phantomsection\label{eq:hazard-discrete}{
h_{\mathrm{obs},d}(t) = -\ln\!\left(1 - \mathrm{MR}_{d,t}\right) = -\ln\!\left(1 - \frac{d_d(t)}{N_d(t)}\right).
}\end{equation}

This transform is standard: it maps an interval event probability into a
continuous-time equivalent hazard under a piecewise-constant hazard
assumption. For rare events,
\(h_{\mathrm{obs},d}(t) \approx \mathrm{MR}_{d,t} = d_d(t)/N_d(t)\), but
the log form remains accurate and stable when weekly risks are not
negligible. The exact transform
\(h_{\mathrm{obs},d}(t) = -\log(1 - \mathrm{MR}_{d,t})\) is used
throughout; the rare-event approximation is provided for intuition only.

\emph{All hazard and cumulative-hazard quantities used in KCOR are
discrete-time integrated hazard estimators derived from fixed-cohort
risk sets; likelihood-based or partial-likelihood formulations are not
used for estimation or for the subsequent frailty-based normalization.}

Observed cumulative hazards are accumulated over event time after an
optional stabilization skip (§2.7):

\begin{equation}\protect\phantomsection\label{eq:cumhazard-observed}{
H_{\mathrm{obs},d}(t) = \sum_{s \le t} h_d^{\mathrm{eff}}(s),
\qquad \Delta t = 1.
}\end{equation}

Discrete binning accommodates tied events and aggregated registry
releases. Bin width is chosen based on diagnostic stability (e.g.,
smoothness and sufficient counts per bin) rather than temporal
resolution alone.

In addition to the primary implementation above,
\(\hat H_{\mathrm{obs},d}(t)\) was computed using the Nelson--Aalen
estimator \(\sum_{s \le t} d_d(s)/N_d(s)\) as a sensitivity check;
results were unchanged.

\subsubsection{2.4 Selection model: gamma frailty and depletion
normalization}\label{selection-model-gamma-frailty-and-depletion-normalization}

\paragraph{2.4.1 Individual hazards with multiplicative
frailty}\label{individual-hazards-with-multiplicative-frailty}

Let \(z_{i,d}\) denote an individual-specific latent frailty term with
mean \(1\) and variance \(\theta_d\), and let \(\tilde h_{0,d}(t)\)
denote the depletion-neutralized baseline hazard for cohort \(d\).
Individual hazards are modeled as:

\begin{equation}\protect\phantomsection\label{eq:individual-hazard-frailty}{
h_{i,d}(t) = z_{i,d}\,\tilde h_{0,d}(t),
\qquad
z_{i,d} \sim \mathrm{Gamma}(\mathrm{mean}=1,\ \mathrm{var}=\theta_d).
}\end{equation}

Here \(\tilde h_{0,d}(t)\) is the cohort's depletion-neutralized
baseline hazard and \(z_{i,d}\) is a latent multiplicative frailty term.
The frailty variance \(\theta_d\) governs the strength of
depletion-induced curvature: larger \(\theta_d\) yields stronger
deceleration at the cohort level due to faster early depletion of
high-frailty individuals.

Gamma frailty is used because it yields a closed-form link between
observed and baseline cumulative hazards via the Laplace
transform.\textsuperscript{1} In KCOR, gamma frailty is a
\textbf{working geometric model} for depletion normalization, not a
claim of biological truth. Adequacy is evaluated empirically via fit
quality, post-normalization linearity, and stability diagnostics.

\paragraph{2.4.2 Gamma-frailty identity and
inversion}\label{gamma-frailty-identity-and-inversion}

Let

\begin{equation}\protect\phantomsection\label{eq:baseline-cumhazard}{
\tilde H_{0,d}(t) = \int_0^t \tilde h_{0,d}(s)\,ds
}\end{equation}

denote the depletion-neutralized baseline cumulative hazard. Let
\(\theta_d\) denote the cohort-specific frailty variance governing
selection-induced depletion. Under a gamma-frailty working model, the
observed cohort-level cumulative hazard satisfies:

\begin{equation}\protect\phantomsection\label{eq:gamma-frailty-identity}{
H_{\mathrm{obs},d}(t) = \frac{1}{\theta_d}\,\log\!\left(1 + \theta_d \tilde H_{0,d}(t)\right),
}\end{equation}

Given an estimate \(\hat{\theta}_d\), the observed cumulative hazard can
be mapped into depletion-neutralized baseline cumulative hazard space
via exact inversion:

\begin{equation}\protect\phantomsection\label{eq:gamma-frailty-inversion}{
\tilde H_{0,d}(t) = \frac{\exp\!\left(\theta_d H_{\mathrm{obs},d}(t)\right) - 1}{\theta_d}.
}\end{equation}

This inversion is the \textbf{normalization operator}: given an estimate
\(\hat{\theta}_d\), it maps the observed cumulative hazard
\(H_{\mathrm{obs},d}(t)\) into a depletion-neutralized cumulative hazard
scale. We use a tilde (e.g., \(\tilde H_{0,d}(t)\)) to denote
depletion-neutralized baseline quantities obtained after frailty
normalization; observed cohort-aggregated quantities are written without
a tilde (e.g., \(H_{\mathrm{obs},d}(t)\)).

\paragraph{2.4.3 Baseline shape used for frailty
identification}\label{baseline-shape-used-for-frailty-identification}

To identify \(\theta_d\), KCOR fits the gamma-frailty model within
prespecified epidemiologically quiet periods. In the reference
specification, the baseline hazard is taken to be constant over the fit
window:

\begin{equation}\protect\phantomsection\label{eq:baseline-shape-default}{
\tilde h_{0,d}(t)=k_d,
\qquad
\tilde H_{0,d}(t)=k_d\,t.
}\end{equation}

This choice intentionally minimizes degrees of freedom: during a quiet
window, curvature is forced to be explained by depletion (via
\(\theta_d\)) rather than by introducing time-varying baseline hazard
terms. If the observed cumulative hazard is near-linear over the fit
window, the model naturally collapses toward
\(\hat{\theta}_d \approx 0\), signaling weak or absent detectable
depletion curvature for that cohort over that window.

\paragraph{2.4.4 Quiet-window validity as the key dataset-specific
requirement}\label{quiet-window-validity-as-the-key-dataset-specific-requirement}

Frailty parameters are estimated using only bins whose corresponding
calendar weeks lie inside a prespecified quiet window (defined in
ISO-week space). The quiet window is prespecified to avoid sharp,
cohort-differential hazard perturbations (e.g., epidemic waves or policy
shocks) that would confound depletion-geometry estimation. A window is
acceptable only if diagnostics indicate (i) good fit in
cumulative-hazard space, (ii) post-normalization linearity within the
window, and (iii) stability of \((\hat{k}_d,\hat{\theta}_d)\) to small
boundary perturbations. If no candidate window passes, diagnostics
indicate non-identifiability and the analysis is treated as not
identified rather than producing a potentially misleading normalized
contrast. All diagnostics are computed over discrete event-time bins
(weekly intervals since enrollment) whose corresponding calendar weeks
fall within the prespecified quiet window.

\paragraph{Quiet-window selection protocol
(operational)}\label{quiet-window-selection-protocol-operational}

Quiet-window selection is prespecified and evaluated using diagnostic
criteria summarized in Supplementary Information §S2 (Tables
\ref{tbl:si_assumptions}--\ref{tbl:si_identifiability}).

\subsubsection{2.5 Estimation during quiet periods (cumulative-hazard
least
squares)}\label{estimation-during-quiet-periods-cumulative-hazard-least-squares}

KCOR estimates \((\hat{k}_d,\hat{\theta}_d)\) independently for each
cohort \(d\) using only time bins that fall inside a prespecified quiet
window in calendar time (see §2.4.4). The quiet window is applied
consistently across cohorts within an analysis. Let \(\mathcal{T}_d\)
denote the set of event-time bins \(t\) whose corresponding calendar
week lies in the quiet window, with \(t\) also satisfying
\(t \ge \mathrm{SKIP\_WEEKS}\).

Under the default baseline shape, the model-implied observed cumulative
hazard is

\begin{equation}\protect\phantomsection\label{eq:hobs-model}{
H^{\mathrm{model}}_{d}(t; k_d,\theta_d)
=
\frac{1}{\theta_d}\log\!\left(1+\theta_d k_d t\right).
}\end{equation}

Identifiability of \((\hat{k}_d,\hat{\theta}_d)\) arises from curvature
in cumulative-hazard space: observed cumulative hazards are nonlinear in
follow-up time when \(\theta_d>0\). When depletion is weak (or the quiet
window is too short to exhibit curvature), the model smoothly approaches
a linear cumulative hazard, since
\(H_{d}^{\mathrm{model}}(t; k_d, \theta_d) \to k_d t\) as
\(\theta_d \to 0\). Operationally, near-linear observed cumulative
hazards naturally drive fitted frailty variance estimates toward zero;
fit diagnostics such as \(n_{\mathrm{obs}}\) and RMSE in \(H\)-space
provide a practical check on whether the selection parameters are being
identified from the quiet-window data. Thus, lack of identifiable
curvature manifests as fitted frailty variance estimates approaching
zero, serving as an internal diagnostic for non-identifiability over
short or sparse follow-up.

In applied analyses, this behavior is most commonly observed in
vaccinated cohorts, whose cumulative hazards during quiet periods are
often close to linear. In such cases, the gamma-frailty fit collapses
naturally, indicating minimal detectable depletion rather than
reflecting a modeling assumption. When residual time-varying risk
contaminates a nominally quiet window, fitted frailty variance estimates
similarly shrink toward zero, signaling limited identifiability rather
than inducing spurious correction.

Parameters are estimated by constrained nonlinear least squares:

\begin{equation}\protect\phantomsection\label{eq:nls-objective}{
(\hat k_d,\hat\theta_d)
=
\arg\min_{k_d>0,\ \theta_d \ge 0}
\sum_{t \in \mathcal{T}_d}
\bigl(
H_{\mathrm{obs},d}(t)
-
H^{\mathrm{model}}_{d}(t; k_d,\theta_d)
\bigr)^2 .
}\end{equation}

Fitting is performed in cumulative-hazard space rather than via
likelihood maximization, as the inputs are discrete-time,
cohort-aggregated hazards rather than individual-level event histories.
Least-squares fitting serves as a stable estimating equation for
selection-induced depletion during quiet periods, emphasizes agreement
in hazard shape rather than instantaneous risk, and yields diagnostics
(e.g., RMSE and residual structure in \(H\)-space) that directly assess
identifiability. Likelihood-based fitting may be used as a sensitivity
analysis but is not required for the gamma-frailty normalization
identity.

All analyses use a prespecified reference implementation with fixed
operational defaults; full details are provided in Supplementary Section
S4.

\subsubsection{2.6 Normalization (depletion-neutralized cumulative
hazards)}\label{normalization-depletion-neutralized-cumulative-hazards}

After fitting, KCOR computes the depletion-neutralized baseline
cumulative hazard for each cohort \(d\) by applying the inversion to the
full post-enrollment trajectory:

\begin{equation}\protect\phantomsection\label{eq:normalized-cumhazard}{
\tilde H_{0,d}(t) = \frac{\exp\!\left(\hat{\theta}_d\,H_{\mathrm{obs},d}(t)\right)-1}{\hat{\theta}_d}.
}\end{equation}

This normalization maps each cohort into a depletion-neutralized
baseline-hazard space in which the contribution of gamma frailty
parameters \((\hat{\theta}_d, \hat{k}_d)\) to hazard curvature has been
factored out. This normalization defines a common comparison scale in
cumulative-hazard space; it is not equivalent to Cox partial-likelihood
baseline anchoring, but serves an analogous geometric role for
cumulative contrasts. In this space, cumulative hazards are directly
comparable across cohorts, and remaining differences reflect real
differences in baseline risk rather than selection-induced depletion.
The core identities used in KCOR are given in Equations
(\ref{eq:hazard-discrete}), (\ref{eq:nls-objective}),
(\ref{eq:normalized-cumhazard}), and (\ref{eq:kcor-estimand}).
Normalization defines a common comparison scale; the scientific estimand
is then computed on that scale (Box 2).

\paragraph{2.6.1 Computational
considerations}\label{computational-considerations}

KCOR operates on aggregated event counts in discrete time and
cumulative-hazard space. Computational complexity scales linearly with
the number of time bins and strata rather than the number of
individuals, making the method feasible for very large population
registries. In practice, KCOR analyses on national-scale datasets
(millions of individuals) are memory-bound rather than CPU-bound and can
be implemented efficiently using standard vectorized numerical
libraries. No iterative optimization over individual-level records is
required.

\paragraph{2.6.2 Internal diagnostics and `self-check'
behavior}\label{internal-diagnostics-and-self-check-behavior}

KCOR includes internal diagnostics intended to make model stress visible
rather than hidden.

\begin{enumerate}
\def\labelenumi{\arabic{enumi}.}
\item
  \textbf{Post-normalization linearity in quiet periods.} Within the
  prespecified quiet window (see §2.4.4), the depletion-neutralized
  cumulative hazard should be approximately linear in event time after
  inversion. Systematic residual curvature indicates window
  contamination (external shocks, secular trends) or misspecified
  depletion geometry for that cohort.
\item
  \textbf{Fit residual structure in cumulative-hazard space.} Define
  residuals over the fit set \(\mathcal{T}_d\):
\end{enumerate}

\begin{equation}\protect\phantomsection\label{eq:si_residuals}{
r_{d}(t)=H_{\mathrm{obs},d}(t)-H_{d}^{\mathrm{model}}(t;\hat{k}_d,\hat{\theta}_d).
}\end{equation}

KCOR expects residuals to be small and not systematically
time-structured. Strongly patterned residuals indicate that the
curvature attributed to depletion is instead being driven by unmodeled
time-varying hazards.

\begin{enumerate}
\def\labelenumi{\arabic{enumi}.}
\setcounter{enumi}{2}
\tightlist
\item
  \textbf{Parameter stability to window perturbations.} Under valid
  quiet-window selection,
\end{enumerate}

\[
(\hat{k}_d,\hat{\theta}_d)
\]

should be stable to small perturbations of the quiet-window boundaries
(e.g., ±4 weeks). Large changes in fitted frailty variance under small
boundary shifts signal that the fitted curvature is sensitive to
transient dynamics rather than stable depletion.

\begin{enumerate}
\def\labelenumi{\arabic{enumi}.}
\setcounter{enumi}{3}
\tightlist
\item
  \textbf{Non-identifiability manifests as:}
\end{enumerate}

\[
\hat{\theta}_d\rightarrow 0.
\]

When the observed cumulative hazard is near-linear (weak curvature) or
events are sparse, \(\theta\) is weakly identified. In such cases, KCOR
should be interpreted primarily as a diagnostic (limited evidence of
detectable depletion curvature) rather than a strong correction.

These diagnostics are reported alongside \(\mathrm{KCOR}(t)\) curves.
The goal is not to assert that a single parametric form is always
correct, but to ensure that when the form is incorrect or the window is
contaminated, the method signals this explicitly rather than silently
producing a misleading `corrected' estimate. When diagnostics indicate
non-identifiability, the depletion-based normalization is inappropriate
and KCOR should not be interpreted.

\subsubsection{2.7 Stabilization (early
weeks)}\label{stabilization-early-weeks}

In many applications, the first few post-enrollment intervals can be
unstable due to immediate post-enrollment artifacts (e.g., rapid
deferral, short-term sorting, administrative effects). KCOR supports a
prespecified stabilization rule by excluding early weeks from
accumulation and from quiet-window fitting. The skip-weeks parameter is
prespecified and evaluated via sensitivity analysis to exclude early
enrollment instability rather than to tune estimates.

In discrete time, define an effective hazard for accumulation:

\begin{equation}\protect\phantomsection\label{eq:effective-hazard-skip}{
h_d^{\mathrm{eff}}(t)=
\begin{cases}
0, & t < \mathrm{SKIP\_WEEKS} \\
h_{\mathrm{obs},d}(t), & t \ge \mathrm{SKIP\_WEEKS}.
\end{cases}
}\end{equation}

Then compute observed cumulative hazards from \(h_d^{\mathrm{eff}}(t)\)
as in §2.3:

\[
H_{\mathrm{obs},d}(t).
\]

\subsubsection{2.8 KCOR estimator}\label{kcor-estimator}

With depletion-neutralized cumulative hazards in hand, the primary KCOR
trajectory is defined as:

\begin{equation}\protect\phantomsection\label{eq:kcor-estimator}{
\mathrm{KCOR}(t) = \frac{\tilde H_{0,A}(t)}{\tilde H_{0,B}(t)}.
}\end{equation}

This ratio is computed after depletion normalization and is interpreted
conditional on the stated assumptions and diagnostics (Box 2; §2.1.2).

\subsubsection{2.9 Uncertainty
quantification}\label{uncertainty-quantification}

Uncertainty is quantified using stratified bootstrap resampling, which
propagates uncertainty through the full pipeline (event counts, frailty
fitting, inversion, and KCOR computation).

\paragraph{2.9.1 Stratified bootstrap
procedure}\label{stratified-bootstrap-procedure}

The stratified bootstrap procedure for KCOR proceeds as follows:

\begin{enumerate}
\def\labelenumi{\arabic{enumi}.}
\item
  \textbf{Resample cohort--time counts (aggregated bootstrap).} Within
  each cohort and stratum (e.g., age group), resample event counts and
  risk-set sizes within time bins, preserving the original
  cohort/stratum structure. When individual-level data are available, an
  equivalent individual-level resampling can be used; in this work, we
  use the aggregated cohort--time bootstrap.
\item
  \textbf{Re-estimate frailty parameters.} For each bootstrap replicate,
  re-estimate \((\hat{k}_d,\hat{\theta}_d)\) independently for each
  cohort \(d\) using the resampled data, applying the same quiet-window
  selection and fitting procedure as in the primary analysis.
\item
  \textbf{Recompute normalized cumulative hazards.} Using the
  bootstrap-estimated frailty parameters, recompute
  \(\tilde H_{0,d}(t)\) for each cohort using Eq.
  (\ref{eq:normalized-cumhazard}) applied to the resampled observed
  cumulative hazards.
\item
  \textbf{Recompute KCOR.} Compute \(\mathrm{KCOR}(t)\) for each
  bootstrap replicate as the ratio of the bootstrap-normalized
  cumulative hazards.
\item
  \textbf{Form percentile intervals.} From the bootstrap distribution of
  \(\mathrm{KCOR}(t)\) values at each time point, form percentile-based
  confidence intervals (e.g., 2.5th and 97.5th percentiles for 95\%
  intervals).
\end{enumerate}

Bootstrap resampling is performed at the cohort-count level in the
aggregated representation by resampling \((d_d(t), N_d(t))\) pairs
within cohort-time strata with replacement, preserving within-cohort
temporal structure, rather than resampling individual-level records.

Uncertainty intervals reflect event stochasticity and model-fit
uncertainty in \((\hat{k}_d,\hat{\theta}_d)\) and are interpreted
conditional on the observed risk sets and modeling assumptions.

\subsubsection{2.10 Algorithm summary and reproducibility
checklist}\label{algorithm-summary-and-reproducibility-checklist}

Table \ref{tbl:KCOR_algorithm} summarizes the complete KCOR pipeline.

\begin{figure}
\centering
\pandocbounded{\includegraphics[keepaspectratio,alt={KCOR as a two-stage framework. (A) Fixed-cohort cumulative hazards exhibit curvature due to selection-induced depletion; late-time curvature is used to estimate frailty parameters for normalization. (B) Gamma-frailty normalization yields approximately linearized cumulative hazards that are directly comparable across cohorts; \textbackslash mathrm\{KCOR\}(t), defined as the ratio of depletion-neutralized baseline cumulative hazards, is near-flat under the null and deviates only under net hazard differences. In the schematic, \textbackslash tilde H\_\{0,d\}(t) denotes the depletion-neutralized baseline cumulative hazard.}]{figures/fig_kcor_workflow.png}}
\caption{\textbf{KCOR as a two-stage framework.} \textbf{(A)}
Fixed-cohort cumulative hazards exhibit curvature due to
selection-induced depletion; late-time curvature is used to estimate
frailty parameters for normalization. \textbf{(B)} Gamma-frailty
normalization yields approximately linearized cumulative hazards that
are directly comparable across cohorts; \(\mathrm{KCOR}(t)\), defined as
the ratio of depletion-neutralized baseline cumulative hazards, is
near-flat under the null and deviates only under net hazard differences.
\emph{In the schematic, \(\tilde H_{0,d}(t)\) denotes the
depletion-neutralized baseline cumulative
hazard.}}\label{fig:kcor_workflow}
\end{figure}

\subsubsection{2.11 Relationship to Cox proportional
hazards}\label{relationship-to-cox-proportional-hazards}

Cox proportional hazards models estimate an instantaneous hazard ratio
under the assumption that hazards differ by a time-invariant
multiplicative factor. Under selective uptake with latent frailty
heterogeneity, this assumption is typically violated, yielding
time-varying hazard ratios induced purely by depletion dynamics. This
reflects an estimand mismatch: Cox targets an instantaneous hazard ratio
conditional on survival, whereas KCOR targets a cumulative hazard
contrast after depletion normalization.

Accordingly, Cox results are presented here as a diagnostic
demonstration of estimand mismatch, not as a competing
intervention-effect estimator. This limitation is consistent with
earlier work by Deeks showing that increasing covariate adjustment in
non-randomized analyses can exacerbate bias and imprecision when
selection effects and measurement error dominate. Deeks further noted
that, despite the widespread reliance on covariate adjustment in
non-randomized studies, there is no empirical evidence that such
adjustment reduces bias on average.\textsuperscript{12}

Even when Cox models are extended with shared frailty to accommodate
heterogeneity, they continue to estimate instantaneous hazard ratios
conditional on survival. KCOR instead uses a parametric working model
only to normalize selection-induced depletion geometry, then computes a
cumulative contrast on the depletion-neutralized scale.

\paragraph{2.11.1 Demonstration: Cox bias under frailty heterogeneity
with no treatment
effect}\label{demonstration-cox-bias-under-frailty-heterogeneity-with-no-treatment-effect}

A controlled synthetic experiment was conducted in which the
\textbf{true effect is known to be zero by construction}, isolating
latent frailty heterogeneity as the sole driver of depletion-induced
non-proportional hazards. Cox and KCOR were applied to the same
simulated datasets under identical information constraints.

\textbf{Data-generating process.}

Two cohorts of equal size were simulated under the same baseline hazard
\(h_0(t)\) over time (constant or Gompertz). Individual hazards were
generated as \(z\,h_0(t)\), with frailty \[
z \sim \text{Gamma}(\theta^{-1}, \theta^{-1}),
\] with mean 1 and variance \(\theta\).

Cohort A was generated with \(\theta = 0\) (no frailty heterogeneity),
while Cohort B was generated with \(\theta > 0\). \textbf{No treatment
or intervention effect was applied}: conditional on frailty, the two
cohorts have identical hazards at all times. Thus, the true hazard ratio
between cohorts is exactly 1 for all \(t\).

Simulations were repeated over a grid of frailty variances
\(\theta \in \{0, 0.5, 1, 2, 5, 10, 20\}\).

\textbf{Cox analysis.}

For each simulated dataset, a standard Cox proportional hazards model
was fitted using partial likelihood (statsmodels \texttt{PHReg}), with
cohort membership as the sole covariate (no time-varying covariates or
interactions). The resulting hazard ratio estimates and confidence
intervals therefore reflect \textbf{only differences induced by
frailty-driven depletion}, not any treatment effect.

\textbf{KCOR analysis.}

The same simulated datasets were analyzed using KCOR. Observed
cumulative hazards were estimated nonparametrically using the
Nelson--Aalen estimator, then normalized using Eq.
(\ref{eq:normalized-cumhazard}) with frailty parameters fitted in the
prespecified quiet window prior to computing \(\mathrm{KCOR}(t)\).
Although the data-generating process specifies individual hazards,
Nelson--Aalen is used to mirror the information available in
observational registry studies rather than exploiting simulator-only
knowledge. Post-normalization slope and asymptotic \(\mathrm{KCOR}(t)\)
values were examined to assess departure from the null.

\textbf{Expected behavior under the null.}

Because the data-generating process includes \textbf{no treatment
effect}, any valid estimator should return a null result. In this
setting:

\begin{itemize}
\tightlist
\item
  \textbf{Cox regression} is expected to produce apparent non-null
  hazard ratios as \(\theta\) increases, reflecting differential
  selection-induced depletion and violation of proportional hazards
  induced by frailty heterogeneity.
\item
  \textbf{KCOR} is expected to remain centered near unity with
  negligible post-normalization slope across all \(\theta\), consistent
  with correct null behavior after depletion normalization.
\end{itemize}

\textbf{Summary of findings.}

Across increasing values of \(\theta\), Cox regression produced
progressively larger apparent deviations from a hazard ratio of 1. The
direction and magnitude of the apparent effect depended on the follow-up
horizon and degree of frailty heterogeneity. In contrast,
\(\mathrm{KCOR}(t)\) trajectories remained stable and centered near
unity, with post-normalization slopes approximately zero across all
simulated conditions.

These results demonstrate that \textbf{frailty heterogeneity alone is
sufficient to induce spurious hazard ratios in Cox regression}, while
KCOR correctly returns a null result under the same conditions.

Table \ref{tbl:cox_bias_demo} reports numerical summaries of the
Cox-vs-KCOR behavior across the frailty grid.

Additional Cox HR results from the same synthetic-null grid are shown in
Figure \ref{fig:cox_bias_hr}.

A compact summary of KCOR bias as a function of frailty variance
\(\theta\) is provided in the Supplementary Information (Figure
\ref{fig:si_kcor_bias_vs_theta}).

\begin{figure}
\centering
\pandocbounded{\includegraphics[keepaspectratio,alt={Cox regression produces spurious non-null hazard ratios under a synthetic null as frailty heterogeneity increases. Hazard ratios (with 95\% confidence intervals) from Cox proportional hazards regression comparing cohort B to cohort A in simulations where the true treatment effect is identically zero and cohorts differ only in frailty variance (\textbackslash theta). Deviations from HR=1 arise solely from frailty-driven depletion and associated non-proportional hazards.}]{figures/fig_cox_bias_hr_vs_theta.png}}
\caption{Cox regression produces spurious non-null hazard ratios under a
\emph{synthetic null} as frailty heterogeneity increases. Hazard ratios
(with 95\% confidence intervals) from Cox proportional hazards
regression comparing cohort B to cohort A in simulations where the true
treatment effect is identically zero and cohorts differ only in frailty
variance (\(\theta\)). Deviations from HR=1 arise solely from
frailty-driven depletion and associated non-proportional
hazards.}\label{fig:cox_bias_hr}
\end{figure}

\begin{figure}
\centering
\pandocbounded{\includegraphics[keepaspectratio,alt={\textbackslash mathrm\{KCOR\}(t) remains null under a synthetic null across increasing frailty heterogeneity. \textbackslash mathrm\{KCOR\}(t) asymptotes remain near 1 across \textbackslash theta in the same simulations, consistent with correct null behavior after depletion normalization. Uncertainty bands (95\% bootstrap intervals) are shown but are narrow due to large sample sizes.}]{figures/fig_cox_bias_kcor_vs_theta.png}}
\caption{\(\mathrm{KCOR}(t)\) remains null under a synthetic null across
increasing frailty heterogeneity. \(\mathrm{KCOR}(t)\) asymptotes remain
near 1 across \(\theta\) in the same simulations, consistent with
correct null behavior after depletion normalization. Uncertainty bands
(95\% bootstrap intervals) are shown but are narrow due to large sample
sizes.}\label{fig:cox_bias_kcor}
\end{figure}

\textbf{Interpretation.}

This controlled synthetic null shows that Cox proportional hazards
regression can report highly statistically significant non-null hazard
ratios even when the true effect is identically zero, purely due to
frailty-driven depletion and induced non-proportional hazards. KCOR
remains near unity under the same conditions because depletion
normalization precedes comparison.

\subsubsection{2.12 Worked example
(descriptive)}\label{worked-example-descriptive}

A brief worked example is included to illustrate the KCOR workflow
end-to-end. This example is descriptive and intended solely to
demonstrate the mechanics of cohort construction, hazard estimation,
frailty fitting, depletion normalization, and KCOR computation.

The example proceeds from aggregated cohort counts through
cumulative-hazard estimation, quiet-window frailty fitting, gamma
inversion, and \(\mathrm{KCOR}(t)\) construction, accompanied by
diagnostic plots assessing post-normalization linearity and parameter
stability.

\subsubsection{2.13 Reproducibility and computational
implementation}\label{reproducibility-and-computational-implementation}

All figures, tables, and simulations can be reproduced from the
accompanying code repository. The manuscript is built from
\texttt{documentation/preprint/paper.md} using the root
\texttt{Makefile} paper target \texttt{make\ paper-full}.

Additional environment and runtime details are provided in the
Supplementary Information (SI); code and archival links are provided in
Code/Data Availability.

\subsection{3. Results}\label{results}

This section is the core validation claim of KCOR:

\begin{itemize}
\tightlist
\item
  \textbf{Negative controls (null under selection):} under a true null
  effect, KCOR remains approximately flat at 1 even when selection
  induces large curvature differences.
\item
  \textbf{Positive controls (detect injected effects):} when known
  harm/benefit is injected into otherwise-null data, KCOR reliably
  detects it.
\item
  \textbf{Failure signaling (diagnostics):} when key assumptions are
  violated or the working model is stressed, KCOR's diagnostics degrade
  (e.g., poor quiet-window fit, post-normalization nonlinearity,
  parameter instability), and the analysis is treated as not identified
  rather than reported as a stable contrast.
\end{itemize}

Throughout, curvature in cumulative hazard plots reflects
selection-induced depletion, while linearity after normalization
indicates successful removal of that curvature.

In vaccinated--unvaccinated comparisons, large early differences in
\(\mathrm{KCOR}(t)\) may reflect baseline risk selection rather than
intervention-attributable effects; in such cases,
\(\mathrm{KCOR}(t; t_0)\) is emphasized to report deviations relative to
an early post-enrollment reference while preserving time-varying
divergence.

\subsubsection{3.1 Negative controls (selection-only
null)}\label{negative-controls-selection-only-null}

\paragraph{3.1.1 Fully synthetic negative control (in-model
gamma-frailty
null)}\label{fully-synthetic-negative-control-in-model-gamma-frailty-null}

We first evaluate KCOR under a fully synthetic, selection-only null in
which the data-generating process exactly matches the working
gamma-frailty model. Two cohorts share the same baseline hazard
\(h_0(t)\) but differ in frailty variance \(\theta\), inducing strong
cohort-level hazard curvature through depletion alone, with no treatment
effect by construction.

Under correct specification, depletion normalization is exact up to
sampling variability. After estimating frailty parameters during quiet
periods and applying gamma-frailty inversion, \(\mathrm{KCOR}(t)\)
remains approximately constant at 1 over follow-up, confirming correct
null behavior under selection-induced curvature. Full design details and
figures are provided in the Supplementary Information (Section S4.2.1;
Figure \ref{fig:neg_control_synthetic}).

\paragraph{3.1.2 Empirical negative control using national registry data
(Czech
Republic)}\label{empirical-negative-control-using-national-registry-data-czech-republic}

This application is presented solely to illustrate KCOR's diagnostic
behavior on real registry data; it uses an age-shift construction
(pseudo-cohorts) that is a negative control by design rather than an
observational treatment-effect analysis. No causal interpretation of
vaccine effects is implied.

The repository includes a pragmatic negative control construction that
repurposes a real dataset by comparing ``like with like'' while inducing
large composition differences (e.g., age band shifts). In this
construction, age strata are remapped into pseudo-doses so that
comparisons are, by construction, within the same underlying category;
the expected differential contrast is near zero, but the baseline
hazards differ strongly.

These age-shift negative controls deliberately induce extreme baseline
mortality differences (10--20 year age gaps) while preserving a
pseudo-null by construction, since all vaccination states are compared
symmetrically. The near-flat \(\mathrm{KCOR}(t)\) trajectories are
consistent with the estimator normalizing selection-induced depletion
curvature without introducing spurious time trends or cumulative drift.

For the empirical age-shift negative control (Figure
\ref{fig:neg_control_10yr}), aggregated weekly cohort summaries derived
from the Czech Republic administrative mortality and vaccination dataset
are used and exported in KCOR\_CMR format.

Notably, KCOR estimates frailty parameters independently for each cohort
without knowledge of exposure status; the observed asymmetry in
depletion correction arises entirely from differences in hazard
curvature rather than from any vaccination-specific assumptions.

Figure \ref{fig:neg_control_10yr} provides a representative
illustration; additional age-shift variants are provided in the
Supplementary Information (SI).

\begin{figure}
\centering
\pandocbounded{\includegraphics[keepaspectratio,alt={Empirical negative control with approximately 10-year age difference between cohorts. Despite large baseline mortality differences, \textbackslash mathrm\{KCOR\}(t) remains near-flat at 1 over follow-up, consistent with the pseudo-null construction. Curves are shown as anchored \textbackslash mathrm\{KCOR\}(t; t\_0), i.e., \textbackslash mathrm\{KCOR\}(t)/\textbackslash mathrm\{KCOR\}(t\_0), which removes pre-existing cumulative differences and displays post-anchor divergence only. KCOR curves are anchored at t\_0 = 4 weeks (i.e., plotted as \textbackslash mathrm\{KCOR\}(t; t\_0)). Uncertainty bands (95\% bootstrap intervals) are shown. Data source: Czech Republic mortality and vaccination dataset processed into KCOR\_CMR aggregated format (negative-control construction; see Supplementary Information (SI)).}]{figures/fig2_neg_control_10yr_age_diff.png}}
\caption{Empirical negative control with approximately 10-year age
difference between cohorts. Despite large baseline mortality
differences, \(\mathrm{KCOR}(t)\) remains near-flat at 1 over follow-up,
consistent with the pseudo-null construction. Curves are shown as
anchored \(\mathrm{KCOR}(t; t_0)\), i.e.,
\(\mathrm{KCOR}(t)/\mathrm{KCOR}(t_0)\), which removes pre-existing
cumulative differences and displays post-anchor divergence only. KCOR
curves are anchored at \(t_0 = 4\) weeks (i.e., plotted as
\(\mathrm{KCOR}(t; t_0)\)). Uncertainty bands (95\% bootstrap intervals)
are shown. Data source: Czech Republic mortality and vaccination dataset
processed into KCOR\_CMR aggregated format (negative-control
construction; see Supplementary Information
(SI)).}\label{fig:neg_control_10yr}
\end{figure}

Table \ref{tbl:neg_control_summary} provides numeric summaries.

\subsubsection{3.2 Positive controls (injected
effects)}\label{positive-controls-injected-effects}

Positive controls (injected harm/benefit) are provided in Supplementary
Section S3. They verify that under a known injected effect, KCOR
deviates in the expected direction and with magnitude consistent with
the injection (up to discretization and sampling noise).

In positive-control simulations with injected multiplicative hazard
shifts, KCOR reliably detects both harm and benefit, with estimated
\(\mathrm{KCOR}(t)\) trajectories tracking the imposed effects; full
results are shown in Supplementary Figure S1.

\subsubsection{3.3 Stress tests (frailty
misspecification)}\label{stress-tests-frailty-misspecification}

\paragraph{3.3.1 Frailty misspecification
robustness}\label{frailty-misspecification-robustness}

To assess robustness to departures from the gamma frailty assumption,
simulations were conducted under alternative frailty distributions while
maintaining the same selection-induced depletion geometry. Simulations
were performed for:

\begin{itemize}
\tightlist
\item
  \textbf{Gamma} (baseline reference)
\item
  \textbf{Lognormal} frailty
\item
  \textbf{Two-point mixture} (discrete frailty)
\item
  \textbf{Bimodal} frailty distributions
\item
  \textbf{Correlated frailty} (within-subgroup correlation)
\end{itemize}

For each frailty specification, bias (deviation from true cumulative
hazard ratio), variance (trajectory stability), coverage (proportion of
simulations where uncertainty intervals contain the true value), and
non-identifiability rate (proportion of simulations where quiet-window
diagnostics indicated non-identifiability) are reported.

Under frailty misspecification, KCOR can degrade gracefully by
attenuating toward unity or by not meeting diagnostic criteria, rather
than producing spurious large effects. When the alternative frailty
distribution produces similar depletion geometry to gamma frailty, KCOR
normalization remains approximately valid, with bias remaining small and
diagnostics indicating successful identification. When the alternative
frailty structure produces substantially different depletion geometry,
KCOR diagnostics (poor cumulative-hazard fit, residual autocorrelation,
parameter instability) correctly signal that the gamma-frailty
approximation is inadequate, and \(\mathrm{KCOR}(t)\) trajectories
either remain near-unity (reflecting attenuation) or are not computed
when diagnostic thresholds are not met. Additional validation
results---including full simulation grids, quiet-window robustness
catalogs, dynamic-selection checks, and extended comparator
analyses---are provided in the Supplementary Information (SI).

Additional derivations, simulation studies, robustness analyses, and
implementation details are provided in the Supplementary Information.

\subsection{4. Discussion}\label{discussion}

\subsubsection{4.1 Limits of attribution and
non-identifiability}\label{limits-of-attribution-and-non-identifiability}

KCOR does not uniquely identify the biological, behavioral, or clinical
mechanisms responsible for observed hazard heterogeneity. In particular,
curvature in the cumulative hazard may arise from multiple sources,
including selection on latent frailty, behavior change, seasonality,
treatment effects, reporting artifacts, or their combination. Depletion
of susceptibles is therefore used as a parsimonious working model whose
adequacy is evaluated through diagnostics and negative controls, rather
than assumed as a substantive truth. KCOR's estimand is whether a
cumulative outcome contrast persists after removal of curvature
consistent with selection-induced depletion, not attribution of that
curvature to a specific mechanism.

\subsubsection{4.2 What KCOR estimates}\label{what-kcor-estimates}

\emph{Table \ref{tbl:positioning} clarifies that KCOR differs from
non-proportional hazards methods not in flexibility, but in estimand and
direction of inference.} KCOR normalizes selection-induced depletion and
then compares depletion-neutralized cumulative hazards;
\(\mathrm{KCOR}(t)\) summarizes cumulative outcome accumulation rather
than instantaneous hazard ratios. It is descriptive rather than causal,
and interpretability is conditional on the stated assumptions and
prespecified diagnostics (quiet-window fit, post-normalization
linearity, parameter stability). When diagnostics fail, contrasts are
not reported and results are treated as non-identifiable rather than as
substantive cumulative effects. Anchored KCOR is used only when
explicitly stated to remove pre-existing level differences and emphasize
post-reference divergence.

\subsubsection{4.3 Relationship to negative control
methods}\label{relationship-to-negative-control-methods}

Negative control outcomes/tests are widely used to \emph{detect}
confounding. KCOR's objective is different: it is an estimator intended
to \emph{normalize away a specific confounding
structure}---selection-induced depletion dynamics---prior to comparison.
Negative and positive controls are nevertheless central to validating
the estimator's behavior.

This asymmetry helps explain why standard observational analyses can
show large mortality differences during periods lacking a plausible
mechanism: vaccinated cohorts are already selection-filtered, while
unvaccinated hazards are suppressed by ongoing frailty depletion.
Unadjusted comparisons therefore systematically understate unvaccinated
baseline risk and exaggerate apparent differences.

\subsubsection{4.4 Practical guidelines for
implementation}\label{practical-guidelines-for-implementation}

This subsection summarizes common operational practices for applying
KCOR in retrospective cohort studies and for assessing when resulting
contrasts are interpretable.

Reporting commonly includes:

\begin{itemize}
\tightlist
\item
  Enrollment definition and justification
\item
  Risk set definitions and event-time binning
\item
  Quiet-window definition and justification
\item
  Baseline-shape choice (default constant baseline over the fit window)
  and fit diagnostics
\item
  Skip/stabilization rule and robustness to nearby values
\item
  Predefined negative/positive controls used for validation
\item
  Sensitivity analysis plan and results
\end{itemize}

KCOR should therefore be applied and reported as a complete
pipeline---from cohort freezing, through depletion normalization, to
cumulative comparison and diagnostics---rather than as a standalone
adjustment step. Scope and interpretation are summarized once in Box 2
(§1.6).

KCOR is diagnostic-first: multiplicity concerns are mitigated by
prespecification of cohorts, quiet windows, and control constructions.
Exploratory stratified analyses should be interpreted descriptively,
with emphasis on diagnostic consistency rather than formal significance
claims.

\subsection{5. Limitations}\label{limitations}

This section summarizes the principal limitations of the KCOR framework,
emphasizing conditions under which interpretation is restricted rather
than situations in which the estimator fails. These limitations are
diagnostic and design-related, reflecting the framework's intentionally
conservative scope.

KCOR is intentionally diagnostic rather than test-based: it does not
attempt to formally test properties such as quiet-window validity or
frailty distributional form, but instead enforces conservative
interpretability gates when prespecified empirical diagnostics fail.
KCOR is not a causal effect estimator and does not identify
counterfactual outcomes under hypothetical interventions.

\begin{itemize}
\tightlist
\item
  \textbf{Model dependence}: Normalization relies on the adequacy of the
  gamma-frailty model and the baseline-shape assumption during the quiet
  window.
\item
  \textbf{Relation to existing non-PH methods}: KCOR is complementary to
  time-varying Cox, flexible parametric, additive hazards, and MSM
  approaches; these methods address different estimands and
  identification strategies, whereas KCOR targets depletion-geometry
  normalization under minimal-data constraints (see §1.3.1).
\item
  \textbf{\(\theta\) estimation is data-derived}: KCOR does not impose
  \(\theta = 0\) for any cohort. The frequent observation that fitted
  frailty variance estimates collapse toward zero for vaccinated cohorts
  is a result of the frailty fit and should not be interpreted as an
  assumption of homogeneity.
\item
  \textbf{Sparse events}: When event counts are small, hazard estimation
  and parameter fitting can be unstable.
\item
  \textbf{Contamination of quiet periods}: External shocks (e.g.,
  epidemic waves) overlapping the quiet window can bias
  selection-parameter estimation.
\item
  \textbf{Applicability to other outcomes}: Although this paper focuses
  on all-cause mortality, KCOR is applicable to other irreversible
  outcomes provided that event timing and risk sets are well defined.
  Application to cause-specific mortality would require explicit
  competing-risk definitions and cause-specific hazards, but the
  normalization logic remains cumulative and descriptive. Extension to
  non-fatal outcomes such as hospitalization is conceptually
  straightforward but may require additional attention to outcome
  definitions, censoring mechanisms, and recurrent events. These
  considerations affect interpretation rather than the core KCOR
  framework.
\item
  \textbf{Non-gamma frailty}: The KCOR framework assumes that selection
  acts approximately multiplicatively through a time-invariant frailty
  distribution, for which the gamma family provides a convenient and
  empirically testable approximation. In settings where depletion
  dynamics are driven by more complex mechanisms---such as time-varying
  frailty variance, interacting risk factors, or shared frailty
  correlations within subgroups---the curvature structure exploited by
  KCOR may be misspecified. In such cases, KCOR diagnostics (e.g., poor
  curvature fit or unstable fitted frailty variance estimates) serve as
  indicators of model inadequacy rather than targets for parameter
  tuning. Extending the framework to accommodate dynamic or correlated
  frailty structures would require explicit model generalization rather
  than modification of KCOR normalization steps and is left to future
  work. Empirically, KCOR's validity depends on curvature removal rather
  than the specific parametric form; alternative frailty distributions
  that generate similar depletion geometry would yield equivalent
  normalization.
\end{itemize}

\subsubsection{5.1 Failure modes and
diagnostics}\label{failure-modes-and-diagnostics}

KCOR is designed to normalize selection-induced depletion curvature
under its stated model and windowing assumptions. Reviewers and readers
should expect the method to degrade when those assumptions are violated.
Common failure modes include:

\begin{itemize}
\tightlist
\item
  \textbf{Mis-specified quiet window}: If the quiet window overlaps
  major external shocks (epidemic waves, policy changes, reporting
  artifacts), the fitted parameters may absorb non-selection dynamics,
  biasing normalization.
\item
  \textbf{External time-varying hazards masquerading as frailty
  depletion}: Strong secular trends, seasonality, or outcome-definition
  changes can introduce curvature that is not well captured by
  gamma-frailty depletion alone. For example, COVID-19 waves
  disproportionately increase mortality among frail individuals; if one
  cohort has higher baseline frailty, such a wave can preferentially
  deplete that cohort, producing the appearance of a benefit in the
  lower-frailty cohort that is actually due to differential
  frailty-specific mortality from the external hazard rather than from
  the intervention under study.
\item
  \textbf{Extremely sparse cohorts}: When events are rare, observed
  cumulative hazards become noisy and \((\hat{k}_d,\hat{\theta}_d)\) can
  be weakly identified, often manifesting as unstable fitted frailty
  variance estimates or wide uncertainty.
\item
  \textbf{Non-frailty-driven curvature}: Administrative censoring,
  cohort-definition drift, changes in risk-set construction, or
  differential loss can induce curvature unrelated to latent frailty.
\end{itemize}

Violations of identifiability assumptions lead to conservative behavior
(e.g., \(\hat{\theta} \to 0\)) rather than spurious non-null contrasts.

Practical diagnostics include:

\begin{itemize}
\tightlist
\item
  \textbf{Quiet-window overlays} on hazard/cumulative-hazard plots to
  confirm the fit window is epidemiologically stable.
\item
  \textbf{Fit residuals in \(H\)-space} (RMSE, residual plots) and
  stability of fitted parameters under small perturbations of the
  quiet-window bounds.
\item
  \textbf{Sensitivity analyses} over plausible quiet windows and
  skip-weeks values.
\item
  \textbf{Prespecified negative controls}: \(\mathrm{KCOR}(t)\) curves
  should remain near-flat at 1 under control constructions designed to
  induce composition differences without true effects.
\end{itemize}

In practice, prespecified negative controls---such as the age-shift
controls presented in §3.1.2---provide a direct empirical check that
KCOR does not generate artifactual cumulative effects under strong
selection-induced curvature.

\subsubsection{5.2 Conservativeness and edge-case detection
limits}\label{conservativeness-and-edge-case-detection-limits}

Because KCOR compares fixed enrollment cohorts, subsequent uptake of the
intervention among initially unexposed individuals (or additional dosing
among exposed cohorts) introduces treatment crossover over time. Such
crossover attenuates between-cohort contrasts and biases
\(\mathrm{KCOR}(t)\) toward unity, making the estimator conservative
with respect to detecting sustained net benefit or harm. Analyses should
therefore restrict follow-up to periods before substantial crossover or
stratify by dosing state when the data permit.

Because KCOR defines explicit diagnostic failure modes---instability,
dose reversals, age incoherence, or absence of asymptotic
convergence---the absence of such failures in the Czech 2021\_24 Dose 0
versus Dose 2 cohorts provides stronger validation than goodness-of-fit
alone.

\textbf{Conservativeness under overlap.}\\
When treatment effects overlap temporally with the quiet window used for
frailty estimation, \(\mathrm{KCOR}(t)\) does not attribute the
resulting curvature to treatment nor amplify it into a spurious
cumulative effect. Instead, overlap manifests as degraded quiet-window
fit, reduced post-normalization linearity, and instability of estimated
frailty parameters, all of which are explicitly surfaced by KCOR's
diagnostics. As a result, KCOR is conservative under temporal
overlap---preferring diagnostic failure and attenuation over
over-interpretation---rather than producing misleading treatment effects
when separability is not supported by the data. See §2.1.1 and
Supplementary Section S7 for the corresponding identifiability
assumptions and stress tests.

KCOR analyses commonly exclude an initial post-enrollment window to
exclude dynamic Healthy Vaccinee Effect artifacts. If an intervention
induces an acute mortality effect concentrated entirely within this
skipped window, that transient signal will not be captured by the
primary analysis. This limitation is addressed by reporting sensitivity
analyses with reduced or zero skip-weeks and/or by separately evaluating
a prespecified acute-risk window.

In degenerate scenarios where an intervention induces a purely
proportional level-shift in hazard that remains constant over time and
does not alter depletion-driven curvature, KCOR's curvature-based
contrast may have limited ability to distinguish such effects from
residual baseline level differences under minimal-data constraints. Such
cases are pathological in the sense that they produce no detectable
depletion signature; in practice, KCOR diagnostics and control tests
help identify when curvature-based inference is not informative.

Simulation results in §3.4 illustrate that when key assumptions are
violated---such as non-gamma frailty geometry, contamination of the
quiet window by external shocks, or extreme event sparsity---frailty
normalization may become weakly identified. In such regimes, KCOR's
diagnostics, including poor cumulative-hazard fit and reduced
post-normalization linearity, explicitly signal that curvature-based
inference is unreliable without model generalization or revised window
selection.

Increasing model complexity within the Cox regression framework---via
random effects, cohort-specific frailty, or information-criterion--based
selection---does not resolve this limitation, because these models
continue to target instantaneous hazard ratios conditional on survival
rather than cumulative counterfactual outcomes. Model-selection criteria
applied within the Cox regression family favor specifications that
improve likelihood fit of instantaneous hazards, but such criteria do
not validate cumulative counterfactual interpretation under
selection-induced non-proportional hazards.

\subsubsection{5.3 Data requirements and external
validation}\label{data-requirements-and-external-validation}

In finite samples, KCOR precision is driven primarily by the number of
events observed over follow-up. In simulation (selection-only null),
cohorts of approximately 5,000 per arm yielded stable KCOR estimates
with narrow uncertainty, whereas smaller cohorts exhibited appreciable
Monte Carlo variability and occasional spurious deviations. Reporting
event counts and conducting a simple cohort-size sensitivity check are
recommended when applying KCOR to sparse outcomes.

\textbf{External validation across interventions.} A natural next step
is to apply KCOR to other vaccines and interventions where large-scale
individual-level event timing data are available. Many RCTs are
underpowered for all-cause mortality and typically do not provide
record-level timing needed for KCOR-style hazard-space normalization,
while large observational studies often publish only aggregated effect
estimates. Where sufficiently detailed time-to-event data exist
(registries, integrated health systems, or open individual-level
datasets), cross-intervention comparisons can help characterize how
often selection-induced depletion dominates observed hazard curvature
and how frequently post-normalization trajectories remain stable under
negative controls.

\subsection{6. Conclusion}\label{conclusion}

KCOR addresses selection-induced hazard curvature in retrospective
cohort comparisons by explicitly modeling and inverting frailty-driven
depletion prior to comparison. Across synthetic and empirical controls,
KCOR remains near-null under selection without effect and reliably
detects injected effects when present. Rather than presuming
identifiability, KCOR enforces its assumptions diagnostically, flagging
violations through degraded fit, instability, or residual curvature
instead of absorbing them into model-dependent estimates.

\newpage

\subsection{Declarations}\label{declarations}

\subsubsection{Ethics approval and consent to
participate}\label{ethics-approval-and-consent-to-participate}

This study used only simulated data and publicly available, aggregated
registry summaries that contain no individual-level or identifiable
information; as such, it did not constitute human subjects research and
was exempt from institutional review board oversight. The primary
validation results use synthetic data. Empirical negative-control
figures (Figures \ref{fig:neg_control_10yr} and
\ref{fig:neg_control_20yr}) use aggregated cohort summaries derived from
Czech Republic administrative data; no record-level data are shared in
this manuscript.\textsuperscript{13}

\subsubsection{Consent for publication}\label{consent-for-publication}

Not applicable.

\subsubsection{Data availability}\label{data-availability}

This study analyzes aggregated cohort-level summaries derived from
administrative health records. The underlying individual-level data from
the Czech Republic are record-level administrative data collected and
maintained by the National Health Information Portal. Access to the
underlying record-level data is subject to the data provider's
governance, approval, and disclosure-control policies.

\subsubsection{Software availability}\label{software-availability}

The complete KCOR reference implementation, simulation code, and
manuscript build instructions are available at
https://github.com/skirsch/KCOR. A citable archival release of the
software is available via Zenodo (DOI: 10.5281/zenodo.18050329).

All synthetic validation datasets used for method development and
evaluation (including negative and positive control simulations), along
with their generation scripts, are publicly available in the project
repository. Sensitivity analysis outputs and example datasets in
KCOR\_CMR format are included to support full computational
reproducibility. A formal specification of the KCOR data formats,
including schema definitions and disclosure-control semantics, is
provided in \texttt{documentation/specs/KCOR\_file\_format.md}.

\subsubsection{Use of artificial intelligence
tools}\label{use-of-artificial-intelligence-tools}

The KCOR method and estimand were developed by the author without the
use of artificial intelligence (AI) tools. Generative AI tools,
including OpenAI's ChatGPT and Cursor Composer 1, were used during
manuscript preparation to assist with drafting and editing text,
mathematical typesetting, refactoring code, and implementing simulation
studies described in this manuscript.

Simulation designs were either specified by the author or proposed
during iterative discussion and subsequently reviewed and approved by
the author prior to implementation. AI assistance was used to draft code
for approved simulations, which the author reviewed, tested, and
validated. Additional large language models (including Gemini, DeepSeek,
and Claude) were used to provide feedback on manuscript wording and
methodological exposition in a role analogous to informal peer review.

All scientific decisions, methodological choices, analyses,
interpretations, and judgments regarding which suggestions to accept or
reject were made solely by the author, who reviewed and understands all
content and takes full responsibility for the manuscript.

\subsubsection{Competing interests}\label{competing-interests}

The author is a board member of the Vaccine Safety Research Foundation.

\subsubsection{Funding}\label{funding}

This research received no external funding.

\subsubsection{Authors' contributions}\label{authors-contributions}

Steven T. Kirsch conceived the method, wrote the code, performed the
analysis, and wrote the manuscript.

\subsubsection{Acknowledgements}\label{acknowledgements}

The author thanks James Lyons-Weiler. Clare Craig, Jasmin Cardinal
Prévost, Alan Mordue, Ben Jackson, and Paul Fischer for helpful
discussions and methodological feedback during the development of this
work. All errors remain the author's responsibility.

\newpage

\subsection{References}\label{references}

\protect\phantomsection\label{refs}
\begin{CSLReferences}{0}{1}
\bibitem[\citeproctext]{ref-vaupel1979}
\CSLLeftMargin{1. }%
\CSLRightInline{Vaupel JW, Manton KG, Stallard E. The impact of
heterogeneity in individual frailty on the dynamics of mortality.
\emph{Demography}. 1979;16(3):439-454.
doi:\href{https://doi.org/10.2307/2061224}{10.2307/2061224}}

\bibitem[\citeproctext]{ref-grambsch1994}
\CSLLeftMargin{2. }%
\CSLRightInline{Grambsch PM, Therneau TM. Proportional hazards tests and
diagnostics based on weighted residuals. \emph{Biometrika}.
1994;81(3):515-526.
doi:\href{https://doi.org/10.1093/biomet/81.3.515}{10.1093/biomet/81.3.515}}

\bibitem[\citeproctext]{ref-andersen1982}
\CSLLeftMargin{3. }%
\CSLRightInline{Andersen PK, Gill RD. Cox's regression model for
counting processes: a large sample study. \emph{Ann Statist}.
1982;10(4).
doi:\href{https://doi.org/10.1214/aos/1176345976}{10.1214/aos/1176345976}}

\bibitem[\citeproctext]{ref-royston2002}
\CSLLeftMargin{4. }%
\CSLRightInline{Royston P, Parmar MKB. Flexible parametric
proportional‐hazards and proportional‐odds models for censored survival
data, with application to prognostic modelling and estimation of
treatment effects. \emph{Statistics in Medicine}. 2002;21(15):2175-2197.
doi:\href{https://doi.org/10.1002/sim.1203}{10.1002/sim.1203}}

\bibitem[\citeproctext]{ref-aalen1989}
\CSLLeftMargin{5. }%
\CSLRightInline{Aalen OO. A linear regression model for the analysis of
life times. \emph{Statistics in Medicine}. 1989;8(8):907-925.
doi:\href{https://doi.org/10.1002/sim.4780080803}{10.1002/sim.4780080803}}

\bibitem[\citeproctext]{ref-lin1994}
\CSLLeftMargin{6. }%
\CSLRightInline{Lin DY, Ying Z. Semiparametric analysis of the additive
risk model. \emph{Biometrika}. 1994;81(1):61-71.
doi:\href{https://doi.org/10.1093/biomet/81.1.61}{10.1093/biomet/81.1.61}}

\bibitem[\citeproctext]{ref-vanhouwelingen2007}
\CSLLeftMargin{7. }%
\CSLRightInline{Van Houwelingen HC. Dynamic prediction by landmarking in
event history analysis. \emph{Scandinavian J Statistics}.
2007;34(1):70-85.
doi:\href{https://doi.org/10.1111/j.1467-9469.2006.00529.x}{10.1111/j.1467-9469.2006.00529.x}}

\bibitem[\citeproctext]{ref-robins2000}
\CSLLeftMargin{8. }%
\CSLRightInline{Robins JM, Hernán MÁ, Brumback B. Marginal structural
models and causal inference in epidemiology: \emph{Epidemiology}.
2000;11(5):550-560.
doi:\href{https://doi.org/10.1097/00001648-200009000-00011}{10.1097/00001648-200009000-00011}}

\bibitem[\citeproctext]{ref-cole2008}
\CSLLeftMargin{9. }%
\CSLRightInline{Cole SR, Hernan MA. Constructing inverse probability
weights for marginal structural models. \emph{American Journal of
Epidemiology}. 2008;168(6):656-664.
doi:\href{https://doi.org/10.1093/aje/kwn164}{10.1093/aje/kwn164}}

\bibitem[\citeproctext]{ref-obel2024}
\CSLLeftMargin{10. }%
\CSLRightInline{Obel N, Fox M, Tetens M, et al. Confounding and negative
control methods in observational study of SARS-CoV-2 vaccine
effectiveness: a nationwide, population-based Danish health registry
study. \emph{CLEP}. 2024;Volume 16:501-512.
doi:\href{https://doi.org/10.2147/CLEP.S468572}{10.2147/CLEP.S468572}}

\bibitem[\citeproctext]{ref-chemaitelly2025}
\CSLLeftMargin{11. }%
\CSLRightInline{Chemaitelly H, Ayoub HH, Coyle P, et al. Assessing
healthy vaccinee effect in COVID-19 vaccine effectiveness studies: a
national cohort study in Qatar. Schiffer JT, Henry D, eds. \emph{eLife}.
2025;14:e103690.
doi:\href{https://doi.org/10.7554/eLife.103690}{10.7554/eLife.103690}}

\bibitem[\citeproctext]{ref-deeks2003}
\CSLLeftMargin{12. }%
\CSLRightInline{Deeks J, Dinnes J, D'Amico R, et al. Evaluating
non-randomised intervention studies. \emph{Health Technol Assess}.
2003;7(27). doi:\href{https://doi.org/10.3310/hta7270}{10.3310/hta7270}}

\bibitem[\citeproctext]{ref-sanca2024}
\CSLLeftMargin{13. }%
\CSLRightInline{Šanca O, Jarkovský J, Klimeš D, et al.
\emph{Vaccination, Positivity, Hospitalization for COVID-19, Deaths,
Long COVID and Comorbidities in People in the Czech Republic}. National
Health Information Portal, Czech Republic; 2024. Accessed December 20,
2024. \url{https://www.nzip.cz/clanek/2135-covid-19-prehled-populace}}

\end{CSLReferences}

\newpage

\subsection{Tables}\label{tables}

\begin{longtable}[]{@{}
  >{\raggedright\arraybackslash}p{(\linewidth - 8\tabcolsep) * \real{0.2000}}
  >{\raggedright\arraybackslash}p{(\linewidth - 8\tabcolsep) * \real{0.2000}}
  >{\raggedright\arraybackslash}p{(\linewidth - 8\tabcolsep) * \real{0.2000}}
  >{\raggedright\arraybackslash}p{(\linewidth - 8\tabcolsep) * \real{0.2000}}
  >{\raggedright\arraybackslash}p{(\linewidth - 8\tabcolsep) * \real{0.2000}}@{}}
\caption{Summary of two large matched observational studies showing
residual confounding / HVE despite meticulous
matching.}\label{tbl:HVE_motivation}\tabularnewline
\toprule\noalign{}
\begin{minipage}[b]{\linewidth}\raggedright
Study
\end{minipage} & \begin{minipage}[b]{\linewidth}\raggedright
Design
\end{minipage} & \begin{minipage}[b]{\linewidth}\raggedright
Matching/adjustment
\end{minipage} & \begin{minipage}[b]{\linewidth}\raggedright
Key control finding
\end{minipage} & \begin{minipage}[b]{\linewidth}\raggedright
Implication for methods
\end{minipage} \\
\midrule\noalign{}
\endfirsthead
\toprule\noalign{}
\begin{minipage}[b]{\linewidth}\raggedright
Study
\end{minipage} & \begin{minipage}[b]{\linewidth}\raggedright
Design
\end{minipage} & \begin{minipage}[b]{\linewidth}\raggedright
Matching/adjustment
\end{minipage} & \begin{minipage}[b]{\linewidth}\raggedright
Key control finding
\end{minipage} & \begin{minipage}[b]{\linewidth}\raggedright
Implication for methods
\end{minipage} \\
\midrule\noalign{}
\endhead
\bottomrule\noalign{}
\endlastfoot
Obel et al.~(Denmark)\textsuperscript{10} & Nationwide registry cohorts
(60--90y) & 1:1 match on age/sex + covariate adjustment; negative
control outcomes & Vaccinated had higher rates of multiple negative
control outcomes, but substantially lower mortality after unrelated
diagnoses & Strong evidence of confounding in observational VE
estimates; ``negative control methods indicate\ldots{} substantial
confounding'' \\
Chemaitelly et al.~(Qatar)\textsuperscript{11} & Matched national
cohorts (primary series and booster) & Exact 1:1 matching on
demographics + coexisting conditions + prior infection; Cox models &
Strong early reduction in non-COVID mortality (HVE), with time-varying
reversal later & Even meticulous matching leaves time-varying residual
differences consistent with selection/frailty depletion \\
\end{longtable}

\newpage

\begin{longtable}[]{@{}
  >{\raggedright\arraybackslash}p{(\linewidth - 6\tabcolsep) * \real{0.4023}}
  >{\raggedright\arraybackslash}p{(\linewidth - 6\tabcolsep) * \real{0.1379}}
  >{\raggedright\arraybackslash}p{(\linewidth - 6\tabcolsep) * \real{0.1954}}
  >{\raggedright\arraybackslash}p{(\linewidth - 6\tabcolsep) * \real{0.2644}}@{}}
\caption{Comparison of Cox proportional hazards, Cox with frailty, and
KCOR across key methodological
dimensions.}\label{tbl:cox_vs_kcor}\tabularnewline
\toprule\noalign{}
\begin{minipage}[b]{\linewidth}\raggedright
Feature
\end{minipage} & \begin{minipage}[b]{\linewidth}\raggedright
Cox PH
\end{minipage} & \begin{minipage}[b]{\linewidth}\raggedright
Cox + frailty
\end{minipage} & \begin{minipage}[b]{\linewidth}\raggedright
KCOR
\end{minipage} \\
\midrule\noalign{}
\endfirsthead
\toprule\noalign{}
\begin{minipage}[b]{\linewidth}\raggedright
Feature
\end{minipage} & \begin{minipage}[b]{\linewidth}\raggedright
Cox PH
\end{minipage} & \begin{minipage}[b]{\linewidth}\raggedright
Cox + frailty
\end{minipage} & \begin{minipage}[b]{\linewidth}\raggedright
KCOR
\end{minipage} \\
\midrule\noalign{}
\endhead
\bottomrule\noalign{}
\endlastfoot
Primary estimand & Hazard ratio & Hazard ratio & Cumulative hazard
ratio \\
Conditions on survival & Yes & Yes & No \\
Assumes PH & Yes & Yes (conditional) & No \\
Frailty role & None & Nuisance & Object of inference \\
Uses partial likelihood & Yes & Yes & No \\
Handles selection-induced curvature & No & Partial & Yes (targeted) \\
Output interpretable under non-PH & No & No & Yes (cumulative) \\
\end{longtable}

Note: KCOR is reported here as a cumulative hazard ratio for
comparability; alternative post-normalization estimands are admissible
within the framework.

\newpage

\begin{longtable}[]{@{}
  >{\raggedright\arraybackslash}p{(\linewidth - 10\tabcolsep) * \real{0.2065}}
  >{\raggedright\arraybackslash}p{(\linewidth - 10\tabcolsep) * \real{0.1685}}
  >{\raggedright\arraybackslash}p{(\linewidth - 10\tabcolsep) * \real{0.1196}}
  >{\raggedright\arraybackslash}p{(\linewidth - 10\tabcolsep) * \real{0.1957}}
  >{\raggedright\arraybackslash}p{(\linewidth - 10\tabcolsep) * \real{0.1304}}
  >{\raggedright\arraybackslash}p{(\linewidth - 10\tabcolsep) * \real{0.1793}}@{}}
\caption{Positioning KCOR relative to non-proportional hazards
methods.}\label{tbl:positioning}\tabularnewline
\toprule\noalign{}
\begin{minipage}[b]{\linewidth}\raggedright
Method class
\end{minipage} & \begin{minipage}[b]{\linewidth}\raggedright
Primary target
\end{minipage} & \begin{minipage}[b]{\linewidth}\raggedright
What is modeled
\end{minipage} & \begin{minipage}[b]{\linewidth}\raggedright
Handles selection-induced depletion?
\end{minipage} & \begin{minipage}[b]{\linewidth}\raggedright
Typical output
\end{minipage} & \begin{minipage}[b]{\linewidth}\raggedright
Failure under latent frailty
\end{minipage} \\
\midrule\noalign{}
\endfirsthead
\toprule\noalign{}
\begin{minipage}[b]{\linewidth}\raggedright
Method class
\end{minipage} & \begin{minipage}[b]{\linewidth}\raggedright
Primary target
\end{minipage} & \begin{minipage}[b]{\linewidth}\raggedright
What is modeled
\end{minipage} & \begin{minipage}[b]{\linewidth}\raggedright
Handles selection-induced depletion?
\end{minipage} & \begin{minipage}[b]{\linewidth}\raggedright
Typical output
\end{minipage} & \begin{minipage}[b]{\linewidth}\raggedright
Failure under latent frailty
\end{minipage} \\
\midrule\noalign{}
\endhead
\bottomrule\noalign{}
\endlastfoot
Cox PH & Instantaneous hazard & Linear predictor & No & HR & Non-PH from
depletion → biased HR \\
Time-varying Cox & Instantaneous hazard & Time-varying \(\beta(t)\) & No
& HR(t) & Fits depletion as signal \\
Flexible parametric survival (splines) & Survival / hazard shape &
Baseline hazard & No & Smooth hazard / survival & Absorbs depletion
curvature \\
Additive hazards (Aalen) & Hazard differences & Additive hazard & No &
\(\Delta h(t)\) & Still conditional on survival \\
RMST & Mean survival & Survival curve & No & RMST & Inherits depletion
bias \\
Frailty regression & Heterogeneity- adjusted HR & Random effects &
Partial & HR & Frailty treated as nuisance \\
\textbf{KCOR (this work)} & \textbf{Cumulative outcome contrast} &
\textbf{Depletion geometry} & \textbf{Yes (targeted)} &
\textbf{\(\mathrm{KCOR}(t)\)} & Diagnostics flag failure \\
\end{longtable}

\newpage

\begin{longtable}[]{@{}
  >{\raggedright\arraybackslash}p{(\linewidth - 2\tabcolsep) * \real{0.4000}}
  >{\raggedright\arraybackslash}p{(\linewidth - 2\tabcolsep) * \real{0.6000}}@{}}
\caption{Notation used throughout the Methods
section.}\label{tbl:notation}\tabularnewline
\toprule\noalign{}
\begin{minipage}[b]{\linewidth}\raggedright
Symbol
\end{minipage} & \begin{minipage}[b]{\linewidth}\raggedright
Definition
\end{minipage} \\
\midrule\noalign{}
\endfirsthead
\toprule\noalign{}
\begin{minipage}[b]{\linewidth}\raggedright
Symbol
\end{minipage} & \begin{minipage}[b]{\linewidth}\raggedright
Definition
\end{minipage} \\
\midrule\noalign{}
\endhead
\bottomrule\noalign{}
\endlastfoot
\(d\) & Cohort index \\
\(A,B\) & Indices of the two cohorts compared in a KCOR contrast \\
\(t\) & Event time since enrollment (discrete bins, e.g., weeks) \\
\(h_{\mathrm{obs},d}(t)\) & Discrete-time cohort hazard (conditional on
\(N_d(t)\)) \\
\(H_{\mathrm{obs},d}(t)\) & Observed cumulative hazard (after
skip/stabilization) \\
\(\tilde h_{0,d}(t)\) & Depletion-neutralized baseline hazard for cohort
\(d\) \\
\(\tilde H_{0,d}(t)\) & Depletion-neutralized baseline cumulative hazard
for cohort \(d\) \\
\(\theta_d\) & Frailty variance (selection strength) for cohort \(d\);
governs curvature in the observed cumulative hazard \\
\(\hat{\theta}_d\) & Estimated frailty variance from quiet-window
fitting \\
\(k_d\) & Baseline hazard level for cohort \(d\) under the default
baseline shape \\
\(\hat{k}_d\) & Estimated baseline hazard level from quiet-window
fitting \\
\(t_0\) & Anchor time for baseline normalization (prespecified) \\
\(\mathrm{KCOR}(t; t_0)\) & Anchored KCOR:
\(\mathrm{KCOR}(t)/\mathrm{KCOR}(t_0)\) \\
\end{longtable}

\newpage

\begin{longtable}[]{@{}
  >{\raggedright\arraybackslash}p{(\linewidth - 8\tabcolsep) * \real{0.2000}}
  >{\raggedright\arraybackslash}p{(\linewidth - 8\tabcolsep) * \real{0.2000}}
  >{\raggedright\arraybackslash}p{(\linewidth - 8\tabcolsep) * \real{0.2000}}
  >{\raggedright\arraybackslash}p{(\linewidth - 8\tabcolsep) * \real{0.2000}}
  >{\raggedright\arraybackslash}p{(\linewidth - 8\tabcolsep) * \real{0.2000}}@{}}
\caption{Step-by-step KCOR algorithm (high-level), with recommended
prespecification and diagnostics. All analysis choices and estimation
procedures are prespecified; numerical parameters such as \(\theta_d\)
are estimated from the data within the prespecified
framework.}\label{tbl:KCOR_algorithm}\tabularnewline
\toprule\noalign{}
\begin{minipage}[b]{\linewidth}\raggedright
Step
\end{minipage} & \begin{minipage}[b]{\linewidth}\raggedright
Operation
\end{minipage} & \begin{minipage}[b]{\linewidth}\raggedright
Output
\end{minipage} & \begin{minipage}[b]{\linewidth}\raggedright
Prespecify?
\end{minipage} & \begin{minipage}[b]{\linewidth}\raggedright
Diagnostics
\end{minipage} \\
\midrule\noalign{}
\endfirsthead
\toprule\noalign{}
\begin{minipage}[b]{\linewidth}\raggedright
Step
\end{minipage} & \begin{minipage}[b]{\linewidth}\raggedright
Operation
\end{minipage} & \begin{minipage}[b]{\linewidth}\raggedright
Output
\end{minipage} & \begin{minipage}[b]{\linewidth}\raggedright
Prespecify?
\end{minipage} & \begin{minipage}[b]{\linewidth}\raggedright
Diagnostics
\end{minipage} \\
\midrule\noalign{}
\endhead
\bottomrule\noalign{}
\endlastfoot
1 & Choose enrollment date and define fixed cohorts & Cohort labels &
Yes & Verify cohort sizes/risk sets \\
2 & Compute discrete-time hazards (observed hazards) & Hazard curves &
Yes (binning/transform) & Check for zeros/sparsity \\
3 & Apply stabilization skip and accumulate observed cumulative hazards
& Observed cumulative hazards & Yes (skip rule) & Plot observed
cumulative hazards \\
4 & Select quiet-window bins in calendar ISO-week space & Fit points
\(\mathcal{T}_d\) & Yes & Overlay quiet window on hazard plots \\
5 & Fit \((\hat{k}_d,\hat{\theta}_d)\) via cumulative-hazard least
squares & Fitted parameters & Yes (estimation procedure) & RMSE,
residuals, fit stability \\
6 & Normalize: invert gamma-frailty identity to depletion-neutralized
cumulative hazards & Depletion-neutralized cumulative hazards & Yes &
Compare pre/post shapes; sanity checks \\
7 & Cumulate and ratio: compute \(\mathrm{KCOR}(t)\) &
\(\mathrm{KCOR}(t)\) curve & Yes (horizon) & Flat under negative
controls \\
8 & Uncertainty & CI / intervals & Yes & Coverage on positive
controls \\
\end{longtable}

\newpage

\begin{longtable}[]{@{}
  >{\raggedleft\arraybackslash}p{(\linewidth - 10\tabcolsep) * \real{0.1364}}
  >{\raggedleft\arraybackslash}p{(\linewidth - 10\tabcolsep) * \real{0.0909}}
  >{\raggedleft\arraybackslash}p{(\linewidth - 10\tabcolsep) * \real{0.0909}}
  >{\raggedleft\arraybackslash}p{(\linewidth - 10\tabcolsep) * \real{0.1667}}
  >{\raggedleft\arraybackslash}p{(\linewidth - 10\tabcolsep) * \real{0.2121}}
  >{\raggedleft\arraybackslash}p{(\linewidth - 10\tabcolsep) * \real{0.3030}}@{}}
\caption{Cox vs KCOR under a synthetic null with increasing frailty
heterogeneity. Two cohorts are simulated with identical baseline hazards
and no treatment effect \emph{(null by construction)}; cohorts differ
only in gamma frailty variance (\(\theta\)). Despite the true hazard
ratio being 1 by construction, Cox regression produces increasingly
non-null hazard ratios as \(\theta\) increases, reflecting
depletion-induced non-proportional hazards. \(\mathrm{KCOR}(t)\) remains
centered near unity with negligible post-normalization slope across
\(\theta\) values. (Exact values depend on simulation seed and follow-up
horizon.)}\label{tbl:cox_bias_demo}\tabularnewline
\toprule\noalign{}
\begin{minipage}[b]{\linewidth}\raggedleft
\(\theta\)
\end{minipage} & \begin{minipage}[b]{\linewidth}\raggedleft
Cox HR
\end{minipage} & \begin{minipage}[b]{\linewidth}\raggedleft
95\% CI
\end{minipage} & \begin{minipage}[b]{\linewidth}\raggedleft
Cox p-value
\end{minipage} & \begin{minipage}[b]{\linewidth}\raggedleft
KCOR asymptote
\end{minipage} & \begin{minipage}[b]{\linewidth}\raggedleft
KCOR post-norm slope
\end{minipage} \\
\midrule\noalign{}
\endfirsthead
\toprule\noalign{}
\begin{minipage}[b]{\linewidth}\raggedleft
\(\theta\)
\end{minipage} & \begin{minipage}[b]{\linewidth}\raggedleft
Cox HR
\end{minipage} & \begin{minipage}[b]{\linewidth}\raggedleft
95\% CI
\end{minipage} & \begin{minipage}[b]{\linewidth}\raggedleft
Cox p-value
\end{minipage} & \begin{minipage}[b]{\linewidth}\raggedleft
KCOR asymptote
\end{minipage} & \begin{minipage}[b]{\linewidth}\raggedleft
KCOR post-norm slope
\end{minipage} \\
\midrule\noalign{}
\endhead
\bottomrule\noalign{}
\endlastfoot
0.0 & 0.988 & {[}0.969, 1.008{]} & 0.234 & 0.988 &
\(7.6 \times 10^{-4}\) \\
0.5 & 0.965 & {[}0.946, 0.985{]} & \(4.9 \times 10^{-4}\) & 0.990 &
\(-3.8 \times 10^{-5}\) \\
1.0 & 0.944 & {[}0.926, 0.963{]} & \(1.7 \times 10^{-8}\) & 0.992 &
\(-3.0 \times 10^{-4}\) \\
2.0 & 0.902 & {[}0.884, 0.921{]} & \(2.4 \times 10^{-23}\) & 0.991 &
\(3.7 \times 10^{-4}\) \\
5.0 & 0.804 & {[}0.787, 0.820{]} & \(1.5 \times 10^{-93}\) & 0.993 &
\(-5.3 \times 10^{-4}\) \\
10.0 & 0.701 & {[}0.686, 0.717{]} & \(<10^{-200}\) & 1.020 &
\(3.2 \times 10^{-4}\) \\
20.0 & 0.551 & {[}0.539, 0.564{]} & \(<10^{-300}\) & 1.024 &
\(-1.6 \times 10^{-4}\) \\
\end{longtable}

\newpage

\begin{longtable}[]{@{}llrl@{}}
\caption{Example end-of-window \(\mathrm{KCOR}(t)\) values from the
empirical negative control (pooled/ASMR summaries), showing near-null
behavior under large composition differences. (Source:
\protect\texttt{test/negative\_control/out/KCOR\_summary.log})}\label{tbl:neg_control_summary}\tabularnewline
\toprule\noalign{}
Enrollment & Dose comparison & KCOR (pooled/ASMR) & 95\% CI \\
\midrule\noalign{}
\endfirsthead
\toprule\noalign{}
Enrollment & Dose comparison & KCOR (pooled/ASMR) & 95\% CI \\
\midrule\noalign{}
\endhead
\bottomrule\noalign{}
\endlastfoot
2021\_24 & 1 vs 0 & 1.0097 & {[}0.992, 1.027{]} \\
2021\_24 & 2 vs 0 & 1.0213 & {[}1.000, 1.043{]} \\
2021\_24 & 2 vs 1 & 1.0115 & {[}0.991, 1.033{]} \\
2022\_06 & 1 vs 0 & 0.9858 & {[}0.970, 1.002{]} \\
2022\_06 & 2 vs 0 & 1.0756 & {[}1.055, 1.097{]} \\
2022\_06 & 2 vs 1 & 1.0911 & {[}1.070, 1.112{]} \\
\end{longtable}

\newpage

\begin{longtable}[]{@{}
  >{\raggedright\arraybackslash}p{(\linewidth - 8\tabcolsep) * \real{0.1765}}
  >{\raggedright\arraybackslash}p{(\linewidth - 8\tabcolsep) * \real{0.1765}}
  >{\raggedleft\arraybackslash}p{(\linewidth - 8\tabcolsep) * \real{0.2353}}
  >{\raggedright\arraybackslash}p{(\linewidth - 8\tabcolsep) * \real{0.1765}}
  >{\raggedleft\arraybackslash}p{(\linewidth - 8\tabcolsep) * \real{0.2353}}@{}}
\caption{Positive control results comparing injected hazard multipliers
to detected KCOR deviations. Both scenarios show KCOR deviating from 1.0
in the expected direction, validating that the estimator can detect true
effects.}\label{tbl:pos_control_summary}\tabularnewline
\toprule\noalign{}
\begin{minipage}[b]{\linewidth}\raggedright
Scenario
\end{minipage} & \begin{minipage}[b]{\linewidth}\raggedright
Effect window
\end{minipage} & \begin{minipage}[b]{\linewidth}\raggedleft
Hazard multiplier \(r\)
\end{minipage} & \begin{minipage}[b]{\linewidth}\raggedright
Expected direction
\end{minipage} & \begin{minipage}[b]{\linewidth}\raggedleft
Observed \(\mathrm{KCOR}(t)\) at week 80
\end{minipage} \\
\midrule\noalign{}
\endfirsthead
\toprule\noalign{}
\begin{minipage}[b]{\linewidth}\raggedright
Scenario
\end{minipage} & \begin{minipage}[b]{\linewidth}\raggedright
Effect window
\end{minipage} & \begin{minipage}[b]{\linewidth}\raggedleft
Hazard multiplier \(r\)
\end{minipage} & \begin{minipage}[b]{\linewidth}\raggedright
Expected direction
\end{minipage} & \begin{minipage}[b]{\linewidth}\raggedleft
Observed \(\mathrm{KCOR}(t)\) at week 80
\end{minipage} \\
\midrule\noalign{}
\endhead
\bottomrule\noalign{}
\endlastfoot
Benefit & week 20--80 & 0.8 & \textless{} 1 & 0.825 \\
Harm & week 20--80 & 1.2 & \textgreater{} 1 & 1.107 \\
\end{longtable}

\newpage

\begin{longtable}[]{@{}
  >{\raggedright\arraybackslash}p{(\linewidth - 14\tabcolsep) * \real{0.0699}}
  >{\raggedright\arraybackslash}p{(\linewidth - 14\tabcolsep) * \real{0.1189}}
  >{\raggedright\arraybackslash}p{(\linewidth - 14\tabcolsep) * \real{0.0559}}
  >{\raggedright\arraybackslash}p{(\linewidth - 14\tabcolsep) * \real{0.1538}}
  >{\raggedright\arraybackslash}p{(\linewidth - 14\tabcolsep) * \real{0.1189}}
  >{\raggedright\arraybackslash}p{(\linewidth - 14\tabcolsep) * \real{0.1469}}
  >{\raggedright\arraybackslash}p{(\linewidth - 14\tabcolsep) * \real{0.1888}}
  >{\raggedright\arraybackslash}p{(\linewidth - 14\tabcolsep) * \real{0.1469}}@{}}
\caption{Comparison of Cox regression, shared frailty Cox models, and
KCOR under selection-only and joint frailty + treatment effect
scenarios. Results are from S7 simulation (joint frailty + treatment)
and gamma-frailty null scenario (selection-only). Standard Cox
regression produces non-null hazard ratios under selection-only
conditions due to depletion dynamics. Shared frailty Cox models
partially mitigate this bias but still exhibit residual non-null
behavior. KCOR remains near-null under selection-only conditions and
correctly detects treatment effects when temporal separability
holds.}\label{tbl:joint_frailty_comparison}\tabularnewline
\toprule\noalign{}
\begin{minipage}[b]{\linewidth}\raggedright
Scenario
\end{minipage} & \begin{minipage}[b]{\linewidth}\raggedright
True effect (r)
\end{minipage} & \begin{minipage}[b]{\linewidth}\raggedright
Cox HR
\end{minipage} & \begin{minipage}[b]{\linewidth}\raggedright
Shared frailty Cox HR
\end{minipage} & \begin{minipage}[b]{\linewidth}\raggedright
KCOR drift/year
\end{minipage} & \begin{minipage}[b]{\linewidth}\raggedright
Cox indicates null?
\end{minipage} & \begin{minipage}[b]{\linewidth}\raggedright
Frailty-Cox indicates null?
\end{minipage} & \begin{minipage}[b]{\linewidth}\raggedright
KCOR indicates null?
\end{minipage} \\
\midrule\noalign{}
\endfirsthead
\toprule\noalign{}
\begin{minipage}[b]{\linewidth}\raggedright
Scenario
\end{minipage} & \begin{minipage}[b]{\linewidth}\raggedright
True effect (r)
\end{minipage} & \begin{minipage}[b]{\linewidth}\raggedright
Cox HR
\end{minipage} & \begin{minipage}[b]{\linewidth}\raggedright
Shared frailty Cox HR
\end{minipage} & \begin{minipage}[b]{\linewidth}\raggedright
KCOR drift/year
\end{minipage} & \begin{minipage}[b]{\linewidth}\raggedright
Cox indicates null?
\end{minipage} & \begin{minipage}[b]{\linewidth}\raggedright
Frailty-Cox indicates null?
\end{minipage} & \begin{minipage}[b]{\linewidth}\raggedright
KCOR indicates null?
\end{minipage} \\
\midrule\noalign{}
\endhead
\bottomrule\noalign{}
\endlastfoot
Gamma-frailty null & 1.0 (null) & 0.87 & 0.94 & \textless{} 0.5\% & No
(HR \(\neq 1\)) & No (HR \(\neq 1\)) & Yes (flat) \\
S7 harm (r=1.2) & 1.2 & 1.18 & 1.19 & +1.8\% & No (detects effect) & No
(detects effect) & No (detects effect) \\
S7 benefit (r=0.8) & 0.8 & 0.83 & 0.82 & -2.1\% & No (detects effect) &
No (detects effect) & No (detects effect) \\
\end{longtable}

\newpage

\begin{longtable}[]{@{}
  >{\raggedright\arraybackslash}p{(\linewidth - 8\tabcolsep) * \real{0.0792}}
  >{\raggedright\arraybackslash}p{(\linewidth - 8\tabcolsep) * \real{0.1584}}
  >{\raggedright\arraybackslash}p{(\linewidth - 8\tabcolsep) * \real{0.3168}}
  >{\raggedright\arraybackslash}p{(\linewidth - 8\tabcolsep) * \real{0.2079}}
  >{\raggedright\arraybackslash}p{(\linewidth - 8\tabcolsep) * \real{0.2376}}@{}}
\caption{Simulation comparison of KCOR and alternative estimands under
selection-induced non-proportional hazards. Results are summarized
across simulation scenarios (null scenarios: gamma-frailty null,
non-gamma frailty, contamination, sparse events; effect scenarios:
injected hazard increase/decrease). KCOR remains stable under
selection-only regimes, while RMST inherits depletion bias and
time-varying Cox captures non-proportional hazards without normalizing
selection geometry. All methods were applied to identical simulation
outputs.}\label{tbl:comparison_estimands}\tabularnewline
\toprule\noalign{}
\begin{minipage}[b]{\linewidth}\raggedright
Method
\end{minipage} & \begin{minipage}[b]{\linewidth}\raggedright
Target estimand
\end{minipage} & \begin{minipage}[b]{\linewidth}\raggedright
Deviation from null (selection-only scenarios)
\end{minipage} & \begin{minipage}[b]{\linewidth}\raggedright
Variance/instability
\end{minipage} & \begin{minipage}[b]{\linewidth}\raggedright
Interpretability notes
\end{minipage} \\
\midrule\noalign{}
\endfirsthead
\toprule\noalign{}
\begin{minipage}[b]{\linewidth}\raggedright
Method
\end{minipage} & \begin{minipage}[b]{\linewidth}\raggedright
Target estimand
\end{minipage} & \begin{minipage}[b]{\linewidth}\raggedright
Deviation from null (selection-only scenarios)
\end{minipage} & \begin{minipage}[b]{\linewidth}\raggedright
Variance/instability
\end{minipage} & \begin{minipage}[b]{\linewidth}\raggedright
Interpretability notes
\end{minipage} \\
\midrule\noalign{}
\endhead
\bottomrule\noalign{}
\endlastfoot
\textbf{KCOR} & Cumulative hazard ratio (depletion-normalized) & Near
zero (median KCOR \(\approx 1.0\)) & Low (stable trajectory) & Stable
under selection-induced depletion; normalization precedes comparison \\
\textbf{RMST} & Restricted mean survival time & Non-zero (depends on
depletion strength) & Moderate (depends on depletion strength) &
Summarizes survival differences that may reflect depletion rather than
treatment effect; does not normalize selection geometry \\
\textbf{Cox} & Time-varying hazard ratio & Non-zero under frailty
heterogeneity & Moderate (HR instability across time windows) & Improves
fit to non-proportional hazards but does not normalize selection
geometry; inherits depletion structure \\
\end{longtable}

\newpage

\begin{longtable}[]{@{}
  >{\raggedright\arraybackslash}p{(\linewidth - 6\tabcolsep) * \real{0.1887}}
  >{\raggedright\arraybackslash}p{(\linewidth - 6\tabcolsep) * \real{0.3208}}
  >{\raggedright\arraybackslash}p{(\linewidth - 6\tabcolsep) * \real{0.3585}}
  >{\raggedright\arraybackslash}p{(\linewidth - 6\tabcolsep) * \real{0.1321}}@{}}
\caption{Bootstrap coverage for KCOR uncertainty intervals. Coverage is
evaluated across simulation scenarios using stratified bootstrap
resampling. Nominal 95\% confidence intervals are compared to empirical
coverage (proportion of simulations where the true value lies within the
interval).}\label{tbl:bootstrap_coverage}\tabularnewline
\toprule\noalign{}
\begin{minipage}[b]{\linewidth}\raggedright
Scenario
\end{minipage} & \begin{minipage}[b]{\linewidth}\raggedright
Nominal coverage
\end{minipage} & \begin{minipage}[b]{\linewidth}\raggedright
Empirical coverage
\end{minipage} & \begin{minipage}[b]{\linewidth}\raggedright
Notes
\end{minipage} \\
\midrule\noalign{}
\endfirsthead
\toprule\noalign{}
\begin{minipage}[b]{\linewidth}\raggedright
Scenario
\end{minipage} & \begin{minipage}[b]{\linewidth}\raggedright
Nominal coverage
\end{minipage} & \begin{minipage}[b]{\linewidth}\raggedright
Empirical coverage
\end{minipage} & \begin{minipage}[b]{\linewidth}\raggedright
Notes
\end{minipage} \\
\midrule\noalign{}
\endhead
\bottomrule\noalign{}
\endlastfoot
Gamma-frailty null & 95\% & 94.2\% & Coverage evaluated under
selection-only conditions \\
Injected effect (harm) & 95\% & 93.8\% & Coverage evaluated under known
treatment effect \\
Injected effect (benefit) & 95\% & 93.5\% & Coverage evaluated under
known treatment effect \\
Non-gamma frailty & 95\% & 89.3\% & Coverage under frailty
misspecification \\
Sparse events & 95\% & 87.6\% & Coverage under reduced event counts \\
\end{longtable}

\section{KCOR Supplementary Information
(SI)}\label{kcor-supplementary-information-si}

% Enforce Supplementary (S) numbering for figures and tables in the SI PDF.
\renewcommand{\thefigure}{S\arabic{figure}}%
\setcounter{figure}{0}%
\renewcommand{\thetable}{S\arabic{table}}%
\setcounter{table}{0}%

This document provides supplementary material supporting the KCOR
methodology described in the main manuscript, including extended
derivations, simulation studies, robustness analyses, and additional
empirical results.

\subsection{S1. Overview}\label{s1.-overview}

This SI is organized as follows:

\begin{itemize}
\tightlist
\item
  \textbf{S1}: Overview
\item
  \textbf{S2}: Extended diagnostics and failure modes
\item
  \textbf{S3}: Positive controls (injected harm/benefit)
\item
  \textbf{S4}: Control-test specifications and simulation parameters
\item
  \textbf{S5}: Additional figures and diagnostics
\item
  \textbf{S6}: Extended Czech empirical application / illustrative
  registry analysis
\end{itemize}

\subsection{S2. Extended diagnostics and failure
modes}\label{s2.-extended-diagnostics-and-failure-modes}

This section describes the \textbf{observable diagnostics and failure
modes} associated with the KCOR working assumptions and the
corresponding diagnostics and identifiability criteria. No additional
assumptions are introduced here. KCOR is designed to \textbf{fail
transparently rather than silently}: when an assumption is violated, the
resulting lack of identifiability or model stress manifests through
explicit diagnostic signals rather than spurious estimates.

The KCOR framework separates \textbf{working assumptions},
\textbf{empirical diagnostics}, and \textbf{identifiability criteria};
these are summarized below in Tables
\ref{tbl:si_assumptions}--\ref{tbl:si_identifiability}.

\begin{longtable}[]{@{}
  >{\raggedright\arraybackslash}p{(\linewidth - 4\tabcolsep) * \real{0.3333}}
  >{\raggedright\arraybackslash}p{(\linewidth - 4\tabcolsep) * \real{0.3333}}
  >{\raggedright\arraybackslash}p{(\linewidth - 4\tabcolsep) * \real{0.3333}}@{}}
\caption{KCOR working
assumptions.}\label{tbl:si_assumptions}\tabularnewline
\toprule\noalign{}
\begin{minipage}[b]{\linewidth}\raggedright
Assumption
\end{minipage} & \begin{minipage}[b]{\linewidth}\raggedright
Description
\end{minipage} & \begin{minipage}[b]{\linewidth}\raggedright
Role in KCOR
\end{minipage} \\
\midrule\noalign{}
\endfirsthead
\toprule\noalign{}
\begin{minipage}[b]{\linewidth}\raggedright
Assumption
\end{minipage} & \begin{minipage}[b]{\linewidth}\raggedright
Description
\end{minipage} & \begin{minipage}[b]{\linewidth}\raggedright
Role in KCOR
\end{minipage} \\
\midrule\noalign{}
\endhead
\bottomrule\noalign{}
\endlastfoot
A1 Cohort stability & Cohorts are fixed at enrollment with no
post-enrollment switching or informative censoring. & Ensures cumulative
hazards are comparable over follow-up \\
A2 Shared external hazard environment & Cohorts experience the same
background hazard over the comparison window. & Prevents confounding by
cohort-specific shocks \\
A3 Time-invariant latent frailty & Selection operates through
time-invariant unobserved heterogeneity inducing depletion. & Enables
geometric normalization of curvature \\
A4 Adequacy of gamma frailty & Gamma frailty provides a reasonable
approximation to observed depletion geometry. & Allows tractable
inversion and normalization \\
A5 Quiet-window validity & A prespecified window exists in which
depletion dominates other curvature sources. & Permits identification of
frailty parameters \\
\end{longtable}

\begin{longtable}[]{@{}
  >{\raggedright\arraybackslash}p{(\linewidth - 4\tabcolsep) * \real{0.3333}}
  >{\raggedright\arraybackslash}p{(\linewidth - 4\tabcolsep) * \real{0.3333}}
  >{\raggedright\arraybackslash}p{(\linewidth - 4\tabcolsep) * \real{0.3333}}@{}}
\caption{Empirical diagnostics associated with KCOR
assumptions.}\label{tbl:si_diagnostics}\tabularnewline
\toprule\noalign{}
\begin{minipage}[b]{\linewidth}\raggedright
Diagnostic
\end{minipage} & \begin{minipage}[b]{\linewidth}\raggedright
Description
\end{minipage} & \begin{minipage}[b]{\linewidth}\raggedright
Observable signal
\end{minipage} \\
\midrule\noalign{}
\endfirsthead
\toprule\noalign{}
\begin{minipage}[b]{\linewidth}\raggedright
Diagnostic
\end{minipage} & \begin{minipage}[b]{\linewidth}\raggedright
Description
\end{minipage} & \begin{minipage}[b]{\linewidth}\raggedright
Observable signal
\end{minipage} \\
\midrule\noalign{}
\endhead
\bottomrule\noalign{}
\endlastfoot
Skip-week sensitivity & Exclude early post-enrollment weeks subject to
dynamic selection. & Stable fitted frailty under varying skip weeks \\
Post-normalization linearity & Assess curvature removal in
cumulative-hazard space. & Approximate linearity after normalization \\
KCOR(t) stability & Inspect KCOR trajectories following anchoring. &
Stabilization rather than drift \\
Quiet-window perturbation & Shift quiet-window boundaries by ± several
weeks. & Parameter and trajectory stability \\
Residual structure & Examine residuals in cumulative-hazard space. & No
systematic curvature or autocorrelation \\
\end{longtable}

\begin{longtable}[]{@{}
  >{\raggedright\arraybackslash}p{(\linewidth - 4\tabcolsep) * \real{0.3333}}
  >{\raggedright\arraybackslash}p{(\linewidth - 4\tabcolsep) * \real{0.3333}}
  >{\raggedright\arraybackslash}p{(\linewidth - 4\tabcolsep) * \real{0.3333}}@{}}
\caption{Identifiability criteria governing KCOR
interpretation.}\label{tbl:si_identifiability}\tabularnewline
\toprule\noalign{}
\begin{minipage}[b]{\linewidth}\raggedright
Criterion
\end{minipage} & \begin{minipage}[b]{\linewidth}\raggedright
Condition
\end{minipage} & \begin{minipage}[b]{\linewidth}\raggedright
Consequence if violated
\end{minipage} \\
\midrule\noalign{}
\endfirsthead
\toprule\noalign{}
\begin{minipage}[b]{\linewidth}\raggedright
Criterion
\end{minipage} & \begin{minipage}[b]{\linewidth}\raggedright
Condition
\end{minipage} & \begin{minipage}[b]{\linewidth}\raggedright
Consequence if violated
\end{minipage} \\
\midrule\noalign{}
\endhead
\bottomrule\noalign{}
\endlastfoot
I1 Diagnostic sufficiency & All required diagnostics pass. & KCOR
interpretable \\
I2 Window alignment & Follow-up overlaps the hypothesized effect window.
& Out-of-window effects not recoverable \\
I3 Stability under perturbation & Estimates robust to tuning of windows
and skips. & Interpretation limited \\
I4 Anchoring validity & Quiet window exhibits post-normalization
linearity. & Anchoring invalid \\
I5 Conservative failure rule & Any failure → diagnostics indicate
non-identifiability. & Analysis treated as not identified; results not
reported \\
\end{longtable}

When diagnostics indicate non-identifiability, the analysis is treated
as not identified and results are not reported; this does not invalidate
the KCOR estimator itself.

\subsection{S3. Positive controls}\label{s3.-positive-controls}

\subsubsection{S3.1 Construction of injected
effects}\label{s3.1-construction-of-injected-effects}

The effect window is a simulation construct used solely for
positive-control validation and does not represent a real-world
intervention period or biological effect window.

Positive controls are constructed by starting from a negative-control
dataset and injecting a known effect into the data-generating process
for one cohort, for example by multiplying the \emph{baseline} hazard by
a constant factor \(r\) over a prespecified interval:

\begin{equation}\protect\phantomsection\label{eq:pos-control-injection}{
h_{0,\mathrm{treated}}(t) = r \cdot h_{0,\mathrm{control}}(t) \quad \text{for } t \in [t_1, t_2],
}\end{equation}

with \(r>1\) for harm and \(0<r<1\) for benefit.

After gamma-frailty normalization (inversion), KCOR should deviate from
1 in the correct direction and with magnitude consistent with the
injected effect (up to discretization and sampling noise). Figure
\ref{fig:pos_control_injected} and Table \ref{tbl:pos_control_summary}
confirm this behavior.

\begin{figure}
\centering
\pandocbounded{\includegraphics[keepaspectratio,alt={Positive control validation: KCOR correctly detects injected effects. Left panels show harm scenario (r=1.2), right panels show benefit scenario (r=0.8). Top row displays cohort hazard curves with effect window shaded. Bottom row shows \textbackslash mathrm\{KCOR\}(t) deviating from 1.0 in the expected direction during the effect window. Uncertainty bands (95\% bootstrap intervals; aggregated cohort--time resampling) are shown. X-axis units are weeks since enrollment.}]{figures/fig_pos_control_injected.png}}
\caption{Positive control validation: KCOR correctly detects injected
effects. Left panels show harm scenario (r=1.2), right panels show
benefit scenario (r=0.8). Top row displays cohort hazard curves with
effect window shaded. Bottom row shows \(\mathrm{KCOR}(t)\) deviating
from 1.0 in the expected direction during the effect window. Uncertainty
bands (95\% bootstrap intervals; aggregated cohort--time resampling) are
shown. X-axis units are weeks since
enrollment.}\label{fig:pos_control_injected}
\end{figure}

\subsection{S4. Control-test specifications and simulation
parameters}\label{s4.-control-test-specifications-and-simulation-parameters}

\subsubsection{S4.0 Summary tables for control-test and simulation
parameters}\label{s4.0-summary-tables-for-control-test-and-simulation-parameters}

\begin{longtable}[]{@{}
  >{\raggedright\arraybackslash}p{(\linewidth - 8\tabcolsep) * \real{0.2000}}
  >{\raggedright\arraybackslash}p{(\linewidth - 8\tabcolsep) * \real{0.2000}}
  >{\raggedright\arraybackslash}p{(\linewidth - 8\tabcolsep) * \real{0.2000}}
  >{\raggedright\arraybackslash}p{(\linewidth - 8\tabcolsep) * \real{0.2000}}
  >{\raggedright\arraybackslash}p{(\linewidth - 8\tabcolsep) * \real{0.2000}}@{}}
\caption{Summary of control-test and simulation parameters referenced in
Sections S4.2--S4.6. Numeric values are fixed unless otherwise noted;
ranges indicate sensitivity
grids.}\label{tbl:si_sim_params}\tabularnewline
\toprule\noalign{}
\begin{minipage}[b]{\linewidth}\raggedright
Section
\end{minipage} & \begin{minipage}[b]{\linewidth}\raggedright
Item
\end{minipage} & \begin{minipage}[b]{\linewidth}\raggedright
Parameter
\end{minipage} & \begin{minipage}[b]{\linewidth}\raggedright
Value
\end{minipage} & \begin{minipage}[b]{\linewidth}\raggedright
Notes
\end{minipage} \\
\midrule\noalign{}
\endfirsthead
\toprule\noalign{}
\begin{minipage}[b]{\linewidth}\raggedright
Section
\end{minipage} & \begin{minipage}[b]{\linewidth}\raggedright
Item
\end{minipage} & \begin{minipage}[b]{\linewidth}\raggedright
Parameter
\end{minipage} & \begin{minipage}[b]{\linewidth}\raggedright
Value
\end{minipage} & \begin{minipage}[b]{\linewidth}\raggedright
Notes
\end{minipage} \\
\midrule\noalign{}
\endhead
\bottomrule\noalign{}
\endlastfoot
S4.2.1 & Synthetic negative control & Data source &
\texttt{example/Frail\_cohort\_mix.xlsx} & Pathological frailty
mixture \\
S4.2.1 & Synthetic negative control & Generation script &
\texttt{code/generate\_pathological\_neg\_control\_figs.py} & \\
S4.2.1 & Synthetic negative control & Cohort A weights & {[}0.20, 0.20,
0.20, 0.20, 0.20{]} & 5 frailty groups \\
S4.2.1 & Synthetic negative control & Cohort B weights & {[}0.30, 0.20,
0.20, 0.20, 0.10{]} & Shifted mixture \\
S4.2.1 & Synthetic negative control & Frailty values & {[}1, 2, 4, 6,
10{]} & Relative multipliers \\
S4.2.1 & Synthetic negative control & Base weekly probability & 0.01
& \\
S4.2.1 & Synthetic negative control & Weekly log-slope & 0.0 & Constant
baseline during quiet periods \\
S4.2.1 & Synthetic negative control & Skip weeks & 2 & \\
S4.2.1 & Synthetic negative control & Normalization weeks & 4 & \\
S4.2.1 & Synthetic negative control & Time horizon & 250 weeks & \\
S4.2.2 & Empirical negative control & Data source & Czech admin registry
data (KCOR\_CMR) & Aggregated cohorts \\
S4.2.2 & Empirical negative control & Generation script &
\texttt{test/negative\_control/code/generate\_negative\_control.py} & \\
S4.2.2 & Empirical negative control & Construction & Age strata remapped
to pseudo-doses & True null preserved \\
S4.2.2 & Empirical negative control & Age mapping & Dose 0→YoB
\{1930,1935\}; Dose 1→\{1940,1945\}; Dose 2→\{1950,1955\} & \\
S4.2.2 & Empirical negative control & Output YoB & 1950 (unvax) or 1940
(vax) & \\
S4.2.2 & Empirical negative control & Sheets processed & 2021\_24,
2022\_06 & \\
S4.3 & Positive control & Generation script &
\texttt{test/positive\_control/code/generate\_positive\_control.py} & \\
S4.3 & Positive control & Initial cohort size & 100,000 per cohort & \\
S4.3 & Positive control & Baseline hazard & 0.002 per week & Constant \\
S4.3 & Positive control & Frailty variance & θ0=0.5 (control), θ1=1.0
(treatment) & \\
S4.3 & Positive control & Effect window & weeks 20--80 & \\
S4.3 & Positive control & Hazard multipliers & r=1.2 (harm); r=0.8
(benefit) & \\
S4.3 & Positive control & Random seed & 42 & \\
S4.3 & Positive control & Enrollment date & 2021-06-14 (ISO week
2021\_24) & \\
S4.4 & Sensitivity analysis & Baseline weeks & {[}2,3,4,5,6,7,8{]} &
Varied \\
S4.4 & Sensitivity analysis & Quiet-start offsets &
{[}-12,-8,-4,0,+4,+8,+12{]} & Weeks from 2023-01 \\
S4.4 & Sensitivity analysis & Quiet-window end & 2023-52 & Fixed \\
S4.4 & Sensitivity analysis & Dose pairs & 1 vs 0; 2 vs 0; 2 vs 1 & \\
S4.4 & Sensitivity analysis & Cohorts & 2021\_24 & \\
S4.5 & Tail-sampling (adversarial) & Generation script &
\texttt{test/sim\_grid/code/generate\_tail\_sampling\_sim.py} & \\
S4.5 & Tail-sampling (adversarial) & Base frailty distribution &
Log-normal, mean 1, variance 0.5 & \\
S4.5 & Tail-sampling (adversarial) & Mid-quantile cohort & 25th--75th
percentile & Renormalized to mean 1 \\
S4.5 & Tail-sampling (adversarial) & Tail-mixture cohort & {[}0--15th{]}
+ {[}85th--100th{]}, equal weights & Weights yield mean 1 \\
S4.5 & Tail-sampling (adversarial) & Baseline hazard & 0.002 per week &
Constant \\
S4.5 & Tail-sampling (adversarial) & Positive-control multiplier & r=1.2
(harm) or r=0.8 (benefit) & \\
S4.5 & Tail-sampling (adversarial) & Effect window & weeks 20--80 & \\
S4.5 & Tail-sampling (adversarial) & Random seed & 42 & \\
S4.6 & Joint frailty + effect & Generation script &
\texttt{test/sim\_grid/code/generate\_s7\_sim.py} & \\
S4.6 & Joint frailty + effect & Time horizon & 260 weeks & \\
S4.6 & Joint frailty + effect & Cohort size & 2,000,000 per cohort & \\
S4.6 & Joint frailty + effect & Frailty distribution & Gamma, mean 1 &
θ0=0.3, θ1=0.8 \\
S4.6 & Joint frailty + effect & Baseline hazard & 0.002 per week &
Constant \\
S4.6 & Joint frailty + effect & Quiet window & weeks 80--140 &
Prespecified for frailty estimation \\
S4.6 & Joint frailty + effect & Effect windows & weeks 10--25 (early),
150--190 (late) & Overlap variant: 70--95 \\
S4.6 & Joint frailty + effect & Effect shapes & step, ramp, bump & \\
S4.6 & Joint frailty + effect & Effect multiplier & r=1.2 (harm); r=0.8
(benefit) & Applied to treated cohort \\
S4.6 & Joint frailty + effect & Skip weeks & 2 & \\
S4.6 & Joint frailty + effect & Random seed & 42 & \\
S4.6 & Joint frailty + effect & Enrollment date & 2021-06-14 & ISO week
2021\_24 \\
\end{longtable}

\subsubsection{S4.1 Reference implementation and default operational
settings}\label{s4.1-reference-implementation-and-default-operational-settings}

\begin{longtable}[]{@{}
  >{\raggedright\arraybackslash}p{(\linewidth - 6\tabcolsep) * \real{0.2500}}
  >{\raggedright\arraybackslash}p{(\linewidth - 6\tabcolsep) * \real{0.2500}}
  >{\raggedright\arraybackslash}p{(\linewidth - 6\tabcolsep) * \real{0.2500}}
  >{\raggedright\arraybackslash}p{(\linewidth - 6\tabcolsep) * \real{0.2500}}@{}}
\caption{Reference implementation and default operational
settings.}\label{tbl:si_defaults}\tabularnewline
\toprule\noalign{}
\begin{minipage}[b]{\linewidth}\raggedright
Component
\end{minipage} & \begin{minipage}[b]{\linewidth}\raggedright
Setting
\end{minipage} & \begin{minipage}[b]{\linewidth}\raggedright
Default value
\end{minipage} & \begin{minipage}[b]{\linewidth}\raggedright
Notes
\end{minipage} \\
\midrule\noalign{}
\endfirsthead
\toprule\noalign{}
\begin{minipage}[b]{\linewidth}\raggedright
Component
\end{minipage} & \begin{minipage}[b]{\linewidth}\raggedright
Setting
\end{minipage} & \begin{minipage}[b]{\linewidth}\raggedright
Default value
\end{minipage} & \begin{minipage}[b]{\linewidth}\raggedright
Notes
\end{minipage} \\
\midrule\noalign{}
\endhead
\bottomrule\noalign{}
\endlastfoot
Cohort construction & Cohort indexing & Enrollment period × YearOfBirth
group × Dose; plus all-ages cohort (YearOfBirth = -2) & Implementation
detail \\
Quiet-period selection & Quiet window & ISO weeks 2023-01 through
2023-52 & Calendar year 2023 \\
Early-period stabilization (dynamic HVE) & \texttt{SKIP\_WEEKS} & 2 &
Weeks \(t < \mathrm{SKIP\_WEEKS}\) are excluded from hazard accumulation
(set \(\Delta H_d(t)=0\) for those weeks). \\
Frailty estimation & Fit method & Nonlinear least squares in
cumulative-hazard space & Constraints: \(k_d>0\), \(\theta_d \ge 0\) \\
\end{longtable}

\subsubsection{S4.2 Negative controls}\label{s4.2-negative-controls}

Negative controls are used to evaluate the behavior of KCOR under
settings where the true effect is known to be null, while allowing
substantial heterogeneity in baseline risk and selection-induced
depletion. Two complementary classes of negative controls are
considered: (i) fully synthetic simulations that induce strong depletion
curvature through frailty-mixture imbalance, and (ii) empirical
registry-based constructions that preserve a true null by repurposing
age strata as pseudo-exposures without selective sampling. Together,
these controls assess whether KCOR remains stable in the presence of
non-proportional hazards arising from selection rather than treatment.

\paragraph{S4.2.1 Synthetic negative control: gamma-frailty
null}\label{s4.2.1-synthetic-negative-control-gamma-frailty-null}

The synthetic negative control (Figure \ref{fig:neg_control_synthetic})
is a fully specified simulation designed to induce \textbf{strong
selection-induced depletion curvature under a true null effect} by
altering only the cohort frailty-mixture weights. KCOR is expected to
remain near 1 after depletion normalization despite large differences in
cohort-level hazard curvature.

Parameter values and scripts are summarized in Table
\ref{tbl:si_sim_params}.

Both cohorts share identical per-frailty-group death probabilities; only
the mixture weights differ. This induces different cohort-level
curvature under the null.

Figure \ref{fig:si_kcor_bias_vs_theta} provides a compact summary of
KCOR bias as a function of frailty variance \(\theta\) under the same
synthetic-null grid used in Table \ref{tbl:cox_bias_demo}.

\begin{figure}
\centering
\pandocbounded{\includegraphics[keepaspectratio,alt={Simulated-null summary: KCOR bias as a function of frailty variance \textbackslash theta. Bias is defined as \textbackslash mathrm\{KCOR\}\_\{\textbackslash mathrm\{asymptote\}\} - 1 at the end of follow-up in the synthetic-null grid (no treatment effect), reflecting cumulative deviation under the null rather than instantaneous hazard bias. Points show single-run estimates from the grid; no error bars are shown.}]{figures/fig_si_kcor_bias_vs_theta.png}}
\caption{Simulated-null summary: KCOR bias as a function of frailty
variance \(\theta\). Bias is defined as
\(\mathrm{KCOR}_{\mathrm{asymptote}} - 1\) at the end of follow-up in
the synthetic-null grid (no treatment effect), reflecting cumulative
deviation under the null rather than instantaneous hazard bias. Points
show single-run estimates from the grid; no error bars are
shown.}\label{fig:si_kcor_bias_vs_theta}
\end{figure}

\begin{figure}
\centering
\pandocbounded{\includegraphics[keepaspectratio,alt={Synthetic negative control under strong selection (different curvature) but no effect: \textbackslash mathrm\{KCOR\}(t) remains flat at 1. Top panel shows cohort hazards with different frailty-mixture weights inducing different curvature. Bottom panel shows \textbackslash mathrm\{KCOR\}(t) remaining near 1.0 after normalization, demonstrating successful depletion-neutralization under the null. Uncertainty bands (95\% bootstrap intervals; aggregated cohort--time resampling) are shown.}]{figures/fig_neg_control_synthetic.png}}
\caption{Synthetic negative control under strong selection (different
curvature) but no effect: \(\mathrm{KCOR}(t)\) remains flat at 1. Top
panel shows cohort hazards with different frailty-mixture weights
inducing different curvature. Bottom panel shows \(\mathrm{KCOR}(t)\)
remaining near 1.0 after normalization, demonstrating successful
depletion-neutralization under the null. Uncertainty bands (95\%
bootstrap intervals; aggregated cohort--time resampling) are
shown.}\label{fig:neg_control_synthetic}
\end{figure}

\paragraph{S4.2.2 Empirical negative control: age-shift
construction}\label{s4.2.2-empirical-negative-control-age-shift-construction}

The empirical negative control (Figures \ref{fig:neg_control_10yr} and
\ref{fig:neg_control_20yr}) repurposes registry cohorts to create a
\textbf{true null comparison} while inducing large baseline hazard
differences via 10--20 year age shifts. Because these are
full-population strata rather than selectively sampled subcohorts,
selection-induced depletion is minimal and no gamma-frailty
normalization is applied.

Parameter values and scripts are summarized in Table
\ref{tbl:si_sim_params}.

This construction ensures that dose comparisons are within the same
underlying vaccination category, preserving a true null while inducing
10--20 year age differences.

This contrasts with the synthetic negative control (Section S4.2.1),
where strong, deliberately induced frailty heterogeneity requires
gamma-frailty normalization to recover the null.

\subsubsection{S4.3 Positive control: injected
effect}\label{s4.3-positive-control-injected-effect}

Positive controls are used to verify that KCOR responds appropriately
when a true effect is present. Starting from a negative-control
simulation with no treatment effect, a known multiplicative hazard shift
is injected into one cohort over a prespecified time window. This
construction allows direct assessment of whether KCOR detects both the
direction and timing of the injected effect while remaining stable
outside the effect window.

Parameter values and scripts are summarized in Table
\ref{tbl:si_sim_params}.

The injection multiplies the treatment cohort's baseline hazard by
factor \(r\) during the effect window, while leaving the control cohort
unchanged.

\subsubsection{S4.4 Sensitivity analysis
parameters}\label{s4.4-sensitivity-analysis-parameters}

Sensitivity analyses evaluate the robustness of KCOR estimates to
reasonable variation in analysis choices that do not alter the
underlying data-generating process. Baseline-window length and
quiet-window placement are perturbed over a prespecified range while
holding all other parameters fixed. These analyses assess whether KCOR
behavior is stable to tuning choices that primarily affect normalization
rather than cohort composition.

Parameter values and scripts are summarized in Table
\ref{tbl:si_sim_params}.

Output grids show KCOR(t) values for each parameter combination.

\begin{figure}
\centering
\pandocbounded{\includegraphics[keepaspectratio,alt={Sensitivity analysis summary showing \textbackslash mathrm\{KCOR\}(t) values across parameter grid. Heatmaps display \textbackslash mathrm\{KCOR\}(t) estimates for different combinations of baseline weeks (rows) and quiet-window start offsets (columns). Across all comparisons, \textbackslash mathrm\{KCOR\}(t) varies smoothly and modestly across a wide range of quiet-start offsets and baseline window lengths, with no qualitative changes in sign or magnitude, indicating robustness to reasonable parameter choices. All panels use a unified color scale centered at 1.0 to enable direct visual comparison across dose comparisons.}]{figures/fig_sensitivity_overview.png}}
\caption{Sensitivity analysis summary showing \(\mathrm{KCOR}(t)\)
values across parameter grid. Heatmaps display \(\mathrm{KCOR}(t)\)
estimates for different combinations of baseline weeks (rows) and
quiet-window start offsets (columns). Across all comparisons,
\(\mathrm{KCOR}(t)\) varies smoothly and modestly across a wide range of
quiet-start offsets and baseline window lengths, with no qualitative
changes in sign or magnitude, indicating robustness to reasonable
parameter choices. All panels use a unified color scale centered at 1.0
to enable direct visual comparison across dose
comparisons.}\label{fig:sensitivity_overview}
\end{figure}

\subsubsection{S4.5 Tail-sampling / bimodal selection (adversarial
selection
geometry)}\label{s4.5-tail-sampling-bimodal-selection-adversarial-selection-geometry}

This adversarial simulation evaluates KCOR under extreme but controlled
violations of typical cohort-selection geometry. Two cohorts are
constructed to share identical mean frailty while differing sharply in
how risk is distributed, using mid-quantile sampling versus a
low/high-tail mixture. This setting stress-tests whether depletion
normalization remains effective when frailty heterogeneity is
concentrated in the tails rather than smoothly distributed.

\begin{itemize}
\tightlist
\item
  \textbf{Mid-sampled cohort}: frailty restricted to central quantiles
  (e.g., 25th--75th percentile) and renormalized to mean 1.
\item
  \textbf{Tail-sampled cohort}: mixture of low and high tails (e.g.,
  0--15th and 85th--100th percentiles) with mixture weights chosen to
  yield mean 1.
\end{itemize}

Parameter values and scripts are summarized in Table
\ref{tbl:si_sim_params}.

Both cohorts share the same baseline hazard \(h_0(t)\) and have no
treatment effect (negative-control version). Positive-control versions
are also generated by applying a known hazard multiplier in a
prespecified window. The evaluation includes (i) KCOR drift, (ii)
quiet-window fit RMSE, (iii) post-normalization linearity, and (iv)
parameter stability under window perturbation.

\subsubsection{S4.6 Joint frailty and treatment-effect
simulation}\label{s4.6-joint-frailty-and-treatment-effect-simulation}

This simulation evaluates KCOR under conditions in which \textbf{both
selection-induced depletion (frailty heterogeneity)} and a \textbf{true
treatment effect (harm or benefit)} are present simultaneously. The
purpose is to assess whether KCOR can (i) correctly identify and
neutralize frailty-driven curvature using a quiet period and (ii) detect
a true treatment effect outside that period without confounding the two
mechanisms.

This joint simulation combines the selection-induced depletion
mechanisms examined in Sections S4.2 and S4.5 with the injected-effect
framework of Section S4.3.

Parameter values and scripts for this joint simulation are summarized in
Table \ref{tbl:si_sim_params}.

\paragraph{Design}\label{design}

Two fixed cohorts are generated with identical baseline hazards but
differing frailty variance. Individual hazards are multiplicatively
scaled by a latent frailty term drawn from a gamma distribution with
unit mean and cohort-specific variance. A treatment effect is then
injected over a prespecified time window that does not overlap the quiet
period used for frailty estimation.

Formally, individual hazards are generated as

\begin{equation}\protect\phantomsection\label{eq:si_individual_hazard_with_effect}{
h_i(t) = z_i \cdot h_0(t) \cdot r(t).
}\end{equation}

where \(z_i\) is individual frailty, \(h_0(t)\) is a shared baseline
hazard, and \(r(t)\) is a time-localized multiplicative treatment effect
applied to one cohort only.

\subsection{S5. Additional figures and
diagnostics}\label{s5.-additional-figures-and-diagnostics}

This section provides diagnostic outputs and evaluation criteria for the
simulations and control-test specifications defined in Section S4.

\subsubsection{S5.1 Fit diagnostics}\label{s5.1-fit-diagnostics}

For each cohort \(d\), the gamma-frailty fit produces diagnostic outputs
including:

\begin{itemize}
\tightlist
\item
  \textbf{RMSE in \(H\)-space}: Root mean squared error between observed
  and model-predicted cumulative hazards over the quiet window. Values
  \textless{} 0.01 indicate excellent fit; values \textgreater{} 0.05
  may warrant investigation.
\item
  \textbf{Fitted parameters}: baseline hazard level and frailty
  variance. Very small frailty variance (\textless{} 0.01) indicates
  minimal detected depletion; very large values (\textgreater{} 5) may
  indicate model stress.
\item
  \textbf{Number of fit points}: \(n_{\mathrm{obs}}\) observations in
  quiet window. Larger \(n_{\mathrm{obs}}\) provides more stable
  estimates.
\end{itemize}

When uncertainty bands or bootstrap intervals are reported in this
supplement, they are computed using an aggregated-data bootstrap at the
cohort--time level (resampling event counts and risk-set sizes within
time bins/strata), not by resampling individuals.

Example diagnostic output from the reference implementation:

\begin{verbatim}
KCOR_FIT,EnrollmentDate=2021_24,YoB=1950,Dose=0,
  k_hat=4.29e-03,theta_hat=8.02e-01,
  RMSE_Hobs=3.37e-03,n_obs=97,success=1
\end{verbatim}

\subsubsection{S5.2 Residual analysis}\label{s5.2-residual-analysis}

Fit residuals should be examined for. Define residuals:

\[
r_{d}(t)=H_{\mathrm{obs},d}(t)-H_{d}^{\mathrm{model}}(t;\hat{k}_d,\hat{\theta}_d).
\]

\begin{itemize}
\tightlist
\item
  \textbf{Systematic patterns}: Residuals should be approximately random
  around zero. Systematic curvature in residuals suggests model
  inadequacy.
\item
  \textbf{Outliers}: Individual weeks with large residuals may indicate
  data quality issues or external shocks.
\item
  \textbf{Autocorrelation}: Strong autocorrelation in residuals suggests
  the model is missing time-varying structure.
\end{itemize}

\subsubsection{S5.3 Parameter stability
checks}\label{s5.3-parameter-stability-checks}

Robustness of fitted parameters is assessed by:

\begin{itemize}
\tightlist
\item
  \textbf{Quiet-window perturbation}: Shift the quiet-window start/end
  by ±4 weeks and re-fit. Stable parameters should vary by \textless{}
  10\%.
\item
  \textbf{Skip-weeks sensitivity}: Vary SKIP\_WEEKS from 0 to 8 and
  verify KCOR(t) trajectories remain qualitatively similar.
\item
  \textbf{Baseline-shape alternatives}: Compare the default constant
  baseline over the fit window to mild linear trends and verify
  normalization is not sensitive to this choice.
\end{itemize}

\subsubsection{S5.4 Quiet-window overlay
plots}\label{s5.4-quiet-window-overlay-plots}

Overlaying the prespecified quiet window on hazard and cumulative-hazard
time series plots provides a visual diagnostic of window placement
relative to mortality dynamics. The fit window should:

\begin{itemize}
\tightlist
\item
  Avoid major epidemic waves or external mortality shocks
\item
  Contain sufficient event counts for stable estimation
\item
  Span a time range where baseline mortality is approximately stationary
\end{itemize}

Visual inspection of quiet-window placement relative to mortality
dynamics is an essential diagnostic step.

\subsubsection{S5.5 Robustness to age
stratification}\label{s5.5-robustness-to-age-stratification}

This subsection illustrates robustness of \(\mathrm{KCOR}(t)\) to narrow
age stratification by repeating the same fixed-cohort comparison in
three single birth-year cohorts spanning advanced ages (1930, 1940,
1950). Across these strata, the trajectories remain qualitatively stable
after depletion normalization, supporting the claim that the observed
behavior is not an artifact of age aggregation.

\begin{figure}
\centering
\pandocbounded{\includegraphics[keepaspectratio,alt={Birth-year cohort 1930: KCOR(t) trajectories comparing dose 2 and dose 3 to dose 0 for cohorts enrolled in ISO week 2022-26 and evaluated over calendar year 2023. KCOR curves are anchored at t\_0 = 4 weeks (i.e., plotted as \textbackslash mathrm\{KCOR\}(t; t\_0)). This figure is presented as an illustrative application demonstrating estimator behavior on registry data and does not support causal inference. X-axis units are weeks since enrollment.}]{figures/supplement/kcor_realdata_yob1930_enroll2022w26_eval2023.png}}
\caption{Birth-year cohort 1930: KCOR(t) trajectories comparing dose 2
and dose 3 to dose 0 for cohorts enrolled in ISO week 2022-26 and
evaluated over calendar year 2023. KCOR curves are anchored at
\(t_0 = 4\) weeks (i.e., plotted as \(\mathrm{KCOR}(t; t_0)\)). This
figure is presented as an illustrative application demonstrating
estimator behavior on registry data and does not support causal
inference. X-axis units are weeks since
enrollment.}\label{fig:si_yob1930}
\end{figure}

\begin{figure}
\centering
\pandocbounded{\includegraphics[keepaspectratio,alt={Birth-year cohort 1940: KCOR(t) trajectories comparing dose 2 and dose 3 to dose 0 for cohorts enrolled in ISO week 2022-26 and evaluated over calendar year 2023. KCOR curves are anchored at t\_0 = 4 weeks (i.e., plotted as \textbackslash mathrm\{KCOR\}(t; t\_0)). This figure is presented as an illustrative application demonstrating estimator behavior on registry data and does not support causal inference. X-axis units are weeks since enrollment.}]{figures/supplement/kcor_realdata_yob1940_enroll2022w26_eval2023.png}}
\caption{Birth-year cohort 1940: KCOR(t) trajectories comparing dose 2
and dose 3 to dose 0 for cohorts enrolled in ISO week 2022-26 and
evaluated over calendar year 2023. KCOR curves are anchored at
\(t_0 = 4\) weeks (i.e., plotted as \(\mathrm{KCOR}(t; t_0)\)). This
figure is presented as an illustrative application demonstrating
estimator behavior on registry data and does not support causal
inference. X-axis units are weeks since
enrollment.}\label{fig:si_yob1940}
\end{figure}

\begin{figure}
\centering
\pandocbounded{\includegraphics[keepaspectratio,alt={Birth-year cohort 1950: KCOR(t) trajectories comparing dose 2 and dose 3 to dose 0 for cohorts enrolled in ISO week 2022-26 and evaluated over calendar year 2023. KCOR curves are anchored at t\_0 = 4 weeks (i.e., plotted as \textbackslash mathrm\{KCOR\}(t; t\_0)). This figure is presented as an illustrative application demonstrating estimator behavior on registry data and does not support causal inference. X-axis units are weeks since enrollment.}]{figures/supplement/kcor_realdata_yob1950_enroll2022w26_eval2023.png}}
\caption{Birth-year cohort 1950: KCOR(t) trajectories comparing dose 2
and dose 3 to dose 0 for cohorts enrolled in ISO week 2022-26 and
evaluated over calendar year 2023. KCOR curves are anchored at
\(t_0 = 4\) weeks (i.e., plotted as \(\mathrm{KCOR}(t; t_0)\)). This
figure is presented as an illustrative application demonstrating
estimator behavior on registry data and does not support causal
inference. X-axis units are weeks since
enrollment.}\label{fig:si_yob1950}
\end{figure}

\textbf{Additional empirical negative-control variant (20-year age
shift).}\\
For completeness, the more extreme 20-year age-shift negative control
referenced in the main text is included:

\begin{figure}
\centering
\pandocbounded{\includegraphics[keepaspectratio,alt={Empirical negative control with approximately 20-year age difference between cohorts. Even under extreme composition differences, \textbackslash mathrm\{KCOR\}(t) exhibits no systematic drift, consistent with robustness to selection-induced curvature. KCOR curves are anchored at t\_0 = 4 weeks (i.e., plotted as \textbackslash mathrm\{KCOR\}(t; t\_0)). Uncertainty bands (95\% bootstrap intervals) are shown. Data source: Czech Republic mortality and vaccination dataset processed into KCOR\_CMR aggregated format (negative-control construction; see Supplementary Information, SI). X-axis units are weeks since enrollment.}]{figures/fig3_neg_control_20yr_age_diff.png}}
\caption{Empirical negative control with approximately 20-year age
difference between cohorts. Even under extreme composition differences,
\(\mathrm{KCOR}(t)\) exhibits no systematic drift, consistent with
robustness to selection-induced curvature. KCOR curves are anchored at
\(t_0 = 4\) weeks (i.e., plotted as \(\mathrm{KCOR}(t; t_0)\)).
Uncertainty bands (95\% bootstrap intervals) are shown. Data source:
Czech Republic mortality and vaccination dataset processed into
KCOR\_CMR aggregated format (negative-control construction; see
Supplementary Information, SI). X-axis units are weeks since
enrollment.}\label{fig:neg_control_20yr}
\end{figure}

\subsection{S6. Extended Czech empirical
application}\label{s6.-extended-czech-empirical-application}

\subsubsection{S6.1 Empirical application with diagnostic validation:
Czech Republic national registry mortality
data}\label{s6.1-empirical-application-with-diagnostic-validation-czech-republic-national-registry-mortality-data}

The Czech results do not validate KCOR; they represent an application
that satisfies all pre-specified diagnostic criteria. Substantive
implications follow only if the identification assumptions hold.
Throughout this subsection, observed divergences are interpreted
strictly as properties of the estimator under real-world selection, not
as intervention effects.

Unless otherwise noted, KCOR curves in the Czech analyses are shown
anchored at \(t_0 = 4\) weeks for interpretability.

\paragraph{S6.1.1 Illustrative empirical context: COVID-19 mortality
data}\label{s6.1.1-illustrative-empirical-context-covid-19-mortality-data}

The COVID-19 vaccination period provides a natural empirical regime
characterized by strong selection heterogeneity and non-proportional
hazards, making it a useful illustration for the KCOR framework. During
this period, vaccine uptake was voluntary, rapidly time-varying, and
correlated with baseline health status, creating clear examples of
selection-induced non-proportional hazards. The Czech Republic national
mortality registry data exemplify this regime: voluntary uptake led to
asymmetric selection at enrollment, with vaccinated cohorts exhibiting
minimal frailty heterogeneity while unvaccinated cohorts retained
substantial heterogeneity. This asymmetric pattern reflects the healthy
vaccinee effect operating through selective uptake rather than
treatment. KCOR normalization removes this selection-induced curvature,
enabling interpretable cumulative comparisons. While these examples
illustrate KCOR's application, the method is general and applies to any
retrospective cohort comparison where selection induces differential
depletion dynamics.

\paragraph{S6.1.2 Frailty normalization behavior under empirical
validation}\label{s6.1.2-frailty-normalization-behavior-under-empirical-validation}

Across examined age strata in the Czech Republic mortality dataset,
fitted frailty parameters exhibit a pronounced asymmetry across cohorts.
Some cohorts show negligible estimated frailty variance:

\begin{equation}\protect\phantomsection\label{eq:si_theta_near_zero}{
\hat{\theta}_d \approx 0.
}\end{equation}

while others exhibit substantial frailty-driven depletion. This pattern
reflects differences in selection-induced hazard curvature at cohort
entry rather than any prespecified cohort identity.

As a consequence, KCOR normalization leaves some cohorts' cumulative
hazards nearly unchanged, while substantially increasing the
depletion-neutralized baseline cumulative hazard for others. This
behavior is consistent with curvature-driven normalization rather than
cohort identity. This pattern is visible directly in
depletion-neutralized versus observed cumulative hazard plots and is
summarized quantitatively in the fitted-parameter logs (see
\texttt{KCOR\_summary.log}).

After frailty normalization, the depletion-neutralized baseline
cumulative hazards are approximately linear in event time. Residual
deviations from linearity reflect real time-varying risk---such as
seasonality or epidemic waves---rather than selection-induced depletion.
This linearization is a diagnostic consistent with successful removal of
depletion-driven curvature under the working model; persistent
nonlinearity or parameter instability indicates model stress or
quiet-window contamination.

Table \ref{tbl:si_diagnostic_gate} summarizes these diagnostic checks
across age strata.

\newpage

\begin{longtable}[]{@{}
  >{\raggedright\arraybackslash}p{(\linewidth - 8\tabcolsep) * \real{0.1702}}
  >{\raggedright\arraybackslash}p{(\linewidth - 8\tabcolsep) * \real{0.1915}}
  >{\raggedright\arraybackslash}p{(\linewidth - 8\tabcolsep) * \real{0.2979}}
  >{\raggedright\arraybackslash}p{(\linewidth - 8\tabcolsep) * \real{0.2021}}
  >{\raggedright\arraybackslash}p{(\linewidth - 8\tabcolsep) * \real{0.1383}}@{}}
\caption{Diagnostic gate for Czech application: KCOR results reported
only where diagnostics
pass.}\label{tbl:si_diagnostic_gate}\tabularnewline
\toprule\noalign{}
\begin{minipage}[b]{\linewidth}\raggedright
Age band (years)
\end{minipage} & \begin{minipage}[b]{\linewidth}\raggedright
Quiet window valid
\end{minipage} & \begin{minipage}[b]{\linewidth}\raggedright
Post-normalization linearity
\end{minipage} & \begin{minipage}[b]{\linewidth}\raggedright
Parameter stability
\end{minipage} & \begin{minipage}[b]{\linewidth}\raggedright
KCOR reported
\end{minipage} \\
\midrule\noalign{}
\endfirsthead
\toprule\noalign{}
\begin{minipage}[b]{\linewidth}\raggedright
Age band (years)
\end{minipage} & \begin{minipage}[b]{\linewidth}\raggedright
Quiet window valid
\end{minipage} & \begin{minipage}[b]{\linewidth}\raggedright
Post-normalization linearity
\end{minipage} & \begin{minipage}[b]{\linewidth}\raggedright
Parameter stability
\end{minipage} & \begin{minipage}[b]{\linewidth}\raggedright
KCOR reported
\end{minipage} \\
\midrule\noalign{}
\endhead
\bottomrule\noalign{}
\endlastfoot
40--49 & Yes & Yes & Yes & Yes \\
50--59 & Yes & Yes & Yes & Yes \\
60--69 & Yes & Yes & Yes & Yes \\
70--79 & Yes & Yes & Yes & Yes \\
80--89 & Yes & Yes & Yes & Yes \\
90--99 & Yes & Yes & Yes & Yes \\
All ages & Yes & Yes & Yes & Yes \\
\end{longtable}

All age strata in the Czech application satisfied the prespecified
diagnostic criteria, permitting KCOR computation and reporting. KCOR
results are not reported for any age stratum where diagnostics indicate
non-identifiability.

\textbf{Interpretation:} In this application, unvaccinated cohorts
exhibit frailty heterogeneity, while Dose 2 cohorts show near-zero
estimated frailty across all age bands, consistent with selective uptake
prior to follow-up:

\begin{equation}\protect\phantomsection\label{eq:si_theta_positive}{
\hat{\theta}_d > 0.
}\end{equation}

for Dose 0 cohorts and

\begin{equation}\protect\phantomsection\label{eq:si_theta_near_zero_dose2}{
\hat{\theta}_d \approx 0.
}\end{equation}

for Dose 2 cohorts. Estimated frailty heterogeneity can appear larger at
younger ages because baseline hazards are low, so proportional
differences across latent risk strata translate into visibly different
short-term hazards before depletion compresses the risk distribution. At
older ages, higher baseline hazard and stronger ongoing depletion can
reduce the apparent dispersion of remaining risk, yielding smaller
fitted \(\theta\) even if latent heterogeneity is not literally smaller.
Frailty variance is largest at younger ages, where low baseline
mortality amplifies the impact of heterogeneity on cumulative hazard
curvature, and declines at older ages where mortality is compressed and
survivors are more homogeneous. Because Table
\ref{tbl:si_frailty_variance} demonstrates selection-induced
heterogeneity, unadjusted cumulative outcome contrasts are expected to
conflate depletion effects with any true treatment differences; see
Table \ref{tbl:si_raw_hazards} for raw cumulative hazards reported as a
pre-normalization diagnostic. KCOR normalization removes the depletion
component, enabling interpretable comparison of the remaining
differences.

These raw contrasts reflect both selection and depletion effects and are
not interpreted causally.

\newpage

\begin{longtable}[]{@{}
  >{\raggedright\arraybackslash}p{(\linewidth - 4\tabcolsep) * \real{0.3077}}
  >{\raggedleft\arraybackslash}p{(\linewidth - 4\tabcolsep) * \real{0.3462}}
  >{\raggedleft\arraybackslash}p{(\linewidth - 4\tabcolsep) * \real{0.3462}}@{}}
\caption{Estimated gamma-frailty variance (fitted frailty variance) by
age band and vaccination status for Czech cohorts enrolled in
2021\_24.}\label{tbl:si_frailty_variance}\tabularnewline
\toprule\noalign{}
\begin{minipage}[b]{\linewidth}\raggedright
Age band (years)
\end{minipage} & \begin{minipage}[b]{\linewidth}\raggedleft
Fitted frailty variance (Dose 0)
\end{minipage} & \begin{minipage}[b]{\linewidth}\raggedleft
Fitted frailty variance (Dose 2)
\end{minipage} \\
\midrule\noalign{}
\endfirsthead
\toprule\noalign{}
\begin{minipage}[b]{\linewidth}\raggedright
Age band (years)
\end{minipage} & \begin{minipage}[b]{\linewidth}\raggedleft
Fitted frailty variance (Dose 0)
\end{minipage} & \begin{minipage}[b]{\linewidth}\raggedleft
Fitted frailty variance (Dose 2)
\end{minipage} \\
\midrule\noalign{}
\endhead
\bottomrule\noalign{}
\endlastfoot
40--49 & 16.79 & \(2.66 \times 10^{-6}\) \\
50--59 & 23.02 & \(1.87 \times 10^{-4}\) \\
60--69 & 13.13 & \(7.01 \times 10^{-18}\) \\
70--79 & 6.98 & \(3.46 \times 10^{-17}\) \\
80--89 & 2.97 & \(2.03 \times 10^{-11}\) \\
90--99 & 0.80 & \(8.66 \times 10^{-16}\) \\
All ages (full population) & 4.98 & \(1.02 \times 10^{-11}\) \\
\end{longtable}

\textbf{Notes:} - The fitted frailty variance quantifies unobserved
frailty heterogeneity and selection-induced depletion within cohorts.
Near-zero values indicate effectively linear cumulative hazards over the
quiet window and are typical of strongly pre-selected cohorts. - Each
entry reports a single fitted gamma-frailty variance for the specified
age band and vaccination status within the 2021\_24 enrollment cohort. -
The ``All ages (full population)'' row corresponds to an independent fit
over the full pooled age range, included as a global diagnostic. - Table
\ref{tbl:si_raw_hazards} reports raw outcome contrasts for ages 40+ (YOB
\(\le 1980\)) where event counts are stable.

\textbf{Diagnostic checks:} - \textbf{Dose ordering:} the fitted frailty
variance is positive for Dose 0 and collapses toward zero for Dose 2
across all age strata, consistent with selective uptake. -
\textbf{Magnitude separation:} Dose 2 estimates are effectively zero
relative to Dose 0, indicating near-linear cumulative hazards rather
than forced curvature. - \textbf{Age coherence:} the fitted frailty
variance decreases at older ages as baseline mortality rises and
survivor populations become more homogeneous; monotonicity is not
imposed. - \textbf{Stability:} No sign reversals, boundary pathologies,
or numerical instabilities are observed. - \textbf{Falsifiability:}
Failure of any one of these checks would constitute evidence against
model adequacy.

\newpage

\begin{longtable}[]{@{}
  >{\raggedright\arraybackslash}p{(\linewidth - 6\tabcolsep) * \real{0.2353}}
  >{\raggedleft\arraybackslash}p{(\linewidth - 6\tabcolsep) * \real{0.3382}}
  >{\raggedleft\arraybackslash}p{(\linewidth - 6\tabcolsep) * \real{0.3529}}
  >{\raggedleft\arraybackslash}p{(\linewidth - 6\tabcolsep) * \real{0.0735}}@{}}
\caption{Ratio of observed cumulative mortality hazards for unvaccinated
(Dose 0) versus fully vaccinated (Dose 2) Czech cohorts enrolled in
2021\_24. (Note: the all-ages row reflects aggregation effects and is
not directly comparable to age-stratified
rows.)}\label{tbl:si_raw_hazards}\tabularnewline
\toprule\noalign{}
\begin{minipage}[b]{\linewidth}\raggedright
Age band (years)
\end{minipage} & \begin{minipage}[b]{\linewidth}\raggedleft
Dose 0 cumulative hazard
\end{minipage} & \begin{minipage}[b]{\linewidth}\raggedleft
Dose 2 cumulative hazard
\end{minipage} & \begin{minipage}[b]{\linewidth}\raggedleft
Ratio
\end{minipage} \\
\midrule\noalign{}
\endfirsthead
\toprule\noalign{}
\begin{minipage}[b]{\linewidth}\raggedright
Age band (years)
\end{minipage} & \begin{minipage}[b]{\linewidth}\raggedleft
Dose 0 cumulative hazard
\end{minipage} & \begin{minipage}[b]{\linewidth}\raggedleft
Dose 2 cumulative hazard
\end{minipage} & \begin{minipage}[b]{\linewidth}\raggedleft
Ratio
\end{minipage} \\
\midrule\noalign{}
\endhead
\bottomrule\noalign{}
\endlastfoot
40--49 & 0.005260 & 0.004117 & 1.2776 \\
50--59 & 0.014969 & 0.009582 & 1.5622 \\
60--69 & 0.045475 & 0.023136 & 1.9655 \\
70--79 & 0.123097 & 0.057675 & 2.1343 \\
80--89 & 0.307169 & 0.167345 & 1.8355 \\
90--99 & 0.776341 & 0.517284 & 1.5008 \\
All ages (full population) & 0.023160 & 0.073323 & 0.3159 \\
\end{longtable}

This table reports unadjusted cumulative hazards derived directly from
the raw data, prior to any frailty normalization or depletion
correction, and is shown to illustrate the magnitude and direction of
selection-induced curvature addressed by KCOR.

Values reflect raw cumulative outcome differences prior to KCOR
normalization and are not interpreted causally due to cohort
non-exchangeability. Cumulative hazards were integrated from cohort
enrollment through the end of available follow-up for the 2021\_24
enrollment window (through week 2024-16), identically for Dose 0 and
Dose 2 cohorts.

\paragraph{S6.1.3 Illustrative application to national registry
mortality
data}\label{s6.1.3-illustrative-application-to-national-registry-mortality-data}

A brief illustrative application is included to demonstrate end-to-end
KCOR behavior on real registry mortality data in a setting that
minimizes timing-driven shocks and window-tuning sensitivity. Cohorts
were enrolled in ISO week 2022-26, and evaluation was restricted to
calendar year 2023, yielding a 26-week post-enrollment buffer before
slope estimation and a prespecified full-year window for assessment.
Frailty parameters were estimated using a prespecified epidemiologically
quiet window (calendar year 2023) to minimize wave-related hazard
variation. This example is intended to illustrate estimator behavior
under real-world selection and heterogeneity and does not support causal
inference.

Figure \ref{fig:si_allages} shows \(\mathrm{KCOR}(t)\) trajectories for
dose 2 and dose 3 relative to dose 0 for an all-ages analysis. An
all-ages analysis is presented as a high-heterogeneity stress test,
since aggregation across age induces substantial baseline hazard and
frailty variation.

\begin{figure}
\centering
\pandocbounded{\includegraphics[keepaspectratio,alt={All-ages stress test: \textbackslash mathrm\{KCOR\}(t) trajectories comparing dose 2 and dose 3 to dose 0 for cohorts enrolled in ISO week 2022-26 and evaluated over calendar year 2023. KCOR curves are anchored at t\_0 = 4 weeks (i.e., plotted as \textbackslash mathrm\{KCOR\}(t; t\_0)). This figure is presented as an illustrative application demonstrating estimator behavior under extreme heterogeneity and does not support causal inference. X-axis units are weeks since enrollment.}]{figures/kcor_realdata_allages_enroll2022w26_eval2023.png}}
\caption{All-ages stress test: \(\mathrm{KCOR}(t)\) trajectories
comparing dose 2 and dose 3 to dose 0 for cohorts enrolled in ISO week
2022-26 and evaluated over calendar year 2023. KCOR curves are anchored
at \(t_0 = 4\) weeks (i.e., plotted as \(\mathrm{KCOR}(t; t_0)\)). This
figure is presented as an illustrative application demonstrating
estimator behavior under extreme heterogeneity and does not support
causal inference. X-axis units are weeks since
enrollment.}\label{fig:si_allages}
\end{figure}

\subsection{S7. Computational environment and runtime
notes}\label{s7.-computational-environment-and-runtime-notes}

\textbf{Environment.} Python 3.11; key dependencies include numpy,
scipy, pandas, and lifelines (for Cox-model comparisons), with plotting
via matplotlib.

\textbf{Compute requirements.} The full simulation grid reproduces in
approximately 1 hour 26 minutes on a 20-core CPU with 128 GB RAM;
smaller subsets reproduce in minutes.

\textbf{Reproduction.} Running \texttt{make\ paper} (or the repository's
top-level build command) regenerates all artifacts from a clean
checkout.

\end{document}
